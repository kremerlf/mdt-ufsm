\chapter{Graphene and silicene nanodomains in a ultra-thin SiC layer for water splitting and hydrogen storage. A first principle study}

First-principles calculations within the density functional theory (DFT) have been addressed to investigate the energetic stability, electronic and optical properties of graphene and silicene nanodomains in a SiC single layer (h-SiC). We observe that graphene domains form a planar structure and give rise to an occupied and an empty electronic levels inside the h-SiC band gap, leading the h-SiC to present a strong optical absorption peak in the visible region. On the other hand, when a silicene nanodomain is present the system is no longer planar and present a corrugated structure similar to the silicene structure. The silicene nanodomain introduce three empty electronic levels within the band gap, leading the h-SiC with optical absorption in the visible region. These results show that a graphene nanodomain in h-SiC is appropriate for optical devices, while silicene nanodomains form almost sp$^3$ quantum dots. This finding suggest that the graphene and silicene nanodomains in a SiC single layer increase the possibility to use h-SiC to produce new electronic and optical devices as well for energy storage by hydrogen adsorption. In fact, we study the H2 and O2 adsorption on the pristine system and on the nanodomains, we observe that the presence of the nanodomais increase the binding energies of the adsorbed molecules \cite{kremer2020graphene}.