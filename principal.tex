\documentclass[12pt,openright,oneside,a4paper,
                chapter=TITLE,section=TITLE,
                brazil]{abntex2}
%
%% PACOTES
%
\usepackage[bottom=2cm,top=3cm,
            left=3cm,right=2cm]{geometry} %% MARGENS
\usepackage[T1]{fontenc}			
\usepackage{newtxtext,newtxmath}	%% times new roman text e math
\usepackage[utf8]{inputenc}			
\usepackage{amsmath}
\usepackage{amsfonts}
\usepackage{amssymb}
\usepackage{makeidx}
\usepackage{color}					
\usepackage{graphicx}				
\usepackage{microtype}				%% MELHORIAS DE JUSTIFICACAO
\usepackage{indentfirst}			
\usepackage[alf,recuo=0.0cm,abnt-etal-cite=2,abnt-etal-list=3,
            abnt-emphasize=bf,abnt-repeated-author-omit=yes,
            abnt-thesis-year=both,abnt-etal-text=default,
            abnt-and-type=e,abnt-full-initials=no,
            bibjustif]{abntex2cite} %% MDT
%
%% utils
%
\usepackage{subfig} 				%% PARA FIGURAS LADO1 A LADO
\usepackage{multirow} 				%% TABELAS MULTICOLUNAS
\usepackage{tablefootnote}	        %% footnote em tabelass

\usepackage{pgf}
\usepackage{lipsum}
\usepackage{multicol}
%
\newtheorem{teo}{Teorema} 			%% DEFINICAO PARA TEOREMAS
%
%% CONFIGURACOES AUXILIARES 
%
%%%%%%%%%%%%%%%%%%%%%%%%%%%%%%%%%%%%%%%%%%%%%%%%%%%%%%%%%%%%%%%%%%%%
% ESTILOS APLICADOS NOS TITULOS
%
\renewcommand{\ABNTEXchapterfont}{\sffamily}
\renewcommand{\ABNTEXchapterfontsize}{\normalfont\bfseries}

\renewcommand{\ABNTEXsectionfont}{\sffamily}
\renewcommand{\ABNTEXsectionfontsize}{\normalfont}

\renewcommand{\ABNTEXsubsectionfont}{\sffamily}
\renewcommand{\ABNTEXsubsectionfontsize}{\normalfont\bfseries}

\renewcommand{\ABNTEXsubsubsectionfont}{\sffamily}
\renewcommand{\ABNTEXsubsubsectionfontsize}{\normalfont\itshape}

\renewcommand{\ABNTEXsubsubsubsectionfont}{\sffamily}
\renewcommand{\ABNTEXsubsubsubsectionfontsize}{\normalsize}

% Estilos aplicados nos titulos que aparecem no sumario
\renewcommand{\cftchapterfont}{\bfseries}
\renewcommand{\cftsectionfont}{\normalfont}
\renewcommand{\cftsubsectionfont}{\bfseries}
\renewcommand{\cftsubsubsectionfont}{\normalfont}
%
%%%%%%%%%%%%%%%%%%%%%%%%%%%%%%%%%%%%%%%%%%%%%%%%%%%%%%%%%%%%%%%%%%%%
% DECLARACOES CABECALHO DO ABNTEX2
% -----
% Cabecalho padrao
\makeevenhead{abntheadings}{\footnotesize\thepage}{}{}
\makeoddhead{abntheadings}{}{}{\footnotesize\thepage}
\makeheadrule{abntheadings}{\textwidth}{0pt}
%
% Cabecalho do inicio do capitulo
\makeoddhead{abntchapfirst}{}{}{}
%
%%%%%%%%%%%%%%%%%%%%%%%%%%%%%%%%%%%%%%%%%%%%%%%%%%%%%%%%%%%%%%%%%%%%
\setlength\parindent{1.25cm}
\setlength\afterchapskip{18pt}
\setlength\aftersecskip{18pt}
\setlength\aftersubsecskip{18pt}
\setlength\aftersubsubsecskip{18pt}
%%%%%%%%%%%%%%%%%%%%%%%%%%%%%%%%5
% caption e fonte MDT style 
\captionsetup{justification=centering,font=small}
% caption para fonte
\newcommand{\source}[1]{\caption*{Fonte: {#1}}} 

%
\pretextual
%
%%	ARQUIVOS EXTERNOS NAO TEXTUAIS
%

%% arquivo completamente opcional, feito pra minha preguiça de digitar

%%%%%%%%%%%%%% \newcommand{\meucomando}{\comandoaserrealizado} --- %
%%%%%%%%%%%%%%---------------------------------------------------- %

% --- DIRAC NOTATION --- %%%%%%%%%%%%%%%%%%%%%%%%%%%%%%%%%%%%%%%%%%%%%%%%%%%%%%%%%%%%
\newcommand{\bra}[1]{\ensuremath{\left\langle#1\right|}}
\newcommand{\ket}[1]{\ensuremath{\left|#1\right\rangle}}
\newcommand{\bracket}[2]{\ensuremath{\left\langle#1 \vphantom{#2}\right| \left. #2 \vphantom{#1}\right\rangle}}
\newcommand{\matrixel}[3]{\ensuremath{\left\langle #1 \vphantom{#2#3} \right| #2 \left| #3 \vphantom{#1#2} \right\rangle}}
%%%%%%%%%%%%%%%%%%%%%%%%%%%%%%%%%%%%%%%%%%%%%%%%%%%%%%%%%%%%%%%%%%%%%%%%%%%
% --- DERIVADAS PRIMEIRAS --- %%%%%%%%%%%%%%%%%%%%%%%%%%%%%%%%%%%%%%%%%%%%%%%%%%%%%%%%%%%%%%%
\newcommand{\ddx}{\frac{\partial}{\partial x}}
\newcommand{\ddy}{\frac{\partial}{\partial y}}
\newcommand{\ddz}{\frac{\partial}{\partial z}}
\newcommand{\ddr}{\frac{\partial}{\partial r}}
\newcommand{\ddtheta}{\frac{\partial}{\partial \theta}}
\newcommand{\ddrho}{\frac{\partial}{\partial \rho}}
\newcommand{\ddt}{\frac{\partial}{\partial t}}
%%%%%%%%%%%%%%%%%%%%%%%%%%%%%%%%%%%%%%%%%%%%%%%%%%%%%%%%%%%%%%%%%%%%%%%%%%%
% --- DERIVADAS SEGUNDAS --- %%%%%%%%%%%%%%%%%%%%%%%%%%%%%%%%%%%%%%%%%%%%%%%%%%%%%%%%%%%%%%%
\newcommand{\dddx}{\frac{\partial^2}{\partial x^2}}
\newcommand{\dddy}{\frac{\partial^2}{\partial y^2}}
\newcommand{\dddz}{\frac{\partial^2}{\partial z^2}}
\newcommand{\dddr}{\frac{\partial^2}{\partial r^2}}
\newcommand{\dddtheta}{\frac{\partial^2}{\partial \theta^2}}
\newcommand{\dddrho}{\frac{\partial^2}{\partial \rho^2}}
\newcommand{\dddt}{\frac{\partial^2}{\partial \t^2}}
%%%%%%%%%%%%%%%%%%%%%%%%%%%%%%%%%%%%%%%%%%%%%%%%%%%%%%%%%%%%%%%%%%%%%%%%%%%
\newcommand{\comut}[2]{\left[#1,#2\right]}

%%%%%%%%%%%%%%
\newcommand{\rR}{(\vec{r},\vec{R})} % coordenadas eletronicas e nucleares
\newcommand{\angstrom}{\textup{\AA}} %% angstrom
\newcommand{\den}{\rho(\vec{r})} 
\newcommand{\deno}{\rho_0(\vec{r})}
\newcommand{\vr}{\vec{r}}
\newcommand{\vrl}{\vec{r}'}

%%%%%%%%%%%% PALAVRAS ESTRANGEIRAS UTEIS
\usepackage{xspace}
\newcommand{\armchair}{\textit{armchair}\xspace}
\newcommand{\zigzag}{\textit{zigzag}\xspace}
\newcommand{\gap}{\textit{gap}\xspace}
\newcommand{\bulk}{\textit{bulk}\xspace}
\newcommand{\exchange}{\textit{exchange}\xspace}
\newcommand{\software}{\textit{software}\xspace}
		%% COMANDOS PERSONALIZADOS DO USUARIO (se tiver um)
\titulo{Sobre os Blas da vida}
%
\tituloestrangeiro{On the Blas of life}
%
\autor{Blaucius Bla}
%
\orientador{Profº Dr. Bleucius Ble}
%
\tipotrabalho{Tese de Doutorado}
%
\data{2022}
%
\local{Santa Maria, RS}
%
%% Alterar conforme curso, pagina 19 MDT
%
\instituicao{%
UNIVERSIDADE FEDERAL DE SANTA MARIA
\\
CENTRO DE CIÊNCIAS NATURAIS E EXATAS
\\
PROGRAMA DE PÓS-GRADUAÇÃO EM FÍSICA}
			%% DADOS SOBRE O TRABALHO
\renewcommand{\imprimircapa}{%
\begin{capa}%
\center
% cabecalho letra 14pt 
\begingroup
    \fontsize{14pt}{14pt}\selectfont
    \imprimirinstituicao
\endgroup
\vspace{8em}

% oito espaços simples depois do cabecalho fonte 14pt
\begingroup
    \fontsize{14pt}{14pt}\selectfont
    \imprimirautor
\endgroup
\vspace{8em}

%titulo fonte 14pt negrito, 8 espacos simples depois autor
% nao pode ter mais de 3 linhas 
\begingroup
    \fontsize{14pt}{14pt}\selectfont
    \textbf{\expandafter\MakeUppercase\expandafter{\imprimirtitulo}}
\endgroup
\vfill

%fica na margem inferior
\begingroup
    \fontsize{14pt}{14pt}\selectfont
    \imprimirlocal\\
    \imprimirdata
\endgroup
\end{capa}
}			%% MODELO JUNTO AO DOCUMENTO PRINCIPAL
\setcounter{page}{1}
\renewcommand{\imprimirfolhaderosto}{%
\begin{folhaderosto}%
\center
\begingroup
    \fontsize{12pt}{12pt}\selectfont
    \imprimirautor
\endgroup
\vspace{7em}

\begingroup
    \fontsize{12pt}{12pt}\selectfont
    \linespread{1.5}
    \textbf{\expandafter\MakeUppercase\expandafter{\imprimirtitulo}}
\endgroup
\vspace{7em}
%
%% Alterar conforme pagina 24 da MDT
%
\begin{flushright}
\begin{minipage}{.50\textwidth}\linespread{1.0}
Tese apresentada ao Programa de Pós-Graduação em Física da Universidade Federal de Santa Maria (UFSM, RS), como requisito parcial para obtenção do grau de Doutor em Física.
\end{minipage}
\end{flushright}
\vspace{10em} 

\begingroup
    \fontsize{12pt}{12pt}\selectfont
    Orientador: \imprimirorientador
\endgroup
\vfill

\begingroup
    \fontsize{12pt}{12pt}\selectfont
 	\imprimirlocal \\
	\imprimirdata
\endgroup

\end{folhaderosto}
}	%% MODELO JUNTO AO DOCUMENTO PRINCIPAL
%
%%
%
\makeindex
\begin{document}
%
\imprimircapa 					%% COMANDO LITERAL ABNTEX2
\imprimirfolhaderosto 			%% COMANDO LITERAL ABNTEX2
% individual, feito no site da biblioteca central
% ver pagina 26 da MDT

\addtocounter{page}{-1}
\begin{fichacatalografica}
O presente trabalho foi realizado com apoio da Coordenação de Aperfeiçoamento de Pessoal de Nível Superior - Brasil (CAPES) - Código de Financiamento 001
\vspace*{\fill}
% \begin{figure}[h!]
% \centering
% \includegraphics[scale=1]{fig/ficha-catalografica}
% \end{figure}
\noindent 2018\\
Todos os direitos autorais reservados a \imprimirautor. A reprodução de partes ou do todo deste trabalho só poderá ser feita mediante a citação da fonte.\\
E-mail: blauciusbla@bla.com
\end{fichacatalografica}

% ATENÇÃO!
% Visando atender à exigência da CAPES, conforme Portaria nº 206, de 4 de setembro de
% 2018, 8 deve ser incluída a seguinte nota:
% O presente trabalho foi realizado com apoio da Coordenação de Aperfeiçoamento de Pessoal
% de Nível Superior - Brasil (CAPES) - Código de Financiamento 001
% This study was financed in part by the Coordenação de Aperfeiçoamento de Pessoal de Nível
% Superior - Brasil (CAPES) - Finance Code 001

\begin{folhadeaprovacao}
\center
\begingroup
    \fontsize{12pt}{12pt}\selectfont
    \textbf{\imprimirautor}
\endgroup
\vspace{7em}

\begingroup
    \fontsize{12pt}{12pt}\selectfont
    \textbf{\expandafter\MakeUppercase\expandafter{\imprimirtitulo}}
\endgroup
\vspace{7em}
%
%% Mesmo texto da folha de rosto, mais informações 
%% nas paginas 29 e 30 da MDT. 
%
\begin{flushright}
\begin{minipage}{.50\textwidth}Tese apresentada ao Programa de Pós-Graduação em Física da Universidade Federal de Santa Maria (UFSM, RS), como requisito parcial para obtenção do grau de \textbf{Doutor em Física}.
\end{minipage}
\end{flushright}
\vspace{4em}


\textbf{Aprovado em XX de Mês de XXXX:}
\vspace{48pt}


\makebox[8cm]{\hrulefill}\\
, Dr. (UFSM)\\
(Presidente/Orientador)
\vspace{1.5em}

\makebox[8cm]{\hrulefill}\\

, Dr. (UFSM)
\vspace{1.5em}

\makebox[8cm]{\hrulefill}\\
, Dr. (UFSM)
\vspace{1.5em}

\makebox[8cm]{\hrulefill}\\

, Dr. (EXT)
\vspace{1.5em}

\makebox[8cm]{\hrulefill}\\
, Dr. (EXT)

\vfill

\begingroup
    \fontsize{12pt}{12pt}\selectfont
    \imprimirlocal\\
    \imprimirdata
\endgroup

\end{folhadeaprovacao}
		%% MODELO JUNTO AO DOCUMENTO PRINCIPAL
\linespread{1.5}
\begin{dedicatoria}
\vspace*{\fill}
\center{Aos meus pais Adelar e Haidi}
\vspace*{\fill}
\end{dedicatoria}			%% MODELO JUNTO AO DOCUMENTO PRINCIPAL
\begin{agradecimentos}[AGRADECIMENTOS]

\lipsum[1]




\end{agradecimentos}		%% MODELO JUNTO AO DOCUMENTO PRINCIPAL
\begin{epigrafe}
\vspace*{\fill}
\begin{flushright}{}


%\begin{minipage}[b]{8cm}

\textit{Speech has allowed the communication of ideas\\
Enabling human beings to work together to build the impossible\\
Mankind's greatest achievements have come about by talking\\
Our greatest hopes could become reality in the future\\
With the technology at our disposal, the possibilities are unbounded\\
All we need to do is make sure we keep talking\\ (\small{Talkin' Hawkin - Pink Floyd })}



%\end{minipage}
\end{flushright}
\end{epigrafe}				%% MODELO JUNTO AO DOCUMENTO PRINCIPAL
\begin{resumo}[RESUMO]
\vspace{2em}
\begin{center}
%se der mais de uma página pode ser escrito em fonte 10pt
\textbf{\expandafter\MakeUppercase\expandafter{\imprimirtitulo}}\\
\vspace{1em}
AUTOR: \imprimirautor\\
ORIENTADOR: \imprimirorientador
\end{center}
\vspace{1em}
% texto
\lipsum[1]

\vspace{2em}
%no minimo tres separadas por .
\textbf{Palavras-chave}: bla. ble. bli. blo. blu.
\end{resumo}				%% MODELO JUNTO AO DOCUMENTO PRINCIPAL
\begin{resumo}[ABSTRACT]
\vspace{2em}
\begin{center}
%se der mais de uma página pode ser escrito em fonte 10pt

\textbf{\expandafter\MakeUppercase\expandafter{\imprimirtituloestrangeiro}}\\
AUTHOR: \imprimirautor\\
ADVISOR: \imprimirorientador
\end{center}
\vspace{1em}
% texto
\lipsum[1]

\vspace{2em}
%no minimo tres separadas por .
\textbf{Keywords}: bla. ble. bli. blo. blu.
\end{resumo}				%% MODELO JUNTO AO DOCUMENTO PRINCIPAL
\listoffigures*					%% O * RETIRA A LISTA DO SUMARIO
\cleardoublepage
\listoftables*
\cleardoublepage
\tableofcontents*				%% IMPRIME SUMARIO
\clearpage						%% retira numeração de pagina do sumario
\textual
%
%% TEXTO
%
\chapter{Introdução}\label{ch:intro}

\lipsum[1]

\begin{figure}[!h]
    \caption{legenda}
    \centering
    %% Creator: Matplotlib, PGF backend
%%
%% To include the figure in your LaTeX document, write
%%   \input{<filename>.pgf}
%%
%% Make sure the required packages are loaded in your preamble
%%   \usepackage{pgf}
%%
%% Also ensure that all the required font packages are loaded; for instance,
%% the lmodern package is sometimes necessary when using math font.
%%   \usepackage{lmodern}
%%
%% Figures using additional raster images can only be included by \input if
%% they are in the same directory as the main LaTeX file. For loading figures
%% from other directories you can use the `import` package
%%   \usepackage{import}
%%
%% and then include the figures with
%%   \import{<path to file>}{<filename>.pgf}
%%
%% Matplotlib used the following preamble
%%   \usepackage{fontspec}
%%
\begingroup%
\makeatletter%
\begin{pgfpicture}%
\pgfpathrectangle{\pgfpointorigin}{\pgfqpoint{3.897177in}{3.045563in}}%
\pgfusepath{use as bounding box, clip}%
\begin{pgfscope}%
\pgfsetbuttcap%
\pgfsetmiterjoin%
\pgfsetlinewidth{0.000000pt}%
\definecolor{currentstroke}{rgb}{1.000000,1.000000,1.000000}%
\pgfsetstrokecolor{currentstroke}%
\pgfsetstrokeopacity{0.000000}%
\pgfsetdash{}{0pt}%
\pgfpathmoveto{\pgfqpoint{0.000000in}{0.000000in}}%
\pgfpathlineto{\pgfqpoint{3.897177in}{0.000000in}}%
\pgfpathlineto{\pgfqpoint{3.897177in}{3.045563in}}%
\pgfpathlineto{\pgfqpoint{0.000000in}{3.045563in}}%
\pgfpathlineto{\pgfqpoint{0.000000in}{0.000000in}}%
\pgfpathclose%
\pgfusepath{}%
\end{pgfscope}%
\begin{pgfscope}%
\pgfsetbuttcap%
\pgfsetmiterjoin%
\definecolor{currentfill}{rgb}{1.000000,1.000000,1.000000}%
\pgfsetfillcolor{currentfill}%
\pgfsetlinewidth{0.000000pt}%
\definecolor{currentstroke}{rgb}{0.000000,0.000000,0.000000}%
\pgfsetstrokecolor{currentstroke}%
\pgfsetstrokeopacity{0.000000}%
\pgfsetdash{}{0pt}%
\pgfpathmoveto{\pgfqpoint{0.664400in}{1.756587in}}%
\pgfpathlineto{\pgfqpoint{3.715581in}{1.756587in}}%
\pgfpathlineto{\pgfqpoint{3.715581in}{2.945563in}}%
\pgfpathlineto{\pgfqpoint{0.664400in}{2.945563in}}%
\pgfpathlineto{\pgfqpoint{0.664400in}{1.756587in}}%
\pgfpathclose%
\pgfusepath{fill}%
\end{pgfscope}%
\begin{pgfscope}%
\pgfsetbuttcap%
\pgfsetroundjoin%
\definecolor{currentfill}{rgb}{0.000000,0.000000,0.000000}%
\pgfsetfillcolor{currentfill}%
\pgfsetlinewidth{0.803000pt}%
\definecolor{currentstroke}{rgb}{0.000000,0.000000,0.000000}%
\pgfsetstrokecolor{currentstroke}%
\pgfsetdash{}{0pt}%
\pgfsys@defobject{currentmarker}{\pgfqpoint{0.000000in}{-0.048611in}}{\pgfqpoint{0.000000in}{0.000000in}}{%
\pgfpathmoveto{\pgfqpoint{0.000000in}{0.000000in}}%
\pgfpathlineto{\pgfqpoint{0.000000in}{-0.048611in}}%
\pgfusepath{stroke,fill}%
}%
\begin{pgfscope}%
\pgfsys@transformshift{0.664400in}{1.756587in}%
\pgfsys@useobject{currentmarker}{}%
\end{pgfscope}%
\end{pgfscope}%
\begin{pgfscope}%
\pgfsetbuttcap%
\pgfsetroundjoin%
\definecolor{currentfill}{rgb}{0.000000,0.000000,0.000000}%
\pgfsetfillcolor{currentfill}%
\pgfsetlinewidth{0.803000pt}%
\definecolor{currentstroke}{rgb}{0.000000,0.000000,0.000000}%
\pgfsetstrokecolor{currentstroke}%
\pgfsetdash{}{0pt}%
\pgfsys@defobject{currentmarker}{\pgfqpoint{0.000000in}{-0.048611in}}{\pgfqpoint{0.000000in}{0.000000in}}{%
\pgfpathmoveto{\pgfqpoint{0.000000in}{0.000000in}}%
\pgfpathlineto{\pgfqpoint{0.000000in}{-0.048611in}}%
\pgfusepath{stroke,fill}%
}%
\begin{pgfscope}%
\pgfsys@transformshift{1.274636in}{1.756587in}%
\pgfsys@useobject{currentmarker}{}%
\end{pgfscope}%
\end{pgfscope}%
\begin{pgfscope}%
\pgfsetbuttcap%
\pgfsetroundjoin%
\definecolor{currentfill}{rgb}{0.000000,0.000000,0.000000}%
\pgfsetfillcolor{currentfill}%
\pgfsetlinewidth{0.803000pt}%
\definecolor{currentstroke}{rgb}{0.000000,0.000000,0.000000}%
\pgfsetstrokecolor{currentstroke}%
\pgfsetdash{}{0pt}%
\pgfsys@defobject{currentmarker}{\pgfqpoint{0.000000in}{-0.048611in}}{\pgfqpoint{0.000000in}{0.000000in}}{%
\pgfpathmoveto{\pgfqpoint{0.000000in}{0.000000in}}%
\pgfpathlineto{\pgfqpoint{0.000000in}{-0.048611in}}%
\pgfusepath{stroke,fill}%
}%
\begin{pgfscope}%
\pgfsys@transformshift{1.884872in}{1.756587in}%
\pgfsys@useobject{currentmarker}{}%
\end{pgfscope}%
\end{pgfscope}%
\begin{pgfscope}%
\pgfsetbuttcap%
\pgfsetroundjoin%
\definecolor{currentfill}{rgb}{0.000000,0.000000,0.000000}%
\pgfsetfillcolor{currentfill}%
\pgfsetlinewidth{0.803000pt}%
\definecolor{currentstroke}{rgb}{0.000000,0.000000,0.000000}%
\pgfsetstrokecolor{currentstroke}%
\pgfsetdash{}{0pt}%
\pgfsys@defobject{currentmarker}{\pgfqpoint{0.000000in}{-0.048611in}}{\pgfqpoint{0.000000in}{0.000000in}}{%
\pgfpathmoveto{\pgfqpoint{0.000000in}{0.000000in}}%
\pgfpathlineto{\pgfqpoint{0.000000in}{-0.048611in}}%
\pgfusepath{stroke,fill}%
}%
\begin{pgfscope}%
\pgfsys@transformshift{2.495109in}{1.756587in}%
\pgfsys@useobject{currentmarker}{}%
\end{pgfscope}%
\end{pgfscope}%
\begin{pgfscope}%
\pgfsetbuttcap%
\pgfsetroundjoin%
\definecolor{currentfill}{rgb}{0.000000,0.000000,0.000000}%
\pgfsetfillcolor{currentfill}%
\pgfsetlinewidth{0.803000pt}%
\definecolor{currentstroke}{rgb}{0.000000,0.000000,0.000000}%
\pgfsetstrokecolor{currentstroke}%
\pgfsetdash{}{0pt}%
\pgfsys@defobject{currentmarker}{\pgfqpoint{0.000000in}{-0.048611in}}{\pgfqpoint{0.000000in}{0.000000in}}{%
\pgfpathmoveto{\pgfqpoint{0.000000in}{0.000000in}}%
\pgfpathlineto{\pgfqpoint{0.000000in}{-0.048611in}}%
\pgfusepath{stroke,fill}%
}%
\begin{pgfscope}%
\pgfsys@transformshift{3.105345in}{1.756587in}%
\pgfsys@useobject{currentmarker}{}%
\end{pgfscope}%
\end{pgfscope}%
\begin{pgfscope}%
\pgfsetbuttcap%
\pgfsetroundjoin%
\definecolor{currentfill}{rgb}{0.000000,0.000000,0.000000}%
\pgfsetfillcolor{currentfill}%
\pgfsetlinewidth{0.803000pt}%
\definecolor{currentstroke}{rgb}{0.000000,0.000000,0.000000}%
\pgfsetstrokecolor{currentstroke}%
\pgfsetdash{}{0pt}%
\pgfsys@defobject{currentmarker}{\pgfqpoint{0.000000in}{-0.048611in}}{\pgfqpoint{0.000000in}{0.000000in}}{%
\pgfpathmoveto{\pgfqpoint{0.000000in}{0.000000in}}%
\pgfpathlineto{\pgfqpoint{0.000000in}{-0.048611in}}%
\pgfusepath{stroke,fill}%
}%
\begin{pgfscope}%
\pgfsys@transformshift{3.715581in}{1.756587in}%
\pgfsys@useobject{currentmarker}{}%
\end{pgfscope}%
\end{pgfscope}%
\begin{pgfscope}%
\pgfsetbuttcap%
\pgfsetroundjoin%
\definecolor{currentfill}{rgb}{0.000000,0.000000,0.000000}%
\pgfsetfillcolor{currentfill}%
\pgfsetlinewidth{0.803000pt}%
\definecolor{currentstroke}{rgb}{0.000000,0.000000,0.000000}%
\pgfsetstrokecolor{currentstroke}%
\pgfsetdash{}{0pt}%
\pgfsys@defobject{currentmarker}{\pgfqpoint{-0.048611in}{0.000000in}}{\pgfqpoint{-0.000000in}{0.000000in}}{%
\pgfpathmoveto{\pgfqpoint{-0.000000in}{0.000000in}}%
\pgfpathlineto{\pgfqpoint{-0.048611in}{0.000000in}}%
\pgfusepath{stroke,fill}%
}%
\begin{pgfscope}%
\pgfsys@transformshift{0.664400in}{1.933247in}%
\pgfsys@useobject{currentmarker}{}%
\end{pgfscope}%
\end{pgfscope}%
\begin{pgfscope}%
\definecolor{textcolor}{rgb}{0.000000,0.000000,0.000000}%
\pgfsetstrokecolor{textcolor}%
\pgfsetfillcolor{textcolor}%
\pgftext[x=0.322389in, y=1.875414in, left, base]{\color{textcolor}\rmfamily\fontsize{12.000000}{14.400000}\selectfont \(\displaystyle {350}\)}%
\end{pgfscope}%
\begin{pgfscope}%
\pgfsetbuttcap%
\pgfsetroundjoin%
\definecolor{currentfill}{rgb}{0.000000,0.000000,0.000000}%
\pgfsetfillcolor{currentfill}%
\pgfsetlinewidth{0.803000pt}%
\definecolor{currentstroke}{rgb}{0.000000,0.000000,0.000000}%
\pgfsetstrokecolor{currentstroke}%
\pgfsetdash{}{0pt}%
\pgfsys@defobject{currentmarker}{\pgfqpoint{-0.048611in}{0.000000in}}{\pgfqpoint{-0.000000in}{0.000000in}}{%
\pgfpathmoveto{\pgfqpoint{-0.000000in}{0.000000in}}%
\pgfpathlineto{\pgfqpoint{-0.048611in}{0.000000in}}%
\pgfusepath{stroke,fill}%
}%
\begin{pgfscope}%
\pgfsys@transformshift{0.664400in}{2.199295in}%
\pgfsys@useobject{currentmarker}{}%
\end{pgfscope}%
\end{pgfscope}%
\begin{pgfscope}%
\definecolor{textcolor}{rgb}{0.000000,0.000000,0.000000}%
\pgfsetstrokecolor{textcolor}%
\pgfsetfillcolor{textcolor}%
\pgftext[x=0.322389in, y=2.141461in, left, base]{\color{textcolor}\rmfamily\fontsize{12.000000}{14.400000}\selectfont \(\displaystyle {600}\)}%
\end{pgfscope}%
\begin{pgfscope}%
\pgfsetbuttcap%
\pgfsetroundjoin%
\definecolor{currentfill}{rgb}{0.000000,0.000000,0.000000}%
\pgfsetfillcolor{currentfill}%
\pgfsetlinewidth{0.803000pt}%
\definecolor{currentstroke}{rgb}{0.000000,0.000000,0.000000}%
\pgfsetstrokecolor{currentstroke}%
\pgfsetdash{}{0pt}%
\pgfsys@defobject{currentmarker}{\pgfqpoint{-0.048611in}{0.000000in}}{\pgfqpoint{-0.000000in}{0.000000in}}{%
\pgfpathmoveto{\pgfqpoint{-0.000000in}{0.000000in}}%
\pgfpathlineto{\pgfqpoint{-0.048611in}{0.000000in}}%
\pgfusepath{stroke,fill}%
}%
\begin{pgfscope}%
\pgfsys@transformshift{0.664400in}{2.465342in}%
\pgfsys@useobject{currentmarker}{}%
\end{pgfscope}%
\end{pgfscope}%
\begin{pgfscope}%
\definecolor{textcolor}{rgb}{0.000000,0.000000,0.000000}%
\pgfsetstrokecolor{textcolor}%
\pgfsetfillcolor{textcolor}%
\pgftext[x=0.322389in, y=2.407509in, left, base]{\color{textcolor}\rmfamily\fontsize{12.000000}{14.400000}\selectfont \(\displaystyle {850}\)}%
\end{pgfscope}%
\begin{pgfscope}%
\definecolor{textcolor}{rgb}{0.000000,0.000000,0.000000}%
\pgfsetstrokecolor{textcolor}%
\pgfsetfillcolor{textcolor}%
\pgftext[x=0.266833in,y=2.351075in,,bottom,rotate=90.000000]{\color{textcolor}\rmfamily\fontsize{12.000000}{14.400000}\selectfont T (K)}%
\end{pgfscope}%
\begin{pgfscope}%
\pgfpathrectangle{\pgfqpoint{0.664400in}{1.756587in}}{\pgfqpoint{3.051181in}{1.188976in}}%
\pgfusepath{clip}%
\pgfsetrectcap%
\pgfsetroundjoin%
\pgfsetlinewidth{1.505625pt}%
\definecolor{currentstroke}{rgb}{0.768627,0.768627,0.768627}%
\pgfsetstrokecolor{currentstroke}%
\pgfsetdash{}{0pt}%
\pgfpathmoveto{\pgfqpoint{0.664705in}{2.199295in}}%
\pgfpathlineto{\pgfqpoint{0.665620in}{2.077456in}}%
\pgfpathlineto{\pgfqpoint{0.666231in}{2.020638in}}%
\pgfpathlineto{\pgfqpoint{0.666536in}{2.060216in}}%
\pgfpathlineto{\pgfqpoint{0.667756in}{2.521191in}}%
\pgfpathlineto{\pgfqpoint{0.668367in}{2.328541in}}%
\pgfpathlineto{\pgfqpoint{0.669587in}{2.008251in}}%
\pgfpathlineto{\pgfqpoint{0.670197in}{2.098516in}}%
\pgfpathlineto{\pgfqpoint{0.671418in}{2.333457in}}%
\pgfpathlineto{\pgfqpoint{0.672333in}{2.296264in}}%
\pgfpathlineto{\pgfqpoint{0.674164in}{2.125951in}}%
\pgfpathlineto{\pgfqpoint{0.675079in}{2.058055in}}%
\pgfpathlineto{\pgfqpoint{0.675384in}{2.081659in}}%
\pgfpathlineto{\pgfqpoint{0.676910in}{2.383166in}}%
\pgfpathlineto{\pgfqpoint{0.677520in}{2.209543in}}%
\pgfpathlineto{\pgfqpoint{0.678741in}{1.970707in}}%
\pgfpathlineto{\pgfqpoint{0.679351in}{2.103837in}}%
\pgfpathlineto{\pgfqpoint{0.680571in}{2.500663in}}%
\pgfpathlineto{\pgfqpoint{0.681181in}{2.317047in}}%
\pgfpathlineto{\pgfqpoint{0.682097in}{2.143212in}}%
\pgfpathlineto{\pgfqpoint{0.683012in}{2.157100in}}%
\pgfpathlineto{\pgfqpoint{0.684233in}{2.111478in}}%
\pgfpathlineto{\pgfqpoint{0.684538in}{2.126611in}}%
\pgfpathlineto{\pgfqpoint{0.685758in}{2.299871in}}%
\pgfpathlineto{\pgfqpoint{0.686369in}{2.245470in}}%
\pgfpathlineto{\pgfqpoint{0.687284in}{2.127387in}}%
\pgfpathlineto{\pgfqpoint{0.687894in}{2.169668in}}%
\pgfpathlineto{\pgfqpoint{0.688504in}{2.235488in}}%
\pgfpathlineto{\pgfqpoint{0.689115in}{2.198241in}}%
\pgfpathlineto{\pgfqpoint{0.690335in}{2.033047in}}%
\pgfpathlineto{\pgfqpoint{0.690640in}{2.086512in}}%
\pgfpathlineto{\pgfqpoint{0.691861in}{2.539208in}}%
\pgfpathlineto{\pgfqpoint{0.692471in}{2.266818in}}%
\pgfpathlineto{\pgfqpoint{0.693386in}{1.940271in}}%
\pgfpathlineto{\pgfqpoint{0.693996in}{2.014487in}}%
\pgfpathlineto{\pgfqpoint{0.695217in}{2.514827in}}%
\pgfpathlineto{\pgfqpoint{0.695827in}{2.298030in}}%
\pgfpathlineto{\pgfqpoint{0.696743in}{1.979710in}}%
\pgfpathlineto{\pgfqpoint{0.697658in}{2.119289in}}%
\pgfpathlineto{\pgfqpoint{0.698573in}{2.354145in}}%
\pgfpathlineto{\pgfqpoint{0.699183in}{2.269319in}}%
\pgfpathlineto{\pgfqpoint{0.700404in}{2.091758in}}%
\pgfpathlineto{\pgfqpoint{0.701014in}{2.174520in}}%
\pgfpathlineto{\pgfqpoint{0.701930in}{2.309290in}}%
\pgfpathlineto{\pgfqpoint{0.702540in}{2.240277in}}%
\pgfpathlineto{\pgfqpoint{0.703150in}{2.153737in}}%
\pgfpathlineto{\pgfqpoint{0.703760in}{2.192005in}}%
\pgfpathlineto{\pgfqpoint{0.704676in}{2.333819in}}%
\pgfpathlineto{\pgfqpoint{0.705286in}{2.252855in}}%
\pgfpathlineto{\pgfqpoint{0.706201in}{2.108562in}}%
\pgfpathlineto{\pgfqpoint{0.707117in}{2.177883in}}%
\pgfpathlineto{\pgfqpoint{0.707727in}{2.209713in}}%
\pgfpathlineto{\pgfqpoint{0.708337in}{2.161931in}}%
\pgfpathlineto{\pgfqpoint{0.708947in}{2.125174in}}%
\pgfpathlineto{\pgfqpoint{0.709557in}{2.187865in}}%
\pgfpathlineto{\pgfqpoint{0.710473in}{2.322943in}}%
\pgfpathlineto{\pgfqpoint{0.711083in}{2.234551in}}%
\pgfpathlineto{\pgfqpoint{0.712304in}{2.061312in}}%
\pgfpathlineto{\pgfqpoint{0.712914in}{2.161112in}}%
\pgfpathlineto{\pgfqpoint{0.713829in}{2.343078in}}%
\pgfpathlineto{\pgfqpoint{0.714439in}{2.250834in}}%
\pgfpathlineto{\pgfqpoint{0.715660in}{2.045913in}}%
\pgfpathlineto{\pgfqpoint{0.716270in}{2.169285in}}%
\pgfpathlineto{\pgfqpoint{0.717185in}{2.381442in}}%
\pgfpathlineto{\pgfqpoint{0.717796in}{2.262859in}}%
\pgfpathlineto{\pgfqpoint{0.718711in}{2.113872in}}%
\pgfpathlineto{\pgfqpoint{0.719321in}{2.174063in}}%
\pgfpathlineto{\pgfqpoint{0.720237in}{2.263253in}}%
\pgfpathlineto{\pgfqpoint{0.720847in}{2.208138in}}%
\pgfpathlineto{\pgfqpoint{0.721762in}{2.129856in}}%
\pgfpathlineto{\pgfqpoint{0.722372in}{2.153428in}}%
\pgfpathlineto{\pgfqpoint{0.725119in}{2.339651in}}%
\pgfpathlineto{\pgfqpoint{0.725424in}{2.308970in}}%
\pgfpathlineto{\pgfqpoint{0.726949in}{2.015030in}}%
\pgfpathlineto{\pgfqpoint{0.727559in}{2.074848in}}%
\pgfpathlineto{\pgfqpoint{0.728780in}{2.385869in}}%
\pgfpathlineto{\pgfqpoint{0.729390in}{2.281408in}}%
\pgfpathlineto{\pgfqpoint{0.730306in}{2.100836in}}%
\pgfpathlineto{\pgfqpoint{0.730916in}{2.150055in}}%
\pgfpathlineto{\pgfqpoint{0.731831in}{2.306438in}}%
\pgfpathlineto{\pgfqpoint{0.732441in}{2.264700in}}%
\pgfpathlineto{\pgfqpoint{0.733662in}{2.100485in}}%
\pgfpathlineto{\pgfqpoint{0.734272in}{2.185216in}}%
\pgfpathlineto{\pgfqpoint{0.735187in}{2.323465in}}%
\pgfpathlineto{\pgfqpoint{0.735798in}{2.225453in}}%
\pgfpathlineto{\pgfqpoint{0.736713in}{2.078169in}}%
\pgfpathlineto{\pgfqpoint{0.737628in}{2.134358in}}%
\pgfpathlineto{\pgfqpoint{0.739764in}{2.311120in}}%
\pgfpathlineto{\pgfqpoint{0.740069in}{2.289102in}}%
\pgfpathlineto{\pgfqpoint{0.741595in}{1.975793in}}%
\pgfpathlineto{\pgfqpoint{0.742205in}{2.038261in}}%
\pgfpathlineto{\pgfqpoint{0.743426in}{2.507474in}}%
\pgfpathlineto{\pgfqpoint{0.744036in}{2.362989in}}%
\pgfpathlineto{\pgfqpoint{0.744951in}{2.101091in}}%
\pgfpathlineto{\pgfqpoint{0.745867in}{2.211661in}}%
\pgfpathlineto{\pgfqpoint{0.746477in}{2.296349in}}%
\pgfpathlineto{\pgfqpoint{0.747087in}{2.235871in}}%
\pgfpathlineto{\pgfqpoint{0.748307in}{2.049148in}}%
\pgfpathlineto{\pgfqpoint{0.748918in}{2.147362in}}%
\pgfpathlineto{\pgfqpoint{0.749833in}{2.426915in}}%
\pgfpathlineto{\pgfqpoint{0.750443in}{2.335894in}}%
\pgfpathlineto{\pgfqpoint{0.751969in}{1.944155in}}%
\pgfpathlineto{\pgfqpoint{0.752579in}{2.019638in}}%
\pgfpathlineto{\pgfqpoint{0.753800in}{2.476538in}}%
\pgfpathlineto{\pgfqpoint{0.754715in}{2.279524in}}%
\pgfpathlineto{\pgfqpoint{0.755630in}{2.132847in}}%
\pgfpathlineto{\pgfqpoint{0.756241in}{2.172860in}}%
\pgfpathlineto{\pgfqpoint{0.757156in}{2.223729in}}%
\pgfpathlineto{\pgfqpoint{0.758071in}{2.207372in}}%
\pgfpathlineto{\pgfqpoint{0.758987in}{2.197209in}}%
\pgfpathlineto{\pgfqpoint{0.760207in}{2.141467in}}%
\pgfpathlineto{\pgfqpoint{0.760817in}{2.175084in}}%
\pgfpathlineto{\pgfqpoint{0.761733in}{2.232221in}}%
\pgfpathlineto{\pgfqpoint{0.762343in}{2.191069in}}%
\pgfpathlineto{\pgfqpoint{0.763563in}{2.098899in}}%
\pgfpathlineto{\pgfqpoint{0.764174in}{2.146436in}}%
\pgfpathlineto{\pgfqpoint{0.765394in}{2.292582in}}%
\pgfpathlineto{\pgfqpoint{0.766004in}{2.211012in}}%
\pgfpathlineto{\pgfqpoint{0.766920in}{2.080254in}}%
\pgfpathlineto{\pgfqpoint{0.767530in}{2.174457in}}%
\pgfpathlineto{\pgfqpoint{0.768445in}{2.512891in}}%
\pgfpathlineto{\pgfqpoint{0.769056in}{2.398181in}}%
\pgfpathlineto{\pgfqpoint{0.770276in}{2.014881in}}%
\pgfpathlineto{\pgfqpoint{0.771191in}{2.103230in}}%
\pgfpathlineto{\pgfqpoint{0.772412in}{2.271245in}}%
\pgfpathlineto{\pgfqpoint{0.773022in}{2.220728in}}%
\pgfpathlineto{\pgfqpoint{0.773937in}{2.163006in}}%
\pgfpathlineto{\pgfqpoint{0.774853in}{2.190238in}}%
\pgfpathlineto{\pgfqpoint{0.775463in}{2.201785in}}%
\pgfpathlineto{\pgfqpoint{0.776073in}{2.189568in}}%
\pgfpathlineto{\pgfqpoint{0.776683in}{2.172318in}}%
\pgfpathlineto{\pgfqpoint{0.777294in}{2.188589in}}%
\pgfpathlineto{\pgfqpoint{0.778514in}{2.293827in}}%
\pgfpathlineto{\pgfqpoint{0.779124in}{2.263008in}}%
\pgfpathlineto{\pgfqpoint{0.780040in}{2.179362in}}%
\pgfpathlineto{\pgfqpoint{0.780650in}{2.205946in}}%
\pgfpathlineto{\pgfqpoint{0.781260in}{2.244161in}}%
\pgfpathlineto{\pgfqpoint{0.781870in}{2.206393in}}%
\pgfpathlineto{\pgfqpoint{0.783091in}{2.081201in}}%
\pgfpathlineto{\pgfqpoint{0.783701in}{2.126515in}}%
\pgfpathlineto{\pgfqpoint{0.784922in}{2.271830in}}%
\pgfpathlineto{\pgfqpoint{0.785532in}{2.185343in}}%
\pgfpathlineto{\pgfqpoint{0.786447in}{2.059779in}}%
\pgfpathlineto{\pgfqpoint{0.787057in}{2.125451in}}%
\pgfpathlineto{\pgfqpoint{0.788278in}{2.399618in}}%
\pgfpathlineto{\pgfqpoint{0.788888in}{2.305160in}}%
\pgfpathlineto{\pgfqpoint{0.790414in}{2.103198in}}%
\pgfpathlineto{\pgfqpoint{0.791024in}{2.136476in}}%
\pgfpathlineto{\pgfqpoint{0.792855in}{2.314217in}}%
\pgfpathlineto{\pgfqpoint{0.793465in}{2.263338in}}%
\pgfpathlineto{\pgfqpoint{0.794991in}{2.103294in}}%
\pgfpathlineto{\pgfqpoint{0.795601in}{2.138317in}}%
\pgfpathlineto{\pgfqpoint{0.796516in}{2.206042in}}%
\pgfpathlineto{\pgfqpoint{0.797126in}{2.174478in}}%
\pgfpathlineto{\pgfqpoint{0.797737in}{2.128281in}}%
\pgfpathlineto{\pgfqpoint{0.798347in}{2.170434in}}%
\pgfpathlineto{\pgfqpoint{0.799567in}{2.384815in}}%
\pgfpathlineto{\pgfqpoint{0.800178in}{2.279737in}}%
\pgfpathlineto{\pgfqpoint{0.801398in}{2.132538in}}%
\pgfpathlineto{\pgfqpoint{0.802008in}{2.160579in}}%
\pgfpathlineto{\pgfqpoint{0.805365in}{2.299297in}}%
\pgfpathlineto{\pgfqpoint{0.805670in}{2.284643in}}%
\pgfpathlineto{\pgfqpoint{0.807195in}{2.046073in}}%
\pgfpathlineto{\pgfqpoint{0.808111in}{2.130122in}}%
\pgfpathlineto{\pgfqpoint{0.809331in}{2.352560in}}%
\pgfpathlineto{\pgfqpoint{0.809941in}{2.237584in}}%
\pgfpathlineto{\pgfqpoint{0.810857in}{2.062099in}}%
\pgfpathlineto{\pgfqpoint{0.811467in}{2.110488in}}%
\pgfpathlineto{\pgfqpoint{0.812687in}{2.328083in}}%
\pgfpathlineto{\pgfqpoint{0.813298in}{2.224963in}}%
\pgfpathlineto{\pgfqpoint{0.814213in}{2.104507in}}%
\pgfpathlineto{\pgfqpoint{0.814823in}{2.205733in}}%
\pgfpathlineto{\pgfqpoint{0.815739in}{2.360754in}}%
\pgfpathlineto{\pgfqpoint{0.816044in}{2.299712in}}%
\pgfpathlineto{\pgfqpoint{0.817569in}{1.942878in}}%
\pgfpathlineto{\pgfqpoint{0.817874in}{1.993714in}}%
\pgfpathlineto{\pgfqpoint{0.819095in}{2.616138in}}%
\pgfpathlineto{\pgfqpoint{0.820010in}{2.251589in}}%
\pgfpathlineto{\pgfqpoint{0.820926in}{1.981497in}}%
\pgfpathlineto{\pgfqpoint{0.821841in}{2.066611in}}%
\pgfpathlineto{\pgfqpoint{0.823977in}{2.378983in}}%
\pgfpathlineto{\pgfqpoint{0.824282in}{2.360860in}}%
\pgfpathlineto{\pgfqpoint{0.826418in}{2.002505in}}%
\pgfpathlineto{\pgfqpoint{0.827028in}{2.082159in}}%
\pgfpathlineto{\pgfqpoint{0.828248in}{2.355252in}}%
\pgfpathlineto{\pgfqpoint{0.829164in}{2.244565in}}%
\pgfpathlineto{\pgfqpoint{0.830079in}{2.165358in}}%
\pgfpathlineto{\pgfqpoint{0.830689in}{2.193516in}}%
\pgfpathlineto{\pgfqpoint{0.831300in}{2.211214in}}%
\pgfpathlineto{\pgfqpoint{0.831910in}{2.183523in}}%
\pgfpathlineto{\pgfqpoint{0.832825in}{2.138625in}}%
\pgfpathlineto{\pgfqpoint{0.833435in}{2.178107in}}%
\pgfpathlineto{\pgfqpoint{0.834351in}{2.296243in}}%
\pgfpathlineto{\pgfqpoint{0.834961in}{2.258549in}}%
\pgfpathlineto{\pgfqpoint{0.836181in}{2.073603in}}%
\pgfpathlineto{\pgfqpoint{0.836792in}{2.125046in}}%
\pgfpathlineto{\pgfqpoint{0.838012in}{2.286952in}}%
\pgfpathlineto{\pgfqpoint{0.838622in}{2.208489in}}%
\pgfpathlineto{\pgfqpoint{0.839233in}{2.151438in}}%
\pgfpathlineto{\pgfqpoint{0.839843in}{2.186940in}}%
\pgfpathlineto{\pgfqpoint{0.840453in}{2.238382in}}%
\pgfpathlineto{\pgfqpoint{0.841369in}{2.188621in}}%
\pgfpathlineto{\pgfqpoint{0.841674in}{2.172743in}}%
\pgfpathlineto{\pgfqpoint{0.842284in}{2.191686in}}%
\pgfpathlineto{\pgfqpoint{0.842894in}{2.216492in}}%
\pgfpathlineto{\pgfqpoint{0.843504in}{2.175127in}}%
\pgfpathlineto{\pgfqpoint{0.844115in}{2.147022in}}%
\pgfpathlineto{\pgfqpoint{0.844420in}{2.178139in}}%
\pgfpathlineto{\pgfqpoint{0.845640in}{2.380239in}}%
\pgfpathlineto{\pgfqpoint{0.845945in}{2.324167in}}%
\pgfpathlineto{\pgfqpoint{0.847471in}{1.994608in}}%
\pgfpathlineto{\pgfqpoint{0.848081in}{2.059109in}}%
\pgfpathlineto{\pgfqpoint{0.849607in}{2.388167in}}%
\pgfpathlineto{\pgfqpoint{0.850217in}{2.268542in}}%
\pgfpathlineto{\pgfqpoint{0.851132in}{2.152204in}}%
\pgfpathlineto{\pgfqpoint{0.852048in}{2.190664in}}%
\pgfpathlineto{\pgfqpoint{0.852353in}{2.192644in}}%
\pgfpathlineto{\pgfqpoint{0.853268in}{2.157163in}}%
\pgfpathlineto{\pgfqpoint{0.853573in}{2.165070in}}%
\pgfpathlineto{\pgfqpoint{0.854794in}{2.364415in}}%
\pgfpathlineto{\pgfqpoint{0.855404in}{2.309832in}}%
\pgfpathlineto{\pgfqpoint{0.856930in}{1.975985in}}%
\pgfpathlineto{\pgfqpoint{0.857540in}{2.020979in}}%
\pgfpathlineto{\pgfqpoint{0.859065in}{2.438152in}}%
\pgfpathlineto{\pgfqpoint{0.859676in}{2.290687in}}%
\pgfpathlineto{\pgfqpoint{0.860896in}{2.066941in}}%
\pgfpathlineto{\pgfqpoint{0.861506in}{2.139658in}}%
\pgfpathlineto{\pgfqpoint{0.862422in}{2.223633in}}%
\pgfpathlineto{\pgfqpoint{0.863032in}{2.164879in}}%
\pgfpathlineto{\pgfqpoint{0.863642in}{2.123184in}}%
\pgfpathlineto{\pgfqpoint{0.863947in}{2.153833in}}%
\pgfpathlineto{\pgfqpoint{0.865168in}{2.484466in}}%
\pgfpathlineto{\pgfqpoint{0.865778in}{2.312291in}}%
\pgfpathlineto{\pgfqpoint{0.866998in}{1.987127in}}%
\pgfpathlineto{\pgfqpoint{0.867609in}{2.064983in}}%
\pgfpathlineto{\pgfqpoint{0.868829in}{2.377334in}}%
\pgfpathlineto{\pgfqpoint{0.869439in}{2.251142in}}%
\pgfpathlineto{\pgfqpoint{0.870355in}{2.051096in}}%
\pgfpathlineto{\pgfqpoint{0.870965in}{2.112265in}}%
\pgfpathlineto{\pgfqpoint{0.871880in}{2.300989in}}%
\pgfpathlineto{\pgfqpoint{0.872796in}{2.183885in}}%
\pgfpathlineto{\pgfqpoint{0.873711in}{2.075317in}}%
\pgfpathlineto{\pgfqpoint{0.874321in}{2.164879in}}%
\pgfpathlineto{\pgfqpoint{0.875542in}{2.461799in}}%
\pgfpathlineto{\pgfqpoint{0.876152in}{2.290826in}}%
\pgfpathlineto{\pgfqpoint{0.877372in}{2.024182in}}%
\pgfpathlineto{\pgfqpoint{0.877983in}{2.100644in}}%
\pgfpathlineto{\pgfqpoint{0.879203in}{2.355869in}}%
\pgfpathlineto{\pgfqpoint{0.879813in}{2.266818in}}%
\pgfpathlineto{\pgfqpoint{0.881034in}{2.072028in}}%
\pgfpathlineto{\pgfqpoint{0.881949in}{2.123333in}}%
\pgfpathlineto{\pgfqpoint{0.883475in}{2.244342in}}%
\pgfpathlineto{\pgfqpoint{0.884390in}{2.225591in}}%
\pgfpathlineto{\pgfqpoint{0.885916in}{2.166390in}}%
\pgfpathlineto{\pgfqpoint{0.886526in}{2.183108in}}%
\pgfpathlineto{\pgfqpoint{0.887746in}{2.237967in}}%
\pgfpathlineto{\pgfqpoint{0.888662in}{2.223111in}}%
\pgfpathlineto{\pgfqpoint{0.889272in}{2.238329in}}%
\pgfpathlineto{\pgfqpoint{0.889882in}{2.262582in}}%
\pgfpathlineto{\pgfqpoint{0.890493in}{2.226251in}}%
\pgfpathlineto{\pgfqpoint{0.891713in}{2.078286in}}%
\pgfpathlineto{\pgfqpoint{0.892323in}{2.109009in}}%
\pgfpathlineto{\pgfqpoint{0.893849in}{2.355710in}}%
\pgfpathlineto{\pgfqpoint{0.894459in}{2.229369in}}%
\pgfpathlineto{\pgfqpoint{0.895680in}{1.978986in}}%
\pgfpathlineto{\pgfqpoint{0.896290in}{2.025683in}}%
\pgfpathlineto{\pgfqpoint{0.897815in}{2.554394in}}%
\pgfpathlineto{\pgfqpoint{0.898426in}{2.293199in}}%
\pgfpathlineto{\pgfqpoint{0.899341in}{2.020245in}}%
\pgfpathlineto{\pgfqpoint{0.899951in}{2.117554in}}%
\pgfpathlineto{\pgfqpoint{0.900867in}{2.401566in}}%
\pgfpathlineto{\pgfqpoint{0.901782in}{2.225729in}}%
\pgfpathlineto{\pgfqpoint{0.903002in}{2.008432in}}%
\pgfpathlineto{\pgfqpoint{0.903613in}{2.080414in}}%
\pgfpathlineto{\pgfqpoint{0.904833in}{2.298797in}}%
\pgfpathlineto{\pgfqpoint{0.905748in}{2.220536in}}%
\pgfpathlineto{\pgfqpoint{0.906969in}{2.120832in}}%
\pgfpathlineto{\pgfqpoint{0.907579in}{2.159356in}}%
\pgfpathlineto{\pgfqpoint{0.909105in}{2.318325in}}%
\pgfpathlineto{\pgfqpoint{0.909715in}{2.257229in}}%
\pgfpathlineto{\pgfqpoint{0.911241in}{2.106412in}}%
\pgfpathlineto{\pgfqpoint{0.911851in}{2.153769in}}%
\pgfpathlineto{\pgfqpoint{0.913071in}{2.292635in}}%
\pgfpathlineto{\pgfqpoint{0.913681in}{2.251312in}}%
\pgfpathlineto{\pgfqpoint{0.914902in}{2.114202in}}%
\pgfpathlineto{\pgfqpoint{0.915512in}{2.134773in}}%
\pgfpathlineto{\pgfqpoint{0.916733in}{2.278173in}}%
\pgfpathlineto{\pgfqpoint{0.917343in}{2.241894in}}%
\pgfpathlineto{\pgfqpoint{0.918563in}{2.101591in}}%
\pgfpathlineto{\pgfqpoint{0.919174in}{2.170923in}}%
\pgfpathlineto{\pgfqpoint{0.920089in}{2.344206in}}%
\pgfpathlineto{\pgfqpoint{0.920699in}{2.292007in}}%
\pgfpathlineto{\pgfqpoint{0.921920in}{2.061216in}}%
\pgfpathlineto{\pgfqpoint{0.922530in}{2.095877in}}%
\pgfpathlineto{\pgfqpoint{0.923750in}{2.369938in}}%
\pgfpathlineto{\pgfqpoint{0.924361in}{2.297296in}}%
\pgfpathlineto{\pgfqpoint{0.925581in}{2.062780in}}%
\pgfpathlineto{\pgfqpoint{0.926191in}{2.124663in}}%
\pgfpathlineto{\pgfqpoint{0.927412in}{2.358083in}}%
\pgfpathlineto{\pgfqpoint{0.928022in}{2.282653in}}%
\pgfpathlineto{\pgfqpoint{0.929243in}{2.088406in}}%
\pgfpathlineto{\pgfqpoint{0.930158in}{2.148480in}}%
\pgfpathlineto{\pgfqpoint{0.931378in}{2.278077in}}%
\pgfpathlineto{\pgfqpoint{0.931989in}{2.205414in}}%
\pgfpathlineto{\pgfqpoint{0.933209in}{2.074295in}}%
\pgfpathlineto{\pgfqpoint{0.933514in}{2.106029in}}%
\pgfpathlineto{\pgfqpoint{0.934735in}{2.387901in}}%
\pgfpathlineto{\pgfqpoint{0.935345in}{2.288123in}}%
\pgfpathlineto{\pgfqpoint{0.936565in}{1.990405in}}%
\pgfpathlineto{\pgfqpoint{0.937176in}{2.053852in}}%
\pgfpathlineto{\pgfqpoint{0.938396in}{2.502366in}}%
\pgfpathlineto{\pgfqpoint{0.939006in}{2.358774in}}%
\pgfpathlineto{\pgfqpoint{0.940227in}{2.036335in}}%
\pgfpathlineto{\pgfqpoint{0.940837in}{2.129303in}}%
\pgfpathlineto{\pgfqpoint{0.941752in}{2.324114in}}%
\pgfpathlineto{\pgfqpoint{0.942668in}{2.216258in}}%
\pgfpathlineto{\pgfqpoint{0.943278in}{2.140519in}}%
\pgfpathlineto{\pgfqpoint{0.944193in}{2.195794in}}%
\pgfpathlineto{\pgfqpoint{0.944498in}{2.213225in}}%
\pgfpathlineto{\pgfqpoint{0.945109in}{2.184077in}}%
\pgfpathlineto{\pgfqpoint{0.946024in}{2.114489in}}%
\pgfpathlineto{\pgfqpoint{0.946634in}{2.155014in}}%
\pgfpathlineto{\pgfqpoint{0.947244in}{2.205393in}}%
\pgfpathlineto{\pgfqpoint{0.947855in}{2.176553in}}%
\pgfpathlineto{\pgfqpoint{0.948465in}{2.125078in}}%
\pgfpathlineto{\pgfqpoint{0.949075in}{2.185875in}}%
\pgfpathlineto{\pgfqpoint{0.949991in}{2.441302in}}%
\pgfpathlineto{\pgfqpoint{0.950601in}{2.354582in}}%
\pgfpathlineto{\pgfqpoint{0.951821in}{2.064792in}}%
\pgfpathlineto{\pgfqpoint{0.952431in}{2.171104in}}%
\pgfpathlineto{\pgfqpoint{0.953347in}{2.355156in}}%
\pgfpathlineto{\pgfqpoint{0.953957in}{2.249248in}}%
\pgfpathlineto{\pgfqpoint{0.955178in}{2.036984in}}%
\pgfpathlineto{\pgfqpoint{0.955788in}{2.086065in}}%
\pgfpathlineto{\pgfqpoint{0.957008in}{2.240181in}}%
\pgfpathlineto{\pgfqpoint{0.957619in}{2.196762in}}%
\pgfpathlineto{\pgfqpoint{0.958229in}{2.163698in}}%
\pgfpathlineto{\pgfqpoint{0.958839in}{2.222569in}}%
\pgfpathlineto{\pgfqpoint{0.959449in}{2.285058in}}%
\pgfpathlineto{\pgfqpoint{0.960059in}{2.223878in}}%
\pgfpathlineto{\pgfqpoint{0.960975in}{2.126259in}}%
\pgfpathlineto{\pgfqpoint{0.961585in}{2.207776in}}%
\pgfpathlineto{\pgfqpoint{0.962195in}{2.299456in}}%
\pgfpathlineto{\pgfqpoint{0.963111in}{2.211203in}}%
\pgfpathlineto{\pgfqpoint{0.963721in}{2.144957in}}%
\pgfpathlineto{\pgfqpoint{0.964331in}{2.181278in}}%
\pgfpathlineto{\pgfqpoint{0.965246in}{2.255410in}}%
\pgfpathlineto{\pgfqpoint{0.965552in}{2.225293in}}%
\pgfpathlineto{\pgfqpoint{0.966467in}{2.119523in}}%
\pgfpathlineto{\pgfqpoint{0.967077in}{2.187695in}}%
\pgfpathlineto{\pgfqpoint{0.967993in}{2.320708in}}%
\pgfpathlineto{\pgfqpoint{0.968603in}{2.215247in}}%
\pgfpathlineto{\pgfqpoint{0.969823in}{1.971324in}}%
\pgfpathlineto{\pgfqpoint{0.970433in}{2.027917in}}%
\pgfpathlineto{\pgfqpoint{0.971654in}{2.550084in}}%
\pgfpathlineto{\pgfqpoint{0.972264in}{2.387167in}}%
\pgfpathlineto{\pgfqpoint{0.973485in}{2.017403in}}%
\pgfpathlineto{\pgfqpoint{0.974095in}{2.165060in}}%
\pgfpathlineto{\pgfqpoint{0.975010in}{2.396447in}}%
\pgfpathlineto{\pgfqpoint{0.975620in}{2.288165in}}%
\pgfpathlineto{\pgfqpoint{0.976841in}{2.098303in}}%
\pgfpathlineto{\pgfqpoint{0.977451in}{2.155205in}}%
\pgfpathlineto{\pgfqpoint{0.978367in}{2.230625in}}%
\pgfpathlineto{\pgfqpoint{0.978977in}{2.167497in}}%
\pgfpathlineto{\pgfqpoint{0.979892in}{2.066888in}}%
\pgfpathlineto{\pgfqpoint{0.980502in}{2.109179in}}%
\pgfpathlineto{\pgfqpoint{0.981723in}{2.375067in}}%
\pgfpathlineto{\pgfqpoint{0.982638in}{2.236541in}}%
\pgfpathlineto{\pgfqpoint{0.983859in}{2.048254in}}%
\pgfpathlineto{\pgfqpoint{0.984469in}{2.113244in}}%
\pgfpathlineto{\pgfqpoint{0.985689in}{2.451912in}}%
\pgfpathlineto{\pgfqpoint{0.986605in}{2.296019in}}%
\pgfpathlineto{\pgfqpoint{0.988435in}{2.039017in}}%
\pgfpathlineto{\pgfqpoint{0.988741in}{2.050798in}}%
\pgfpathlineto{\pgfqpoint{0.989961in}{2.239436in}}%
\pgfpathlineto{\pgfqpoint{0.990876in}{2.348324in}}%
\pgfpathlineto{\pgfqpoint{0.991487in}{2.242182in}}%
\pgfpathlineto{\pgfqpoint{0.992707in}{2.074518in}}%
\pgfpathlineto{\pgfqpoint{0.993317in}{2.162080in}}%
\pgfpathlineto{\pgfqpoint{0.994233in}{2.313748in}}%
\pgfpathlineto{\pgfqpoint{0.994843in}{2.242448in}}%
\pgfpathlineto{\pgfqpoint{0.995758in}{2.090492in}}%
\pgfpathlineto{\pgfqpoint{0.996369in}{2.132676in}}%
\pgfpathlineto{\pgfqpoint{0.997589in}{2.411824in}}%
\pgfpathlineto{\pgfqpoint{0.998199in}{2.302936in}}%
\pgfpathlineto{\pgfqpoint{1.000335in}{2.096451in}}%
\pgfpathlineto{\pgfqpoint{1.000640in}{2.094472in}}%
\pgfpathlineto{\pgfqpoint{1.001556in}{2.184875in}}%
\pgfpathlineto{\pgfqpoint{1.002471in}{2.323539in}}%
\pgfpathlineto{\pgfqpoint{1.003081in}{2.238095in}}%
\pgfpathlineto{\pgfqpoint{1.003996in}{2.065281in}}%
\pgfpathlineto{\pgfqpoint{1.004912in}{2.156301in}}%
\pgfpathlineto{\pgfqpoint{1.005827in}{2.245960in}}%
\pgfpathlineto{\pgfqpoint{1.006437in}{2.179128in}}%
\pgfpathlineto{\pgfqpoint{1.006743in}{2.159866in}}%
\pgfpathlineto{\pgfqpoint{1.007353in}{2.211735in}}%
\pgfpathlineto{\pgfqpoint{1.008268in}{2.442015in}}%
\pgfpathlineto{\pgfqpoint{1.008878in}{2.348973in}}%
\pgfpathlineto{\pgfqpoint{1.010404in}{1.952977in}}%
\pgfpathlineto{\pgfqpoint{1.011014in}{2.050308in}}%
\pgfpathlineto{\pgfqpoint{1.012235in}{2.506867in}}%
\pgfpathlineto{\pgfqpoint{1.012845in}{2.354294in}}%
\pgfpathlineto{\pgfqpoint{1.014065in}{1.953115in}}%
\pgfpathlineto{\pgfqpoint{1.014676in}{2.006442in}}%
\pgfpathlineto{\pgfqpoint{1.015896in}{2.373524in}}%
\pgfpathlineto{\pgfqpoint{1.016811in}{2.258240in}}%
\pgfpathlineto{\pgfqpoint{1.018337in}{2.117437in}}%
\pgfpathlineto{\pgfqpoint{1.018642in}{2.128271in}}%
\pgfpathlineto{\pgfqpoint{1.020168in}{2.418805in}}%
\pgfpathlineto{\pgfqpoint{1.020778in}{2.260475in}}%
\pgfpathlineto{\pgfqpoint{1.021998in}{1.973857in}}%
\pgfpathlineto{\pgfqpoint{1.022609in}{2.113979in}}%
\pgfpathlineto{\pgfqpoint{1.023524in}{2.492128in}}%
\pgfpathlineto{\pgfqpoint{1.024134in}{2.331691in}}%
\pgfpathlineto{\pgfqpoint{1.025355in}{1.990532in}}%
\pgfpathlineto{\pgfqpoint{1.025965in}{2.018989in}}%
\pgfpathlineto{\pgfqpoint{1.027796in}{2.394670in}}%
\pgfpathlineto{\pgfqpoint{1.028406in}{2.275959in}}%
\pgfpathlineto{\pgfqpoint{1.030542in}{2.060279in}}%
\pgfpathlineto{\pgfqpoint{1.031152in}{2.156238in}}%
\pgfpathlineto{\pgfqpoint{1.032067in}{2.513157in}}%
\pgfpathlineto{\pgfqpoint{1.032678in}{2.305395in}}%
\pgfpathlineto{\pgfqpoint{1.033593in}{1.982157in}}%
\pgfpathlineto{\pgfqpoint{1.034203in}{2.066462in}}%
\pgfpathlineto{\pgfqpoint{1.035424in}{2.357050in}}%
\pgfpathlineto{\pgfqpoint{1.036034in}{2.227677in}}%
\pgfpathlineto{\pgfqpoint{1.037254in}{2.038229in}}%
\pgfpathlineto{\pgfqpoint{1.037865in}{2.126046in}}%
\pgfpathlineto{\pgfqpoint{1.038780in}{2.353358in}}%
\pgfpathlineto{\pgfqpoint{1.039390in}{2.265328in}}%
\pgfpathlineto{\pgfqpoint{1.040306in}{2.074486in}}%
\pgfpathlineto{\pgfqpoint{1.040916in}{2.158738in}}%
\pgfpathlineto{\pgfqpoint{1.041831in}{2.359498in}}%
\pgfpathlineto{\pgfqpoint{1.042441in}{2.225133in}}%
\pgfpathlineto{\pgfqpoint{1.043052in}{2.149193in}}%
\pgfpathlineto{\pgfqpoint{1.043662in}{2.258421in}}%
\pgfpathlineto{\pgfqpoint{1.043967in}{2.304873in}}%
\pgfpathlineto{\pgfqpoint{1.044577in}{2.236116in}}%
\pgfpathlineto{\pgfqpoint{1.045493in}{2.081531in}}%
\pgfpathlineto{\pgfqpoint{1.046103in}{2.163921in}}%
\pgfpathlineto{\pgfqpoint{1.047018in}{2.281950in}}%
\pgfpathlineto{\pgfqpoint{1.047628in}{2.165656in}}%
\pgfpathlineto{\pgfqpoint{1.048544in}{2.013210in}}%
\pgfpathlineto{\pgfqpoint{1.049154in}{2.136486in}}%
\pgfpathlineto{\pgfqpoint{1.050069in}{2.565057in}}%
\pgfpathlineto{\pgfqpoint{1.050680in}{2.369289in}}%
\pgfpathlineto{\pgfqpoint{1.051900in}{1.896756in}}%
\pgfpathlineto{\pgfqpoint{1.052510in}{1.994108in}}%
\pgfpathlineto{\pgfqpoint{1.053731in}{2.481486in}}%
\pgfpathlineto{\pgfqpoint{1.054341in}{2.271170in}}%
\pgfpathlineto{\pgfqpoint{1.054951in}{2.119268in}}%
\pgfpathlineto{\pgfqpoint{1.055867in}{2.231997in}}%
\pgfpathlineto{\pgfqpoint{1.056172in}{2.235935in}}%
\pgfpathlineto{\pgfqpoint{1.057392in}{2.028332in}}%
\pgfpathlineto{\pgfqpoint{1.058002in}{2.123354in}}%
\pgfpathlineto{\pgfqpoint{1.058918in}{2.525789in}}%
\pgfpathlineto{\pgfqpoint{1.059528in}{2.358476in}}%
\pgfpathlineto{\pgfqpoint{1.060748in}{1.967663in}}%
\pgfpathlineto{\pgfqpoint{1.061359in}{2.087033in}}%
\pgfpathlineto{\pgfqpoint{1.062274in}{2.326285in}}%
\pgfpathlineto{\pgfqpoint{1.063189in}{2.192186in}}%
\pgfpathlineto{\pgfqpoint{1.063800in}{2.130016in}}%
\pgfpathlineto{\pgfqpoint{1.064410in}{2.216865in}}%
\pgfpathlineto{\pgfqpoint{1.065020in}{2.307193in}}%
\pgfpathlineto{\pgfqpoint{1.065630in}{2.233019in}}%
\pgfpathlineto{\pgfqpoint{1.066546in}{2.099420in}}%
\pgfpathlineto{\pgfqpoint{1.067156in}{2.154205in}}%
\pgfpathlineto{\pgfqpoint{1.067766in}{2.218493in}}%
\pgfpathlineto{\pgfqpoint{1.068376in}{2.181789in}}%
\pgfpathlineto{\pgfqpoint{1.069292in}{2.116043in}}%
\pgfpathlineto{\pgfqpoint{1.069597in}{2.147926in}}%
\pgfpathlineto{\pgfqpoint{1.070817in}{2.469961in}}%
\pgfpathlineto{\pgfqpoint{1.071428in}{2.288155in}}%
\pgfpathlineto{\pgfqpoint{1.072343in}{2.001494in}}%
\pgfpathlineto{\pgfqpoint{1.072953in}{2.076030in}}%
\pgfpathlineto{\pgfqpoint{1.074174in}{2.424914in}}%
\pgfpathlineto{\pgfqpoint{1.074784in}{2.206138in}}%
\pgfpathlineto{\pgfqpoint{1.075699in}{1.981221in}}%
\pgfpathlineto{\pgfqpoint{1.076309in}{2.076530in}}%
\pgfpathlineto{\pgfqpoint{1.077225in}{2.362989in}}%
\pgfpathlineto{\pgfqpoint{1.077835in}{2.277949in}}%
\pgfpathlineto{\pgfqpoint{1.078750in}{2.054299in}}%
\pgfpathlineto{\pgfqpoint{1.079666in}{2.172467in}}%
\pgfpathlineto{\pgfqpoint{1.080581in}{2.266786in}}%
\pgfpathlineto{\pgfqpoint{1.081191in}{2.220398in}}%
\pgfpathlineto{\pgfqpoint{1.081496in}{2.211437in}}%
\pgfpathlineto{\pgfqpoint{1.081802in}{2.224122in}}%
\pgfpathlineto{\pgfqpoint{1.082717in}{2.286846in}}%
\pgfpathlineto{\pgfqpoint{1.083327in}{2.228624in}}%
\pgfpathlineto{\pgfqpoint{1.084243in}{2.122258in}}%
\pgfpathlineto{\pgfqpoint{1.084853in}{2.158621in}}%
\pgfpathlineto{\pgfqpoint{1.085768in}{2.291071in}}%
\pgfpathlineto{\pgfqpoint{1.086378in}{2.231444in}}%
\pgfpathlineto{\pgfqpoint{1.087294in}{2.104635in}}%
\pgfpathlineto{\pgfqpoint{1.087904in}{2.141115in}}%
\pgfpathlineto{\pgfqpoint{1.088819in}{2.224835in}}%
\pgfpathlineto{\pgfqpoint{1.089430in}{2.154918in}}%
\pgfpathlineto{\pgfqpoint{1.090345in}{2.052490in}}%
\pgfpathlineto{\pgfqpoint{1.090650in}{2.085394in}}%
\pgfpathlineto{\pgfqpoint{1.091870in}{2.430767in}}%
\pgfpathlineto{\pgfqpoint{1.092481in}{2.300638in}}%
\pgfpathlineto{\pgfqpoint{1.093396in}{2.107572in}}%
\pgfpathlineto{\pgfqpoint{1.094006in}{2.164932in}}%
\pgfpathlineto{\pgfqpoint{1.096142in}{2.263763in}}%
\pgfpathlineto{\pgfqpoint{1.097057in}{2.306161in}}%
\pgfpathlineto{\pgfqpoint{1.097363in}{2.286601in}}%
\pgfpathlineto{\pgfqpoint{1.099193in}{1.971111in}}%
\pgfpathlineto{\pgfqpoint{1.099804in}{2.083383in}}%
\pgfpathlineto{\pgfqpoint{1.101024in}{2.443420in}}%
\pgfpathlineto{\pgfqpoint{1.101634in}{2.233966in}}%
\pgfpathlineto{\pgfqpoint{1.102855in}{1.965247in}}%
\pgfpathlineto{\pgfqpoint{1.103465in}{2.097654in}}%
\pgfpathlineto{\pgfqpoint{1.104380in}{2.466268in}}%
\pgfpathlineto{\pgfqpoint{1.105296in}{2.240085in}}%
\pgfpathlineto{\pgfqpoint{1.105906in}{2.107679in}}%
\pgfpathlineto{\pgfqpoint{1.106821in}{2.232487in}}%
\pgfpathlineto{\pgfqpoint{1.107431in}{2.268680in}}%
\pgfpathlineto{\pgfqpoint{1.108042in}{2.199518in}}%
\pgfpathlineto{\pgfqpoint{1.108652in}{2.162006in}}%
\pgfpathlineto{\pgfqpoint{1.109262in}{2.199369in}}%
\pgfpathlineto{\pgfqpoint{1.109567in}{2.219780in}}%
\pgfpathlineto{\pgfqpoint{1.110178in}{2.191196in}}%
\pgfpathlineto{\pgfqpoint{1.111093in}{2.083085in}}%
\pgfpathlineto{\pgfqpoint{1.111703in}{2.131719in}}%
\pgfpathlineto{\pgfqpoint{1.112619in}{2.295604in}}%
\pgfpathlineto{\pgfqpoint{1.113229in}{2.216056in}}%
\pgfpathlineto{\pgfqpoint{1.114144in}{2.019287in}}%
\pgfpathlineto{\pgfqpoint{1.114754in}{2.093259in}}%
\pgfpathlineto{\pgfqpoint{1.115975in}{2.509592in}}%
\pgfpathlineto{\pgfqpoint{1.116585in}{2.234488in}}%
\pgfpathlineto{\pgfqpoint{1.117195in}{2.020287in}}%
\pgfpathlineto{\pgfqpoint{1.118111in}{2.233604in}}%
\pgfpathlineto{\pgfqpoint{1.118721in}{2.465672in}}%
\pgfpathlineto{\pgfqpoint{1.119331in}{2.337820in}}%
\pgfpathlineto{\pgfqpoint{1.120552in}{2.021820in}}%
\pgfpathlineto{\pgfqpoint{1.121162in}{2.100995in}}%
\pgfpathlineto{\pgfqpoint{1.122077in}{2.263987in}}%
\pgfpathlineto{\pgfqpoint{1.122687in}{2.208415in}}%
\pgfpathlineto{\pgfqpoint{1.123603in}{2.060141in}}%
\pgfpathlineto{\pgfqpoint{1.124213in}{2.116128in}}%
\pgfpathlineto{\pgfqpoint{1.125433in}{2.342673in}}%
\pgfpathlineto{\pgfqpoint{1.126044in}{2.199891in}}%
\pgfpathlineto{\pgfqpoint{1.126654in}{2.103528in}}%
\pgfpathlineto{\pgfqpoint{1.127264in}{2.203956in}}%
\pgfpathlineto{\pgfqpoint{1.128180in}{2.356220in}}%
\pgfpathlineto{\pgfqpoint{1.128790in}{2.217439in}}%
\pgfpathlineto{\pgfqpoint{1.129400in}{2.113627in}}%
\pgfpathlineto{\pgfqpoint{1.130315in}{2.193548in}}%
\pgfpathlineto{\pgfqpoint{1.130926in}{2.252887in}}%
\pgfpathlineto{\pgfqpoint{1.131536in}{2.207159in}}%
\pgfpathlineto{\pgfqpoint{1.132451in}{2.111488in}}%
\pgfpathlineto{\pgfqpoint{1.133061in}{2.174457in}}%
\pgfpathlineto{\pgfqpoint{1.133977in}{2.290677in}}%
\pgfpathlineto{\pgfqpoint{1.134587in}{2.187504in}}%
\pgfpathlineto{\pgfqpoint{1.135502in}{2.022054in}}%
\pgfpathlineto{\pgfqpoint{1.136113in}{2.087768in}}%
\pgfpathlineto{\pgfqpoint{1.137333in}{2.333926in}}%
\pgfpathlineto{\pgfqpoint{1.137943in}{2.199359in}}%
\pgfpathlineto{\pgfqpoint{1.138554in}{2.121630in}}%
\pgfpathlineto{\pgfqpoint{1.139164in}{2.238340in}}%
\pgfpathlineto{\pgfqpoint{1.139774in}{2.358849in}}%
\pgfpathlineto{\pgfqpoint{1.140384in}{2.234807in}}%
\pgfpathlineto{\pgfqpoint{1.140994in}{2.078030in}}%
\pgfpathlineto{\pgfqpoint{1.141605in}{2.153524in}}%
\pgfpathlineto{\pgfqpoint{1.142520in}{2.526512in}}%
\pgfpathlineto{\pgfqpoint{1.143130in}{2.356912in}}%
\pgfpathlineto{\pgfqpoint{1.144351in}{1.970600in}}%
\pgfpathlineto{\pgfqpoint{1.144961in}{2.013913in}}%
\pgfpathlineto{\pgfqpoint{1.146487in}{2.232093in}}%
\pgfpathlineto{\pgfqpoint{1.147402in}{2.179948in}}%
\pgfpathlineto{\pgfqpoint{1.148012in}{2.236786in}}%
\pgfpathlineto{\pgfqpoint{1.148928in}{2.346866in}}%
\pgfpathlineto{\pgfqpoint{1.149233in}{2.299009in}}%
\pgfpathlineto{\pgfqpoint{1.150758in}{1.980295in}}%
\pgfpathlineto{\pgfqpoint{1.151369in}{2.099399in}}%
\pgfpathlineto{\pgfqpoint{1.152589in}{2.592364in}}%
\pgfpathlineto{\pgfqpoint{1.153199in}{2.295881in}}%
\pgfpathlineto{\pgfqpoint{1.154420in}{1.964949in}}%
\pgfpathlineto{\pgfqpoint{1.155030in}{2.051042in}}%
\pgfpathlineto{\pgfqpoint{1.156250in}{2.302649in}}%
\pgfpathlineto{\pgfqpoint{1.157166in}{2.230380in}}%
\pgfpathlineto{\pgfqpoint{1.157471in}{2.224442in}}%
\pgfpathlineto{\pgfqpoint{1.157776in}{2.240138in}}%
\pgfpathlineto{\pgfqpoint{1.158386in}{2.259336in}}%
\pgfpathlineto{\pgfqpoint{1.158691in}{2.229465in}}%
\pgfpathlineto{\pgfqpoint{1.160217in}{2.038559in}}%
\pgfpathlineto{\pgfqpoint{1.160827in}{2.125717in}}%
\pgfpathlineto{\pgfqpoint{1.162048in}{2.402076in}}%
\pgfpathlineto{\pgfqpoint{1.162658in}{2.280727in}}%
\pgfpathlineto{\pgfqpoint{1.163573in}{2.116437in}}%
\pgfpathlineto{\pgfqpoint{1.164489in}{2.192292in}}%
\pgfpathlineto{\pgfqpoint{1.165099in}{2.229188in}}%
\pgfpathlineto{\pgfqpoint{1.165709in}{2.201264in}}%
\pgfpathlineto{\pgfqpoint{1.166319in}{2.183758in}}%
\pgfpathlineto{\pgfqpoint{1.166624in}{2.206574in}}%
\pgfpathlineto{\pgfqpoint{1.167235in}{2.274682in}}%
\pgfpathlineto{\pgfqpoint{1.167845in}{2.229039in}}%
\pgfpathlineto{\pgfqpoint{1.168760in}{2.047254in}}%
\pgfpathlineto{\pgfqpoint{1.169370in}{2.115841in}}%
\pgfpathlineto{\pgfqpoint{1.170286in}{2.412803in}}%
\pgfpathlineto{\pgfqpoint{1.170896in}{2.293965in}}%
\pgfpathlineto{\pgfqpoint{1.172117in}{1.996918in}}%
\pgfpathlineto{\pgfqpoint{1.172727in}{2.157887in}}%
\pgfpathlineto{\pgfqpoint{1.173642in}{2.452902in}}%
\pgfpathlineto{\pgfqpoint{1.174252in}{2.289847in}}%
\pgfpathlineto{\pgfqpoint{1.175168in}{2.023576in}}%
\pgfpathlineto{\pgfqpoint{1.176083in}{2.162484in}}%
\pgfpathlineto{\pgfqpoint{1.176998in}{2.382751in}}%
\pgfpathlineto{\pgfqpoint{1.177609in}{2.265413in}}%
\pgfpathlineto{\pgfqpoint{1.178829in}{2.039432in}}%
\pgfpathlineto{\pgfqpoint{1.179439in}{2.096962in}}%
\pgfpathlineto{\pgfqpoint{1.180660in}{2.306182in}}%
\pgfpathlineto{\pgfqpoint{1.181575in}{2.227326in}}%
\pgfpathlineto{\pgfqpoint{1.182796in}{2.105657in}}%
\pgfpathlineto{\pgfqpoint{1.183406in}{2.157557in}}%
\pgfpathlineto{\pgfqpoint{1.184626in}{2.362712in}}%
\pgfpathlineto{\pgfqpoint{1.185237in}{2.306491in}}%
\pgfpathlineto{\pgfqpoint{1.186762in}{2.079307in}}%
\pgfpathlineto{\pgfqpoint{1.187372in}{2.128335in}}%
\pgfpathlineto{\pgfqpoint{1.190424in}{2.311971in}}%
\pgfpathlineto{\pgfqpoint{1.191034in}{2.261997in}}%
\pgfpathlineto{\pgfqpoint{1.192254in}{2.091279in}}%
\pgfpathlineto{\pgfqpoint{1.192865in}{2.138327in}}%
\pgfpathlineto{\pgfqpoint{1.195306in}{2.276885in}}%
\pgfpathlineto{\pgfqpoint{1.196221in}{2.207904in}}%
\pgfpathlineto{\pgfqpoint{1.197746in}{2.007081in}}%
\pgfpathlineto{\pgfqpoint{1.198357in}{2.061876in}}%
\pgfpathlineto{\pgfqpoint{1.199882in}{2.515732in}}%
\pgfpathlineto{\pgfqpoint{1.200493in}{2.323943in}}%
\pgfpathlineto{\pgfqpoint{1.201713in}{2.000908in}}%
\pgfpathlineto{\pgfqpoint{1.202323in}{2.043497in}}%
\pgfpathlineto{\pgfqpoint{1.203849in}{2.336618in}}%
\pgfpathlineto{\pgfqpoint{1.204459in}{2.235914in}}%
\pgfpathlineto{\pgfqpoint{1.205374in}{2.134166in}}%
\pgfpathlineto{\pgfqpoint{1.205985in}{2.202466in}}%
\pgfpathlineto{\pgfqpoint{1.206900in}{2.278130in}}%
\pgfpathlineto{\pgfqpoint{1.207510in}{2.209224in}}%
\pgfpathlineto{\pgfqpoint{1.208426in}{2.126270in}}%
\pgfpathlineto{\pgfqpoint{1.209341in}{2.138923in}}%
\pgfpathlineto{\pgfqpoint{1.209646in}{2.138498in}}%
\pgfpathlineto{\pgfqpoint{1.209951in}{2.139711in}}%
\pgfpathlineto{\pgfqpoint{1.210561in}{2.192133in}}%
\pgfpathlineto{\pgfqpoint{1.211477in}{2.395319in}}%
\pgfpathlineto{\pgfqpoint{1.212087in}{2.321911in}}%
\pgfpathlineto{\pgfqpoint{1.213613in}{1.960724in}}%
\pgfpathlineto{\pgfqpoint{1.214223in}{2.100123in}}%
\pgfpathlineto{\pgfqpoint{1.215138in}{2.506537in}}%
\pgfpathlineto{\pgfqpoint{1.215748in}{2.384922in}}%
\pgfpathlineto{\pgfqpoint{1.216664in}{2.053309in}}%
\pgfpathlineto{\pgfqpoint{1.217579in}{2.220132in}}%
\pgfpathlineto{\pgfqpoint{1.218189in}{2.342663in}}%
\pgfpathlineto{\pgfqpoint{1.218800in}{2.210490in}}%
\pgfpathlineto{\pgfqpoint{1.219715in}{2.025385in}}%
\pgfpathlineto{\pgfqpoint{1.220325in}{2.187120in}}%
\pgfpathlineto{\pgfqpoint{1.220935in}{2.420412in}}%
\pgfpathlineto{\pgfqpoint{1.221546in}{2.313823in}}%
\pgfpathlineto{\pgfqpoint{1.222766in}{1.932896in}}%
\pgfpathlineto{\pgfqpoint{1.223376in}{2.052766in}}%
\pgfpathlineto{\pgfqpoint{1.224597in}{2.508251in}}%
\pgfpathlineto{\pgfqpoint{1.225207in}{2.300382in}}%
\pgfpathlineto{\pgfqpoint{1.226122in}{2.043572in}}%
\pgfpathlineto{\pgfqpoint{1.227038in}{2.159494in}}%
\pgfpathlineto{\pgfqpoint{1.227953in}{2.278662in}}%
\pgfpathlineto{\pgfqpoint{1.228563in}{2.205393in}}%
\pgfpathlineto{\pgfqpoint{1.229174in}{2.158696in}}%
\pgfpathlineto{\pgfqpoint{1.229784in}{2.215077in}}%
\pgfpathlineto{\pgfqpoint{1.230089in}{2.242160in}}%
\pgfpathlineto{\pgfqpoint{1.230699in}{2.206733in}}%
\pgfpathlineto{\pgfqpoint{1.231615in}{2.105220in}}%
\pgfpathlineto{\pgfqpoint{1.232225in}{2.198273in}}%
\pgfpathlineto{\pgfqpoint{1.232835in}{2.330307in}}%
\pgfpathlineto{\pgfqpoint{1.233445in}{2.266850in}}%
\pgfpathlineto{\pgfqpoint{1.234666in}{1.993331in}}%
\pgfpathlineto{\pgfqpoint{1.235276in}{2.107721in}}%
\pgfpathlineto{\pgfqpoint{1.236191in}{2.473611in}}%
\pgfpathlineto{\pgfqpoint{1.236802in}{2.324912in}}%
\pgfpathlineto{\pgfqpoint{1.237717in}{2.037176in}}%
\pgfpathlineto{\pgfqpoint{1.238327in}{2.213938in}}%
\pgfpathlineto{\pgfqpoint{1.238937in}{2.440451in}}%
\pgfpathlineto{\pgfqpoint{1.239548in}{2.308193in}}%
\pgfpathlineto{\pgfqpoint{1.240768in}{1.952541in}}%
\pgfpathlineto{\pgfqpoint{1.241378in}{2.094312in}}%
\pgfpathlineto{\pgfqpoint{1.242294in}{2.423605in}}%
\pgfpathlineto{\pgfqpoint{1.242904in}{2.298020in}}%
\pgfpathlineto{\pgfqpoint{1.243819in}{2.078360in}}%
\pgfpathlineto{\pgfqpoint{1.244735in}{2.179341in}}%
\pgfpathlineto{\pgfqpoint{1.245040in}{2.204445in}}%
\pgfpathlineto{\pgfqpoint{1.245650in}{2.154748in}}%
\pgfpathlineto{\pgfqpoint{1.246260in}{2.079467in}}%
\pgfpathlineto{\pgfqpoint{1.246870in}{2.184109in}}%
\pgfpathlineto{\pgfqpoint{1.247786in}{2.502195in}}%
\pgfpathlineto{\pgfqpoint{1.248396in}{2.272224in}}%
\pgfpathlineto{\pgfqpoint{1.249311in}{1.967748in}}%
\pgfpathlineto{\pgfqpoint{1.249922in}{2.079807in}}%
\pgfpathlineto{\pgfqpoint{1.250837in}{2.502430in}}%
\pgfpathlineto{\pgfqpoint{1.251447in}{2.361978in}}%
\pgfpathlineto{\pgfqpoint{1.252668in}{1.940239in}}%
\pgfpathlineto{\pgfqpoint{1.253278in}{2.089736in}}%
\pgfpathlineto{\pgfqpoint{1.254193in}{2.465428in}}%
\pgfpathlineto{\pgfqpoint{1.254804in}{2.281046in}}%
\pgfpathlineto{\pgfqpoint{1.256024in}{1.911527in}}%
\pgfpathlineto{\pgfqpoint{1.256634in}{2.043987in}}%
\pgfpathlineto{\pgfqpoint{1.257855in}{2.501333in}}%
\pgfpathlineto{\pgfqpoint{1.258465in}{2.254963in}}%
\pgfpathlineto{\pgfqpoint{1.259380in}{2.090300in}}%
\pgfpathlineto{\pgfqpoint{1.259991in}{2.167529in}}%
\pgfpathlineto{\pgfqpoint{1.261211in}{2.299754in}}%
\pgfpathlineto{\pgfqpoint{1.261821in}{2.279194in}}%
\pgfpathlineto{\pgfqpoint{1.263957in}{2.077988in}}%
\pgfpathlineto{\pgfqpoint{1.264567in}{2.126674in}}%
\pgfpathlineto{\pgfqpoint{1.265788in}{2.307321in}}%
\pgfpathlineto{\pgfqpoint{1.266093in}{2.259805in}}%
\pgfpathlineto{\pgfqpoint{1.267313in}{2.019127in}}%
\pgfpathlineto{\pgfqpoint{1.267924in}{2.167890in}}%
\pgfpathlineto{\pgfqpoint{1.268839in}{2.524586in}}%
\pgfpathlineto{\pgfqpoint{1.269449in}{2.269521in}}%
\pgfpathlineto{\pgfqpoint{1.270365in}{1.932332in}}%
\pgfpathlineto{\pgfqpoint{1.270975in}{2.010209in}}%
\pgfpathlineto{\pgfqpoint{1.272195in}{2.525118in}}%
\pgfpathlineto{\pgfqpoint{1.272806in}{2.302330in}}%
\pgfpathlineto{\pgfqpoint{1.273721in}{1.986627in}}%
\pgfpathlineto{\pgfqpoint{1.274331in}{2.064291in}}%
\pgfpathlineto{\pgfqpoint{1.275552in}{2.422221in}}%
\pgfpathlineto{\pgfqpoint{1.276162in}{2.254728in}}%
\pgfpathlineto{\pgfqpoint{1.277382in}{2.018361in}}%
\pgfpathlineto{\pgfqpoint{1.277993in}{2.045360in}}%
\pgfpathlineto{\pgfqpoint{1.280128in}{2.422551in}}%
\pgfpathlineto{\pgfqpoint{1.280739in}{2.261848in}}%
\pgfpathlineto{\pgfqpoint{1.281654in}{2.033398in}}%
\pgfpathlineto{\pgfqpoint{1.282264in}{2.098260in}}%
\pgfpathlineto{\pgfqpoint{1.283485in}{2.431225in}}%
\pgfpathlineto{\pgfqpoint{1.284095in}{2.268116in}}%
\pgfpathlineto{\pgfqpoint{1.285315in}{2.013487in}}%
\pgfpathlineto{\pgfqpoint{1.285926in}{2.054384in}}%
\pgfpathlineto{\pgfqpoint{1.287451in}{2.364202in}}%
\pgfpathlineto{\pgfqpoint{1.288367in}{2.254207in}}%
\pgfpathlineto{\pgfqpoint{1.290502in}{2.102155in}}%
\pgfpathlineto{\pgfqpoint{1.291113in}{2.158281in}}%
\pgfpathlineto{\pgfqpoint{1.292333in}{2.395500in}}%
\pgfpathlineto{\pgfqpoint{1.292638in}{2.332500in}}%
\pgfpathlineto{\pgfqpoint{1.293859in}{2.010858in}}%
\pgfpathlineto{\pgfqpoint{1.294469in}{2.115628in}}%
\pgfpathlineto{\pgfqpoint{1.295384in}{2.465364in}}%
\pgfpathlineto{\pgfqpoint{1.295994in}{2.357114in}}%
\pgfpathlineto{\pgfqpoint{1.297215in}{1.954212in}}%
\pgfpathlineto{\pgfqpoint{1.297825in}{2.012072in}}%
\pgfpathlineto{\pgfqpoint{1.299046in}{2.478240in}}%
\pgfpathlineto{\pgfqpoint{1.299656in}{2.316122in}}%
\pgfpathlineto{\pgfqpoint{1.300876in}{1.986159in}}%
\pgfpathlineto{\pgfqpoint{1.301487in}{2.103688in}}%
\pgfpathlineto{\pgfqpoint{1.302402in}{2.379654in}}%
\pgfpathlineto{\pgfqpoint{1.303012in}{2.297445in}}%
\pgfpathlineto{\pgfqpoint{1.303928in}{2.067175in}}%
\pgfpathlineto{\pgfqpoint{1.304843in}{2.201721in}}%
\pgfpathlineto{\pgfqpoint{1.305758in}{2.406482in}}%
\pgfpathlineto{\pgfqpoint{1.306369in}{2.268765in}}%
\pgfpathlineto{\pgfqpoint{1.307589in}{2.052096in}}%
\pgfpathlineto{\pgfqpoint{1.308199in}{2.121800in}}%
\pgfpathlineto{\pgfqpoint{1.309725in}{2.273203in}}%
\pgfpathlineto{\pgfqpoint{1.310335in}{2.252409in}}%
\pgfpathlineto{\pgfqpoint{1.312166in}{2.026491in}}%
\pgfpathlineto{\pgfqpoint{1.312471in}{2.074008in}}%
\pgfpathlineto{\pgfqpoint{1.313691in}{2.570538in}}%
\pgfpathlineto{\pgfqpoint{1.314302in}{2.322262in}}%
\pgfpathlineto{\pgfqpoint{1.315522in}{1.896266in}}%
\pgfpathlineto{\pgfqpoint{1.316132in}{2.014956in}}%
\pgfpathlineto{\pgfqpoint{1.317353in}{2.621087in}}%
\pgfpathlineto{\pgfqpoint{1.317963in}{2.292369in}}%
\pgfpathlineto{\pgfqpoint{1.318878in}{1.938419in}}%
\pgfpathlineto{\pgfqpoint{1.319489in}{2.029035in}}%
\pgfpathlineto{\pgfqpoint{1.320709in}{2.386944in}}%
\pgfpathlineto{\pgfqpoint{1.321319in}{2.220015in}}%
\pgfpathlineto{\pgfqpoint{1.322235in}{2.051521in}}%
\pgfpathlineto{\pgfqpoint{1.322845in}{2.145606in}}%
\pgfpathlineto{\pgfqpoint{1.323760in}{2.329254in}}%
\pgfpathlineto{\pgfqpoint{1.324370in}{2.272905in}}%
\pgfpathlineto{\pgfqpoint{1.325286in}{2.132272in}}%
\pgfpathlineto{\pgfqpoint{1.326201in}{2.200508in}}%
\pgfpathlineto{\pgfqpoint{1.326811in}{2.251600in}}%
\pgfpathlineto{\pgfqpoint{1.327422in}{2.217280in}}%
\pgfpathlineto{\pgfqpoint{1.330168in}{2.099218in}}%
\pgfpathlineto{\pgfqpoint{1.330473in}{2.104731in}}%
\pgfpathlineto{\pgfqpoint{1.331693in}{2.373364in}}%
\pgfpathlineto{\pgfqpoint{1.332609in}{2.209926in}}%
\pgfpathlineto{\pgfqpoint{1.333524in}{2.051425in}}%
\pgfpathlineto{\pgfqpoint{1.334439in}{2.124599in}}%
\pgfpathlineto{\pgfqpoint{1.335965in}{2.358264in}}%
\pgfpathlineto{\pgfqpoint{1.336575in}{2.312248in}}%
\pgfpathlineto{\pgfqpoint{1.337796in}{2.126185in}}%
\pgfpathlineto{\pgfqpoint{1.338406in}{2.165602in}}%
\pgfpathlineto{\pgfqpoint{1.339626in}{2.266435in}}%
\pgfpathlineto{\pgfqpoint{1.340237in}{2.208128in}}%
\pgfpathlineto{\pgfqpoint{1.341457in}{2.045977in}}%
\pgfpathlineto{\pgfqpoint{1.342067in}{2.099303in}}%
\pgfpathlineto{\pgfqpoint{1.343288in}{2.416996in}}%
\pgfpathlineto{\pgfqpoint{1.343898in}{2.322815in}}%
\pgfpathlineto{\pgfqpoint{1.345424in}{2.000259in}}%
\pgfpathlineto{\pgfqpoint{1.346034in}{2.111233in}}%
\pgfpathlineto{\pgfqpoint{1.347254in}{2.477730in}}%
\pgfpathlineto{\pgfqpoint{1.347865in}{2.229582in}}%
\pgfpathlineto{\pgfqpoint{1.348780in}{1.989926in}}%
\pgfpathlineto{\pgfqpoint{1.349390in}{2.110541in}}%
\pgfpathlineto{\pgfqpoint{1.350306in}{2.416837in}}%
\pgfpathlineto{\pgfqpoint{1.350916in}{2.316430in}}%
\pgfpathlineto{\pgfqpoint{1.352441in}{2.069070in}}%
\pgfpathlineto{\pgfqpoint{1.353052in}{2.075487in}}%
\pgfpathlineto{\pgfqpoint{1.353967in}{2.166198in}}%
\pgfpathlineto{\pgfqpoint{1.354882in}{2.382485in}}%
\pgfpathlineto{\pgfqpoint{1.355493in}{2.316675in}}%
\pgfpathlineto{\pgfqpoint{1.356713in}{2.090726in}}%
\pgfpathlineto{\pgfqpoint{1.357323in}{2.192814in}}%
\pgfpathlineto{\pgfqpoint{1.358239in}{2.331723in}}%
\pgfpathlineto{\pgfqpoint{1.358849in}{2.240724in}}%
\pgfpathlineto{\pgfqpoint{1.360069in}{2.086288in}}%
\pgfpathlineto{\pgfqpoint{1.360680in}{2.141754in}}%
\pgfpathlineto{\pgfqpoint{1.361595in}{2.251727in}}%
\pgfpathlineto{\pgfqpoint{1.362205in}{2.213927in}}%
\pgfpathlineto{\pgfqpoint{1.362815in}{2.159345in}}%
\pgfpathlineto{\pgfqpoint{1.363426in}{2.184896in}}%
\pgfpathlineto{\pgfqpoint{1.364341in}{2.252919in}}%
\pgfpathlineto{\pgfqpoint{1.364951in}{2.185716in}}%
\pgfpathlineto{\pgfqpoint{1.365561in}{2.144319in}}%
\pgfpathlineto{\pgfqpoint{1.366172in}{2.197166in}}%
\pgfpathlineto{\pgfqpoint{1.366782in}{2.246939in}}%
\pgfpathlineto{\pgfqpoint{1.367697in}{2.212384in}}%
\pgfpathlineto{\pgfqpoint{1.368307in}{2.231072in}}%
\pgfpathlineto{\pgfqpoint{1.368613in}{2.243405in}}%
\pgfpathlineto{\pgfqpoint{1.369223in}{2.213970in}}%
\pgfpathlineto{\pgfqpoint{1.370138in}{2.126579in}}%
\pgfpathlineto{\pgfqpoint{1.370748in}{2.161452in}}%
\pgfpathlineto{\pgfqpoint{1.371969in}{2.270404in}}%
\pgfpathlineto{\pgfqpoint{1.372579in}{2.241926in}}%
\pgfpathlineto{\pgfqpoint{1.374105in}{2.138859in}}%
\pgfpathlineto{\pgfqpoint{1.375020in}{2.030408in}}%
\pgfpathlineto{\pgfqpoint{1.375630in}{2.068069in}}%
\pgfpathlineto{\pgfqpoint{1.376851in}{2.444740in}}%
\pgfpathlineto{\pgfqpoint{1.377461in}{2.339108in}}%
\pgfpathlineto{\pgfqpoint{1.378681in}{1.968706in}}%
\pgfpathlineto{\pgfqpoint{1.379292in}{2.108221in}}%
\pgfpathlineto{\pgfqpoint{1.380207in}{2.625599in}}%
\pgfpathlineto{\pgfqpoint{1.380817in}{2.413931in}}%
\pgfpathlineto{\pgfqpoint{1.382038in}{1.927043in}}%
\pgfpathlineto{\pgfqpoint{1.382648in}{2.002004in}}%
\pgfpathlineto{\pgfqpoint{1.383869in}{2.377525in}}%
\pgfpathlineto{\pgfqpoint{1.384479in}{2.265966in}}%
\pgfpathlineto{\pgfqpoint{1.385699in}{2.015232in}}%
\pgfpathlineto{\pgfqpoint{1.386309in}{2.159962in}}%
\pgfpathlineto{\pgfqpoint{1.387225in}{2.469333in}}%
\pgfpathlineto{\pgfqpoint{1.387835in}{2.305693in}}%
\pgfpathlineto{\pgfqpoint{1.389056in}{1.995587in}}%
\pgfpathlineto{\pgfqpoint{1.389666in}{2.110871in}}%
\pgfpathlineto{\pgfqpoint{1.390886in}{2.479943in}}%
\pgfpathlineto{\pgfqpoint{1.391496in}{2.356763in}}%
\pgfpathlineto{\pgfqpoint{1.393327in}{1.966056in}}%
\pgfpathlineto{\pgfqpoint{1.393937in}{2.031834in}}%
\pgfpathlineto{\pgfqpoint{1.395158in}{2.462107in}}%
\pgfpathlineto{\pgfqpoint{1.396073in}{2.231923in}}%
\pgfpathlineto{\pgfqpoint{1.396989in}{1.998908in}}%
\pgfpathlineto{\pgfqpoint{1.397599in}{2.100006in}}%
\pgfpathlineto{\pgfqpoint{1.398514in}{2.434034in}}%
\pgfpathlineto{\pgfqpoint{1.399124in}{2.321592in}}%
\pgfpathlineto{\pgfqpoint{1.400345in}{1.971611in}}%
\pgfpathlineto{\pgfqpoint{1.400955in}{2.143095in}}%
\pgfpathlineto{\pgfqpoint{1.401870in}{2.514710in}}%
\pgfpathlineto{\pgfqpoint{1.402481in}{2.271905in}}%
\pgfpathlineto{\pgfqpoint{1.403396in}{1.978305in}}%
\pgfpathlineto{\pgfqpoint{1.404006in}{2.066548in}}%
\pgfpathlineto{\pgfqpoint{1.405227in}{2.390030in}}%
\pgfpathlineto{\pgfqpoint{1.405837in}{2.229624in}}%
\pgfpathlineto{\pgfqpoint{1.406752in}{2.015073in}}%
\pgfpathlineto{\pgfqpoint{1.407668in}{2.122194in}}%
\pgfpathlineto{\pgfqpoint{1.408583in}{2.300680in}}%
\pgfpathlineto{\pgfqpoint{1.409498in}{2.249195in}}%
\pgfpathlineto{\pgfqpoint{1.409804in}{2.243884in}}%
\pgfpathlineto{\pgfqpoint{1.410109in}{2.254239in}}%
\pgfpathlineto{\pgfqpoint{1.410414in}{2.264391in}}%
\pgfpathlineto{\pgfqpoint{1.410719in}{2.257102in}}%
\pgfpathlineto{\pgfqpoint{1.411939in}{2.120566in}}%
\pgfpathlineto{\pgfqpoint{1.412550in}{2.157770in}}%
\pgfpathlineto{\pgfqpoint{1.413770in}{2.338991in}}%
\pgfpathlineto{\pgfqpoint{1.414380in}{2.224250in}}%
\pgfpathlineto{\pgfqpoint{1.415296in}{2.054277in}}%
\pgfpathlineto{\pgfqpoint{1.415906in}{2.092099in}}%
\pgfpathlineto{\pgfqpoint{1.417126in}{2.358625in}}%
\pgfpathlineto{\pgfqpoint{1.417737in}{2.230454in}}%
\pgfpathlineto{\pgfqpoint{1.418957in}{1.939281in}}%
\pgfpathlineto{\pgfqpoint{1.419567in}{2.071070in}}%
\pgfpathlineto{\pgfqpoint{1.420788in}{2.630665in}}%
\pgfpathlineto{\pgfqpoint{1.421398in}{2.285973in}}%
\pgfpathlineto{\pgfqpoint{1.422313in}{2.024033in}}%
\pgfpathlineto{\pgfqpoint{1.423229in}{2.117895in}}%
\pgfpathlineto{\pgfqpoint{1.424754in}{2.274054in}}%
\pgfpathlineto{\pgfqpoint{1.425365in}{2.241479in}}%
\pgfpathlineto{\pgfqpoint{1.427195in}{2.040124in}}%
\pgfpathlineto{\pgfqpoint{1.427500in}{2.065366in}}%
\pgfpathlineto{\pgfqpoint{1.428721in}{2.571432in}}%
\pgfpathlineto{\pgfqpoint{1.429331in}{2.403939in}}%
\pgfpathlineto{\pgfqpoint{1.430552in}{1.844664in}}%
\pgfpathlineto{\pgfqpoint{1.431162in}{1.909909in}}%
\pgfpathlineto{\pgfqpoint{1.432382in}{2.739372in}}%
\pgfpathlineto{\pgfqpoint{1.432993in}{2.398884in}}%
\pgfpathlineto{\pgfqpoint{1.433908in}{1.881240in}}%
\pgfpathlineto{\pgfqpoint{1.434823in}{2.117235in}}%
\pgfpathlineto{\pgfqpoint{1.435739in}{2.532802in}}%
\pgfpathlineto{\pgfqpoint{1.436349in}{2.275863in}}%
\pgfpathlineto{\pgfqpoint{1.437569in}{1.936397in}}%
\pgfpathlineto{\pgfqpoint{1.438180in}{2.053330in}}%
\pgfpathlineto{\pgfqpoint{1.439400in}{2.456361in}}%
\pgfpathlineto{\pgfqpoint{1.440010in}{2.333617in}}%
\pgfpathlineto{\pgfqpoint{1.441536in}{2.083458in}}%
\pgfpathlineto{\pgfqpoint{1.442146in}{2.129473in}}%
\pgfpathlineto{\pgfqpoint{1.443367in}{2.334915in}}%
\pgfpathlineto{\pgfqpoint{1.443977in}{2.259954in}}%
\pgfpathlineto{\pgfqpoint{1.445197in}{2.070943in}}%
\pgfpathlineto{\pgfqpoint{1.445807in}{2.146309in}}%
\pgfpathlineto{\pgfqpoint{1.446723in}{2.382197in}}%
\pgfpathlineto{\pgfqpoint{1.447333in}{2.306789in}}%
\pgfpathlineto{\pgfqpoint{1.448554in}{2.012551in}}%
\pgfpathlineto{\pgfqpoint{1.449164in}{2.130122in}}%
\pgfpathlineto{\pgfqpoint{1.450079in}{2.435290in}}%
\pgfpathlineto{\pgfqpoint{1.450689in}{2.300244in}}%
\pgfpathlineto{\pgfqpoint{1.451910in}{1.896490in}}%
\pgfpathlineto{\pgfqpoint{1.452520in}{1.964853in}}%
\pgfpathlineto{\pgfqpoint{1.453741in}{2.679511in}}%
\pgfpathlineto{\pgfqpoint{1.454351in}{2.415145in}}%
\pgfpathlineto{\pgfqpoint{1.455571in}{2.017520in}}%
\pgfpathlineto{\pgfqpoint{1.456181in}{2.046605in}}%
\pgfpathlineto{\pgfqpoint{1.458012in}{2.356955in}}%
\pgfpathlineto{\pgfqpoint{1.458622in}{2.404162in}}%
\pgfpathlineto{\pgfqpoint{1.458928in}{2.351357in}}%
\pgfpathlineto{\pgfqpoint{1.460758in}{1.910761in}}%
\pgfpathlineto{\pgfqpoint{1.461369in}{2.024810in}}%
\pgfpathlineto{\pgfqpoint{1.462589in}{2.654673in}}%
\pgfpathlineto{\pgfqpoint{1.463199in}{2.261933in}}%
\pgfpathlineto{\pgfqpoint{1.463809in}{1.979444in}}%
\pgfpathlineto{\pgfqpoint{1.464725in}{2.232051in}}%
\pgfpathlineto{\pgfqpoint{1.465335in}{2.433044in}}%
\pgfpathlineto{\pgfqpoint{1.465945in}{2.234519in}}%
\pgfpathlineto{\pgfqpoint{1.466861in}{1.968025in}}%
\pgfpathlineto{\pgfqpoint{1.467471in}{2.051553in}}%
\pgfpathlineto{\pgfqpoint{1.468691in}{2.397777in}}%
\pgfpathlineto{\pgfqpoint{1.469302in}{2.320283in}}%
\pgfpathlineto{\pgfqpoint{1.470827in}{2.079382in}}%
\pgfpathlineto{\pgfqpoint{1.471437in}{2.118799in}}%
\pgfpathlineto{\pgfqpoint{1.473268in}{2.401587in}}%
\pgfpathlineto{\pgfqpoint{1.473878in}{2.293018in}}%
\pgfpathlineto{\pgfqpoint{1.475404in}{1.904535in}}%
\pgfpathlineto{\pgfqpoint{1.475709in}{1.956170in}}%
\pgfpathlineto{\pgfqpoint{1.476930in}{2.720142in}}%
\pgfpathlineto{\pgfqpoint{1.477540in}{2.438163in}}%
\pgfpathlineto{\pgfqpoint{1.478760in}{1.878409in}}%
\pgfpathlineto{\pgfqpoint{1.479370in}{2.013019in}}%
\pgfpathlineto{\pgfqpoint{1.480591in}{2.492596in}}%
\pgfpathlineto{\pgfqpoint{1.481201in}{2.218706in}}%
\pgfpathlineto{\pgfqpoint{1.482117in}{1.949902in}}%
\pgfpathlineto{\pgfqpoint{1.482727in}{2.047233in}}%
\pgfpathlineto{\pgfqpoint{1.483947in}{2.454360in}}%
\pgfpathlineto{\pgfqpoint{1.484557in}{2.295934in}}%
\pgfpathlineto{\pgfqpoint{1.485778in}{2.009166in}}%
\pgfpathlineto{\pgfqpoint{1.486388in}{2.070485in}}%
\pgfpathlineto{\pgfqpoint{1.487609in}{2.512561in}}%
\pgfpathlineto{\pgfqpoint{1.488524in}{2.250504in}}%
\pgfpathlineto{\pgfqpoint{1.489744in}{1.989511in}}%
\pgfpathlineto{\pgfqpoint{1.490355in}{2.037548in}}%
\pgfpathlineto{\pgfqpoint{1.491880in}{2.473239in}}%
\pgfpathlineto{\pgfqpoint{1.492491in}{2.262699in}}%
\pgfpathlineto{\pgfqpoint{1.493711in}{1.898342in}}%
\pgfpathlineto{\pgfqpoint{1.494321in}{2.087086in}}%
\pgfpathlineto{\pgfqpoint{1.495237in}{2.697102in}}%
\pgfpathlineto{\pgfqpoint{1.495847in}{2.385305in}}%
\pgfpathlineto{\pgfqpoint{1.497067in}{1.908462in}}%
\pgfpathlineto{\pgfqpoint{1.497678in}{2.053192in}}%
\pgfpathlineto{\pgfqpoint{1.498898in}{2.484232in}}%
\pgfpathlineto{\pgfqpoint{1.499508in}{2.264487in}}%
\pgfpathlineto{\pgfqpoint{1.500729in}{1.995896in}}%
\pgfpathlineto{\pgfqpoint{1.501339in}{2.095579in}}%
\pgfpathlineto{\pgfqpoint{1.502559in}{2.301170in}}%
\pgfpathlineto{\pgfqpoint{1.503170in}{2.210682in}}%
\pgfpathlineto{\pgfqpoint{1.504085in}{2.128260in}}%
\pgfpathlineto{\pgfqpoint{1.504695in}{2.192537in}}%
\pgfpathlineto{\pgfqpoint{1.505916in}{2.324316in}}%
\pgfpathlineto{\pgfqpoint{1.506526in}{2.276640in}}%
\pgfpathlineto{\pgfqpoint{1.508357in}{2.117554in}}%
\pgfpathlineto{\pgfqpoint{1.508967in}{2.133879in}}%
\pgfpathlineto{\pgfqpoint{1.510187in}{2.236446in}}%
\pgfpathlineto{\pgfqpoint{1.510798in}{2.211959in}}%
\pgfpathlineto{\pgfqpoint{1.511713in}{2.102443in}}%
\pgfpathlineto{\pgfqpoint{1.512323in}{2.168337in}}%
\pgfpathlineto{\pgfqpoint{1.513239in}{2.449092in}}%
\pgfpathlineto{\pgfqpoint{1.513849in}{2.324603in}}%
\pgfpathlineto{\pgfqpoint{1.515069in}{1.974186in}}%
\pgfpathlineto{\pgfqpoint{1.515680in}{2.070719in}}%
\pgfpathlineto{\pgfqpoint{1.516900in}{2.401278in}}%
\pgfpathlineto{\pgfqpoint{1.517510in}{2.236722in}}%
\pgfpathlineto{\pgfqpoint{1.518426in}{2.066760in}}%
\pgfpathlineto{\pgfqpoint{1.519036in}{2.161644in}}%
\pgfpathlineto{\pgfqpoint{1.519951in}{2.296541in}}%
\pgfpathlineto{\pgfqpoint{1.520867in}{2.235839in}}%
\pgfpathlineto{\pgfqpoint{1.521782in}{2.228251in}}%
\pgfpathlineto{\pgfqpoint{1.523613in}{2.059258in}}%
\pgfpathlineto{\pgfqpoint{1.524223in}{2.136699in}}%
\pgfpathlineto{\pgfqpoint{1.525443in}{2.352634in}}%
\pgfpathlineto{\pgfqpoint{1.526054in}{2.245342in}}%
\pgfpathlineto{\pgfqpoint{1.526969in}{2.142754in}}%
\pgfpathlineto{\pgfqpoint{1.527579in}{2.203935in}}%
\pgfpathlineto{\pgfqpoint{1.528189in}{2.247066in}}%
\pgfpathlineto{\pgfqpoint{1.528800in}{2.186322in}}%
\pgfpathlineto{\pgfqpoint{1.529715in}{2.090896in}}%
\pgfpathlineto{\pgfqpoint{1.530325in}{2.180235in}}%
\pgfpathlineto{\pgfqpoint{1.531241in}{2.404875in}}%
\pgfpathlineto{\pgfqpoint{1.531851in}{2.292135in}}%
\pgfpathlineto{\pgfqpoint{1.533376in}{1.967067in}}%
\pgfpathlineto{\pgfqpoint{1.533987in}{2.073475in}}%
\pgfpathlineto{\pgfqpoint{1.535207in}{2.516168in}}%
\pgfpathlineto{\pgfqpoint{1.535817in}{2.326082in}}%
\pgfpathlineto{\pgfqpoint{1.537038in}{1.941090in}}%
\pgfpathlineto{\pgfqpoint{1.537648in}{2.015764in}}%
\pgfpathlineto{\pgfqpoint{1.539174in}{2.519340in}}%
\pgfpathlineto{\pgfqpoint{1.539784in}{2.349388in}}%
\pgfpathlineto{\pgfqpoint{1.541615in}{2.083936in}}%
\pgfpathlineto{\pgfqpoint{1.541920in}{2.092078in}}%
\pgfpathlineto{\pgfqpoint{1.543445in}{2.188429in}}%
\pgfpathlineto{\pgfqpoint{1.544056in}{2.160175in}}%
\pgfpathlineto{\pgfqpoint{1.544361in}{2.149022in}}%
\pgfpathlineto{\pgfqpoint{1.544971in}{2.170806in}}%
\pgfpathlineto{\pgfqpoint{1.546191in}{2.315717in}}%
\pgfpathlineto{\pgfqpoint{1.546802in}{2.253803in}}%
\pgfpathlineto{\pgfqpoint{1.548022in}{2.068378in}}%
\pgfpathlineto{\pgfqpoint{1.548632in}{2.165294in}}%
\pgfpathlineto{\pgfqpoint{1.549548in}{2.394808in}}%
\pgfpathlineto{\pgfqpoint{1.550158in}{2.268648in}}%
\pgfpathlineto{\pgfqpoint{1.551378in}{2.011678in}}%
\pgfpathlineto{\pgfqpoint{1.551989in}{2.146213in}}%
\pgfpathlineto{\pgfqpoint{1.552904in}{2.401949in}}%
\pgfpathlineto{\pgfqpoint{1.553514in}{2.299201in}}%
\pgfpathlineto{\pgfqpoint{1.554735in}{2.080435in}}%
\pgfpathlineto{\pgfqpoint{1.555345in}{2.130623in}}%
\pgfpathlineto{\pgfqpoint{1.556565in}{2.292784in}}%
\pgfpathlineto{\pgfqpoint{1.557176in}{2.232146in}}%
\pgfpathlineto{\pgfqpoint{1.558701in}{2.048456in}}%
\pgfpathlineto{\pgfqpoint{1.559311in}{2.121928in}}%
\pgfpathlineto{\pgfqpoint{1.560532in}{2.487882in}}%
\pgfpathlineto{\pgfqpoint{1.561142in}{2.353453in}}%
\pgfpathlineto{\pgfqpoint{1.562363in}{1.993895in}}%
\pgfpathlineto{\pgfqpoint{1.562973in}{2.060035in}}%
\pgfpathlineto{\pgfqpoint{1.564193in}{2.430458in}}%
\pgfpathlineto{\pgfqpoint{1.564804in}{2.324859in}}%
\pgfpathlineto{\pgfqpoint{1.566329in}{2.002058in}}%
\pgfpathlineto{\pgfqpoint{1.566939in}{2.102347in}}%
\pgfpathlineto{\pgfqpoint{1.568160in}{2.399192in}}%
\pgfpathlineto{\pgfqpoint{1.568770in}{2.225282in}}%
\pgfpathlineto{\pgfqpoint{1.569685in}{1.997726in}}%
\pgfpathlineto{\pgfqpoint{1.570296in}{2.082904in}}%
\pgfpathlineto{\pgfqpoint{1.571516in}{2.527906in}}%
\pgfpathlineto{\pgfqpoint{1.572126in}{2.286399in}}%
\pgfpathlineto{\pgfqpoint{1.573042in}{2.011827in}}%
\pgfpathlineto{\pgfqpoint{1.573957in}{2.137625in}}%
\pgfpathlineto{\pgfqpoint{1.574872in}{2.313025in}}%
\pgfpathlineto{\pgfqpoint{1.575483in}{2.251440in}}%
\pgfpathlineto{\pgfqpoint{1.576398in}{2.122982in}}%
\pgfpathlineto{\pgfqpoint{1.577313in}{2.170434in}}%
\pgfpathlineto{\pgfqpoint{1.577924in}{2.189930in}}%
\pgfpathlineto{\pgfqpoint{1.578534in}{2.161090in}}%
\pgfpathlineto{\pgfqpoint{1.578839in}{2.153247in}}%
\pgfpathlineto{\pgfqpoint{1.579144in}{2.161037in}}%
\pgfpathlineto{\pgfqpoint{1.580365in}{2.239085in}}%
\pgfpathlineto{\pgfqpoint{1.581280in}{2.221632in}}%
\pgfpathlineto{\pgfqpoint{1.582195in}{2.219323in}}%
\pgfpathlineto{\pgfqpoint{1.583111in}{2.192495in}}%
\pgfpathlineto{\pgfqpoint{1.583721in}{2.204701in}}%
\pgfpathlineto{\pgfqpoint{1.584636in}{2.246736in}}%
\pgfpathlineto{\pgfqpoint{1.585246in}{2.230422in}}%
\pgfpathlineto{\pgfqpoint{1.587687in}{2.132368in}}%
\pgfpathlineto{\pgfqpoint{1.587993in}{2.137827in}}%
\pgfpathlineto{\pgfqpoint{1.589213in}{2.276768in}}%
\pgfpathlineto{\pgfqpoint{1.589823in}{2.347207in}}%
\pgfpathlineto{\pgfqpoint{1.590433in}{2.289379in}}%
\pgfpathlineto{\pgfqpoint{1.591959in}{1.978635in}}%
\pgfpathlineto{\pgfqpoint{1.592569in}{2.047743in}}%
\pgfpathlineto{\pgfqpoint{1.593790in}{2.515860in}}%
\pgfpathlineto{\pgfqpoint{1.594400in}{2.338906in}}%
\pgfpathlineto{\pgfqpoint{1.595620in}{1.959075in}}%
\pgfpathlineto{\pgfqpoint{1.596231in}{2.119661in}}%
\pgfpathlineto{\pgfqpoint{1.597146in}{2.547902in}}%
\pgfpathlineto{\pgfqpoint{1.597756in}{2.371577in}}%
\pgfpathlineto{\pgfqpoint{1.598977in}{1.964066in}}%
\pgfpathlineto{\pgfqpoint{1.599587in}{2.006485in}}%
\pgfpathlineto{\pgfqpoint{1.601113in}{2.346781in}}%
\pgfpathlineto{\pgfqpoint{1.601723in}{2.250163in}}%
\pgfpathlineto{\pgfqpoint{1.602638in}{2.139732in}}%
\pgfpathlineto{\pgfqpoint{1.603248in}{2.194985in}}%
\pgfpathlineto{\pgfqpoint{1.604164in}{2.268606in}}%
\pgfpathlineto{\pgfqpoint{1.604774in}{2.245630in}}%
\pgfpathlineto{\pgfqpoint{1.606300in}{2.091801in}}%
\pgfpathlineto{\pgfqpoint{1.606910in}{2.169721in}}%
\pgfpathlineto{\pgfqpoint{1.607825in}{2.376099in}}%
\pgfpathlineto{\pgfqpoint{1.608435in}{2.306778in}}%
\pgfpathlineto{\pgfqpoint{1.609656in}{2.043199in}}%
\pgfpathlineto{\pgfqpoint{1.610266in}{2.084096in}}%
\pgfpathlineto{\pgfqpoint{1.611487in}{2.300329in}}%
\pgfpathlineto{\pgfqpoint{1.612097in}{2.239053in}}%
\pgfpathlineto{\pgfqpoint{1.613317in}{2.079116in}}%
\pgfpathlineto{\pgfqpoint{1.613928in}{2.168870in}}%
\pgfpathlineto{\pgfqpoint{1.614843in}{2.427989in}}%
\pgfpathlineto{\pgfqpoint{1.615453in}{2.348313in}}%
\pgfpathlineto{\pgfqpoint{1.616674in}{2.038059in}}%
\pgfpathlineto{\pgfqpoint{1.617284in}{2.127643in}}%
\pgfpathlineto{\pgfqpoint{1.618199in}{2.361201in}}%
\pgfpathlineto{\pgfqpoint{1.618809in}{2.278875in}}%
\pgfpathlineto{\pgfqpoint{1.620030in}{2.034047in}}%
\pgfpathlineto{\pgfqpoint{1.620640in}{2.136752in}}%
\pgfpathlineto{\pgfqpoint{1.621556in}{2.368991in}}%
\pgfpathlineto{\pgfqpoint{1.622166in}{2.279173in}}%
\pgfpathlineto{\pgfqpoint{1.623386in}{2.064483in}}%
\pgfpathlineto{\pgfqpoint{1.623996in}{2.092812in}}%
\pgfpathlineto{\pgfqpoint{1.626132in}{2.354220in}}%
\pgfpathlineto{\pgfqpoint{1.626743in}{2.279258in}}%
\pgfpathlineto{\pgfqpoint{1.628268in}{2.039294in}}%
\pgfpathlineto{\pgfqpoint{1.628573in}{2.069985in}}%
\pgfpathlineto{\pgfqpoint{1.630099in}{2.441419in}}%
\pgfpathlineto{\pgfqpoint{1.630709in}{2.256559in}}%
\pgfpathlineto{\pgfqpoint{1.631624in}{2.055544in}}%
\pgfpathlineto{\pgfqpoint{1.632540in}{2.121790in}}%
\pgfpathlineto{\pgfqpoint{1.633760in}{2.265881in}}%
\pgfpathlineto{\pgfqpoint{1.634370in}{2.227719in}}%
\pgfpathlineto{\pgfqpoint{1.635591in}{2.116235in}}%
\pgfpathlineto{\pgfqpoint{1.636201in}{2.159526in}}%
\pgfpathlineto{\pgfqpoint{1.637117in}{2.253483in}}%
\pgfpathlineto{\pgfqpoint{1.637727in}{2.212118in}}%
\pgfpathlineto{\pgfqpoint{1.638337in}{2.151885in}}%
\pgfpathlineto{\pgfqpoint{1.638947in}{2.213619in}}%
\pgfpathlineto{\pgfqpoint{1.639863in}{2.436907in}}%
\pgfpathlineto{\pgfqpoint{1.640473in}{2.325306in}}%
\pgfpathlineto{\pgfqpoint{1.641998in}{2.043731in}}%
\pgfpathlineto{\pgfqpoint{1.642609in}{2.070847in}}%
\pgfpathlineto{\pgfqpoint{1.644439in}{2.225591in}}%
\pgfpathlineto{\pgfqpoint{1.645355in}{2.345706in}}%
\pgfpathlineto{\pgfqpoint{1.645965in}{2.254111in}}%
\pgfpathlineto{\pgfqpoint{1.646880in}{2.117405in}}%
\pgfpathlineto{\pgfqpoint{1.647491in}{2.215683in}}%
\pgfpathlineto{\pgfqpoint{1.648406in}{2.398788in}}%
\pgfpathlineto{\pgfqpoint{1.649016in}{2.236850in}}%
\pgfpathlineto{\pgfqpoint{1.650237in}{1.979933in}}%
\pgfpathlineto{\pgfqpoint{1.650847in}{2.115000in}}%
\pgfpathlineto{\pgfqpoint{1.651762in}{2.428745in}}%
\pgfpathlineto{\pgfqpoint{1.652372in}{2.294806in}}%
\pgfpathlineto{\pgfqpoint{1.653288in}{2.011125in}}%
\pgfpathlineto{\pgfqpoint{1.654203in}{2.166965in}}%
\pgfpathlineto{\pgfqpoint{1.655119in}{2.420636in}}%
\pgfpathlineto{\pgfqpoint{1.655729in}{2.275757in}}%
\pgfpathlineto{\pgfqpoint{1.656949in}{2.048435in}}%
\pgfpathlineto{\pgfqpoint{1.657559in}{2.102347in}}%
\pgfpathlineto{\pgfqpoint{1.658475in}{2.203392in}}%
\pgfpathlineto{\pgfqpoint{1.659390in}{2.152215in}}%
\pgfpathlineto{\pgfqpoint{1.659695in}{2.135965in}}%
\pgfpathlineto{\pgfqpoint{1.660000in}{2.144531in}}%
\pgfpathlineto{\pgfqpoint{1.661526in}{2.472792in}}%
\pgfpathlineto{\pgfqpoint{1.662441in}{2.288559in}}%
\pgfpathlineto{\pgfqpoint{1.663967in}{2.045711in}}%
\pgfpathlineto{\pgfqpoint{1.664577in}{2.060929in}}%
\pgfpathlineto{\pgfqpoint{1.666408in}{2.326987in}}%
\pgfpathlineto{\pgfqpoint{1.667323in}{2.216822in}}%
\pgfpathlineto{\pgfqpoint{1.668544in}{2.007134in}}%
\pgfpathlineto{\pgfqpoint{1.669154in}{2.054703in}}%
\pgfpathlineto{\pgfqpoint{1.670680in}{2.525076in}}%
\pgfpathlineto{\pgfqpoint{1.671290in}{2.301244in}}%
\pgfpathlineto{\pgfqpoint{1.672205in}{2.043370in}}%
\pgfpathlineto{\pgfqpoint{1.673120in}{2.143372in}}%
\pgfpathlineto{\pgfqpoint{1.675561in}{2.283376in}}%
\pgfpathlineto{\pgfqpoint{1.675867in}{2.306161in}}%
\pgfpathlineto{\pgfqpoint{1.676477in}{2.241926in}}%
\pgfpathlineto{\pgfqpoint{1.677697in}{2.055437in}}%
\pgfpathlineto{\pgfqpoint{1.678307in}{2.101677in}}%
\pgfpathlineto{\pgfqpoint{1.679833in}{2.293039in}}%
\pgfpathlineto{\pgfqpoint{1.680748in}{2.254079in}}%
\pgfpathlineto{\pgfqpoint{1.681969in}{2.176649in}}%
\pgfpathlineto{\pgfqpoint{1.683189in}{2.073263in}}%
\pgfpathlineto{\pgfqpoint{1.683800in}{2.140562in}}%
\pgfpathlineto{\pgfqpoint{1.685020in}{2.403481in}}%
\pgfpathlineto{\pgfqpoint{1.685630in}{2.282908in}}%
\pgfpathlineto{\pgfqpoint{1.686546in}{2.116916in}}%
\pgfpathlineto{\pgfqpoint{1.687461in}{2.174265in}}%
\pgfpathlineto{\pgfqpoint{1.687766in}{2.185950in}}%
\pgfpathlineto{\pgfqpoint{1.688376in}{2.161590in}}%
\pgfpathlineto{\pgfqpoint{1.688987in}{2.129154in}}%
\pgfpathlineto{\pgfqpoint{1.689597in}{2.161069in}}%
\pgfpathlineto{\pgfqpoint{1.690817in}{2.269574in}}%
\pgfpathlineto{\pgfqpoint{1.691122in}{2.242192in}}%
\pgfpathlineto{\pgfqpoint{1.692343in}{2.112989in}}%
\pgfpathlineto{\pgfqpoint{1.692953in}{2.173499in}}%
\pgfpathlineto{\pgfqpoint{1.693869in}{2.300925in}}%
\pgfpathlineto{\pgfqpoint{1.694784in}{2.256931in}}%
\pgfpathlineto{\pgfqpoint{1.695089in}{2.260166in}}%
\pgfpathlineto{\pgfqpoint{1.695699in}{2.305075in}}%
\pgfpathlineto{\pgfqpoint{1.696309in}{2.262486in}}%
\pgfpathlineto{\pgfqpoint{1.697835in}{2.013008in}}%
\pgfpathlineto{\pgfqpoint{1.698140in}{2.051660in}}%
\pgfpathlineto{\pgfqpoint{1.699361in}{2.392413in}}%
\pgfpathlineto{\pgfqpoint{1.699971in}{2.285760in}}%
\pgfpathlineto{\pgfqpoint{1.701496in}{1.942846in}}%
\pgfpathlineto{\pgfqpoint{1.702107in}{2.034132in}}%
\pgfpathlineto{\pgfqpoint{1.703327in}{2.640306in}}%
\pgfpathlineto{\pgfqpoint{1.703937in}{2.442686in}}%
\pgfpathlineto{\pgfqpoint{1.705158in}{2.009060in}}%
\pgfpathlineto{\pgfqpoint{1.705768in}{2.076657in}}%
\pgfpathlineto{\pgfqpoint{1.706989in}{2.272532in}}%
\pgfpathlineto{\pgfqpoint{1.707599in}{2.215768in}}%
\pgfpathlineto{\pgfqpoint{1.708209in}{2.152034in}}%
\pgfpathlineto{\pgfqpoint{1.708819in}{2.184087in}}%
\pgfpathlineto{\pgfqpoint{1.709735in}{2.260773in}}%
\pgfpathlineto{\pgfqpoint{1.710345in}{2.199433in}}%
\pgfpathlineto{\pgfqpoint{1.711260in}{2.113021in}}%
\pgfpathlineto{\pgfqpoint{1.711870in}{2.141701in}}%
\pgfpathlineto{\pgfqpoint{1.713091in}{2.271936in}}%
\pgfpathlineto{\pgfqpoint{1.713701in}{2.241352in}}%
\pgfpathlineto{\pgfqpoint{1.714922in}{2.147884in}}%
\pgfpathlineto{\pgfqpoint{1.715532in}{2.174265in}}%
\pgfpathlineto{\pgfqpoint{1.717668in}{2.301787in}}%
\pgfpathlineto{\pgfqpoint{1.717973in}{2.294093in}}%
\pgfpathlineto{\pgfqpoint{1.719804in}{2.015860in}}%
\pgfpathlineto{\pgfqpoint{1.720414in}{2.116267in}}%
\pgfpathlineto{\pgfqpoint{1.721634in}{2.415272in}}%
\pgfpathlineto{\pgfqpoint{1.722244in}{2.260677in}}%
\pgfpathlineto{\pgfqpoint{1.723160in}{2.113862in}}%
\pgfpathlineto{\pgfqpoint{1.724075in}{2.148490in}}%
\pgfpathlineto{\pgfqpoint{1.726211in}{2.244502in}}%
\pgfpathlineto{\pgfqpoint{1.726821in}{2.269021in}}%
\pgfpathlineto{\pgfqpoint{1.727431in}{2.231742in}}%
\pgfpathlineto{\pgfqpoint{1.728652in}{2.099612in}}%
\pgfpathlineto{\pgfqpoint{1.729262in}{2.169434in}}%
\pgfpathlineto{\pgfqpoint{1.730178in}{2.341386in}}%
\pgfpathlineto{\pgfqpoint{1.730788in}{2.249674in}}%
\pgfpathlineto{\pgfqpoint{1.731703in}{2.080095in}}%
\pgfpathlineto{\pgfqpoint{1.732313in}{2.126791in}}%
\pgfpathlineto{\pgfqpoint{1.733229in}{2.251802in}}%
\pgfpathlineto{\pgfqpoint{1.734144in}{2.181629in}}%
\pgfpathlineto{\pgfqpoint{1.734754in}{2.127036in}}%
\pgfpathlineto{\pgfqpoint{1.735365in}{2.159451in}}%
\pgfpathlineto{\pgfqpoint{1.736585in}{2.379122in}}%
\pgfpathlineto{\pgfqpoint{1.737195in}{2.280939in}}%
\pgfpathlineto{\pgfqpoint{1.738416in}{2.004676in}}%
\pgfpathlineto{\pgfqpoint{1.739026in}{2.101804in}}%
\pgfpathlineto{\pgfqpoint{1.740246in}{2.450316in}}%
\pgfpathlineto{\pgfqpoint{1.740857in}{2.256101in}}%
\pgfpathlineto{\pgfqpoint{1.742077in}{2.039528in}}%
\pgfpathlineto{\pgfqpoint{1.742687in}{2.087076in}}%
\pgfpathlineto{\pgfqpoint{1.744823in}{2.291709in}}%
\pgfpathlineto{\pgfqpoint{1.745433in}{2.259666in}}%
\pgfpathlineto{\pgfqpoint{1.746654in}{2.198422in}}%
\pgfpathlineto{\pgfqpoint{1.747569in}{2.202636in}}%
\pgfpathlineto{\pgfqpoint{1.748180in}{2.188323in}}%
\pgfpathlineto{\pgfqpoint{1.749400in}{2.130548in}}%
\pgfpathlineto{\pgfqpoint{1.750010in}{2.157727in}}%
\pgfpathlineto{\pgfqpoint{1.751231in}{2.242341in}}%
\pgfpathlineto{\pgfqpoint{1.751536in}{2.220110in}}%
\pgfpathlineto{\pgfqpoint{1.752451in}{2.105029in}}%
\pgfpathlineto{\pgfqpoint{1.753061in}{2.147224in}}%
\pgfpathlineto{\pgfqpoint{1.754282in}{2.424446in}}%
\pgfpathlineto{\pgfqpoint{1.754892in}{2.260028in}}%
\pgfpathlineto{\pgfqpoint{1.755807in}{2.067186in}}%
\pgfpathlineto{\pgfqpoint{1.756723in}{2.142329in}}%
\pgfpathlineto{\pgfqpoint{1.757638in}{2.207000in}}%
\pgfpathlineto{\pgfqpoint{1.758248in}{2.173233in}}%
\pgfpathlineto{\pgfqpoint{1.758554in}{2.163389in}}%
\pgfpathlineto{\pgfqpoint{1.758859in}{2.173626in}}%
\pgfpathlineto{\pgfqpoint{1.760079in}{2.327072in}}%
\pgfpathlineto{\pgfqpoint{1.760689in}{2.254707in}}%
\pgfpathlineto{\pgfqpoint{1.761910in}{2.057672in}}%
\pgfpathlineto{\pgfqpoint{1.762520in}{2.107668in}}%
\pgfpathlineto{\pgfqpoint{1.764046in}{2.319633in}}%
\pgfpathlineto{\pgfqpoint{1.764656in}{2.254122in}}%
\pgfpathlineto{\pgfqpoint{1.767097in}{2.065441in}}%
\pgfpathlineto{\pgfqpoint{1.767402in}{2.087097in}}%
\pgfpathlineto{\pgfqpoint{1.768928in}{2.545582in}}%
\pgfpathlineto{\pgfqpoint{1.769538in}{2.318548in}}%
\pgfpathlineto{\pgfqpoint{1.770758in}{2.008815in}}%
\pgfpathlineto{\pgfqpoint{1.771369in}{2.075827in}}%
\pgfpathlineto{\pgfqpoint{1.772894in}{2.271426in}}%
\pgfpathlineto{\pgfqpoint{1.773504in}{2.224005in}}%
\pgfpathlineto{\pgfqpoint{1.775335in}{2.084415in}}%
\pgfpathlineto{\pgfqpoint{1.775945in}{2.123918in}}%
\pgfpathlineto{\pgfqpoint{1.777471in}{2.427202in}}%
\pgfpathlineto{\pgfqpoint{1.778081in}{2.249237in}}%
\pgfpathlineto{\pgfqpoint{1.779302in}{1.986669in}}%
\pgfpathlineto{\pgfqpoint{1.779912in}{2.123077in}}%
\pgfpathlineto{\pgfqpoint{1.780827in}{2.491170in}}%
\pgfpathlineto{\pgfqpoint{1.781437in}{2.347164in}}%
\pgfpathlineto{\pgfqpoint{1.782658in}{2.021309in}}%
\pgfpathlineto{\pgfqpoint{1.783268in}{2.135252in}}%
\pgfpathlineto{\pgfqpoint{1.784183in}{2.309428in}}%
\pgfpathlineto{\pgfqpoint{1.784794in}{2.208277in}}%
\pgfpathlineto{\pgfqpoint{1.785709in}{2.024203in}}%
\pgfpathlineto{\pgfqpoint{1.786319in}{2.083447in}}%
\pgfpathlineto{\pgfqpoint{1.787540in}{2.529396in}}%
\pgfpathlineto{\pgfqpoint{1.788150in}{2.315068in}}%
\pgfpathlineto{\pgfqpoint{1.789370in}{1.955063in}}%
\pgfpathlineto{\pgfqpoint{1.789981in}{2.061397in}}%
\pgfpathlineto{\pgfqpoint{1.791201in}{2.459458in}}%
\pgfpathlineto{\pgfqpoint{1.791811in}{2.282068in}}%
\pgfpathlineto{\pgfqpoint{1.793032in}{2.000951in}}%
\pgfpathlineto{\pgfqpoint{1.793642in}{2.101251in}}%
\pgfpathlineto{\pgfqpoint{1.794863in}{2.467258in}}%
\pgfpathlineto{\pgfqpoint{1.795473in}{2.304756in}}%
\pgfpathlineto{\pgfqpoint{1.796693in}{2.011252in}}%
\pgfpathlineto{\pgfqpoint{1.797304in}{2.060397in}}%
\pgfpathlineto{\pgfqpoint{1.798829in}{2.264785in}}%
\pgfpathlineto{\pgfqpoint{1.799439in}{2.201338in}}%
\pgfpathlineto{\pgfqpoint{1.800050in}{2.157110in}}%
\pgfpathlineto{\pgfqpoint{1.800660in}{2.230997in}}%
\pgfpathlineto{\pgfqpoint{1.801575in}{2.376866in}}%
\pgfpathlineto{\pgfqpoint{1.802185in}{2.269765in}}%
\pgfpathlineto{\pgfqpoint{1.803406in}{2.062695in}}%
\pgfpathlineto{\pgfqpoint{1.804016in}{2.118853in}}%
\pgfpathlineto{\pgfqpoint{1.805237in}{2.302330in}}%
\pgfpathlineto{\pgfqpoint{1.805847in}{2.214002in}}%
\pgfpathlineto{\pgfqpoint{1.806762in}{2.055384in}}%
\pgfpathlineto{\pgfqpoint{1.807372in}{2.098665in}}%
\pgfpathlineto{\pgfqpoint{1.808593in}{2.342460in}}%
\pgfpathlineto{\pgfqpoint{1.809203in}{2.262486in}}%
\pgfpathlineto{\pgfqpoint{1.810729in}{2.067410in}}%
\pgfpathlineto{\pgfqpoint{1.811339in}{2.119534in}}%
\pgfpathlineto{\pgfqpoint{1.812865in}{2.417805in}}%
\pgfpathlineto{\pgfqpoint{1.813475in}{2.354273in}}%
\pgfpathlineto{\pgfqpoint{1.815306in}{2.074561in}}%
\pgfpathlineto{\pgfqpoint{1.815916in}{2.104827in}}%
\pgfpathlineto{\pgfqpoint{1.817441in}{2.317750in}}%
\pgfpathlineto{\pgfqpoint{1.818052in}{2.207297in}}%
\pgfpathlineto{\pgfqpoint{1.819272in}{2.013891in}}%
\pgfpathlineto{\pgfqpoint{1.819882in}{2.090811in}}%
\pgfpathlineto{\pgfqpoint{1.821103in}{2.444335in}}%
\pgfpathlineto{\pgfqpoint{1.821713in}{2.354071in}}%
\pgfpathlineto{\pgfqpoint{1.823239in}{2.008028in}}%
\pgfpathlineto{\pgfqpoint{1.823849in}{2.106348in}}%
\pgfpathlineto{\pgfqpoint{1.825069in}{2.393201in}}%
\pgfpathlineto{\pgfqpoint{1.825680in}{2.260624in}}%
\pgfpathlineto{\pgfqpoint{1.826595in}{2.089970in}}%
\pgfpathlineto{\pgfqpoint{1.827510in}{2.167454in}}%
\pgfpathlineto{\pgfqpoint{1.828426in}{2.249227in}}%
\pgfpathlineto{\pgfqpoint{1.829036in}{2.194240in}}%
\pgfpathlineto{\pgfqpoint{1.829646in}{2.132432in}}%
\pgfpathlineto{\pgfqpoint{1.830256in}{2.171700in}}%
\pgfpathlineto{\pgfqpoint{1.831477in}{2.391019in}}%
\pgfpathlineto{\pgfqpoint{1.831782in}{2.331925in}}%
\pgfpathlineto{\pgfqpoint{1.833307in}{1.953722in}}%
\pgfpathlineto{\pgfqpoint{1.833918in}{2.043359in}}%
\pgfpathlineto{\pgfqpoint{1.835138in}{2.521191in}}%
\pgfpathlineto{\pgfqpoint{1.835748in}{2.305959in}}%
\pgfpathlineto{\pgfqpoint{1.836664in}{1.982732in}}%
\pgfpathlineto{\pgfqpoint{1.837579in}{2.142339in}}%
\pgfpathlineto{\pgfqpoint{1.838494in}{2.471760in}}%
\pgfpathlineto{\pgfqpoint{1.839105in}{2.339906in}}%
\pgfpathlineto{\pgfqpoint{1.840325in}{2.007464in}}%
\pgfpathlineto{\pgfqpoint{1.840935in}{2.053564in}}%
\pgfpathlineto{\pgfqpoint{1.842461in}{2.297732in}}%
\pgfpathlineto{\pgfqpoint{1.843376in}{2.228901in}}%
\pgfpathlineto{\pgfqpoint{1.844292in}{2.171530in}}%
\pgfpathlineto{\pgfqpoint{1.844597in}{2.196379in}}%
\pgfpathlineto{\pgfqpoint{1.845817in}{2.321804in}}%
\pgfpathlineto{\pgfqpoint{1.846122in}{2.289964in}}%
\pgfpathlineto{\pgfqpoint{1.847953in}{2.033047in}}%
\pgfpathlineto{\pgfqpoint{1.848563in}{2.081574in}}%
\pgfpathlineto{\pgfqpoint{1.850089in}{2.380133in}}%
\pgfpathlineto{\pgfqpoint{1.850699in}{2.290922in}}%
\pgfpathlineto{\pgfqpoint{1.851920in}{2.083905in}}%
\pgfpathlineto{\pgfqpoint{1.852530in}{2.125110in}}%
\pgfpathlineto{\pgfqpoint{1.853750in}{2.293986in}}%
\pgfpathlineto{\pgfqpoint{1.854361in}{2.225453in}}%
\pgfpathlineto{\pgfqpoint{1.855276in}{2.126217in}}%
\pgfpathlineto{\pgfqpoint{1.855886in}{2.175233in}}%
\pgfpathlineto{\pgfqpoint{1.856802in}{2.249450in}}%
\pgfpathlineto{\pgfqpoint{1.857412in}{2.182970in}}%
\pgfpathlineto{\pgfqpoint{1.858022in}{2.117256in}}%
\pgfpathlineto{\pgfqpoint{1.858632in}{2.191005in}}%
\pgfpathlineto{\pgfqpoint{1.859548in}{2.383347in}}%
\pgfpathlineto{\pgfqpoint{1.860158in}{2.211576in}}%
\pgfpathlineto{\pgfqpoint{1.861073in}{2.009454in}}%
\pgfpathlineto{\pgfqpoint{1.861683in}{2.151811in}}%
\pgfpathlineto{\pgfqpoint{1.862599in}{2.386741in}}%
\pgfpathlineto{\pgfqpoint{1.863209in}{2.222622in}}%
\pgfpathlineto{\pgfqpoint{1.863819in}{2.087140in}}%
\pgfpathlineto{\pgfqpoint{1.864430in}{2.157983in}}%
\pgfpathlineto{\pgfqpoint{1.865345in}{2.336160in}}%
\pgfpathlineto{\pgfqpoint{1.865955in}{2.216662in}}%
\pgfpathlineto{\pgfqpoint{1.866870in}{2.030440in}}%
\pgfpathlineto{\pgfqpoint{1.867481in}{2.098473in}}%
\pgfpathlineto{\pgfqpoint{1.868701in}{2.380973in}}%
\pgfpathlineto{\pgfqpoint{1.869311in}{2.239926in}}%
\pgfpathlineto{\pgfqpoint{1.870227in}{2.040933in}}%
\pgfpathlineto{\pgfqpoint{1.870837in}{2.104699in}}%
\pgfpathlineto{\pgfqpoint{1.871752in}{2.357955in}}%
\pgfpathlineto{\pgfqpoint{1.872668in}{2.190047in}}%
\pgfpathlineto{\pgfqpoint{1.873278in}{2.054618in}}%
\pgfpathlineto{\pgfqpoint{1.874193in}{2.187248in}}%
\pgfpathlineto{\pgfqpoint{1.875109in}{2.326764in}}%
\pgfpathlineto{\pgfqpoint{1.875719in}{2.233232in}}%
\pgfpathlineto{\pgfqpoint{1.876634in}{2.143776in}}%
\pgfpathlineto{\pgfqpoint{1.877244in}{2.185513in}}%
\pgfpathlineto{\pgfqpoint{1.879075in}{2.256452in}}%
\pgfpathlineto{\pgfqpoint{1.879380in}{2.255559in}}%
\pgfpathlineto{\pgfqpoint{1.881211in}{2.057789in}}%
\pgfpathlineto{\pgfqpoint{1.882126in}{2.162814in}}%
\pgfpathlineto{\pgfqpoint{1.883042in}{2.342322in}}%
\pgfpathlineto{\pgfqpoint{1.883652in}{2.270819in}}%
\pgfpathlineto{\pgfqpoint{1.884872in}{2.078126in}}%
\pgfpathlineto{\pgfqpoint{1.885483in}{2.193420in}}%
\pgfpathlineto{\pgfqpoint{1.886398in}{2.396223in}}%
\pgfpathlineto{\pgfqpoint{1.887008in}{2.279120in}}%
\pgfpathlineto{\pgfqpoint{1.888229in}{2.066143in}}%
\pgfpathlineto{\pgfqpoint{1.888839in}{2.132112in}}%
\pgfpathlineto{\pgfqpoint{1.890059in}{2.354475in}}%
\pgfpathlineto{\pgfqpoint{1.890670in}{2.224729in}}%
\pgfpathlineto{\pgfqpoint{1.891585in}{2.049744in}}%
\pgfpathlineto{\pgfqpoint{1.892195in}{2.110967in}}%
\pgfpathlineto{\pgfqpoint{1.893111in}{2.306980in}}%
\pgfpathlineto{\pgfqpoint{1.894026in}{2.191079in}}%
\pgfpathlineto{\pgfqpoint{1.894636in}{2.119672in}}%
\pgfpathlineto{\pgfqpoint{1.895246in}{2.178671in}}%
\pgfpathlineto{\pgfqpoint{1.896162in}{2.279513in}}%
\pgfpathlineto{\pgfqpoint{1.896772in}{2.200636in}}%
\pgfpathlineto{\pgfqpoint{1.897382in}{2.123269in}}%
\pgfpathlineto{\pgfqpoint{1.897993in}{2.155886in}}%
\pgfpathlineto{\pgfqpoint{1.898908in}{2.282632in}}%
\pgfpathlineto{\pgfqpoint{1.899518in}{2.232306in}}%
\pgfpathlineto{\pgfqpoint{1.900433in}{2.111659in}}%
\pgfpathlineto{\pgfqpoint{1.901044in}{2.138817in}}%
\pgfpathlineto{\pgfqpoint{1.902264in}{2.299627in}}%
\pgfpathlineto{\pgfqpoint{1.902874in}{2.252036in}}%
\pgfpathlineto{\pgfqpoint{1.904095in}{2.099112in}}%
\pgfpathlineto{\pgfqpoint{1.904705in}{2.167209in}}%
\pgfpathlineto{\pgfqpoint{1.905620in}{2.273575in}}%
\pgfpathlineto{\pgfqpoint{1.906231in}{2.205552in}}%
\pgfpathlineto{\pgfqpoint{1.907146in}{2.108711in}}%
\pgfpathlineto{\pgfqpoint{1.907756in}{2.182395in}}%
\pgfpathlineto{\pgfqpoint{1.908672in}{2.366458in}}%
\pgfpathlineto{\pgfqpoint{1.909282in}{2.315217in}}%
\pgfpathlineto{\pgfqpoint{1.910807in}{2.077573in}}%
\pgfpathlineto{\pgfqpoint{1.911418in}{2.152651in}}%
\pgfpathlineto{\pgfqpoint{1.912333in}{2.306012in}}%
\pgfpathlineto{\pgfqpoint{1.912943in}{2.255559in}}%
\pgfpathlineto{\pgfqpoint{1.914164in}{2.068974in}}%
\pgfpathlineto{\pgfqpoint{1.914774in}{2.128643in}}%
\pgfpathlineto{\pgfqpoint{1.915994in}{2.374503in}}%
\pgfpathlineto{\pgfqpoint{1.916605in}{2.252877in}}%
\pgfpathlineto{\pgfqpoint{1.917825in}{2.001313in}}%
\pgfpathlineto{\pgfqpoint{1.918435in}{2.089939in}}%
\pgfpathlineto{\pgfqpoint{1.919656in}{2.448347in}}%
\pgfpathlineto{\pgfqpoint{1.920266in}{2.322613in}}%
\pgfpathlineto{\pgfqpoint{1.921487in}{2.030078in}}%
\pgfpathlineto{\pgfqpoint{1.922097in}{2.109945in}}%
\pgfpathlineto{\pgfqpoint{1.923317in}{2.404875in}}%
\pgfpathlineto{\pgfqpoint{1.923928in}{2.218993in}}%
\pgfpathlineto{\pgfqpoint{1.924843in}{2.003803in}}%
\pgfpathlineto{\pgfqpoint{1.925453in}{2.075338in}}%
\pgfpathlineto{\pgfqpoint{1.926674in}{2.357412in}}%
\pgfpathlineto{\pgfqpoint{1.927284in}{2.284111in}}%
\pgfpathlineto{\pgfqpoint{1.928504in}{2.118948in}}%
\pgfpathlineto{\pgfqpoint{1.929420in}{2.142733in}}%
\pgfpathlineto{\pgfqpoint{1.931556in}{2.262071in}}%
\pgfpathlineto{\pgfqpoint{1.932166in}{2.295476in}}%
\pgfpathlineto{\pgfqpoint{1.932776in}{2.264115in}}%
\pgfpathlineto{\pgfqpoint{1.934302in}{2.137274in}}%
\pgfpathlineto{\pgfqpoint{1.934912in}{2.176234in}}%
\pgfpathlineto{\pgfqpoint{1.935827in}{2.257048in}}%
\pgfpathlineto{\pgfqpoint{1.936437in}{2.196155in}}%
\pgfpathlineto{\pgfqpoint{1.937048in}{2.129335in}}%
\pgfpathlineto{\pgfqpoint{1.937658in}{2.159302in}}%
\pgfpathlineto{\pgfqpoint{1.938878in}{2.321453in}}%
\pgfpathlineto{\pgfqpoint{1.939489in}{2.230007in}}%
\pgfpathlineto{\pgfqpoint{1.940709in}{2.071624in}}%
\pgfpathlineto{\pgfqpoint{1.941319in}{2.133464in}}%
\pgfpathlineto{\pgfqpoint{1.942845in}{2.289900in}}%
\pgfpathlineto{\pgfqpoint{1.943455in}{2.256814in}}%
\pgfpathlineto{\pgfqpoint{1.944981in}{2.083915in}}%
\pgfpathlineto{\pgfqpoint{1.945591in}{2.163698in}}%
\pgfpathlineto{\pgfqpoint{1.946506in}{2.407812in}}%
\pgfpathlineto{\pgfqpoint{1.947117in}{2.291890in}}%
\pgfpathlineto{\pgfqpoint{1.948337in}{1.932651in}}%
\pgfpathlineto{\pgfqpoint{1.948947in}{2.031983in}}%
\pgfpathlineto{\pgfqpoint{1.950168in}{2.544093in}}%
\pgfpathlineto{\pgfqpoint{1.950778in}{2.261135in}}%
\pgfpathlineto{\pgfqpoint{1.951693in}{1.984892in}}%
\pgfpathlineto{\pgfqpoint{1.952304in}{2.106636in}}%
\pgfpathlineto{\pgfqpoint{1.953219in}{2.403481in}}%
\pgfpathlineto{\pgfqpoint{1.953829in}{2.289517in}}%
\pgfpathlineto{\pgfqpoint{1.955050in}{1.958149in}}%
\pgfpathlineto{\pgfqpoint{1.955660in}{2.096739in}}%
\pgfpathlineto{\pgfqpoint{1.956575in}{2.604039in}}%
\pgfpathlineto{\pgfqpoint{1.957185in}{2.410547in}}%
\pgfpathlineto{\pgfqpoint{1.958406in}{1.929714in}}%
\pgfpathlineto{\pgfqpoint{1.959016in}{2.007410in}}%
\pgfpathlineto{\pgfqpoint{1.960237in}{2.415262in}}%
\pgfpathlineto{\pgfqpoint{1.961152in}{2.205233in}}%
\pgfpathlineto{\pgfqpoint{1.961762in}{2.079850in}}%
\pgfpathlineto{\pgfqpoint{1.962678in}{2.185407in}}%
\pgfpathlineto{\pgfqpoint{1.963593in}{2.283823in}}%
\pgfpathlineto{\pgfqpoint{1.964203in}{2.203243in}}%
\pgfpathlineto{\pgfqpoint{1.964813in}{2.138019in}}%
\pgfpathlineto{\pgfqpoint{1.965729in}{2.194708in}}%
\pgfpathlineto{\pgfqpoint{1.966339in}{2.235275in}}%
\pgfpathlineto{\pgfqpoint{1.966949in}{2.197975in}}%
\pgfpathlineto{\pgfqpoint{1.967559in}{2.151172in}}%
\pgfpathlineto{\pgfqpoint{1.968170in}{2.177734in}}%
\pgfpathlineto{\pgfqpoint{1.969390in}{2.267775in}}%
\pgfpathlineto{\pgfqpoint{1.970000in}{2.237638in}}%
\pgfpathlineto{\pgfqpoint{1.972441in}{2.131059in}}%
\pgfpathlineto{\pgfqpoint{1.972746in}{2.143531in}}%
\pgfpathlineto{\pgfqpoint{1.973662in}{2.205627in}}%
\pgfpathlineto{\pgfqpoint{1.974272in}{2.176627in}}%
\pgfpathlineto{\pgfqpoint{1.974577in}{2.158696in}}%
\pgfpathlineto{\pgfqpoint{1.975187in}{2.195900in}}%
\pgfpathlineto{\pgfqpoint{1.976103in}{2.328051in}}%
\pgfpathlineto{\pgfqpoint{1.976713in}{2.232934in}}%
\pgfpathlineto{\pgfqpoint{1.977628in}{2.056991in}}%
\pgfpathlineto{\pgfqpoint{1.978239in}{2.098856in}}%
\pgfpathlineto{\pgfqpoint{1.979459in}{2.366415in}}%
\pgfpathlineto{\pgfqpoint{1.980069in}{2.265477in}}%
\pgfpathlineto{\pgfqpoint{1.980985in}{2.083064in}}%
\pgfpathlineto{\pgfqpoint{1.981900in}{2.172328in}}%
\pgfpathlineto{\pgfqpoint{1.982815in}{2.263849in}}%
\pgfpathlineto{\pgfqpoint{1.983731in}{2.225761in}}%
\pgfpathlineto{\pgfqpoint{1.984646in}{2.245172in}}%
\pgfpathlineto{\pgfqpoint{1.984951in}{2.232008in}}%
\pgfpathlineto{\pgfqpoint{1.985867in}{2.155748in}}%
\pgfpathlineto{\pgfqpoint{1.986782in}{2.194591in}}%
\pgfpathlineto{\pgfqpoint{1.987087in}{2.206882in}}%
\pgfpathlineto{\pgfqpoint{1.987697in}{2.176191in}}%
\pgfpathlineto{\pgfqpoint{1.988307in}{2.129037in}}%
\pgfpathlineto{\pgfqpoint{1.988918in}{2.158802in}}%
\pgfpathlineto{\pgfqpoint{1.990138in}{2.299765in}}%
\pgfpathlineto{\pgfqpoint{1.990748in}{2.244097in}}%
\pgfpathlineto{\pgfqpoint{1.991969in}{2.091545in}}%
\pgfpathlineto{\pgfqpoint{1.992579in}{2.128622in}}%
\pgfpathlineto{\pgfqpoint{1.993800in}{2.387539in}}%
\pgfpathlineto{\pgfqpoint{1.994410in}{2.299808in}}%
\pgfpathlineto{\pgfqpoint{1.995630in}{1.993533in}}%
\pgfpathlineto{\pgfqpoint{1.996241in}{2.040017in}}%
\pgfpathlineto{\pgfqpoint{1.997461in}{2.346355in}}%
\pgfpathlineto{\pgfqpoint{1.998376in}{2.243320in}}%
\pgfpathlineto{\pgfqpoint{1.998681in}{2.236935in}}%
\pgfpathlineto{\pgfqpoint{1.999597in}{2.349420in}}%
\pgfpathlineto{\pgfqpoint{2.000207in}{2.253622in}}%
\pgfpathlineto{\pgfqpoint{2.001733in}{1.938909in}}%
\pgfpathlineto{\pgfqpoint{2.002343in}{2.029599in}}%
\pgfpathlineto{\pgfqpoint{2.003563in}{2.572432in}}%
\pgfpathlineto{\pgfqpoint{2.004174in}{2.350186in}}%
\pgfpathlineto{\pgfqpoint{2.005089in}{1.994779in}}%
\pgfpathlineto{\pgfqpoint{2.006004in}{2.150491in}}%
\pgfpathlineto{\pgfqpoint{2.006920in}{2.340492in}}%
\pgfpathlineto{\pgfqpoint{2.007530in}{2.245076in}}%
\pgfpathlineto{\pgfqpoint{2.008445in}{2.146255in}}%
\pgfpathlineto{\pgfqpoint{2.009056in}{2.194708in}}%
\pgfpathlineto{\pgfqpoint{2.009666in}{2.245906in}}%
\pgfpathlineto{\pgfqpoint{2.010276in}{2.187248in}}%
\pgfpathlineto{\pgfqpoint{2.011191in}{2.051798in}}%
\pgfpathlineto{\pgfqpoint{2.011802in}{2.167422in}}%
\pgfpathlineto{\pgfqpoint{2.012717in}{2.530535in}}%
\pgfpathlineto{\pgfqpoint{2.013327in}{2.313216in}}%
\pgfpathlineto{\pgfqpoint{2.014548in}{1.893244in}}%
\pgfpathlineto{\pgfqpoint{2.015158in}{1.964566in}}%
\pgfpathlineto{\pgfqpoint{2.016683in}{2.480890in}}%
\pgfpathlineto{\pgfqpoint{2.017294in}{2.288463in}}%
\pgfpathlineto{\pgfqpoint{2.017904in}{2.187631in}}%
\pgfpathlineto{\pgfqpoint{2.018819in}{2.241777in}}%
\pgfpathlineto{\pgfqpoint{2.020040in}{2.063121in}}%
\pgfpathlineto{\pgfqpoint{2.020650in}{2.024991in}}%
\pgfpathlineto{\pgfqpoint{2.020955in}{2.052138in}}%
\pgfpathlineto{\pgfqpoint{2.022481in}{2.430522in}}%
\pgfpathlineto{\pgfqpoint{2.023091in}{2.257612in}}%
\pgfpathlineto{\pgfqpoint{2.024006in}{2.038985in}}%
\pgfpathlineto{\pgfqpoint{2.024617in}{2.103401in}}%
\pgfpathlineto{\pgfqpoint{2.025532in}{2.275746in}}%
\pgfpathlineto{\pgfqpoint{2.026142in}{2.217663in}}%
\pgfpathlineto{\pgfqpoint{2.026752in}{2.122003in}}%
\pgfpathlineto{\pgfqpoint{2.027363in}{2.165890in}}%
\pgfpathlineto{\pgfqpoint{2.028278in}{2.370810in}}%
\pgfpathlineto{\pgfqpoint{2.028888in}{2.303660in}}%
\pgfpathlineto{\pgfqpoint{2.030109in}{2.045711in}}%
\pgfpathlineto{\pgfqpoint{2.030719in}{2.099005in}}%
\pgfpathlineto{\pgfqpoint{2.031939in}{2.377281in}}%
\pgfpathlineto{\pgfqpoint{2.032855in}{2.243225in}}%
\pgfpathlineto{\pgfqpoint{2.034991in}{2.104018in}}%
\pgfpathlineto{\pgfqpoint{2.035601in}{2.090173in}}%
\pgfpathlineto{\pgfqpoint{2.035906in}{2.111808in}}%
\pgfpathlineto{\pgfqpoint{2.037126in}{2.452232in}}%
\pgfpathlineto{\pgfqpoint{2.037737in}{2.343769in}}%
\pgfpathlineto{\pgfqpoint{2.038957in}{1.968131in}}%
\pgfpathlineto{\pgfqpoint{2.039567in}{2.027109in}}%
\pgfpathlineto{\pgfqpoint{2.040788in}{2.373120in}}%
\pgfpathlineto{\pgfqpoint{2.041703in}{2.255144in}}%
\pgfpathlineto{\pgfqpoint{2.042008in}{2.229816in}}%
\pgfpathlineto{\pgfqpoint{2.042619in}{2.290507in}}%
\pgfpathlineto{\pgfqpoint{2.042924in}{2.324901in}}%
\pgfpathlineto{\pgfqpoint{2.043534in}{2.261220in}}%
\pgfpathlineto{\pgfqpoint{2.045059in}{1.936844in}}%
\pgfpathlineto{\pgfqpoint{2.045670in}{2.016488in}}%
\pgfpathlineto{\pgfqpoint{2.046890in}{2.619725in}}%
\pgfpathlineto{\pgfqpoint{2.047500in}{2.379792in}}%
\pgfpathlineto{\pgfqpoint{2.048721in}{1.914570in}}%
\pgfpathlineto{\pgfqpoint{2.049331in}{2.013040in}}%
\pgfpathlineto{\pgfqpoint{2.050552in}{2.446698in}}%
\pgfpathlineto{\pgfqpoint{2.051162in}{2.302500in}}%
\pgfpathlineto{\pgfqpoint{2.052077in}{2.111605in}}%
\pgfpathlineto{\pgfqpoint{2.052687in}{2.171307in}}%
\pgfpathlineto{\pgfqpoint{2.053298in}{2.225772in}}%
\pgfpathlineto{\pgfqpoint{2.054213in}{2.173031in}}%
\pgfpathlineto{\pgfqpoint{2.054518in}{2.157855in}}%
\pgfpathlineto{\pgfqpoint{2.055128in}{2.186429in}}%
\pgfpathlineto{\pgfqpoint{2.056349in}{2.291220in}}%
\pgfpathlineto{\pgfqpoint{2.056959in}{2.237701in}}%
\pgfpathlineto{\pgfqpoint{2.059705in}{2.098324in}}%
\pgfpathlineto{\pgfqpoint{2.060010in}{2.098910in}}%
\pgfpathlineto{\pgfqpoint{2.060926in}{2.251951in}}%
\pgfpathlineto{\pgfqpoint{2.061536in}{2.385081in}}%
\pgfpathlineto{\pgfqpoint{2.062146in}{2.319814in}}%
\pgfpathlineto{\pgfqpoint{2.063672in}{2.075178in}}%
\pgfpathlineto{\pgfqpoint{2.064282in}{2.114734in}}%
\pgfpathlineto{\pgfqpoint{2.066113in}{2.334447in}}%
\pgfpathlineto{\pgfqpoint{2.067028in}{2.270745in}}%
\pgfpathlineto{\pgfqpoint{2.068859in}{2.028854in}}%
\pgfpathlineto{\pgfqpoint{2.069469in}{2.068963in}}%
\pgfpathlineto{\pgfqpoint{2.070994in}{2.323720in}}%
\pgfpathlineto{\pgfqpoint{2.071605in}{2.255559in}}%
\pgfpathlineto{\pgfqpoint{2.072215in}{2.197571in}}%
\pgfpathlineto{\pgfqpoint{2.072825in}{2.243693in}}%
\pgfpathlineto{\pgfqpoint{2.073435in}{2.306012in}}%
\pgfpathlineto{\pgfqpoint{2.074046in}{2.222366in}}%
\pgfpathlineto{\pgfqpoint{2.075266in}{2.000791in}}%
\pgfpathlineto{\pgfqpoint{2.075876in}{2.135156in}}%
\pgfpathlineto{\pgfqpoint{2.076792in}{2.485115in}}%
\pgfpathlineto{\pgfqpoint{2.077402in}{2.338597in}}%
\pgfpathlineto{\pgfqpoint{2.078622in}{2.030227in}}%
\pgfpathlineto{\pgfqpoint{2.079233in}{2.098111in}}%
\pgfpathlineto{\pgfqpoint{2.080453in}{2.275693in}}%
\pgfpathlineto{\pgfqpoint{2.081063in}{2.238712in}}%
\pgfpathlineto{\pgfqpoint{2.082894in}{2.142850in}}%
\pgfpathlineto{\pgfqpoint{2.083199in}{2.144468in}}%
\pgfpathlineto{\pgfqpoint{2.084115in}{2.219184in}}%
\pgfpathlineto{\pgfqpoint{2.085030in}{2.338182in}}%
\pgfpathlineto{\pgfqpoint{2.085640in}{2.289485in}}%
\pgfpathlineto{\pgfqpoint{2.086861in}{2.043816in}}%
\pgfpathlineto{\pgfqpoint{2.087471in}{2.112670in}}%
\pgfpathlineto{\pgfqpoint{2.088386in}{2.364095in}}%
\pgfpathlineto{\pgfqpoint{2.089302in}{2.199997in}}%
\pgfpathlineto{\pgfqpoint{2.090217in}{2.042422in}}%
\pgfpathlineto{\pgfqpoint{2.090827in}{2.157515in}}%
\pgfpathlineto{\pgfqpoint{2.091743in}{2.397000in}}%
\pgfpathlineto{\pgfqpoint{2.092353in}{2.272032in}}%
\pgfpathlineto{\pgfqpoint{2.093573in}{2.002664in}}%
\pgfpathlineto{\pgfqpoint{2.094183in}{2.106018in}}%
\pgfpathlineto{\pgfqpoint{2.095099in}{2.325231in}}%
\pgfpathlineto{\pgfqpoint{2.095709in}{2.234424in}}%
\pgfpathlineto{\pgfqpoint{2.096319in}{2.130867in}}%
\pgfpathlineto{\pgfqpoint{2.096930in}{2.245470in}}%
\pgfpathlineto{\pgfqpoint{2.097540in}{2.401480in}}%
\pgfpathlineto{\pgfqpoint{2.098150in}{2.281355in}}%
\pgfpathlineto{\pgfqpoint{2.099065in}{2.064792in}}%
\pgfpathlineto{\pgfqpoint{2.099981in}{2.160558in}}%
\pgfpathlineto{\pgfqpoint{2.100591in}{2.204020in}}%
\pgfpathlineto{\pgfqpoint{2.101201in}{2.164304in}}%
\pgfpathlineto{\pgfqpoint{2.101506in}{2.151289in}}%
\pgfpathlineto{\pgfqpoint{2.101811in}{2.168178in}}%
\pgfpathlineto{\pgfqpoint{2.103032in}{2.376727in}}%
\pgfpathlineto{\pgfqpoint{2.103337in}{2.314259in}}%
\pgfpathlineto{\pgfqpoint{2.104557in}{1.983200in}}%
\pgfpathlineto{\pgfqpoint{2.105168in}{2.065079in}}%
\pgfpathlineto{\pgfqpoint{2.106388in}{2.464214in}}%
\pgfpathlineto{\pgfqpoint{2.106998in}{2.282951in}}%
\pgfpathlineto{\pgfqpoint{2.107914in}{2.028865in}}%
\pgfpathlineto{\pgfqpoint{2.108524in}{2.093131in}}%
\pgfpathlineto{\pgfqpoint{2.109744in}{2.321730in}}%
\pgfpathlineto{\pgfqpoint{2.110355in}{2.239372in}}%
\pgfpathlineto{\pgfqpoint{2.113101in}{2.068889in}}%
\pgfpathlineto{\pgfqpoint{2.114016in}{2.259017in}}%
\pgfpathlineto{\pgfqpoint{2.114931in}{2.480497in}}%
\pgfpathlineto{\pgfqpoint{2.115237in}{2.380388in}}%
\pgfpathlineto{\pgfqpoint{2.116457in}{1.943729in}}%
\pgfpathlineto{\pgfqpoint{2.117067in}{2.064664in}}%
\pgfpathlineto{\pgfqpoint{2.117983in}{2.447858in}}%
\pgfpathlineto{\pgfqpoint{2.118593in}{2.309119in}}%
\pgfpathlineto{\pgfqpoint{2.119508in}{1.976006in}}%
\pgfpathlineto{\pgfqpoint{2.120119in}{2.063025in}}%
\pgfpathlineto{\pgfqpoint{2.121339in}{2.564344in}}%
\pgfpathlineto{\pgfqpoint{2.121949in}{2.230241in}}%
\pgfpathlineto{\pgfqpoint{2.122865in}{1.930182in}}%
\pgfpathlineto{\pgfqpoint{2.123475in}{2.002143in}}%
\pgfpathlineto{\pgfqpoint{2.125000in}{2.413123in}}%
\pgfpathlineto{\pgfqpoint{2.125611in}{2.300425in}}%
\pgfpathlineto{\pgfqpoint{2.127136in}{2.032951in}}%
\pgfpathlineto{\pgfqpoint{2.127746in}{2.058896in}}%
\pgfpathlineto{\pgfqpoint{2.129272in}{2.503611in}}%
\pgfpathlineto{\pgfqpoint{2.129882in}{2.345014in}}%
\pgfpathlineto{\pgfqpoint{2.131103in}{1.984530in}}%
\pgfpathlineto{\pgfqpoint{2.131713in}{2.141296in}}%
\pgfpathlineto{\pgfqpoint{2.132628in}{2.407280in}}%
\pgfpathlineto{\pgfqpoint{2.133239in}{2.229411in}}%
\pgfpathlineto{\pgfqpoint{2.134154in}{2.000930in}}%
\pgfpathlineto{\pgfqpoint{2.134764in}{2.103603in}}%
\pgfpathlineto{\pgfqpoint{2.135680in}{2.415475in}}%
\pgfpathlineto{\pgfqpoint{2.136290in}{2.321985in}}%
\pgfpathlineto{\pgfqpoint{2.137510in}{1.993097in}}%
\pgfpathlineto{\pgfqpoint{2.138120in}{2.048052in}}%
\pgfpathlineto{\pgfqpoint{2.139341in}{2.540006in}}%
\pgfpathlineto{\pgfqpoint{2.139951in}{2.371938in}}%
\pgfpathlineto{\pgfqpoint{2.141172in}{1.956191in}}%
\pgfpathlineto{\pgfqpoint{2.141782in}{2.015828in}}%
\pgfpathlineto{\pgfqpoint{2.143307in}{2.373364in}}%
\pgfpathlineto{\pgfqpoint{2.144223in}{2.237733in}}%
\pgfpathlineto{\pgfqpoint{2.145443in}{2.135262in}}%
\pgfpathlineto{\pgfqpoint{2.146054in}{2.182715in}}%
\pgfpathlineto{\pgfqpoint{2.146969in}{2.278056in}}%
\pgfpathlineto{\pgfqpoint{2.147579in}{2.209777in}}%
\pgfpathlineto{\pgfqpoint{2.148800in}{1.982594in}}%
\pgfpathlineto{\pgfqpoint{2.149410in}{2.107178in}}%
\pgfpathlineto{\pgfqpoint{2.150325in}{2.670870in}}%
\pgfpathlineto{\pgfqpoint{2.150935in}{2.424084in}}%
\pgfpathlineto{\pgfqpoint{2.152156in}{1.854295in}}%
\pgfpathlineto{\pgfqpoint{2.152766in}{1.931959in}}%
\pgfpathlineto{\pgfqpoint{2.153987in}{2.662026in}}%
\pgfpathlineto{\pgfqpoint{2.154902in}{2.227475in}}%
\pgfpathlineto{\pgfqpoint{2.155817in}{1.879878in}}%
\pgfpathlineto{\pgfqpoint{2.156733in}{2.043221in}}%
\pgfpathlineto{\pgfqpoint{2.157953in}{2.558417in}}%
\pgfpathlineto{\pgfqpoint{2.158563in}{2.247758in}}%
\pgfpathlineto{\pgfqpoint{2.159479in}{1.958926in}}%
\pgfpathlineto{\pgfqpoint{2.160089in}{2.044796in}}%
\pgfpathlineto{\pgfqpoint{2.161309in}{2.363680in}}%
\pgfpathlineto{\pgfqpoint{2.161920in}{2.279460in}}%
\pgfpathlineto{\pgfqpoint{2.163750in}{2.125812in}}%
\pgfpathlineto{\pgfqpoint{2.164056in}{2.135273in}}%
\pgfpathlineto{\pgfqpoint{2.165276in}{2.328573in}}%
\pgfpathlineto{\pgfqpoint{2.165886in}{2.239032in}}%
\pgfpathlineto{\pgfqpoint{2.167107in}{2.029620in}}%
\pgfpathlineto{\pgfqpoint{2.167717in}{2.096387in}}%
\pgfpathlineto{\pgfqpoint{2.168937in}{2.430426in}}%
\pgfpathlineto{\pgfqpoint{2.169548in}{2.339098in}}%
\pgfpathlineto{\pgfqpoint{2.170768in}{2.030855in}}%
\pgfpathlineto{\pgfqpoint{2.171378in}{2.127206in}}%
\pgfpathlineto{\pgfqpoint{2.172294in}{2.330499in}}%
\pgfpathlineto{\pgfqpoint{2.172904in}{2.210831in}}%
\pgfpathlineto{\pgfqpoint{2.173819in}{2.015913in}}%
\pgfpathlineto{\pgfqpoint{2.174430in}{2.164464in}}%
\pgfpathlineto{\pgfqpoint{2.175345in}{2.469195in}}%
\pgfpathlineto{\pgfqpoint{2.175955in}{2.296998in}}%
\pgfpathlineto{\pgfqpoint{2.177176in}{2.001217in}}%
\pgfpathlineto{\pgfqpoint{2.177786in}{2.082808in}}%
\pgfpathlineto{\pgfqpoint{2.179006in}{2.342077in}}%
\pgfpathlineto{\pgfqpoint{2.179617in}{2.275289in}}%
\pgfpathlineto{\pgfqpoint{2.182057in}{2.066569in}}%
\pgfpathlineto{\pgfqpoint{2.182363in}{2.072582in}}%
\pgfpathlineto{\pgfqpoint{2.183888in}{2.542198in}}%
\pgfpathlineto{\pgfqpoint{2.184498in}{2.244055in}}%
\pgfpathlineto{\pgfqpoint{2.185414in}{1.963459in}}%
\pgfpathlineto{\pgfqpoint{2.186329in}{2.092163in}}%
\pgfpathlineto{\pgfqpoint{2.187550in}{2.374865in}}%
\pgfpathlineto{\pgfqpoint{2.188160in}{2.250738in}}%
\pgfpathlineto{\pgfqpoint{2.189380in}{2.037783in}}%
\pgfpathlineto{\pgfqpoint{2.189991in}{2.153045in}}%
\pgfpathlineto{\pgfqpoint{2.190906in}{2.404811in}}%
\pgfpathlineto{\pgfqpoint{2.191516in}{2.300765in}}%
\pgfpathlineto{\pgfqpoint{2.192737in}{1.992459in}}%
\pgfpathlineto{\pgfqpoint{2.193347in}{2.058790in}}%
\pgfpathlineto{\pgfqpoint{2.194567in}{2.541975in}}%
\pgfpathlineto{\pgfqpoint{2.195178in}{2.397256in}}%
\pgfpathlineto{\pgfqpoint{2.196703in}{1.965226in}}%
\pgfpathlineto{\pgfqpoint{2.197313in}{2.029790in}}%
\pgfpathlineto{\pgfqpoint{2.198839in}{2.512688in}}%
\pgfpathlineto{\pgfqpoint{2.199449in}{2.298573in}}%
\pgfpathlineto{\pgfqpoint{2.200670in}{1.886391in}}%
\pgfpathlineto{\pgfqpoint{2.201280in}{1.955797in}}%
\pgfpathlineto{\pgfqpoint{2.202500in}{2.616990in}}%
\pgfpathlineto{\pgfqpoint{2.203111in}{2.351176in}}%
\pgfpathlineto{\pgfqpoint{2.204026in}{1.987436in}}%
\pgfpathlineto{\pgfqpoint{2.204636in}{2.114649in}}%
\pgfpathlineto{\pgfqpoint{2.205552in}{2.363691in}}%
\pgfpathlineto{\pgfqpoint{2.206162in}{2.223441in}}%
\pgfpathlineto{\pgfqpoint{2.207077in}{2.025683in}}%
\pgfpathlineto{\pgfqpoint{2.207687in}{2.118597in}}%
\pgfpathlineto{\pgfqpoint{2.208603in}{2.439366in}}%
\pgfpathlineto{\pgfqpoint{2.209213in}{2.343248in}}%
\pgfpathlineto{\pgfqpoint{2.210433in}{1.997077in}}%
\pgfpathlineto{\pgfqpoint{2.211044in}{2.068378in}}%
\pgfpathlineto{\pgfqpoint{2.213485in}{2.323592in}}%
\pgfpathlineto{\pgfqpoint{2.213790in}{2.331340in}}%
\pgfpathlineto{\pgfqpoint{2.215010in}{2.024650in}}%
\pgfpathlineto{\pgfqpoint{2.215620in}{1.943644in}}%
\pgfpathlineto{\pgfqpoint{2.216231in}{2.074433in}}%
\pgfpathlineto{\pgfqpoint{2.217451in}{2.647287in}}%
\pgfpathlineto{\pgfqpoint{2.218061in}{2.251249in}}%
\pgfpathlineto{\pgfqpoint{2.219282in}{1.880123in}}%
\pgfpathlineto{\pgfqpoint{2.219892in}{2.032940in}}%
\pgfpathlineto{\pgfqpoint{2.221113in}{2.496598in}}%
\pgfpathlineto{\pgfqpoint{2.221723in}{2.254250in}}%
\pgfpathlineto{\pgfqpoint{2.222638in}{2.040975in}}%
\pgfpathlineto{\pgfqpoint{2.223248in}{2.101017in}}%
\pgfpathlineto{\pgfqpoint{2.226300in}{2.370129in}}%
\pgfpathlineto{\pgfqpoint{2.227520in}{2.173297in}}%
\pgfpathlineto{\pgfqpoint{2.228435in}{2.100251in}}%
\pgfpathlineto{\pgfqpoint{2.229046in}{2.123812in}}%
\pgfpathlineto{\pgfqpoint{2.232097in}{2.312684in}}%
\pgfpathlineto{\pgfqpoint{2.232402in}{2.287282in}}%
\pgfpathlineto{\pgfqpoint{2.233928in}{1.972792in}}%
\pgfpathlineto{\pgfqpoint{2.234538in}{2.090098in}}%
\pgfpathlineto{\pgfqpoint{2.235453in}{2.566962in}}%
\pgfpathlineto{\pgfqpoint{2.236063in}{2.396702in}}%
\pgfpathlineto{\pgfqpoint{2.237284in}{1.848633in}}%
\pgfpathlineto{\pgfqpoint{2.237894in}{2.014424in}}%
\pgfpathlineto{\pgfqpoint{2.238809in}{2.730039in}}%
\pgfpathlineto{\pgfqpoint{2.239420in}{2.435588in}}%
\pgfpathlineto{\pgfqpoint{2.240640in}{1.891478in}}%
\pgfpathlineto{\pgfqpoint{2.241250in}{2.061259in}}%
\pgfpathlineto{\pgfqpoint{2.242166in}{2.437897in}}%
\pgfpathlineto{\pgfqpoint{2.242776in}{2.324305in}}%
\pgfpathlineto{\pgfqpoint{2.243996in}{2.026513in}}%
\pgfpathlineto{\pgfqpoint{2.244607in}{2.106848in}}%
\pgfpathlineto{\pgfqpoint{2.245827in}{2.349856in}}%
\pgfpathlineto{\pgfqpoint{2.246743in}{2.264743in}}%
\pgfpathlineto{\pgfqpoint{2.248573in}{2.034654in}}%
\pgfpathlineto{\pgfqpoint{2.249183in}{2.091077in}}%
\pgfpathlineto{\pgfqpoint{2.250404in}{2.505292in}}%
\pgfpathlineto{\pgfqpoint{2.251014in}{2.347058in}}%
\pgfpathlineto{\pgfqpoint{2.252235in}{2.036250in}}%
\pgfpathlineto{\pgfqpoint{2.252845in}{2.131665in}}%
\pgfpathlineto{\pgfqpoint{2.253760in}{2.256527in}}%
\pgfpathlineto{\pgfqpoint{2.254370in}{2.203807in}}%
\pgfpathlineto{\pgfqpoint{2.255286in}{2.114830in}}%
\pgfpathlineto{\pgfqpoint{2.255896in}{2.147830in}}%
\pgfpathlineto{\pgfqpoint{2.257422in}{2.322049in}}%
\pgfpathlineto{\pgfqpoint{2.258032in}{2.247247in}}%
\pgfpathlineto{\pgfqpoint{2.258947in}{2.122577in}}%
\pgfpathlineto{\pgfqpoint{2.259557in}{2.161388in}}%
\pgfpathlineto{\pgfqpoint{2.260473in}{2.259145in}}%
\pgfpathlineto{\pgfqpoint{2.261388in}{2.220898in}}%
\pgfpathlineto{\pgfqpoint{2.262914in}{2.138764in}}%
\pgfpathlineto{\pgfqpoint{2.263524in}{2.098984in}}%
\pgfpathlineto{\pgfqpoint{2.264134in}{2.154748in}}%
\pgfpathlineto{\pgfqpoint{2.265355in}{2.364638in}}%
\pgfpathlineto{\pgfqpoint{2.265965in}{2.226368in}}%
\pgfpathlineto{\pgfqpoint{2.267185in}{1.979156in}}%
\pgfpathlineto{\pgfqpoint{2.267796in}{2.068325in}}%
\pgfpathlineto{\pgfqpoint{2.269016in}{2.513486in}}%
\pgfpathlineto{\pgfqpoint{2.269626in}{2.239713in}}%
\pgfpathlineto{\pgfqpoint{2.270542in}{1.982562in}}%
\pgfpathlineto{\pgfqpoint{2.271152in}{2.236786in}}%
\pgfpathlineto{\pgfqpoint{2.271762in}{2.498524in}}%
\pgfpathlineto{\pgfqpoint{2.272372in}{2.270244in}}%
\pgfpathlineto{\pgfqpoint{2.273288in}{1.898225in}}%
\pgfpathlineto{\pgfqpoint{2.273898in}{1.988979in}}%
\pgfpathlineto{\pgfqpoint{2.275119in}{2.603411in}}%
\pgfpathlineto{\pgfqpoint{2.275729in}{2.251717in}}%
\pgfpathlineto{\pgfqpoint{2.276644in}{1.934290in}}%
\pgfpathlineto{\pgfqpoint{2.277254in}{2.056789in}}%
\pgfpathlineto{\pgfqpoint{2.278170in}{2.361967in}}%
\pgfpathlineto{\pgfqpoint{2.279085in}{2.216854in}}%
\pgfpathlineto{\pgfqpoint{2.279695in}{2.132442in}}%
\pgfpathlineto{\pgfqpoint{2.280611in}{2.186716in}}%
\pgfpathlineto{\pgfqpoint{2.280916in}{2.187663in}}%
\pgfpathlineto{\pgfqpoint{2.281526in}{2.150150in}}%
\pgfpathlineto{\pgfqpoint{2.282136in}{2.203381in}}%
\pgfpathlineto{\pgfqpoint{2.283052in}{2.427308in}}%
\pgfpathlineto{\pgfqpoint{2.283662in}{2.258932in}}%
\pgfpathlineto{\pgfqpoint{2.284577in}{1.983041in}}%
\pgfpathlineto{\pgfqpoint{2.285187in}{2.047722in}}%
\pgfpathlineto{\pgfqpoint{2.286408in}{2.429022in}}%
\pgfpathlineto{\pgfqpoint{2.287018in}{2.230241in}}%
\pgfpathlineto{\pgfqpoint{2.287933in}{1.966769in}}%
\pgfpathlineto{\pgfqpoint{2.288544in}{2.086746in}}%
\pgfpathlineto{\pgfqpoint{2.289459in}{2.522319in}}%
\pgfpathlineto{\pgfqpoint{2.290069in}{2.349644in}}%
\pgfpathlineto{\pgfqpoint{2.291290in}{1.927564in}}%
\pgfpathlineto{\pgfqpoint{2.291900in}{2.084107in}}%
\pgfpathlineto{\pgfqpoint{2.292815in}{2.485456in}}%
\pgfpathlineto{\pgfqpoint{2.293426in}{2.310567in}}%
\pgfpathlineto{\pgfqpoint{2.294341in}{2.031929in}}%
\pgfpathlineto{\pgfqpoint{2.294951in}{2.138753in}}%
\pgfpathlineto{\pgfqpoint{2.295867in}{2.337533in}}%
\pgfpathlineto{\pgfqpoint{2.296477in}{2.214055in}}%
\pgfpathlineto{\pgfqpoint{2.297392in}{2.071230in}}%
\pgfpathlineto{\pgfqpoint{2.298002in}{2.163314in}}%
\pgfpathlineto{\pgfqpoint{2.298918in}{2.275065in}}%
\pgfpathlineto{\pgfqpoint{2.299528in}{2.169444in}}%
\pgfpathlineto{\pgfqpoint{2.300138in}{2.087108in}}%
\pgfpathlineto{\pgfqpoint{2.300748in}{2.154748in}}%
\pgfpathlineto{\pgfqpoint{2.301664in}{2.338619in}}%
\pgfpathlineto{\pgfqpoint{2.302274in}{2.273139in}}%
\pgfpathlineto{\pgfqpoint{2.303189in}{2.167156in}}%
\pgfpathlineto{\pgfqpoint{2.303800in}{2.230561in}}%
\pgfpathlineto{\pgfqpoint{2.304410in}{2.282068in}}%
\pgfpathlineto{\pgfqpoint{2.305020in}{2.226591in}}%
\pgfpathlineto{\pgfqpoint{2.306241in}{2.023618in}}%
\pgfpathlineto{\pgfqpoint{2.306851in}{2.092482in}}%
\pgfpathlineto{\pgfqpoint{2.308071in}{2.486839in}}%
\pgfpathlineto{\pgfqpoint{2.308681in}{2.207255in}}%
\pgfpathlineto{\pgfqpoint{2.309597in}{1.937993in}}%
\pgfpathlineto{\pgfqpoint{2.310207in}{2.068165in}}%
\pgfpathlineto{\pgfqpoint{2.311428in}{2.475388in}}%
\pgfpathlineto{\pgfqpoint{2.312038in}{2.257048in}}%
\pgfpathlineto{\pgfqpoint{2.313258in}{1.996971in}}%
\pgfpathlineto{\pgfqpoint{2.313869in}{2.057651in}}%
\pgfpathlineto{\pgfqpoint{2.315089in}{2.378888in}}%
\pgfpathlineto{\pgfqpoint{2.316004in}{2.232391in}}%
\pgfpathlineto{\pgfqpoint{2.316920in}{2.126589in}}%
\pgfpathlineto{\pgfqpoint{2.317530in}{2.159686in}}%
\pgfpathlineto{\pgfqpoint{2.318750in}{2.315866in}}%
\pgfpathlineto{\pgfqpoint{2.319361in}{2.268361in}}%
\pgfpathlineto{\pgfqpoint{2.320886in}{1.986893in}}%
\pgfpathlineto{\pgfqpoint{2.321496in}{2.095813in}}%
\pgfpathlineto{\pgfqpoint{2.322717in}{2.527725in}}%
\pgfpathlineto{\pgfqpoint{2.323022in}{2.400033in}}%
\pgfpathlineto{\pgfqpoint{2.324243in}{1.876238in}}%
\pgfpathlineto{\pgfqpoint{2.324853in}{1.963396in}}%
\pgfpathlineto{\pgfqpoint{2.326073in}{2.807267in}}%
\pgfpathlineto{\pgfqpoint{2.326683in}{2.282876in}}%
\pgfpathlineto{\pgfqpoint{2.327904in}{1.809684in}}%
\pgfpathlineto{\pgfqpoint{2.328514in}{1.992246in}}%
\pgfpathlineto{\pgfqpoint{2.329430in}{2.607646in}}%
\pgfpathlineto{\pgfqpoint{2.330345in}{2.198188in}}%
\pgfpathlineto{\pgfqpoint{2.331260in}{1.899012in}}%
\pgfpathlineto{\pgfqpoint{2.331870in}{2.076530in}}%
\pgfpathlineto{\pgfqpoint{2.332786in}{2.552202in}}%
\pgfpathlineto{\pgfqpoint{2.333396in}{2.330392in}}%
\pgfpathlineto{\pgfqpoint{2.334311in}{2.013200in}}%
\pgfpathlineto{\pgfqpoint{2.334922in}{2.162218in}}%
\pgfpathlineto{\pgfqpoint{2.335837in}{2.396692in}}%
\pgfpathlineto{\pgfqpoint{2.336447in}{2.208181in}}%
\pgfpathlineto{\pgfqpoint{2.337363in}{1.991809in}}%
\pgfpathlineto{\pgfqpoint{2.337973in}{2.044742in}}%
\pgfpathlineto{\pgfqpoint{2.339193in}{2.300169in}}%
\pgfpathlineto{\pgfqpoint{2.340414in}{2.277896in}}%
\pgfpathlineto{\pgfqpoint{2.340719in}{2.292316in}}%
\pgfpathlineto{\pgfqpoint{2.341329in}{2.259347in}}%
\pgfpathlineto{\pgfqpoint{2.342550in}{2.069857in}}%
\pgfpathlineto{\pgfqpoint{2.343160in}{2.124014in}}%
\pgfpathlineto{\pgfqpoint{2.344075in}{2.287048in}}%
\pgfpathlineto{\pgfqpoint{2.344685in}{2.235041in}}%
\pgfpathlineto{\pgfqpoint{2.345906in}{2.049457in}}%
\pgfpathlineto{\pgfqpoint{2.346516in}{2.166613in}}%
\pgfpathlineto{\pgfqpoint{2.347431in}{2.461128in}}%
\pgfpathlineto{\pgfqpoint{2.348042in}{2.324454in}}%
\pgfpathlineto{\pgfqpoint{2.349262in}{2.015105in}}%
\pgfpathlineto{\pgfqpoint{2.349872in}{2.127526in}}%
\pgfpathlineto{\pgfqpoint{2.350788in}{2.358147in}}%
\pgfpathlineto{\pgfqpoint{2.351398in}{2.241192in}}%
\pgfpathlineto{\pgfqpoint{2.352313in}{2.084256in}}%
\pgfpathlineto{\pgfqpoint{2.352924in}{2.171892in}}%
\pgfpathlineto{\pgfqpoint{2.353839in}{2.287750in}}%
\pgfpathlineto{\pgfqpoint{2.354449in}{2.216322in}}%
\pgfpathlineto{\pgfqpoint{2.354754in}{2.190792in}}%
\pgfpathlineto{\pgfqpoint{2.355365in}{2.235669in}}%
\pgfpathlineto{\pgfqpoint{2.355975in}{2.285707in}}%
\pgfpathlineto{\pgfqpoint{2.356280in}{2.245821in}}%
\pgfpathlineto{\pgfqpoint{2.357500in}{1.943729in}}%
\pgfpathlineto{\pgfqpoint{2.358111in}{2.037878in}}%
\pgfpathlineto{\pgfqpoint{2.359331in}{2.694165in}}%
\pgfpathlineto{\pgfqpoint{2.359941in}{2.239191in}}%
\pgfpathlineto{\pgfqpoint{2.360857in}{1.862053in}}%
\pgfpathlineto{\pgfqpoint{2.361467in}{1.961437in}}%
\pgfpathlineto{\pgfqpoint{2.362687in}{2.503994in}}%
\pgfpathlineto{\pgfqpoint{2.363298in}{2.258645in}}%
\pgfpathlineto{\pgfqpoint{2.364213in}{1.928724in}}%
\pgfpathlineto{\pgfqpoint{2.364823in}{2.017542in}}%
\pgfpathlineto{\pgfqpoint{2.366044in}{2.579849in}}%
\pgfpathlineto{\pgfqpoint{2.366654in}{2.372279in}}%
\pgfpathlineto{\pgfqpoint{2.367874in}{2.054065in}}%
\pgfpathlineto{\pgfqpoint{2.368485in}{2.086969in}}%
\pgfpathlineto{\pgfqpoint{2.370315in}{2.245991in}}%
\pgfpathlineto{\pgfqpoint{2.370926in}{2.217929in}}%
\pgfpathlineto{\pgfqpoint{2.372146in}{2.048169in}}%
\pgfpathlineto{\pgfqpoint{2.372756in}{2.110626in}}%
\pgfpathlineto{\pgfqpoint{2.373977in}{2.531620in}}%
\pgfpathlineto{\pgfqpoint{2.374587in}{2.293178in}}%
\pgfpathlineto{\pgfqpoint{2.375807in}{1.909526in}}%
\pgfpathlineto{\pgfqpoint{2.376418in}{2.080073in}}%
\pgfpathlineto{\pgfqpoint{2.377333in}{2.617958in}}%
\pgfpathlineto{\pgfqpoint{2.377943in}{2.337980in}}%
\pgfpathlineto{\pgfqpoint{2.378859in}{1.906472in}}%
\pgfpathlineto{\pgfqpoint{2.379469in}{1.994715in}}%
\pgfpathlineto{\pgfqpoint{2.380689in}{2.538218in}}%
\pgfpathlineto{\pgfqpoint{2.381300in}{2.242586in}}%
\pgfpathlineto{\pgfqpoint{2.382215in}{1.893329in}}%
\pgfpathlineto{\pgfqpoint{2.382825in}{1.993661in}}%
\pgfpathlineto{\pgfqpoint{2.384046in}{2.627270in}}%
\pgfpathlineto{\pgfqpoint{2.384656in}{2.300616in}}%
\pgfpathlineto{\pgfqpoint{2.385571in}{1.958511in}}%
\pgfpathlineto{\pgfqpoint{2.386487in}{2.077519in}}%
\pgfpathlineto{\pgfqpoint{2.387707in}{2.352411in}}%
\pgfpathlineto{\pgfqpoint{2.388317in}{2.244353in}}%
\pgfpathlineto{\pgfqpoint{2.389233in}{2.111978in}}%
\pgfpathlineto{\pgfqpoint{2.389843in}{2.143329in}}%
\pgfpathlineto{\pgfqpoint{2.391063in}{2.284590in}}%
\pgfpathlineto{\pgfqpoint{2.391979in}{2.234041in}}%
\pgfpathlineto{\pgfqpoint{2.393504in}{2.081755in}}%
\pgfpathlineto{\pgfqpoint{2.394115in}{2.122779in}}%
\pgfpathlineto{\pgfqpoint{2.395335in}{2.420125in}}%
\pgfpathlineto{\pgfqpoint{2.395945in}{2.224803in}}%
\pgfpathlineto{\pgfqpoint{2.397166in}{1.899991in}}%
\pgfpathlineto{\pgfqpoint{2.397471in}{1.966652in}}%
\pgfpathlineto{\pgfqpoint{2.398691in}{2.707169in}}%
\pgfpathlineto{\pgfqpoint{2.399302in}{2.347686in}}%
\pgfpathlineto{\pgfqpoint{2.400217in}{1.940792in}}%
\pgfpathlineto{\pgfqpoint{2.401132in}{2.134986in}}%
\pgfpathlineto{\pgfqpoint{2.402048in}{2.404673in}}%
\pgfpathlineto{\pgfqpoint{2.402658in}{2.275714in}}%
\pgfpathlineto{\pgfqpoint{2.403573in}{2.097037in}}%
\pgfpathlineto{\pgfqpoint{2.404489in}{2.139711in}}%
\pgfpathlineto{\pgfqpoint{2.404794in}{2.150289in}}%
\pgfpathlineto{\pgfqpoint{2.405404in}{2.118406in}}%
\pgfpathlineto{\pgfqpoint{2.406014in}{2.084213in}}%
\pgfpathlineto{\pgfqpoint{2.406319in}{2.122130in}}%
\pgfpathlineto{\pgfqpoint{2.407540in}{2.563376in}}%
\pgfpathlineto{\pgfqpoint{2.408150in}{2.253600in}}%
\pgfpathlineto{\pgfqpoint{2.409065in}{1.916305in}}%
\pgfpathlineto{\pgfqpoint{2.409676in}{2.018350in}}%
\pgfpathlineto{\pgfqpoint{2.410896in}{2.509421in}}%
\pgfpathlineto{\pgfqpoint{2.411506in}{2.261146in}}%
\pgfpathlineto{\pgfqpoint{2.412422in}{1.989255in}}%
\pgfpathlineto{\pgfqpoint{2.413032in}{2.071326in}}%
\pgfpathlineto{\pgfqpoint{2.414252in}{2.350910in}}%
\pgfpathlineto{\pgfqpoint{2.414863in}{2.180182in}}%
\pgfpathlineto{\pgfqpoint{2.415778in}{1.983402in}}%
\pgfpathlineto{\pgfqpoint{2.416388in}{2.102836in}}%
\pgfpathlineto{\pgfqpoint{2.417609in}{2.537112in}}%
\pgfpathlineto{\pgfqpoint{2.418219in}{2.298296in}}%
\pgfpathlineto{\pgfqpoint{2.419439in}{1.975517in}}%
\pgfpathlineto{\pgfqpoint{2.420050in}{2.008358in}}%
\pgfpathlineto{\pgfqpoint{2.421575in}{2.463033in}}%
\pgfpathlineto{\pgfqpoint{2.422491in}{2.213236in}}%
\pgfpathlineto{\pgfqpoint{2.423711in}{1.959820in}}%
\pgfpathlineto{\pgfqpoint{2.424321in}{2.056938in}}%
\pgfpathlineto{\pgfqpoint{2.425542in}{2.692334in}}%
\pgfpathlineto{\pgfqpoint{2.426152in}{2.353294in}}%
\pgfpathlineto{\pgfqpoint{2.427372in}{1.867480in}}%
\pgfpathlineto{\pgfqpoint{2.427983in}{1.959447in}}%
\pgfpathlineto{\pgfqpoint{2.429203in}{2.482040in}}%
\pgfpathlineto{\pgfqpoint{2.430119in}{2.230869in}}%
\pgfpathlineto{\pgfqpoint{2.430729in}{2.104082in}}%
\pgfpathlineto{\pgfqpoint{2.431644in}{2.221111in}}%
\pgfpathlineto{\pgfqpoint{2.431949in}{2.255931in}}%
\pgfpathlineto{\pgfqpoint{2.432559in}{2.211980in}}%
\pgfpathlineto{\pgfqpoint{2.433475in}{2.065856in}}%
\pgfpathlineto{\pgfqpoint{2.434085in}{2.127802in}}%
\pgfpathlineto{\pgfqpoint{2.435306in}{2.396936in}}%
\pgfpathlineto{\pgfqpoint{2.435916in}{2.221941in}}%
\pgfpathlineto{\pgfqpoint{2.436831in}{2.060737in}}%
\pgfpathlineto{\pgfqpoint{2.437441in}{2.122748in}}%
\pgfpathlineto{\pgfqpoint{2.438357in}{2.218397in}}%
\pgfpathlineto{\pgfqpoint{2.438967in}{2.169008in}}%
\pgfpathlineto{\pgfqpoint{2.439577in}{2.122779in}}%
\pgfpathlineto{\pgfqpoint{2.440187in}{2.200348in}}%
\pgfpathlineto{\pgfqpoint{2.441103in}{2.372396in}}%
\pgfpathlineto{\pgfqpoint{2.441713in}{2.236967in}}%
\pgfpathlineto{\pgfqpoint{2.442628in}{2.085682in}}%
\pgfpathlineto{\pgfqpoint{2.443239in}{2.239947in}}%
\pgfpathlineto{\pgfqpoint{2.443849in}{2.405194in}}%
\pgfpathlineto{\pgfqpoint{2.444459in}{2.290592in}}%
\pgfpathlineto{\pgfqpoint{2.445985in}{1.941824in}}%
\pgfpathlineto{\pgfqpoint{2.446595in}{2.052596in}}%
\pgfpathlineto{\pgfqpoint{2.447815in}{2.542667in}}%
\pgfpathlineto{\pgfqpoint{2.448426in}{2.323922in}}%
\pgfpathlineto{\pgfqpoint{2.449341in}{2.022788in}}%
\pgfpathlineto{\pgfqpoint{2.449951in}{2.094174in}}%
\pgfpathlineto{\pgfqpoint{2.451172in}{2.374610in}}%
\pgfpathlineto{\pgfqpoint{2.451782in}{2.236041in}}%
\pgfpathlineto{\pgfqpoint{2.452697in}{2.054756in}}%
\pgfpathlineto{\pgfqpoint{2.453613in}{2.140371in}}%
\pgfpathlineto{\pgfqpoint{2.455748in}{2.255729in}}%
\pgfpathlineto{\pgfqpoint{2.456054in}{2.274224in}}%
\pgfpathlineto{\pgfqpoint{2.456664in}{2.231976in}}%
\pgfpathlineto{\pgfqpoint{2.457579in}{2.104741in}}%
\pgfpathlineto{\pgfqpoint{2.458189in}{2.181033in}}%
\pgfpathlineto{\pgfqpoint{2.459105in}{2.367150in}}%
\pgfpathlineto{\pgfqpoint{2.459715in}{2.246460in}}%
\pgfpathlineto{\pgfqpoint{2.460630in}{2.043423in}}%
\pgfpathlineto{\pgfqpoint{2.461546in}{2.145564in}}%
\pgfpathlineto{\pgfqpoint{2.462461in}{2.298382in}}%
\pgfpathlineto{\pgfqpoint{2.463071in}{2.225719in}}%
\pgfpathlineto{\pgfqpoint{2.463987in}{2.083745in}}%
\pgfpathlineto{\pgfqpoint{2.464597in}{2.143999in}}%
\pgfpathlineto{\pgfqpoint{2.465512in}{2.349197in}}%
\pgfpathlineto{\pgfqpoint{2.466428in}{2.216854in}}%
\pgfpathlineto{\pgfqpoint{2.467038in}{2.140658in}}%
\pgfpathlineto{\pgfqpoint{2.467648in}{2.191665in}}%
\pgfpathlineto{\pgfqpoint{2.468258in}{2.229369in}}%
\pgfpathlineto{\pgfqpoint{2.468869in}{2.167539in}}%
\pgfpathlineto{\pgfqpoint{2.469479in}{2.117533in}}%
\pgfpathlineto{\pgfqpoint{2.470089in}{2.184534in}}%
\pgfpathlineto{\pgfqpoint{2.471004in}{2.307778in}}%
\pgfpathlineto{\pgfqpoint{2.471615in}{2.188036in}}%
\pgfpathlineto{\pgfqpoint{2.472530in}{2.027045in}}%
\pgfpathlineto{\pgfqpoint{2.473140in}{2.179948in}}%
\pgfpathlineto{\pgfqpoint{2.474056in}{2.543954in}}%
\pgfpathlineto{\pgfqpoint{2.474666in}{2.273288in}}%
\pgfpathlineto{\pgfqpoint{2.475886in}{1.890435in}}%
\pgfpathlineto{\pgfqpoint{2.476496in}{2.027013in}}%
\pgfpathlineto{\pgfqpoint{2.477717in}{2.579669in}}%
\pgfpathlineto{\pgfqpoint{2.478327in}{2.287857in}}%
\pgfpathlineto{\pgfqpoint{2.479243in}{2.003909in}}%
\pgfpathlineto{\pgfqpoint{2.479853in}{2.089130in}}%
\pgfpathlineto{\pgfqpoint{2.480768in}{2.289410in}}%
\pgfpathlineto{\pgfqpoint{2.481683in}{2.178181in}}%
\pgfpathlineto{\pgfqpoint{2.482294in}{2.106306in}}%
\pgfpathlineto{\pgfqpoint{2.482904in}{2.181736in}}%
\pgfpathlineto{\pgfqpoint{2.483819in}{2.369012in}}%
\pgfpathlineto{\pgfqpoint{2.484430in}{2.273192in}}%
\pgfpathlineto{\pgfqpoint{2.485955in}{2.051181in}}%
\pgfpathlineto{\pgfqpoint{2.486565in}{2.069761in}}%
\pgfpathlineto{\pgfqpoint{2.487481in}{2.193953in}}%
\pgfpathlineto{\pgfqpoint{2.488396in}{2.499343in}}%
\pgfpathlineto{\pgfqpoint{2.489006in}{2.399352in}}%
\pgfpathlineto{\pgfqpoint{2.490227in}{1.964534in}}%
\pgfpathlineto{\pgfqpoint{2.490837in}{2.017871in}}%
\pgfpathlineto{\pgfqpoint{2.492057in}{2.445336in}}%
\pgfpathlineto{\pgfqpoint{2.492668in}{2.301457in}}%
\pgfpathlineto{\pgfqpoint{2.493888in}{2.006932in}}%
\pgfpathlineto{\pgfqpoint{2.494498in}{2.132751in}}%
\pgfpathlineto{\pgfqpoint{2.495414in}{2.399341in}}%
\pgfpathlineto{\pgfqpoint{2.496024in}{2.257229in}}%
\pgfpathlineto{\pgfqpoint{2.496939in}{2.034664in}}%
\pgfpathlineto{\pgfqpoint{2.497550in}{2.156035in}}%
\pgfpathlineto{\pgfqpoint{2.498465in}{2.394042in}}%
\pgfpathlineto{\pgfqpoint{2.499075in}{2.232157in}}%
\pgfpathlineto{\pgfqpoint{2.499991in}{2.045381in}}%
\pgfpathlineto{\pgfqpoint{2.500601in}{2.125312in}}%
\pgfpathlineto{\pgfqpoint{2.501516in}{2.295785in}}%
\pgfpathlineto{\pgfqpoint{2.502431in}{2.220046in}}%
\pgfpathlineto{\pgfqpoint{2.502737in}{2.195432in}}%
\pgfpathlineto{\pgfqpoint{2.503347in}{2.220919in}}%
\pgfpathlineto{\pgfqpoint{2.503957in}{2.271468in}}%
\pgfpathlineto{\pgfqpoint{2.504567in}{2.206744in}}%
\pgfpathlineto{\pgfqpoint{2.505788in}{1.970898in}}%
\pgfpathlineto{\pgfqpoint{2.506398in}{2.069942in}}%
\pgfpathlineto{\pgfqpoint{2.507619in}{2.645180in}}%
\pgfpathlineto{\pgfqpoint{2.508229in}{2.287708in}}%
\pgfpathlineto{\pgfqpoint{2.509449in}{1.924521in}}%
\pgfpathlineto{\pgfqpoint{2.510059in}{2.028322in}}%
\pgfpathlineto{\pgfqpoint{2.511280in}{2.342492in}}%
\pgfpathlineto{\pgfqpoint{2.512195in}{2.245066in}}%
\pgfpathlineto{\pgfqpoint{2.513416in}{2.160505in}}%
\pgfpathlineto{\pgfqpoint{2.514026in}{2.184332in}}%
\pgfpathlineto{\pgfqpoint{2.515246in}{2.321389in}}%
\pgfpathlineto{\pgfqpoint{2.515857in}{2.266903in}}%
\pgfpathlineto{\pgfqpoint{2.517077in}{2.065505in}}%
\pgfpathlineto{\pgfqpoint{2.517687in}{2.120768in}}%
\pgfpathlineto{\pgfqpoint{2.518908in}{2.303734in}}%
\pgfpathlineto{\pgfqpoint{2.519518in}{2.200848in}}%
\pgfpathlineto{\pgfqpoint{2.520433in}{2.073252in}}%
\pgfpathlineto{\pgfqpoint{2.521044in}{2.164879in}}%
\pgfpathlineto{\pgfqpoint{2.521959in}{2.395851in}}%
\pgfpathlineto{\pgfqpoint{2.522569in}{2.281333in}}%
\pgfpathlineto{\pgfqpoint{2.523790in}{1.965928in}}%
\pgfpathlineto{\pgfqpoint{2.524400in}{2.060950in}}%
\pgfpathlineto{\pgfqpoint{2.525620in}{2.548158in}}%
\pgfpathlineto{\pgfqpoint{2.526231in}{2.309790in}}%
\pgfpathlineto{\pgfqpoint{2.527146in}{2.007996in}}%
\pgfpathlineto{\pgfqpoint{2.528061in}{2.160760in}}%
\pgfpathlineto{\pgfqpoint{2.528977in}{2.316547in}}%
\pgfpathlineto{\pgfqpoint{2.529587in}{2.176255in}}%
\pgfpathlineto{\pgfqpoint{2.530197in}{2.054033in}}%
\pgfpathlineto{\pgfqpoint{2.530807in}{2.107466in}}%
\pgfpathlineto{\pgfqpoint{2.531723in}{2.317856in}}%
\pgfpathlineto{\pgfqpoint{2.532638in}{2.225421in}}%
\pgfpathlineto{\pgfqpoint{2.533248in}{2.173882in}}%
\pgfpathlineto{\pgfqpoint{2.534164in}{2.215513in}}%
\pgfpathlineto{\pgfqpoint{2.534774in}{2.187227in}}%
\pgfpathlineto{\pgfqpoint{2.535384in}{2.138402in}}%
\pgfpathlineto{\pgfqpoint{2.535994in}{2.162389in}}%
\pgfpathlineto{\pgfqpoint{2.536910in}{2.266786in}}%
\pgfpathlineto{\pgfqpoint{2.537520in}{2.236499in}}%
\pgfpathlineto{\pgfqpoint{2.538435in}{2.124397in}}%
\pgfpathlineto{\pgfqpoint{2.539046in}{2.156429in}}%
\pgfpathlineto{\pgfqpoint{2.540266in}{2.355848in}}%
\pgfpathlineto{\pgfqpoint{2.540876in}{2.256027in}}%
\pgfpathlineto{\pgfqpoint{2.542097in}{2.037878in}}%
\pgfpathlineto{\pgfqpoint{2.542707in}{2.093280in}}%
\pgfpathlineto{\pgfqpoint{2.543928in}{2.262242in}}%
\pgfpathlineto{\pgfqpoint{2.544538in}{2.201636in}}%
\pgfpathlineto{\pgfqpoint{2.544843in}{2.180214in}}%
\pgfpathlineto{\pgfqpoint{2.545453in}{2.243235in}}%
\pgfpathlineto{\pgfqpoint{2.546063in}{2.364617in}}%
\pgfpathlineto{\pgfqpoint{2.546674in}{2.285611in}}%
\pgfpathlineto{\pgfqpoint{2.547894in}{2.010337in}}%
\pgfpathlineto{\pgfqpoint{2.548504in}{2.138019in}}%
\pgfpathlineto{\pgfqpoint{2.549420in}{2.361786in}}%
\pgfpathlineto{\pgfqpoint{2.550030in}{2.271756in}}%
\pgfpathlineto{\pgfqpoint{2.550945in}{2.147862in}}%
\pgfpathlineto{\pgfqpoint{2.551556in}{2.191122in}}%
\pgfpathlineto{\pgfqpoint{2.551861in}{2.208798in}}%
\pgfpathlineto{\pgfqpoint{2.552471in}{2.178298in}}%
\pgfpathlineto{\pgfqpoint{2.553691in}{2.050010in}}%
\pgfpathlineto{\pgfqpoint{2.553996in}{2.077009in}}%
\pgfpathlineto{\pgfqpoint{2.555217in}{2.503983in}}%
\pgfpathlineto{\pgfqpoint{2.556132in}{2.249950in}}%
\pgfpathlineto{\pgfqpoint{2.557048in}{2.007017in}}%
\pgfpathlineto{\pgfqpoint{2.557963in}{2.118406in}}%
\pgfpathlineto{\pgfqpoint{2.559183in}{2.306704in}}%
\pgfpathlineto{\pgfqpoint{2.559794in}{2.201413in}}%
\pgfpathlineto{\pgfqpoint{2.560404in}{2.107508in}}%
\pgfpathlineto{\pgfqpoint{2.561319in}{2.191824in}}%
\pgfpathlineto{\pgfqpoint{2.561930in}{2.251568in}}%
\pgfpathlineto{\pgfqpoint{2.562845in}{2.198486in}}%
\pgfpathlineto{\pgfqpoint{2.564065in}{2.364787in}}%
\pgfpathlineto{\pgfqpoint{2.564676in}{2.277300in}}%
\pgfpathlineto{\pgfqpoint{2.565896in}{2.012167in}}%
\pgfpathlineto{\pgfqpoint{2.566506in}{2.093131in}}%
\pgfpathlineto{\pgfqpoint{2.567422in}{2.270723in}}%
\pgfpathlineto{\pgfqpoint{2.568337in}{2.164506in}}%
\pgfpathlineto{\pgfqpoint{2.568642in}{2.134773in}}%
\pgfpathlineto{\pgfqpoint{2.569252in}{2.197230in}}%
\pgfpathlineto{\pgfqpoint{2.570168in}{2.358498in}}%
\pgfpathlineto{\pgfqpoint{2.570778in}{2.247747in}}%
\pgfpathlineto{\pgfqpoint{2.571998in}{2.056278in}}%
\pgfpathlineto{\pgfqpoint{2.572609in}{2.135475in}}%
\pgfpathlineto{\pgfqpoint{2.573829in}{2.342386in}}%
\pgfpathlineto{\pgfqpoint{2.574439in}{2.254537in}}%
\pgfpathlineto{\pgfqpoint{2.575355in}{2.130974in}}%
\pgfpathlineto{\pgfqpoint{2.575965in}{2.176670in}}%
\pgfpathlineto{\pgfqpoint{2.576575in}{2.236935in}}%
\pgfpathlineto{\pgfqpoint{2.577185in}{2.195559in}}%
\pgfpathlineto{\pgfqpoint{2.578101in}{2.074976in}}%
\pgfpathlineto{\pgfqpoint{2.578711in}{2.144957in}}%
\pgfpathlineto{\pgfqpoint{2.579626in}{2.393946in}}%
\pgfpathlineto{\pgfqpoint{2.580237in}{2.315675in}}%
\pgfpathlineto{\pgfqpoint{2.581457in}{2.062557in}}%
\pgfpathlineto{\pgfqpoint{2.582067in}{2.127536in}}%
\pgfpathlineto{\pgfqpoint{2.582983in}{2.242852in}}%
\pgfpathlineto{\pgfqpoint{2.583898in}{2.220355in}}%
\pgfpathlineto{\pgfqpoint{2.584813in}{2.221089in}}%
\pgfpathlineto{\pgfqpoint{2.585729in}{2.198805in}}%
\pgfpathlineto{\pgfqpoint{2.587254in}{2.130335in}}%
\pgfpathlineto{\pgfqpoint{2.587865in}{2.172264in}}%
\pgfpathlineto{\pgfqpoint{2.589085in}{2.380463in}}%
\pgfpathlineto{\pgfqpoint{2.589695in}{2.263104in}}%
\pgfpathlineto{\pgfqpoint{2.590916in}{2.056427in}}%
\pgfpathlineto{\pgfqpoint{2.591526in}{2.080286in}}%
\pgfpathlineto{\pgfqpoint{2.593357in}{2.298222in}}%
\pgfpathlineto{\pgfqpoint{2.593967in}{2.238180in}}%
\pgfpathlineto{\pgfqpoint{2.595187in}{2.140275in}}%
\pgfpathlineto{\pgfqpoint{2.595798in}{2.183162in}}%
\pgfpathlineto{\pgfqpoint{2.597018in}{2.296817in}}%
\pgfpathlineto{\pgfqpoint{2.597628in}{2.235530in}}%
\pgfpathlineto{\pgfqpoint{2.598849in}{2.103113in}}%
\pgfpathlineto{\pgfqpoint{2.599459in}{2.164698in}}%
\pgfpathlineto{\pgfqpoint{2.600374in}{2.241511in}}%
\pgfpathlineto{\pgfqpoint{2.600985in}{2.186769in}}%
\pgfpathlineto{\pgfqpoint{2.601595in}{2.141275in}}%
\pgfpathlineto{\pgfqpoint{2.602205in}{2.181161in}}%
\pgfpathlineto{\pgfqpoint{2.603120in}{2.299414in}}%
\pgfpathlineto{\pgfqpoint{2.603731in}{2.253888in}}%
\pgfpathlineto{\pgfqpoint{2.604951in}{2.158994in}}%
\pgfpathlineto{\pgfqpoint{2.605867in}{2.163389in}}%
\pgfpathlineto{\pgfqpoint{2.606477in}{2.172190in}}%
\pgfpathlineto{\pgfqpoint{2.608613in}{2.260007in}}%
\pgfpathlineto{\pgfqpoint{2.609223in}{2.235786in}}%
\pgfpathlineto{\pgfqpoint{2.610443in}{2.108115in}}%
\pgfpathlineto{\pgfqpoint{2.611054in}{2.135837in}}%
\pgfpathlineto{\pgfqpoint{2.612274in}{2.295934in}}%
\pgfpathlineto{\pgfqpoint{2.612884in}{2.230263in}}%
\pgfpathlineto{\pgfqpoint{2.613800in}{2.162112in}}%
\pgfpathlineto{\pgfqpoint{2.614410in}{2.208202in}}%
\pgfpathlineto{\pgfqpoint{2.614715in}{2.222313in}}%
\pgfpathlineto{\pgfqpoint{2.615020in}{2.209873in}}%
\pgfpathlineto{\pgfqpoint{2.616241in}{2.091258in}}%
\pgfpathlineto{\pgfqpoint{2.616546in}{2.126078in}}%
\pgfpathlineto{\pgfqpoint{2.617766in}{2.343280in}}%
\pgfpathlineto{\pgfqpoint{2.618376in}{2.214683in}}%
\pgfpathlineto{\pgfqpoint{2.619292in}{2.080020in}}%
\pgfpathlineto{\pgfqpoint{2.619902in}{2.146553in}}%
\pgfpathlineto{\pgfqpoint{2.621428in}{2.329073in}}%
\pgfpathlineto{\pgfqpoint{2.622038in}{2.266573in}}%
\pgfpathlineto{\pgfqpoint{2.623258in}{2.084745in}}%
\pgfpathlineto{\pgfqpoint{2.624174in}{2.146224in}}%
\pgfpathlineto{\pgfqpoint{2.625089in}{2.225187in}}%
\pgfpathlineto{\pgfqpoint{2.626004in}{2.188014in}}%
\pgfpathlineto{\pgfqpoint{2.626309in}{2.192346in}}%
\pgfpathlineto{\pgfqpoint{2.627530in}{2.306246in}}%
\pgfpathlineto{\pgfqpoint{2.627835in}{2.270702in}}%
\pgfpathlineto{\pgfqpoint{2.629056in}{2.097941in}}%
\pgfpathlineto{\pgfqpoint{2.629666in}{2.149427in}}%
\pgfpathlineto{\pgfqpoint{2.630276in}{2.193889in}}%
\pgfpathlineto{\pgfqpoint{2.630886in}{2.168050in}}%
\pgfpathlineto{\pgfqpoint{2.631191in}{2.146458in}}%
\pgfpathlineto{\pgfqpoint{2.631802in}{2.205818in}}%
\pgfpathlineto{\pgfqpoint{2.632717in}{2.460862in}}%
\pgfpathlineto{\pgfqpoint{2.633327in}{2.266967in}}%
\pgfpathlineto{\pgfqpoint{2.634243in}{1.951136in}}%
\pgfpathlineto{\pgfqpoint{2.634853in}{2.024480in}}%
\pgfpathlineto{\pgfqpoint{2.636073in}{2.458925in}}%
\pgfpathlineto{\pgfqpoint{2.636683in}{2.249354in}}%
\pgfpathlineto{\pgfqpoint{2.637599in}{2.007485in}}%
\pgfpathlineto{\pgfqpoint{2.638209in}{2.079286in}}%
\pgfpathlineto{\pgfqpoint{2.639430in}{2.353145in}}%
\pgfpathlineto{\pgfqpoint{2.640040in}{2.250897in}}%
\pgfpathlineto{\pgfqpoint{2.640955in}{2.140913in}}%
\pgfpathlineto{\pgfqpoint{2.641565in}{2.196475in}}%
\pgfpathlineto{\pgfqpoint{2.642176in}{2.247024in}}%
\pgfpathlineto{\pgfqpoint{2.642786in}{2.209649in}}%
\pgfpathlineto{\pgfqpoint{2.643701in}{2.137050in}}%
\pgfpathlineto{\pgfqpoint{2.644311in}{2.175925in}}%
\pgfpathlineto{\pgfqpoint{2.645227in}{2.235626in}}%
\pgfpathlineto{\pgfqpoint{2.645837in}{2.174116in}}%
\pgfpathlineto{\pgfqpoint{2.646447in}{2.108317in}}%
\pgfpathlineto{\pgfqpoint{2.647057in}{2.172850in}}%
\pgfpathlineto{\pgfqpoint{2.647973in}{2.413591in}}%
\pgfpathlineto{\pgfqpoint{2.648583in}{2.304703in}}%
\pgfpathlineto{\pgfqpoint{2.649804in}{2.000121in}}%
\pgfpathlineto{\pgfqpoint{2.650414in}{2.069144in}}%
\pgfpathlineto{\pgfqpoint{2.651634in}{2.350995in}}%
\pgfpathlineto{\pgfqpoint{2.652550in}{2.226112in}}%
\pgfpathlineto{\pgfqpoint{2.653770in}{2.095887in}}%
\pgfpathlineto{\pgfqpoint{2.654380in}{2.144872in}}%
\pgfpathlineto{\pgfqpoint{2.655906in}{2.402119in}}%
\pgfpathlineto{\pgfqpoint{2.656516in}{2.277960in}}%
\pgfpathlineto{\pgfqpoint{2.658042in}{2.024108in}}%
\pgfpathlineto{\pgfqpoint{2.658652in}{2.082287in}}%
\pgfpathlineto{\pgfqpoint{2.659872in}{2.222781in}}%
\pgfpathlineto{\pgfqpoint{2.660483in}{2.160079in}}%
\pgfpathlineto{\pgfqpoint{2.660788in}{2.141499in}}%
\pgfpathlineto{\pgfqpoint{2.661093in}{2.165485in}}%
\pgfpathlineto{\pgfqpoint{2.662313in}{2.512550in}}%
\pgfpathlineto{\pgfqpoint{2.662924in}{2.302606in}}%
\pgfpathlineto{\pgfqpoint{2.664144in}{1.984030in}}%
\pgfpathlineto{\pgfqpoint{2.664754in}{2.046317in}}%
\pgfpathlineto{\pgfqpoint{2.665975in}{2.305969in}}%
\pgfpathlineto{\pgfqpoint{2.666890in}{2.198050in}}%
\pgfpathlineto{\pgfqpoint{2.667806in}{2.108008in}}%
\pgfpathlineto{\pgfqpoint{2.668416in}{2.185769in}}%
\pgfpathlineto{\pgfqpoint{2.669331in}{2.350272in}}%
\pgfpathlineto{\pgfqpoint{2.669941in}{2.287740in}}%
\pgfpathlineto{\pgfqpoint{2.672077in}{2.173648in}}%
\pgfpathlineto{\pgfqpoint{2.673298in}{2.023884in}}%
\pgfpathlineto{\pgfqpoint{2.673908in}{2.080212in}}%
\pgfpathlineto{\pgfqpoint{2.675128in}{2.555362in}}%
\pgfpathlineto{\pgfqpoint{2.675739in}{2.345898in}}%
\pgfpathlineto{\pgfqpoint{2.676959in}{1.977932in}}%
\pgfpathlineto{\pgfqpoint{2.677569in}{2.092141in}}%
\pgfpathlineto{\pgfqpoint{2.678485in}{2.322049in}}%
\pgfpathlineto{\pgfqpoint{2.679095in}{2.239064in}}%
\pgfpathlineto{\pgfqpoint{2.680010in}{2.115745in}}%
\pgfpathlineto{\pgfqpoint{2.680620in}{2.194761in}}%
\pgfpathlineto{\pgfqpoint{2.681536in}{2.299084in}}%
\pgfpathlineto{\pgfqpoint{2.682146in}{2.254377in}}%
\pgfpathlineto{\pgfqpoint{2.683367in}{2.135901in}}%
\pgfpathlineto{\pgfqpoint{2.683977in}{2.163900in}}%
\pgfpathlineto{\pgfqpoint{2.684892in}{2.265871in}}%
\pgfpathlineto{\pgfqpoint{2.685502in}{2.215396in}}%
\pgfpathlineto{\pgfqpoint{2.686418in}{2.102475in}}%
\pgfpathlineto{\pgfqpoint{2.687028in}{2.173850in}}%
\pgfpathlineto{\pgfqpoint{2.687943in}{2.280833in}}%
\pgfpathlineto{\pgfqpoint{2.688554in}{2.229145in}}%
\pgfpathlineto{\pgfqpoint{2.690994in}{2.088672in}}%
\pgfpathlineto{\pgfqpoint{2.691300in}{2.120906in}}%
\pgfpathlineto{\pgfqpoint{2.692520in}{2.366437in}}%
\pgfpathlineto{\pgfqpoint{2.693130in}{2.238170in}}%
\pgfpathlineto{\pgfqpoint{2.694046in}{2.080925in}}%
\pgfpathlineto{\pgfqpoint{2.694656in}{2.185205in}}%
\pgfpathlineto{\pgfqpoint{2.695571in}{2.370672in}}%
\pgfpathlineto{\pgfqpoint{2.696181in}{2.223931in}}%
\pgfpathlineto{\pgfqpoint{2.697097in}{2.018670in}}%
\pgfpathlineto{\pgfqpoint{2.697707in}{2.090077in}}%
\pgfpathlineto{\pgfqpoint{2.698928in}{2.296466in}}%
\pgfpathlineto{\pgfqpoint{2.699538in}{2.216247in}}%
\pgfpathlineto{\pgfqpoint{2.700148in}{2.167667in}}%
\pgfpathlineto{\pgfqpoint{2.700758in}{2.233455in}}%
\pgfpathlineto{\pgfqpoint{2.701369in}{2.304235in}}%
\pgfpathlineto{\pgfqpoint{2.701979in}{2.235243in}}%
\pgfpathlineto{\pgfqpoint{2.703199in}{2.057449in}}%
\pgfpathlineto{\pgfqpoint{2.703809in}{2.154205in}}%
\pgfpathlineto{\pgfqpoint{2.705030in}{2.400693in}}%
\pgfpathlineto{\pgfqpoint{2.705640in}{2.257549in}}%
\pgfpathlineto{\pgfqpoint{2.706861in}{1.986531in}}%
\pgfpathlineto{\pgfqpoint{2.707471in}{2.035707in}}%
\pgfpathlineto{\pgfqpoint{2.708691in}{2.424020in}}%
\pgfpathlineto{\pgfqpoint{2.709302in}{2.275438in}}%
\pgfpathlineto{\pgfqpoint{2.709912in}{2.097590in}}%
\pgfpathlineto{\pgfqpoint{2.710827in}{2.238531in}}%
\pgfpathlineto{\pgfqpoint{2.711437in}{2.344929in}}%
\pgfpathlineto{\pgfqpoint{2.712048in}{2.240990in}}%
\pgfpathlineto{\pgfqpoint{2.713268in}{1.998844in}}%
\pgfpathlineto{\pgfqpoint{2.713878in}{2.049382in}}%
\pgfpathlineto{\pgfqpoint{2.715099in}{2.506186in}}%
\pgfpathlineto{\pgfqpoint{2.716014in}{2.238372in}}%
\pgfpathlineto{\pgfqpoint{2.716930in}{2.036069in}}%
\pgfpathlineto{\pgfqpoint{2.717540in}{2.094770in}}%
\pgfpathlineto{\pgfqpoint{2.719065in}{2.302362in}}%
\pgfpathlineto{\pgfqpoint{2.719676in}{2.252228in}}%
\pgfpathlineto{\pgfqpoint{2.720896in}{2.051755in}}%
\pgfpathlineto{\pgfqpoint{2.721506in}{2.099984in}}%
\pgfpathlineto{\pgfqpoint{2.722727in}{2.323709in}}%
\pgfpathlineto{\pgfqpoint{2.723642in}{2.291145in}}%
\pgfpathlineto{\pgfqpoint{2.727609in}{2.128686in}}%
\pgfpathlineto{\pgfqpoint{2.728524in}{2.191367in}}%
\pgfpathlineto{\pgfqpoint{2.730965in}{2.390870in}}%
\pgfpathlineto{\pgfqpoint{2.731270in}{2.371023in}}%
\pgfpathlineto{\pgfqpoint{2.733406in}{1.968525in}}%
\pgfpathlineto{\pgfqpoint{2.734016in}{2.023959in}}%
\pgfpathlineto{\pgfqpoint{2.735237in}{2.568856in}}%
\pgfpathlineto{\pgfqpoint{2.735847in}{2.343727in}}%
\pgfpathlineto{\pgfqpoint{2.737067in}{1.878165in}}%
\pgfpathlineto{\pgfqpoint{2.737678in}{2.040666in}}%
\pgfpathlineto{\pgfqpoint{2.738593in}{2.666826in}}%
\pgfpathlineto{\pgfqpoint{2.739508in}{2.245002in}}%
\pgfpathlineto{\pgfqpoint{2.740424in}{1.922084in}}%
\pgfpathlineto{\pgfqpoint{2.741034in}{1.991299in}}%
\pgfpathlineto{\pgfqpoint{2.742559in}{2.360020in}}%
\pgfpathlineto{\pgfqpoint{2.743170in}{2.227230in}}%
\pgfpathlineto{\pgfqpoint{2.744085in}{2.102336in}}%
\pgfpathlineto{\pgfqpoint{2.744695in}{2.227602in}}%
\pgfpathlineto{\pgfqpoint{2.745306in}{2.389168in}}%
\pgfpathlineto{\pgfqpoint{2.745916in}{2.308832in}}%
\pgfpathlineto{\pgfqpoint{2.747136in}{2.048669in}}%
\pgfpathlineto{\pgfqpoint{2.747746in}{2.111722in}}%
\pgfpathlineto{\pgfqpoint{2.748967in}{2.280226in}}%
\pgfpathlineto{\pgfqpoint{2.749577in}{2.228188in}}%
\pgfpathlineto{\pgfqpoint{2.750493in}{2.125578in}}%
\pgfpathlineto{\pgfqpoint{2.751103in}{2.152151in}}%
\pgfpathlineto{\pgfqpoint{2.752323in}{2.295317in}}%
\pgfpathlineto{\pgfqpoint{2.753239in}{2.261177in}}%
\pgfpathlineto{\pgfqpoint{2.753849in}{2.271692in}}%
\pgfpathlineto{\pgfqpoint{2.754154in}{2.247226in}}%
\pgfpathlineto{\pgfqpoint{2.755680in}{1.967365in}}%
\pgfpathlineto{\pgfqpoint{2.756290in}{2.030099in}}%
\pgfpathlineto{\pgfqpoint{2.757815in}{2.530705in}}%
\pgfpathlineto{\pgfqpoint{2.758426in}{2.326944in}}%
\pgfpathlineto{\pgfqpoint{2.759646in}{1.977847in}}%
\pgfpathlineto{\pgfqpoint{2.760256in}{2.042518in}}%
\pgfpathlineto{\pgfqpoint{2.761477in}{2.443750in}}%
\pgfpathlineto{\pgfqpoint{2.762087in}{2.289421in}}%
\pgfpathlineto{\pgfqpoint{2.763002in}{2.029929in}}%
\pgfpathlineto{\pgfqpoint{2.763918in}{2.129111in}}%
\pgfpathlineto{\pgfqpoint{2.765138in}{2.305299in}}%
\pgfpathlineto{\pgfqpoint{2.765748in}{2.259187in}}%
\pgfpathlineto{\pgfqpoint{2.766969in}{2.115511in}}%
\pgfpathlineto{\pgfqpoint{2.767579in}{2.151991in}}%
\pgfpathlineto{\pgfqpoint{2.768494in}{2.298978in}}%
\pgfpathlineto{\pgfqpoint{2.769105in}{2.240734in}}%
\pgfpathlineto{\pgfqpoint{2.770325in}{2.050393in}}%
\pgfpathlineto{\pgfqpoint{2.770630in}{2.101932in}}%
\pgfpathlineto{\pgfqpoint{2.771851in}{2.527853in}}%
\pgfpathlineto{\pgfqpoint{2.772461in}{2.267073in}}%
\pgfpathlineto{\pgfqpoint{2.773681in}{1.900800in}}%
\pgfpathlineto{\pgfqpoint{2.774292in}{2.057002in}}%
\pgfpathlineto{\pgfqpoint{2.775207in}{2.509932in}}%
\pgfpathlineto{\pgfqpoint{2.776122in}{2.242011in}}%
\pgfpathlineto{\pgfqpoint{2.777038in}{2.035356in}}%
\pgfpathlineto{\pgfqpoint{2.777648in}{2.088257in}}%
\pgfpathlineto{\pgfqpoint{2.779784in}{2.323316in}}%
\pgfpathlineto{\pgfqpoint{2.780089in}{2.285643in}}%
\pgfpathlineto{\pgfqpoint{2.781309in}{2.053352in}}%
\pgfpathlineto{\pgfqpoint{2.781920in}{2.202764in}}%
\pgfpathlineto{\pgfqpoint{2.782835in}{2.434077in}}%
\pgfpathlineto{\pgfqpoint{2.783445in}{2.223122in}}%
\pgfpathlineto{\pgfqpoint{2.784666in}{1.982445in}}%
\pgfpathlineto{\pgfqpoint{2.785276in}{2.076689in}}%
\pgfpathlineto{\pgfqpoint{2.786496in}{2.400672in}}%
\pgfpathlineto{\pgfqpoint{2.787107in}{2.323411in}}%
\pgfpathlineto{\pgfqpoint{2.788327in}{2.138178in}}%
\pgfpathlineto{\pgfqpoint{2.788937in}{2.168401in}}%
\pgfpathlineto{\pgfqpoint{2.789243in}{2.175276in}}%
\pgfpathlineto{\pgfqpoint{2.789853in}{2.156844in}}%
\pgfpathlineto{\pgfqpoint{2.790158in}{2.149288in}}%
\pgfpathlineto{\pgfqpoint{2.790463in}{2.158313in}}%
\pgfpathlineto{\pgfqpoint{2.791683in}{2.270702in}}%
\pgfpathlineto{\pgfqpoint{2.792294in}{2.216545in}}%
\pgfpathlineto{\pgfqpoint{2.792599in}{2.193325in}}%
\pgfpathlineto{\pgfqpoint{2.793514in}{2.240011in}}%
\pgfpathlineto{\pgfqpoint{2.793819in}{2.244353in}}%
\pgfpathlineto{\pgfqpoint{2.795040in}{2.099835in}}%
\pgfpathlineto{\pgfqpoint{2.795650in}{2.153886in}}%
\pgfpathlineto{\pgfqpoint{2.796565in}{2.318622in}}%
\pgfpathlineto{\pgfqpoint{2.797176in}{2.223761in}}%
\pgfpathlineto{\pgfqpoint{2.797786in}{2.140860in}}%
\pgfpathlineto{\pgfqpoint{2.798701in}{2.187418in}}%
\pgfpathlineto{\pgfqpoint{2.799006in}{2.191845in}}%
\pgfpathlineto{\pgfqpoint{2.799311in}{2.177298in}}%
\pgfpathlineto{\pgfqpoint{2.799922in}{2.136295in}}%
\pgfpathlineto{\pgfqpoint{2.800532in}{2.185439in}}%
\pgfpathlineto{\pgfqpoint{2.801447in}{2.347419in}}%
\pgfpathlineto{\pgfqpoint{2.802057in}{2.258070in}}%
\pgfpathlineto{\pgfqpoint{2.803278in}{2.037218in}}%
\pgfpathlineto{\pgfqpoint{2.803888in}{2.100080in}}%
\pgfpathlineto{\pgfqpoint{2.805109in}{2.389370in}}%
\pgfpathlineto{\pgfqpoint{2.806024in}{2.279503in}}%
\pgfpathlineto{\pgfqpoint{2.807855in}{2.080925in}}%
\pgfpathlineto{\pgfqpoint{2.808160in}{2.090715in}}%
\pgfpathlineto{\pgfqpoint{2.811516in}{2.455584in}}%
\pgfpathlineto{\pgfqpoint{2.811821in}{2.510241in}}%
\pgfpathlineto{\pgfqpoint{2.812126in}{2.463118in}}%
\pgfpathlineto{\pgfqpoint{2.813957in}{1.892105in}}%
\pgfpathlineto{\pgfqpoint{2.814567in}{2.012306in}}%
\pgfpathlineto{\pgfqpoint{2.815788in}{2.561003in}}%
\pgfpathlineto{\pgfqpoint{2.816398in}{2.270702in}}%
\pgfpathlineto{\pgfqpoint{2.817313in}{2.020713in}}%
\pgfpathlineto{\pgfqpoint{2.817924in}{2.164145in}}%
\pgfpathlineto{\pgfqpoint{2.818839in}{2.420785in}}%
\pgfpathlineto{\pgfqpoint{2.819449in}{2.285335in}}%
\pgfpathlineto{\pgfqpoint{2.820670in}{2.039358in}}%
\pgfpathlineto{\pgfqpoint{2.821280in}{2.096792in}}%
\pgfpathlineto{\pgfqpoint{2.822195in}{2.191303in}}%
\pgfpathlineto{\pgfqpoint{2.822806in}{2.146000in}}%
\pgfpathlineto{\pgfqpoint{2.823111in}{2.122620in}}%
\pgfpathlineto{\pgfqpoint{2.823416in}{2.137582in}}%
\pgfpathlineto{\pgfqpoint{2.824636in}{2.515977in}}%
\pgfpathlineto{\pgfqpoint{2.825246in}{2.297445in}}%
\pgfpathlineto{\pgfqpoint{2.826162in}{2.019830in}}%
\pgfpathlineto{\pgfqpoint{2.827077in}{2.144031in}}%
\pgfpathlineto{\pgfqpoint{2.827687in}{2.233966in}}%
\pgfpathlineto{\pgfqpoint{2.828298in}{2.183300in}}%
\pgfpathlineto{\pgfqpoint{2.828908in}{2.109307in}}%
\pgfpathlineto{\pgfqpoint{2.829518in}{2.163815in}}%
\pgfpathlineto{\pgfqpoint{2.830739in}{2.372822in}}%
\pgfpathlineto{\pgfqpoint{2.831349in}{2.262550in}}%
\pgfpathlineto{\pgfqpoint{2.832569in}{2.090694in}}%
\pgfpathlineto{\pgfqpoint{2.833180in}{2.111382in}}%
\pgfpathlineto{\pgfqpoint{2.835315in}{2.256176in}}%
\pgfpathlineto{\pgfqpoint{2.835926in}{2.380516in}}%
\pgfpathlineto{\pgfqpoint{2.836536in}{2.268691in}}%
\pgfpathlineto{\pgfqpoint{2.837451in}{2.054299in}}%
\pgfpathlineto{\pgfqpoint{2.838061in}{2.252025in}}%
\pgfpathlineto{\pgfqpoint{2.838672in}{2.489223in}}%
\pgfpathlineto{\pgfqpoint{2.839282in}{2.296572in}}%
\pgfpathlineto{\pgfqpoint{2.840502in}{1.879473in}}%
\pgfpathlineto{\pgfqpoint{2.841113in}{2.026385in}}%
\pgfpathlineto{\pgfqpoint{2.842028in}{2.557150in}}%
\pgfpathlineto{\pgfqpoint{2.842638in}{2.361179in}}%
\pgfpathlineto{\pgfqpoint{2.843554in}{1.965098in}}%
\pgfpathlineto{\pgfqpoint{2.844164in}{2.100644in}}%
\pgfpathlineto{\pgfqpoint{2.845079in}{2.485222in}}%
\pgfpathlineto{\pgfqpoint{2.845689in}{2.297105in}}%
\pgfpathlineto{\pgfqpoint{2.846910in}{1.938153in}}%
\pgfpathlineto{\pgfqpoint{2.847520in}{2.099410in}}%
\pgfpathlineto{\pgfqpoint{2.848435in}{2.451167in}}%
\pgfpathlineto{\pgfqpoint{2.849046in}{2.293763in}}%
\pgfpathlineto{\pgfqpoint{2.849656in}{2.122737in}}%
\pgfpathlineto{\pgfqpoint{2.850571in}{2.225506in}}%
\pgfpathlineto{\pgfqpoint{2.850876in}{2.255622in}}%
\pgfpathlineto{\pgfqpoint{2.851487in}{2.220706in}}%
\pgfpathlineto{\pgfqpoint{2.852707in}{2.143701in}}%
\pgfpathlineto{\pgfqpoint{2.853317in}{2.159345in}}%
\pgfpathlineto{\pgfqpoint{2.853928in}{2.170508in}}%
\pgfpathlineto{\pgfqpoint{2.854538in}{2.159377in}}%
\pgfpathlineto{\pgfqpoint{2.854843in}{2.152949in}}%
\pgfpathlineto{\pgfqpoint{2.855148in}{2.157344in}}%
\pgfpathlineto{\pgfqpoint{2.856674in}{2.328381in}}%
\pgfpathlineto{\pgfqpoint{2.857284in}{2.259219in}}%
\pgfpathlineto{\pgfqpoint{2.858809in}{2.105752in}}%
\pgfpathlineto{\pgfqpoint{2.859420in}{2.143467in}}%
\pgfpathlineto{\pgfqpoint{2.860945in}{2.332233in}}%
\pgfpathlineto{\pgfqpoint{2.861556in}{2.272500in}}%
\pgfpathlineto{\pgfqpoint{2.863081in}{1.981391in}}%
\pgfpathlineto{\pgfqpoint{2.863691in}{2.072592in}}%
\pgfpathlineto{\pgfqpoint{2.864912in}{2.620768in}}%
\pgfpathlineto{\pgfqpoint{2.865522in}{2.296179in}}%
\pgfpathlineto{\pgfqpoint{2.866743in}{1.928224in}}%
\pgfpathlineto{\pgfqpoint{2.867353in}{2.014168in}}%
\pgfpathlineto{\pgfqpoint{2.868878in}{2.511167in}}%
\pgfpathlineto{\pgfqpoint{2.869489in}{2.282632in}}%
\pgfpathlineto{\pgfqpoint{2.870404in}{1.971526in}}%
\pgfpathlineto{\pgfqpoint{2.871014in}{2.066611in}}%
\pgfpathlineto{\pgfqpoint{2.871930in}{2.420487in}}%
\pgfpathlineto{\pgfqpoint{2.872540in}{2.314046in}}%
\pgfpathlineto{\pgfqpoint{2.873760in}{1.974761in}}%
\pgfpathlineto{\pgfqpoint{2.874370in}{2.108668in}}%
\pgfpathlineto{\pgfqpoint{2.875286in}{2.528375in}}%
\pgfpathlineto{\pgfqpoint{2.875896in}{2.360541in}}%
\pgfpathlineto{\pgfqpoint{2.877117in}{1.891722in}}%
\pgfpathlineto{\pgfqpoint{2.877727in}{2.002505in}}%
\pgfpathlineto{\pgfqpoint{2.878947in}{2.642381in}}%
\pgfpathlineto{\pgfqpoint{2.879557in}{2.308236in}}%
\pgfpathlineto{\pgfqpoint{2.880778in}{1.921381in}}%
\pgfpathlineto{\pgfqpoint{2.881388in}{2.006144in}}%
\pgfpathlineto{\pgfqpoint{2.882914in}{2.406578in}}%
\pgfpathlineto{\pgfqpoint{2.883524in}{2.313961in}}%
\pgfpathlineto{\pgfqpoint{2.885660in}{2.074486in}}%
\pgfpathlineto{\pgfqpoint{2.885965in}{2.094887in}}%
\pgfpathlineto{\pgfqpoint{2.887185in}{2.370332in}}%
\pgfpathlineto{\pgfqpoint{2.887796in}{2.271777in}}%
\pgfpathlineto{\pgfqpoint{2.888711in}{2.077030in}}%
\pgfpathlineto{\pgfqpoint{2.889626in}{2.169114in}}%
\pgfpathlineto{\pgfqpoint{2.890542in}{2.226112in}}%
\pgfpathlineto{\pgfqpoint{2.891457in}{2.220451in}}%
\pgfpathlineto{\pgfqpoint{2.892678in}{2.275512in}}%
\pgfpathlineto{\pgfqpoint{2.893288in}{2.239117in}}%
\pgfpathlineto{\pgfqpoint{2.894203in}{2.119395in}}%
\pgfpathlineto{\pgfqpoint{2.894813in}{2.174308in}}%
\pgfpathlineto{\pgfqpoint{2.895729in}{2.305299in}}%
\pgfpathlineto{\pgfqpoint{2.896034in}{2.242895in}}%
\pgfpathlineto{\pgfqpoint{2.897254in}{1.920424in}}%
\pgfpathlineto{\pgfqpoint{2.897559in}{1.977954in}}%
\pgfpathlineto{\pgfqpoint{2.898780in}{2.763476in}}%
\pgfpathlineto{\pgfqpoint{2.899390in}{2.261273in}}%
\pgfpathlineto{\pgfqpoint{2.900611in}{1.772129in}}%
\pgfpathlineto{\pgfqpoint{2.901221in}{2.005687in}}%
\pgfpathlineto{\pgfqpoint{2.902136in}{2.878281in}}%
\pgfpathlineto{\pgfqpoint{2.902746in}{2.406110in}}%
\pgfpathlineto{\pgfqpoint{2.903967in}{1.766893in}}%
\pgfpathlineto{\pgfqpoint{2.904577in}{1.911814in}}%
\pgfpathlineto{\pgfqpoint{2.905798in}{2.742447in}}%
\pgfpathlineto{\pgfqpoint{2.906408in}{2.432012in}}%
\pgfpathlineto{\pgfqpoint{2.907628in}{1.977241in}}%
\pgfpathlineto{\pgfqpoint{2.908239in}{2.088917in}}%
\pgfpathlineto{\pgfqpoint{2.909154in}{2.281855in}}%
\pgfpathlineto{\pgfqpoint{2.909764in}{2.208191in}}%
\pgfpathlineto{\pgfqpoint{2.910680in}{2.111180in}}%
\pgfpathlineto{\pgfqpoint{2.911290in}{2.186195in}}%
\pgfpathlineto{\pgfqpoint{2.913731in}{2.403747in}}%
\pgfpathlineto{\pgfqpoint{2.914036in}{2.340087in}}%
\pgfpathlineto{\pgfqpoint{2.915561in}{1.950519in}}%
\pgfpathlineto{\pgfqpoint{2.916172in}{2.038804in}}%
\pgfpathlineto{\pgfqpoint{2.917392in}{2.460809in}}%
\pgfpathlineto{\pgfqpoint{2.918307in}{2.253984in}}%
\pgfpathlineto{\pgfqpoint{2.919223in}{2.030237in}}%
\pgfpathlineto{\pgfqpoint{2.919833in}{2.077477in}}%
\pgfpathlineto{\pgfqpoint{2.920748in}{2.346015in}}%
\pgfpathlineto{\pgfqpoint{2.921664in}{2.182470in}}%
\pgfpathlineto{\pgfqpoint{2.921969in}{2.125302in}}%
\pgfpathlineto{\pgfqpoint{2.922884in}{2.234477in}}%
\pgfpathlineto{\pgfqpoint{2.923189in}{2.264093in}}%
\pgfpathlineto{\pgfqpoint{2.923494in}{2.239926in}}%
\pgfpathlineto{\pgfqpoint{2.924410in}{2.055480in}}%
\pgfpathlineto{\pgfqpoint{2.925020in}{2.184205in}}%
\pgfpathlineto{\pgfqpoint{2.925935in}{2.518201in}}%
\pgfpathlineto{\pgfqpoint{2.926546in}{2.197858in}}%
\pgfpathlineto{\pgfqpoint{2.927461in}{1.843812in}}%
\pgfpathlineto{\pgfqpoint{2.928071in}{1.963672in}}%
\pgfpathlineto{\pgfqpoint{2.929292in}{2.663250in}}%
\pgfpathlineto{\pgfqpoint{2.929902in}{2.176713in}}%
\pgfpathlineto{\pgfqpoint{2.930817in}{1.803384in}}%
\pgfpathlineto{\pgfqpoint{2.931428in}{2.021277in}}%
\pgfpathlineto{\pgfqpoint{2.932343in}{2.800967in}}%
\pgfpathlineto{\pgfqpoint{2.932953in}{2.346930in}}%
\pgfpathlineto{\pgfqpoint{2.933869in}{1.857370in}}%
\pgfpathlineto{\pgfqpoint{2.934479in}{1.983956in}}%
\pgfpathlineto{\pgfqpoint{2.935699in}{2.420572in}}%
\pgfpathlineto{\pgfqpoint{2.936309in}{2.250376in}}%
\pgfpathlineto{\pgfqpoint{2.936920in}{2.150874in}}%
\pgfpathlineto{\pgfqpoint{2.937835in}{2.215183in}}%
\pgfpathlineto{\pgfqpoint{2.938140in}{2.221494in}}%
\pgfpathlineto{\pgfqpoint{2.938445in}{2.211309in}}%
\pgfpathlineto{\pgfqpoint{2.939666in}{2.134762in}}%
\pgfpathlineto{\pgfqpoint{2.940276in}{2.164879in}}%
\pgfpathlineto{\pgfqpoint{2.941191in}{2.252526in}}%
\pgfpathlineto{\pgfqpoint{2.941802in}{2.178511in}}%
\pgfpathlineto{\pgfqpoint{2.942412in}{2.113925in}}%
\pgfpathlineto{\pgfqpoint{2.942717in}{2.158547in}}%
\pgfpathlineto{\pgfqpoint{2.943632in}{2.480326in}}%
\pgfpathlineto{\pgfqpoint{2.944243in}{2.295168in}}%
\pgfpathlineto{\pgfqpoint{2.945158in}{1.926138in}}%
\pgfpathlineto{\pgfqpoint{2.945768in}{2.016765in}}%
\pgfpathlineto{\pgfqpoint{2.946683in}{2.459692in}}%
\pgfpathlineto{\pgfqpoint{2.947599in}{2.159217in}}%
\pgfpathlineto{\pgfqpoint{2.948209in}{1.948380in}}%
\pgfpathlineto{\pgfqpoint{2.948819in}{2.031685in}}%
\pgfpathlineto{\pgfqpoint{2.949735in}{2.443378in}}%
\pgfpathlineto{\pgfqpoint{2.950650in}{2.171338in}}%
\pgfpathlineto{\pgfqpoint{2.950955in}{2.093887in}}%
\pgfpathlineto{\pgfqpoint{2.951565in}{2.170657in}}%
\pgfpathlineto{\pgfqpoint{2.952176in}{2.310833in}}%
\pgfpathlineto{\pgfqpoint{2.952786in}{2.190100in}}%
\pgfpathlineto{\pgfqpoint{2.953701in}{2.014243in}}%
\pgfpathlineto{\pgfqpoint{2.954006in}{2.089853in}}%
\pgfpathlineto{\pgfqpoint{2.954922in}{2.506580in}}%
\pgfpathlineto{\pgfqpoint{2.955532in}{2.337405in}}%
\pgfpathlineto{\pgfqpoint{2.956752in}{1.931715in}}%
\pgfpathlineto{\pgfqpoint{2.957363in}{2.056672in}}%
\pgfpathlineto{\pgfqpoint{2.958583in}{2.489627in}}%
\pgfpathlineto{\pgfqpoint{2.959193in}{2.218642in}}%
\pgfpathlineto{\pgfqpoint{2.960109in}{1.983402in}}%
\pgfpathlineto{\pgfqpoint{2.960719in}{2.097069in}}%
\pgfpathlineto{\pgfqpoint{2.961634in}{2.358668in}}%
\pgfpathlineto{\pgfqpoint{2.962550in}{2.198646in}}%
\pgfpathlineto{\pgfqpoint{2.963465in}{2.075859in}}%
\pgfpathlineto{\pgfqpoint{2.964075in}{2.162484in}}%
\pgfpathlineto{\pgfqpoint{2.964991in}{2.385124in}}%
\pgfpathlineto{\pgfqpoint{2.965601in}{2.321719in}}%
\pgfpathlineto{\pgfqpoint{2.966516in}{2.124163in}}%
\pgfpathlineto{\pgfqpoint{2.967431in}{2.206446in}}%
\pgfpathlineto{\pgfqpoint{2.967737in}{2.234104in}}%
\pgfpathlineto{\pgfqpoint{2.968347in}{2.186546in}}%
\pgfpathlineto{\pgfqpoint{2.969262in}{2.094844in}}%
\pgfpathlineto{\pgfqpoint{2.969567in}{2.137604in}}%
\pgfpathlineto{\pgfqpoint{2.970483in}{2.311141in}}%
\pgfpathlineto{\pgfqpoint{2.971093in}{2.172286in}}%
\pgfpathlineto{\pgfqpoint{2.972008in}{1.987968in}}%
\pgfpathlineto{\pgfqpoint{2.972313in}{2.062770in}}%
\pgfpathlineto{\pgfqpoint{2.973229in}{2.600335in}}%
\pgfpathlineto{\pgfqpoint{2.973839in}{2.405450in}}%
\pgfpathlineto{\pgfqpoint{2.975059in}{1.928245in}}%
\pgfpathlineto{\pgfqpoint{2.975670in}{2.078818in}}%
\pgfpathlineto{\pgfqpoint{2.976585in}{2.408557in}}%
\pgfpathlineto{\pgfqpoint{2.977195in}{2.278098in}}%
\pgfpathlineto{\pgfqpoint{2.978111in}{2.020117in}}%
\pgfpathlineto{\pgfqpoint{2.978721in}{2.121311in}}%
\pgfpathlineto{\pgfqpoint{2.979636in}{2.409015in}}%
\pgfpathlineto{\pgfqpoint{2.980246in}{2.249695in}}%
\pgfpathlineto{\pgfqpoint{2.981162in}{2.037612in}}%
\pgfpathlineto{\pgfqpoint{2.981772in}{2.163570in}}%
\pgfpathlineto{\pgfqpoint{2.982382in}{2.271192in}}%
\pgfpathlineto{\pgfqpoint{2.982993in}{2.166315in}}%
\pgfpathlineto{\pgfqpoint{2.983603in}{2.064589in}}%
\pgfpathlineto{\pgfqpoint{2.984213in}{2.230337in}}%
\pgfpathlineto{\pgfqpoint{2.984823in}{2.528641in}}%
\pgfpathlineto{\pgfqpoint{2.985433in}{2.374769in}}%
\pgfpathlineto{\pgfqpoint{2.986654in}{1.891648in}}%
\pgfpathlineto{\pgfqpoint{2.987264in}{2.098942in}}%
\pgfpathlineto{\pgfqpoint{2.988180in}{2.667847in}}%
\pgfpathlineto{\pgfqpoint{2.988790in}{2.327455in}}%
\pgfpathlineto{\pgfqpoint{2.990010in}{1.818368in}}%
\pgfpathlineto{\pgfqpoint{2.990620in}{1.952285in}}%
\pgfpathlineto{\pgfqpoint{2.991841in}{2.698837in}}%
\pgfpathlineto{\pgfqpoint{2.992451in}{2.269659in}}%
\pgfpathlineto{\pgfqpoint{2.993367in}{1.922775in}}%
\pgfpathlineto{\pgfqpoint{2.993977in}{2.018042in}}%
\pgfpathlineto{\pgfqpoint{2.995502in}{2.405918in}}%
\pgfpathlineto{\pgfqpoint{2.996113in}{2.318814in}}%
\pgfpathlineto{\pgfqpoint{2.997638in}{2.065483in}}%
\pgfpathlineto{\pgfqpoint{2.998248in}{2.125961in}}%
\pgfpathlineto{\pgfqpoint{2.999469in}{2.349590in}}%
\pgfpathlineto{\pgfqpoint{3.000079in}{2.307576in}}%
\pgfpathlineto{\pgfqpoint{3.001605in}{2.057161in}}%
\pgfpathlineto{\pgfqpoint{3.002215in}{2.152875in}}%
\pgfpathlineto{\pgfqpoint{3.003130in}{2.310322in}}%
\pgfpathlineto{\pgfqpoint{3.003741in}{2.189792in}}%
\pgfpathlineto{\pgfqpoint{3.004656in}{1.989787in}}%
\pgfpathlineto{\pgfqpoint{3.005266in}{2.090641in}}%
\pgfpathlineto{\pgfqpoint{3.006487in}{2.556459in}}%
\pgfpathlineto{\pgfqpoint{3.007097in}{2.310162in}}%
\pgfpathlineto{\pgfqpoint{3.008317in}{2.002462in}}%
\pgfpathlineto{\pgfqpoint{3.008928in}{2.058364in}}%
\pgfpathlineto{\pgfqpoint{3.010148in}{2.269138in}}%
\pgfpathlineto{\pgfqpoint{3.010758in}{2.215119in}}%
\pgfpathlineto{\pgfqpoint{3.011369in}{2.173658in}}%
\pgfpathlineto{\pgfqpoint{3.011979in}{2.225314in}}%
\pgfpathlineto{\pgfqpoint{3.012589in}{2.270085in}}%
\pgfpathlineto{\pgfqpoint{3.013199in}{2.209266in}}%
\pgfpathlineto{\pgfqpoint{3.013809in}{2.144297in}}%
\pgfpathlineto{\pgfqpoint{3.014420in}{2.223218in}}%
\pgfpathlineto{\pgfqpoint{3.015030in}{2.339406in}}%
\pgfpathlineto{\pgfqpoint{3.015640in}{2.236563in}}%
\pgfpathlineto{\pgfqpoint{3.016556in}{1.991118in}}%
\pgfpathlineto{\pgfqpoint{3.017166in}{2.057587in}}%
\pgfpathlineto{\pgfqpoint{3.018386in}{2.501184in}}%
\pgfpathlineto{\pgfqpoint{3.018996in}{2.241288in}}%
\pgfpathlineto{\pgfqpoint{3.019912in}{1.918210in}}%
\pgfpathlineto{\pgfqpoint{3.020522in}{2.030897in}}%
\pgfpathlineto{\pgfqpoint{3.021743in}{2.606912in}}%
\pgfpathlineto{\pgfqpoint{3.022353in}{2.205020in}}%
\pgfpathlineto{\pgfqpoint{3.023268in}{1.846452in}}%
\pgfpathlineto{\pgfqpoint{3.023878in}{1.943112in}}%
\pgfpathlineto{\pgfqpoint{3.025099in}{2.627195in}}%
\pgfpathlineto{\pgfqpoint{3.025709in}{2.321198in}}%
\pgfpathlineto{\pgfqpoint{3.026624in}{1.915826in}}%
\pgfpathlineto{\pgfqpoint{3.027235in}{2.021990in}}%
\pgfpathlineto{\pgfqpoint{3.028455in}{2.493458in}}%
\pgfpathlineto{\pgfqpoint{3.029065in}{2.233040in}}%
\pgfpathlineto{\pgfqpoint{3.029981in}{2.028641in}}%
\pgfpathlineto{\pgfqpoint{3.030591in}{2.085905in}}%
\pgfpathlineto{\pgfqpoint{3.032117in}{2.396074in}}%
\pgfpathlineto{\pgfqpoint{3.032727in}{2.319165in}}%
\pgfpathlineto{\pgfqpoint{3.033947in}{1.942186in}}%
\pgfpathlineto{\pgfqpoint{3.034557in}{1.997226in}}%
\pgfpathlineto{\pgfqpoint{3.035778in}{2.529183in}}%
\pgfpathlineto{\pgfqpoint{3.036693in}{2.226900in}}%
\pgfpathlineto{\pgfqpoint{3.037304in}{2.112521in}}%
\pgfpathlineto{\pgfqpoint{3.038219in}{2.220398in}}%
\pgfpathlineto{\pgfqpoint{3.038524in}{2.218897in}}%
\pgfpathlineto{\pgfqpoint{3.039744in}{2.059098in}}%
\pgfpathlineto{\pgfqpoint{3.040355in}{2.141222in}}%
\pgfpathlineto{\pgfqpoint{3.041270in}{2.263423in}}%
\pgfpathlineto{\pgfqpoint{3.042185in}{2.226581in}}%
\pgfpathlineto{\pgfqpoint{3.043101in}{2.322241in}}%
\pgfpathlineto{\pgfqpoint{3.043711in}{2.246981in}}%
\pgfpathlineto{\pgfqpoint{3.044931in}{2.021554in}}%
\pgfpathlineto{\pgfqpoint{3.045542in}{2.110786in}}%
\pgfpathlineto{\pgfqpoint{3.046762in}{2.423211in}}%
\pgfpathlineto{\pgfqpoint{3.047372in}{2.299286in}}%
\pgfpathlineto{\pgfqpoint{3.048593in}{2.015732in}}%
\pgfpathlineto{\pgfqpoint{3.049203in}{2.046466in}}%
\pgfpathlineto{\pgfqpoint{3.051034in}{2.426276in}}%
\pgfpathlineto{\pgfqpoint{3.051644in}{2.319623in}}%
\pgfpathlineto{\pgfqpoint{3.053170in}{1.916263in}}%
\pgfpathlineto{\pgfqpoint{3.053780in}{1.978656in}}%
\pgfpathlineto{\pgfqpoint{3.055000in}{2.748790in}}%
\pgfpathlineto{\pgfqpoint{3.055611in}{2.409696in}}%
\pgfpathlineto{\pgfqpoint{3.056831in}{1.853518in}}%
\pgfpathlineto{\pgfqpoint{3.057441in}{1.934811in}}%
\pgfpathlineto{\pgfqpoint{3.058662in}{2.546828in}}%
\pgfpathlineto{\pgfqpoint{3.059577in}{2.195474in}}%
\pgfpathlineto{\pgfqpoint{3.060493in}{1.945709in}}%
\pgfpathlineto{\pgfqpoint{3.061103in}{2.189451in}}%
\pgfpathlineto{\pgfqpoint{3.062018in}{2.560726in}}%
\pgfpathlineto{\pgfqpoint{3.062628in}{2.208585in}}%
\pgfpathlineto{\pgfqpoint{3.063544in}{1.910793in}}%
\pgfpathlineto{\pgfqpoint{3.064154in}{2.012582in}}%
\pgfpathlineto{\pgfqpoint{3.065374in}{2.494874in}}%
\pgfpathlineto{\pgfqpoint{3.065985in}{2.322496in}}%
\pgfpathlineto{\pgfqpoint{3.067205in}{2.062536in}}%
\pgfpathlineto{\pgfqpoint{3.067815in}{2.123578in}}%
\pgfpathlineto{\pgfqpoint{3.070867in}{2.426297in}}%
\pgfpathlineto{\pgfqpoint{3.072392in}{1.933886in}}%
\pgfpathlineto{\pgfqpoint{3.073002in}{2.063440in}}%
\pgfpathlineto{\pgfqpoint{3.073918in}{2.514955in}}%
\pgfpathlineto{\pgfqpoint{3.074833in}{2.221983in}}%
\pgfpathlineto{\pgfqpoint{3.075748in}{1.959128in}}%
\pgfpathlineto{\pgfqpoint{3.076359in}{2.052383in}}%
\pgfpathlineto{\pgfqpoint{3.077579in}{2.443282in}}%
\pgfpathlineto{\pgfqpoint{3.078189in}{2.240639in}}%
\pgfpathlineto{\pgfqpoint{3.079105in}{2.029918in}}%
\pgfpathlineto{\pgfqpoint{3.079715in}{2.178202in}}%
\pgfpathlineto{\pgfqpoint{3.080630in}{2.408717in}}%
\pgfpathlineto{\pgfqpoint{3.081241in}{2.212512in}}%
\pgfpathlineto{\pgfqpoint{3.082461in}{1.938706in}}%
\pgfpathlineto{\pgfqpoint{3.083071in}{2.018052in}}%
\pgfpathlineto{\pgfqpoint{3.084597in}{2.625429in}}%
\pgfpathlineto{\pgfqpoint{3.085207in}{2.322688in}}%
\pgfpathlineto{\pgfqpoint{3.086428in}{1.975283in}}%
\pgfpathlineto{\pgfqpoint{3.087038in}{2.037687in}}%
\pgfpathlineto{\pgfqpoint{3.088258in}{2.434704in}}%
\pgfpathlineto{\pgfqpoint{3.088869in}{2.331393in}}%
\pgfpathlineto{\pgfqpoint{3.090089in}{1.900842in}}%
\pgfpathlineto{\pgfqpoint{3.090699in}{1.975027in}}%
\pgfpathlineto{\pgfqpoint{3.091920in}{2.631122in}}%
\pgfpathlineto{\pgfqpoint{3.092530in}{2.341715in}}%
\pgfpathlineto{\pgfqpoint{3.093445in}{1.987095in}}%
\pgfpathlineto{\pgfqpoint{3.094056in}{2.090769in}}%
\pgfpathlineto{\pgfqpoint{3.094971in}{2.392531in}}%
\pgfpathlineto{\pgfqpoint{3.095886in}{2.208415in}}%
\pgfpathlineto{\pgfqpoint{3.097107in}{1.987436in}}%
\pgfpathlineto{\pgfqpoint{3.097717in}{2.091865in}}%
\pgfpathlineto{\pgfqpoint{3.098937in}{2.419167in}}%
\pgfpathlineto{\pgfqpoint{3.099548in}{2.239670in}}%
\pgfpathlineto{\pgfqpoint{3.100158in}{2.134219in}}%
\pgfpathlineto{\pgfqpoint{3.101073in}{2.232104in}}%
\pgfpathlineto{\pgfqpoint{3.101683in}{2.169721in}}%
\pgfpathlineto{\pgfqpoint{3.102599in}{2.056991in}}%
\pgfpathlineto{\pgfqpoint{3.102904in}{2.105614in}}%
\pgfpathlineto{\pgfqpoint{3.104124in}{2.445942in}}%
\pgfpathlineto{\pgfqpoint{3.104735in}{2.214981in}}%
\pgfpathlineto{\pgfqpoint{3.105650in}{1.967354in}}%
\pgfpathlineto{\pgfqpoint{3.106260in}{2.096079in}}%
\pgfpathlineto{\pgfqpoint{3.107176in}{2.467418in}}%
\pgfpathlineto{\pgfqpoint{3.107786in}{2.299403in}}%
\pgfpathlineto{\pgfqpoint{3.109006in}{1.908537in}}%
\pgfpathlineto{\pgfqpoint{3.109617in}{2.095185in}}%
\pgfpathlineto{\pgfqpoint{3.110532in}{2.710074in}}%
\pgfpathlineto{\pgfqpoint{3.111142in}{2.384294in}}%
\pgfpathlineto{\pgfqpoint{3.112363in}{1.822465in}}%
\pgfpathlineto{\pgfqpoint{3.112973in}{1.926404in}}%
\pgfpathlineto{\pgfqpoint{3.114193in}{2.706350in}}%
\pgfpathlineto{\pgfqpoint{3.114804in}{2.448273in}}%
\pgfpathlineto{\pgfqpoint{3.116024in}{1.919115in}}%
\pgfpathlineto{\pgfqpoint{3.116634in}{1.956691in}}%
\pgfpathlineto{\pgfqpoint{3.118465in}{2.549967in}}%
\pgfpathlineto{\pgfqpoint{3.119075in}{2.292731in}}%
\pgfpathlineto{\pgfqpoint{3.120296in}{1.889370in}}%
\pgfpathlineto{\pgfqpoint{3.120906in}{1.971760in}}%
\pgfpathlineto{\pgfqpoint{3.122126in}{2.583744in}}%
\pgfpathlineto{\pgfqpoint{3.122737in}{2.283419in}}%
\pgfpathlineto{\pgfqpoint{3.123652in}{1.925212in}}%
\pgfpathlineto{\pgfqpoint{3.124262in}{2.179171in}}%
\pgfpathlineto{\pgfqpoint{3.125178in}{2.615510in}}%
\pgfpathlineto{\pgfqpoint{3.125788in}{2.186439in}}%
\pgfpathlineto{\pgfqpoint{3.126703in}{1.819251in}}%
\pgfpathlineto{\pgfqpoint{3.127313in}{1.905408in}}%
\pgfpathlineto{\pgfqpoint{3.128534in}{2.725420in}}%
\pgfpathlineto{\pgfqpoint{3.129144in}{2.412048in}}%
\pgfpathlineto{\pgfqpoint{3.130365in}{1.875930in}}%
\pgfpathlineto{\pgfqpoint{3.130975in}{1.978379in}}%
\pgfpathlineto{\pgfqpoint{3.132195in}{2.502376in}}%
\pgfpathlineto{\pgfqpoint{3.132806in}{2.343737in}}%
\pgfpathlineto{\pgfqpoint{3.133721in}{2.065047in}}%
\pgfpathlineto{\pgfqpoint{3.134331in}{2.147405in}}%
\pgfpathlineto{\pgfqpoint{3.135246in}{2.324337in}}%
\pgfpathlineto{\pgfqpoint{3.135857in}{2.225644in}}%
\pgfpathlineto{\pgfqpoint{3.136772in}{2.092131in}}%
\pgfpathlineto{\pgfqpoint{3.137687in}{2.147022in}}%
\pgfpathlineto{\pgfqpoint{3.139518in}{2.294263in}}%
\pgfpathlineto{\pgfqpoint{3.140128in}{2.256080in}}%
\pgfpathlineto{\pgfqpoint{3.141349in}{2.047594in}}%
\pgfpathlineto{\pgfqpoint{3.141959in}{2.135081in}}%
\pgfpathlineto{\pgfqpoint{3.143180in}{2.440994in}}%
\pgfpathlineto{\pgfqpoint{3.143790in}{2.260539in}}%
\pgfpathlineto{\pgfqpoint{3.145010in}{1.946879in}}%
\pgfpathlineto{\pgfqpoint{3.145620in}{2.008900in}}%
\pgfpathlineto{\pgfqpoint{3.146841in}{2.513061in}}%
\pgfpathlineto{\pgfqpoint{3.147756in}{2.245449in}}%
\pgfpathlineto{\pgfqpoint{3.148367in}{2.128154in}}%
\pgfpathlineto{\pgfqpoint{3.149282in}{2.178469in}}%
\pgfpathlineto{\pgfqpoint{3.149587in}{2.175659in}}%
\pgfpathlineto{\pgfqpoint{3.150502in}{2.115820in}}%
\pgfpathlineto{\pgfqpoint{3.151113in}{2.138902in}}%
\pgfpathlineto{\pgfqpoint{3.152028in}{2.198922in}}%
\pgfpathlineto{\pgfqpoint{3.152943in}{2.179884in}}%
\pgfpathlineto{\pgfqpoint{3.153248in}{2.179203in}}%
\pgfpathlineto{\pgfqpoint{3.154164in}{2.261944in}}%
\pgfpathlineto{\pgfqpoint{3.155079in}{2.358860in}}%
\pgfpathlineto{\pgfqpoint{3.155689in}{2.261539in}}%
\pgfpathlineto{\pgfqpoint{3.156910in}{1.984030in}}%
\pgfpathlineto{\pgfqpoint{3.157520in}{2.051447in}}%
\pgfpathlineto{\pgfqpoint{3.158741in}{2.545774in}}%
\pgfpathlineto{\pgfqpoint{3.159351in}{2.229614in}}%
\pgfpathlineto{\pgfqpoint{3.160266in}{1.860722in}}%
\pgfpathlineto{\pgfqpoint{3.160876in}{1.962310in}}%
\pgfpathlineto{\pgfqpoint{3.162097in}{2.711149in}}%
\pgfpathlineto{\pgfqpoint{3.162707in}{2.259720in}}%
\pgfpathlineto{\pgfqpoint{3.163622in}{1.829105in}}%
\pgfpathlineto{\pgfqpoint{3.164233in}{1.960767in}}%
\pgfpathlineto{\pgfqpoint{3.165453in}{2.683800in}}%
\pgfpathlineto{\pgfqpoint{3.166063in}{2.199167in}}%
\pgfpathlineto{\pgfqpoint{3.166979in}{1.785995in}}%
\pgfpathlineto{\pgfqpoint{3.167589in}{1.877728in}}%
\pgfpathlineto{\pgfqpoint{3.168809in}{2.879621in}}%
\pgfpathlineto{\pgfqpoint{3.169420in}{2.376493in}}%
\pgfpathlineto{\pgfqpoint{3.170640in}{1.840194in}}%
\pgfpathlineto{\pgfqpoint{3.171250in}{2.004069in}}%
\pgfpathlineto{\pgfqpoint{3.172471in}{2.584127in}}%
\pgfpathlineto{\pgfqpoint{3.173081in}{2.331723in}}%
\pgfpathlineto{\pgfqpoint{3.174912in}{2.014083in}}%
\pgfpathlineto{\pgfqpoint{3.175217in}{2.017520in}}%
\pgfpathlineto{\pgfqpoint{3.175827in}{2.064334in}}%
\pgfpathlineto{\pgfqpoint{3.177048in}{2.537750in}}%
\pgfpathlineto{\pgfqpoint{3.177963in}{2.233466in}}%
\pgfpathlineto{\pgfqpoint{3.178573in}{2.033398in}}%
\pgfpathlineto{\pgfqpoint{3.179489in}{2.175936in}}%
\pgfpathlineto{\pgfqpoint{3.180099in}{2.270393in}}%
\pgfpathlineto{\pgfqpoint{3.180709in}{2.184024in}}%
\pgfpathlineto{\pgfqpoint{3.181319in}{2.098111in}}%
\pgfpathlineto{\pgfqpoint{3.181930in}{2.151459in}}%
\pgfpathlineto{\pgfqpoint{3.182845in}{2.294848in}}%
\pgfpathlineto{\pgfqpoint{3.183760in}{2.224527in}}%
\pgfpathlineto{\pgfqpoint{3.184676in}{2.179320in}}%
\pgfpathlineto{\pgfqpoint{3.185286in}{2.193655in}}%
\pgfpathlineto{\pgfqpoint{3.185896in}{2.203051in}}%
\pgfpathlineto{\pgfqpoint{3.186201in}{2.189291in}}%
\pgfpathlineto{\pgfqpoint{3.187117in}{2.127228in}}%
\pgfpathlineto{\pgfqpoint{3.187727in}{2.185290in}}%
\pgfpathlineto{\pgfqpoint{3.188642in}{2.313578in}}%
\pgfpathlineto{\pgfqpoint{3.188947in}{2.263263in}}%
\pgfpathlineto{\pgfqpoint{3.190168in}{1.987074in}}%
\pgfpathlineto{\pgfqpoint{3.190778in}{2.093237in}}%
\pgfpathlineto{\pgfqpoint{3.191998in}{2.566462in}}%
\pgfpathlineto{\pgfqpoint{3.192609in}{2.371215in}}%
\pgfpathlineto{\pgfqpoint{3.194134in}{2.006165in}}%
\pgfpathlineto{\pgfqpoint{3.194744in}{2.054458in}}%
\pgfpathlineto{\pgfqpoint{3.195965in}{2.296338in}}%
\pgfpathlineto{\pgfqpoint{3.196575in}{2.185035in}}%
\pgfpathlineto{\pgfqpoint{3.197185in}{2.066707in}}%
\pgfpathlineto{\pgfqpoint{3.197796in}{2.180118in}}%
\pgfpathlineto{\pgfqpoint{3.198711in}{2.507080in}}%
\pgfpathlineto{\pgfqpoint{3.199321in}{2.258336in}}%
\pgfpathlineto{\pgfqpoint{3.200542in}{1.900619in}}%
\pgfpathlineto{\pgfqpoint{3.201152in}{2.068027in}}%
\pgfpathlineto{\pgfqpoint{3.202067in}{2.575912in}}%
\pgfpathlineto{\pgfqpoint{3.202678in}{2.393297in}}%
\pgfpathlineto{\pgfqpoint{3.203898in}{1.952605in}}%
\pgfpathlineto{\pgfqpoint{3.204508in}{2.060173in}}%
\pgfpathlineto{\pgfqpoint{3.205729in}{2.468471in}}%
\pgfpathlineto{\pgfqpoint{3.206339in}{2.325678in}}%
\pgfpathlineto{\pgfqpoint{3.207559in}{2.041912in}}%
\pgfpathlineto{\pgfqpoint{3.208170in}{2.078860in}}%
\pgfpathlineto{\pgfqpoint{3.209695in}{2.274895in}}%
\pgfpathlineto{\pgfqpoint{3.210611in}{2.206169in}}%
\pgfpathlineto{\pgfqpoint{3.211526in}{2.127494in}}%
\pgfpathlineto{\pgfqpoint{3.212136in}{2.199050in}}%
\pgfpathlineto{\pgfqpoint{3.213052in}{2.355231in}}%
\pgfpathlineto{\pgfqpoint{3.213662in}{2.272171in}}%
\pgfpathlineto{\pgfqpoint{3.214882in}{2.027098in}}%
\pgfpathlineto{\pgfqpoint{3.215493in}{2.076604in}}%
\pgfpathlineto{\pgfqpoint{3.216713in}{2.384847in}}%
\pgfpathlineto{\pgfqpoint{3.217323in}{2.300393in}}%
\pgfpathlineto{\pgfqpoint{3.218239in}{2.106551in}}%
\pgfpathlineto{\pgfqpoint{3.219154in}{2.185705in}}%
\pgfpathlineto{\pgfqpoint{3.219764in}{2.223090in}}%
\pgfpathlineto{\pgfqpoint{3.220374in}{2.186897in}}%
\pgfpathlineto{\pgfqpoint{3.220680in}{2.178809in}}%
\pgfpathlineto{\pgfqpoint{3.220985in}{2.188355in}}%
\pgfpathlineto{\pgfqpoint{3.223120in}{2.248716in}}%
\pgfpathlineto{\pgfqpoint{3.223426in}{2.248045in}}%
\pgfpathlineto{\pgfqpoint{3.224341in}{2.168774in}}%
\pgfpathlineto{\pgfqpoint{3.225867in}{2.021362in}}%
\pgfpathlineto{\pgfqpoint{3.226172in}{2.047594in}}%
\pgfpathlineto{\pgfqpoint{3.227697in}{2.531791in}}%
\pgfpathlineto{\pgfqpoint{3.228613in}{2.273043in}}%
\pgfpathlineto{\pgfqpoint{3.229833in}{2.012710in}}%
\pgfpathlineto{\pgfqpoint{3.230443in}{2.044742in}}%
\pgfpathlineto{\pgfqpoint{3.231969in}{2.360903in}}%
\pgfpathlineto{\pgfqpoint{3.232884in}{2.222622in}}%
\pgfpathlineto{\pgfqpoint{3.233800in}{2.072699in}}%
\pgfpathlineto{\pgfqpoint{3.234410in}{2.139125in}}%
\pgfpathlineto{\pgfqpoint{3.235325in}{2.328945in}}%
\pgfpathlineto{\pgfqpoint{3.236241in}{2.208234in}}%
\pgfpathlineto{\pgfqpoint{3.237156in}{2.087321in}}%
\pgfpathlineto{\pgfqpoint{3.237766in}{2.134837in}}%
\pgfpathlineto{\pgfqpoint{3.238987in}{2.255292in}}%
\pgfpathlineto{\pgfqpoint{3.239597in}{2.186993in}}%
\pgfpathlineto{\pgfqpoint{3.240512in}{2.102155in}}%
\pgfpathlineto{\pgfqpoint{3.240817in}{2.134560in}}%
\pgfpathlineto{\pgfqpoint{3.242038in}{2.447900in}}%
\pgfpathlineto{\pgfqpoint{3.242648in}{2.351357in}}%
\pgfpathlineto{\pgfqpoint{3.244174in}{2.017190in}}%
\pgfpathlineto{\pgfqpoint{3.244784in}{2.055214in}}%
\pgfpathlineto{\pgfqpoint{3.246309in}{2.381878in}}%
\pgfpathlineto{\pgfqpoint{3.246920in}{2.206776in}}%
\pgfpathlineto{\pgfqpoint{3.247835in}{1.974750in}}%
\pgfpathlineto{\pgfqpoint{3.248445in}{2.026513in}}%
\pgfpathlineto{\pgfqpoint{3.249666in}{2.554607in}}%
\pgfpathlineto{\pgfqpoint{3.250276in}{2.327285in}}%
\pgfpathlineto{\pgfqpoint{3.251496in}{1.915805in}}%
\pgfpathlineto{\pgfqpoint{3.252107in}{2.064781in}}%
\pgfpathlineto{\pgfqpoint{3.253327in}{2.600548in}}%
\pgfpathlineto{\pgfqpoint{3.253937in}{2.291826in}}%
\pgfpathlineto{\pgfqpoint{3.255158in}{1.944357in}}%
\pgfpathlineto{\pgfqpoint{3.255768in}{2.049999in}}%
\pgfpathlineto{\pgfqpoint{3.256989in}{2.392967in}}%
\pgfpathlineto{\pgfqpoint{3.257599in}{2.268052in}}%
\pgfpathlineto{\pgfqpoint{3.258514in}{2.095142in}}%
\pgfpathlineto{\pgfqpoint{3.259124in}{2.147564in}}%
\pgfpathlineto{\pgfqpoint{3.260040in}{2.286069in}}%
\pgfpathlineto{\pgfqpoint{3.260650in}{2.228145in}}%
\pgfpathlineto{\pgfqpoint{3.261870in}{2.047052in}}%
\pgfpathlineto{\pgfqpoint{3.262481in}{2.124046in}}%
\pgfpathlineto{\pgfqpoint{3.263701in}{2.422860in}}%
\pgfpathlineto{\pgfqpoint{3.264311in}{2.319069in}}%
\pgfpathlineto{\pgfqpoint{3.265532in}{2.074199in}}%
\pgfpathlineto{\pgfqpoint{3.266142in}{2.134411in}}%
\pgfpathlineto{\pgfqpoint{3.267363in}{2.341364in}}%
\pgfpathlineto{\pgfqpoint{3.267973in}{2.214619in}}%
\pgfpathlineto{\pgfqpoint{3.269193in}{1.988287in}}%
\pgfpathlineto{\pgfqpoint{3.269804in}{2.054171in}}%
\pgfpathlineto{\pgfqpoint{3.271329in}{2.522458in}}%
\pgfpathlineto{\pgfqpoint{3.271939in}{2.296594in}}%
\pgfpathlineto{\pgfqpoint{3.273465in}{1.928937in}}%
\pgfpathlineto{\pgfqpoint{3.274075in}{1.983338in}}%
\pgfpathlineto{\pgfqpoint{3.275601in}{2.633878in}}%
\pgfpathlineto{\pgfqpoint{3.276516in}{2.244778in}}%
\pgfpathlineto{\pgfqpoint{3.277431in}{1.960171in}}%
\pgfpathlineto{\pgfqpoint{3.278347in}{2.082255in}}%
\pgfpathlineto{\pgfqpoint{3.279567in}{2.379484in}}%
\pgfpathlineto{\pgfqpoint{3.280178in}{2.277194in}}%
\pgfpathlineto{\pgfqpoint{3.281398in}{1.994459in}}%
\pgfpathlineto{\pgfqpoint{3.282008in}{2.067324in}}%
\pgfpathlineto{\pgfqpoint{3.283229in}{2.463714in}}%
\pgfpathlineto{\pgfqpoint{3.283839in}{2.224878in}}%
\pgfpathlineto{\pgfqpoint{3.284449in}{2.023256in}}%
\pgfpathlineto{\pgfqpoint{3.285365in}{2.226804in}}%
\pgfpathlineto{\pgfqpoint{3.285975in}{2.447017in}}%
\pgfpathlineto{\pgfqpoint{3.286585in}{2.333351in}}%
\pgfpathlineto{\pgfqpoint{3.288111in}{1.936769in}}%
\pgfpathlineto{\pgfqpoint{3.288721in}{1.987106in}}%
\pgfpathlineto{\pgfqpoint{3.290552in}{2.699922in}}%
\pgfpathlineto{\pgfqpoint{3.291162in}{2.366766in}}%
\pgfpathlineto{\pgfqpoint{3.292687in}{1.876909in}}%
\pgfpathlineto{\pgfqpoint{3.293298in}{1.928267in}}%
\pgfpathlineto{\pgfqpoint{3.294823in}{2.601282in}}%
\pgfpathlineto{\pgfqpoint{3.295739in}{2.210618in}}%
\pgfpathlineto{\pgfqpoint{3.296654in}{1.937631in}}%
\pgfpathlineto{\pgfqpoint{3.297264in}{2.075529in}}%
\pgfpathlineto{\pgfqpoint{3.298180in}{2.556618in}}%
\pgfpathlineto{\pgfqpoint{3.298790in}{2.368916in}}%
\pgfpathlineto{\pgfqpoint{3.300010in}{1.910303in}}%
\pgfpathlineto{\pgfqpoint{3.300620in}{2.054512in}}%
\pgfpathlineto{\pgfqpoint{3.301536in}{2.473260in}}%
\pgfpathlineto{\pgfqpoint{3.302146in}{2.335841in}}%
\pgfpathlineto{\pgfqpoint{3.303367in}{1.939270in}}%
\pgfpathlineto{\pgfqpoint{3.303977in}{2.078094in}}%
\pgfpathlineto{\pgfqpoint{3.304892in}{2.592439in}}%
\pgfpathlineto{\pgfqpoint{3.305502in}{2.438184in}}%
\pgfpathlineto{\pgfqpoint{3.306723in}{1.922116in}}%
\pgfpathlineto{\pgfqpoint{3.307333in}{1.997056in}}%
\pgfpathlineto{\pgfqpoint{3.308554in}{2.428373in}}%
\pgfpathlineto{\pgfqpoint{3.309164in}{2.274054in}}%
\pgfpathlineto{\pgfqpoint{3.310079in}{2.053032in}}%
\pgfpathlineto{\pgfqpoint{3.310689in}{2.149001in}}%
\pgfpathlineto{\pgfqpoint{3.311605in}{2.332989in}}%
\pgfpathlineto{\pgfqpoint{3.312215in}{2.266381in}}%
\pgfpathlineto{\pgfqpoint{3.313130in}{2.174904in}}%
\pgfpathlineto{\pgfqpoint{3.314046in}{2.196539in}}%
\pgfpathlineto{\pgfqpoint{3.315266in}{2.062961in}}%
\pgfpathlineto{\pgfqpoint{3.315876in}{2.139689in}}%
\pgfpathlineto{\pgfqpoint{3.316792in}{2.492671in}}%
\pgfpathlineto{\pgfqpoint{3.317402in}{2.337097in}}%
\pgfpathlineto{\pgfqpoint{3.318622in}{1.940952in}}%
\pgfpathlineto{\pgfqpoint{3.319233in}{2.055991in}}%
\pgfpathlineto{\pgfqpoint{3.320453in}{2.474250in}}%
\pgfpathlineto{\pgfqpoint{3.321063in}{2.261890in}}%
\pgfpathlineto{\pgfqpoint{3.321979in}{1.984488in}}%
\pgfpathlineto{\pgfqpoint{3.322589in}{2.064653in}}%
\pgfpathlineto{\pgfqpoint{3.323809in}{2.449720in}}%
\pgfpathlineto{\pgfqpoint{3.324420in}{2.194453in}}%
\pgfpathlineto{\pgfqpoint{3.325335in}{1.967886in}}%
\pgfpathlineto{\pgfqpoint{3.325945in}{2.138402in}}%
\pgfpathlineto{\pgfqpoint{3.326861in}{2.428851in}}%
\pgfpathlineto{\pgfqpoint{3.327471in}{2.254952in}}%
\pgfpathlineto{\pgfqpoint{3.328386in}{2.108966in}}%
\pgfpathlineto{\pgfqpoint{3.328996in}{2.166528in}}%
\pgfpathlineto{\pgfqpoint{3.330217in}{2.227262in}}%
\pgfpathlineto{\pgfqpoint{3.330827in}{2.204041in}}%
\pgfpathlineto{\pgfqpoint{3.332353in}{2.094057in}}%
\pgfpathlineto{\pgfqpoint{3.332658in}{2.118938in}}%
\pgfpathlineto{\pgfqpoint{3.334183in}{2.399597in}}%
\pgfpathlineto{\pgfqpoint{3.334794in}{2.271894in}}%
\pgfpathlineto{\pgfqpoint{3.336014in}{2.036431in}}%
\pgfpathlineto{\pgfqpoint{3.336624in}{2.122779in}}%
\pgfpathlineto{\pgfqpoint{3.337845in}{2.362616in}}%
\pgfpathlineto{\pgfqpoint{3.338455in}{2.269148in}}%
\pgfpathlineto{\pgfqpoint{3.339981in}{2.031738in}}%
\pgfpathlineto{\pgfqpoint{3.340591in}{2.079648in}}%
\pgfpathlineto{\pgfqpoint{3.341811in}{2.455009in}}%
\pgfpathlineto{\pgfqpoint{3.342422in}{2.354635in}}%
\pgfpathlineto{\pgfqpoint{3.343642in}{2.034398in}}%
\pgfpathlineto{\pgfqpoint{3.344252in}{2.185939in}}%
\pgfpathlineto{\pgfqpoint{3.344863in}{2.354592in}}%
\pgfpathlineto{\pgfqpoint{3.345473in}{2.250035in}}%
\pgfpathlineto{\pgfqpoint{3.346693in}{1.958096in}}%
\pgfpathlineto{\pgfqpoint{3.347304in}{2.132676in}}%
\pgfpathlineto{\pgfqpoint{3.348219in}{2.622587in}}%
\pgfpathlineto{\pgfqpoint{3.348829in}{2.405131in}}%
\pgfpathlineto{\pgfqpoint{3.350355in}{1.972260in}}%
\pgfpathlineto{\pgfqpoint{3.350965in}{2.008751in}}%
\pgfpathlineto{\pgfqpoint{3.352491in}{2.326051in}}%
\pgfpathlineto{\pgfqpoint{3.353101in}{2.495417in}}%
\pgfpathlineto{\pgfqpoint{3.353711in}{2.390892in}}%
\pgfpathlineto{\pgfqpoint{3.355237in}{2.003154in}}%
\pgfpathlineto{\pgfqpoint{3.355847in}{2.073380in}}%
\pgfpathlineto{\pgfqpoint{3.357372in}{2.389551in}}%
\pgfpathlineto{\pgfqpoint{3.357983in}{2.317388in}}%
\pgfpathlineto{\pgfqpoint{3.359508in}{2.066782in}}%
\pgfpathlineto{\pgfqpoint{3.360119in}{2.106625in}}%
\pgfpathlineto{\pgfqpoint{3.363170in}{2.398490in}}%
\pgfpathlineto{\pgfqpoint{3.363475in}{2.345696in}}%
\pgfpathlineto{\pgfqpoint{3.365306in}{1.961331in}}%
\pgfpathlineto{\pgfqpoint{3.365611in}{1.997205in}}%
\pgfpathlineto{\pgfqpoint{3.367136in}{2.529396in}}%
\pgfpathlineto{\pgfqpoint{3.367746in}{2.317973in}}%
\pgfpathlineto{\pgfqpoint{3.368662in}{2.026906in}}%
\pgfpathlineto{\pgfqpoint{3.369577in}{2.146724in}}%
\pgfpathlineto{\pgfqpoint{3.370493in}{2.314972in}}%
\pgfpathlineto{\pgfqpoint{3.371103in}{2.252887in}}%
\pgfpathlineto{\pgfqpoint{3.372323in}{2.143457in}}%
\pgfpathlineto{\pgfqpoint{3.372933in}{2.160345in}}%
\pgfpathlineto{\pgfqpoint{3.374459in}{2.213906in}}%
\pgfpathlineto{\pgfqpoint{3.374764in}{2.200178in}}%
\pgfpathlineto{\pgfqpoint{3.375985in}{2.080180in}}%
\pgfpathlineto{\pgfqpoint{3.376595in}{2.137029in}}%
\pgfpathlineto{\pgfqpoint{3.377815in}{2.496481in}}%
\pgfpathlineto{\pgfqpoint{3.378426in}{2.347068in}}%
\pgfpathlineto{\pgfqpoint{3.379951in}{1.967844in}}%
\pgfpathlineto{\pgfqpoint{3.380561in}{2.021330in}}%
\pgfpathlineto{\pgfqpoint{3.382087in}{2.448901in}}%
\pgfpathlineto{\pgfqpoint{3.382697in}{2.291113in}}%
\pgfpathlineto{\pgfqpoint{3.383613in}{2.088342in}}%
\pgfpathlineto{\pgfqpoint{3.384528in}{2.179362in}}%
\pgfpathlineto{\pgfqpoint{3.385443in}{2.274299in}}%
\pgfpathlineto{\pgfqpoint{3.386054in}{2.196507in}}%
\pgfpathlineto{\pgfqpoint{3.387274in}{2.035548in}}%
\pgfpathlineto{\pgfqpoint{3.387579in}{2.073529in}}%
\pgfpathlineto{\pgfqpoint{3.388800in}{2.491117in}}%
\pgfpathlineto{\pgfqpoint{3.389410in}{2.334298in}}%
\pgfpathlineto{\pgfqpoint{3.390630in}{2.046434in}}%
\pgfpathlineto{\pgfqpoint{3.391241in}{2.143340in}}%
\pgfpathlineto{\pgfqpoint{3.392156in}{2.252079in}}%
\pgfpathlineto{\pgfqpoint{3.392766in}{2.201455in}}%
\pgfpathlineto{\pgfqpoint{3.393376in}{2.176298in}}%
\pgfpathlineto{\pgfqpoint{3.393681in}{2.197922in}}%
\pgfpathlineto{\pgfqpoint{3.394597in}{2.271085in}}%
\pgfpathlineto{\pgfqpoint{3.395207in}{2.204903in}}%
\pgfpathlineto{\pgfqpoint{3.396122in}{2.087395in}}%
\pgfpathlineto{\pgfqpoint{3.397038in}{2.151928in}}%
\pgfpathlineto{\pgfqpoint{3.398258in}{2.303043in}}%
\pgfpathlineto{\pgfqpoint{3.399174in}{2.260826in}}%
\pgfpathlineto{\pgfqpoint{3.401309in}{2.042380in}}%
\pgfpathlineto{\pgfqpoint{3.401615in}{2.074635in}}%
\pgfpathlineto{\pgfqpoint{3.402835in}{2.502398in}}%
\pgfpathlineto{\pgfqpoint{3.403750in}{2.250812in}}%
\pgfpathlineto{\pgfqpoint{3.404666in}{1.996981in}}%
\pgfpathlineto{\pgfqpoint{3.405581in}{2.114649in}}%
\pgfpathlineto{\pgfqpoint{3.406802in}{2.329307in}}%
\pgfpathlineto{\pgfqpoint{3.407412in}{2.231572in}}%
\pgfpathlineto{\pgfqpoint{3.408327in}{2.108434in}}%
\pgfpathlineto{\pgfqpoint{3.409243in}{2.168401in}}%
\pgfpathlineto{\pgfqpoint{3.410768in}{2.268212in}}%
\pgfpathlineto{\pgfqpoint{3.411378in}{2.258677in}}%
\pgfpathlineto{\pgfqpoint{3.412904in}{2.109211in}}%
\pgfpathlineto{\pgfqpoint{3.413209in}{2.089704in}}%
\pgfpathlineto{\pgfqpoint{3.413819in}{2.132325in}}%
\pgfpathlineto{\pgfqpoint{3.414735in}{2.327136in}}%
\pgfpathlineto{\pgfqpoint{3.415345in}{2.264317in}}%
\pgfpathlineto{\pgfqpoint{3.416260in}{2.057928in}}%
\pgfpathlineto{\pgfqpoint{3.416870in}{2.183598in}}%
\pgfpathlineto{\pgfqpoint{3.417786in}{2.449773in}}%
\pgfpathlineto{\pgfqpoint{3.418396in}{2.214215in}}%
\pgfpathlineto{\pgfqpoint{3.419311in}{1.938355in}}%
\pgfpathlineto{\pgfqpoint{3.419922in}{2.037783in}}%
\pgfpathlineto{\pgfqpoint{3.421142in}{2.525650in}}%
\pgfpathlineto{\pgfqpoint{3.421752in}{2.258581in}}%
\pgfpathlineto{\pgfqpoint{3.422973in}{1.938345in}}%
\pgfpathlineto{\pgfqpoint{3.423583in}{2.083085in}}%
\pgfpathlineto{\pgfqpoint{3.424804in}{2.428511in}}%
\pgfpathlineto{\pgfqpoint{3.425414in}{2.259028in}}%
\pgfpathlineto{\pgfqpoint{3.426024in}{2.149044in}}%
\pgfpathlineto{\pgfqpoint{3.426939in}{2.209554in}}%
\pgfpathlineto{\pgfqpoint{3.427244in}{2.211065in}}%
\pgfpathlineto{\pgfqpoint{3.428465in}{2.101740in}}%
\pgfpathlineto{\pgfqpoint{3.429075in}{2.173062in}}%
\pgfpathlineto{\pgfqpoint{3.429991in}{2.259038in}}%
\pgfpathlineto{\pgfqpoint{3.430601in}{2.219238in}}%
\pgfpathlineto{\pgfqpoint{3.430906in}{2.210256in}}%
\pgfpathlineto{\pgfqpoint{3.431211in}{2.224080in}}%
\pgfpathlineto{\pgfqpoint{3.432126in}{2.309939in}}%
\pgfpathlineto{\pgfqpoint{3.432737in}{2.227443in}}%
\pgfpathlineto{\pgfqpoint{3.433957in}{1.948774in}}%
\pgfpathlineto{\pgfqpoint{3.434567in}{2.014572in}}%
\pgfpathlineto{\pgfqpoint{3.435788in}{2.681448in}}%
\pgfpathlineto{\pgfqpoint{3.436398in}{2.315206in}}%
\pgfpathlineto{\pgfqpoint{3.437313in}{1.880484in}}%
\pgfpathlineto{\pgfqpoint{3.437924in}{1.969951in}}%
\pgfpathlineto{\pgfqpoint{3.439144in}{2.497896in}}%
\pgfpathlineto{\pgfqpoint{3.439754in}{2.245949in}}%
\pgfpathlineto{\pgfqpoint{3.440670in}{2.037304in}}%
\pgfpathlineto{\pgfqpoint{3.441280in}{2.175904in}}%
\pgfpathlineto{\pgfqpoint{3.442195in}{2.342343in}}%
\pgfpathlineto{\pgfqpoint{3.442806in}{2.259709in}}%
\pgfpathlineto{\pgfqpoint{3.444331in}{2.095941in}}%
\pgfpathlineto{\pgfqpoint{3.444941in}{2.104411in}}%
\pgfpathlineto{\pgfqpoint{3.445246in}{2.103752in}}%
\pgfpathlineto{\pgfqpoint{3.445552in}{2.098580in}}%
\pgfpathlineto{\pgfqpoint{3.445857in}{2.100368in}}%
\pgfpathlineto{\pgfqpoint{3.446467in}{2.214619in}}%
\pgfpathlineto{\pgfqpoint{3.447382in}{2.586362in}}%
\pgfpathlineto{\pgfqpoint{3.447993in}{2.291358in}}%
\pgfpathlineto{\pgfqpoint{3.448908in}{1.870141in}}%
\pgfpathlineto{\pgfqpoint{3.449518in}{1.959884in}}%
\pgfpathlineto{\pgfqpoint{3.450739in}{2.625599in}}%
\pgfpathlineto{\pgfqpoint{3.451349in}{2.277034in}}%
\pgfpathlineto{\pgfqpoint{3.452264in}{1.904950in}}%
\pgfpathlineto{\pgfqpoint{3.452874in}{1.985637in}}%
\pgfpathlineto{\pgfqpoint{3.454095in}{2.489340in}}%
\pgfpathlineto{\pgfqpoint{3.454705in}{2.307608in}}%
\pgfpathlineto{\pgfqpoint{3.455620in}{2.019085in}}%
\pgfpathlineto{\pgfqpoint{3.456231in}{2.104741in}}%
\pgfpathlineto{\pgfqpoint{3.457146in}{2.350697in}}%
\pgfpathlineto{\pgfqpoint{3.457756in}{2.252142in}}%
\pgfpathlineto{\pgfqpoint{3.458672in}{2.046892in}}%
\pgfpathlineto{\pgfqpoint{3.459587in}{2.156089in}}%
\pgfpathlineto{\pgfqpoint{3.461723in}{2.401427in}}%
\pgfpathlineto{\pgfqpoint{3.462028in}{2.373258in}}%
\pgfpathlineto{\pgfqpoint{3.463859in}{1.911176in}}%
\pgfpathlineto{\pgfqpoint{3.464469in}{1.980263in}}%
\pgfpathlineto{\pgfqpoint{3.465689in}{2.660441in}}%
\pgfpathlineto{\pgfqpoint{3.466300in}{2.375684in}}%
\pgfpathlineto{\pgfqpoint{3.467215in}{1.927660in}}%
\pgfpathlineto{\pgfqpoint{3.468130in}{2.115671in}}%
\pgfpathlineto{\pgfqpoint{3.469046in}{2.411633in}}%
\pgfpathlineto{\pgfqpoint{3.469656in}{2.222122in}}%
\pgfpathlineto{\pgfqpoint{3.470571in}{2.024012in}}%
\pgfpathlineto{\pgfqpoint{3.471181in}{2.156748in}}%
\pgfpathlineto{\pgfqpoint{3.472097in}{2.361158in}}%
\pgfpathlineto{\pgfqpoint{3.472707in}{2.239319in}}%
\pgfpathlineto{\pgfqpoint{3.473622in}{2.070517in}}%
\pgfpathlineto{\pgfqpoint{3.474233in}{2.142733in}}%
\pgfpathlineto{\pgfqpoint{3.475148in}{2.390615in}}%
\pgfpathlineto{\pgfqpoint{3.475758in}{2.292954in}}%
\pgfpathlineto{\pgfqpoint{3.476674in}{2.073465in}}%
\pgfpathlineto{\pgfqpoint{3.477589in}{2.180969in}}%
\pgfpathlineto{\pgfqpoint{3.478199in}{2.249854in}}%
\pgfpathlineto{\pgfqpoint{3.478809in}{2.195432in}}%
\pgfpathlineto{\pgfqpoint{3.479725in}{2.110573in}}%
\pgfpathlineto{\pgfqpoint{3.480335in}{2.193080in}}%
\pgfpathlineto{\pgfqpoint{3.481250in}{2.334649in}}%
\pgfpathlineto{\pgfqpoint{3.481861in}{2.193772in}}%
\pgfpathlineto{\pgfqpoint{3.482776in}{2.004101in}}%
\pgfpathlineto{\pgfqpoint{3.483081in}{2.063472in}}%
\pgfpathlineto{\pgfqpoint{3.484302in}{2.568292in}}%
\pgfpathlineto{\pgfqpoint{3.484607in}{2.407589in}}%
\pgfpathlineto{\pgfqpoint{3.485827in}{1.901949in}}%
\pgfpathlineto{\pgfqpoint{3.486437in}{2.048957in}}%
\pgfpathlineto{\pgfqpoint{3.487353in}{2.485637in}}%
\pgfpathlineto{\pgfqpoint{3.487963in}{2.313355in}}%
\pgfpathlineto{\pgfqpoint{3.488878in}{1.974921in}}%
\pgfpathlineto{\pgfqpoint{3.489489in}{2.077307in}}%
\pgfpathlineto{\pgfqpoint{3.490709in}{2.409143in}}%
\pgfpathlineto{\pgfqpoint{3.491319in}{2.278215in}}%
\pgfpathlineto{\pgfqpoint{3.493455in}{2.058907in}}%
\pgfpathlineto{\pgfqpoint{3.493760in}{2.060439in}}%
\pgfpathlineto{\pgfqpoint{3.494676in}{2.246215in}}%
\pgfpathlineto{\pgfqpoint{3.495591in}{2.455818in}}%
\pgfpathlineto{\pgfqpoint{3.495896in}{2.360456in}}%
\pgfpathlineto{\pgfqpoint{3.497117in}{1.963470in}}%
\pgfpathlineto{\pgfqpoint{3.497727in}{2.124940in}}%
\pgfpathlineto{\pgfqpoint{3.498642in}{2.506740in}}%
\pgfpathlineto{\pgfqpoint{3.499252in}{2.313557in}}%
\pgfpathlineto{\pgfqpoint{3.500473in}{1.922371in}}%
\pgfpathlineto{\pgfqpoint{3.501083in}{2.067229in}}%
\pgfpathlineto{\pgfqpoint{3.501998in}{2.541570in}}%
\pgfpathlineto{\pgfqpoint{3.502609in}{2.359647in}}%
\pgfpathlineto{\pgfqpoint{3.503829in}{1.945389in}}%
\pgfpathlineto{\pgfqpoint{3.504439in}{2.059130in}}%
\pgfpathlineto{\pgfqpoint{3.505660in}{2.380271in}}%
\pgfpathlineto{\pgfqpoint{3.506270in}{2.229784in}}%
\pgfpathlineto{\pgfqpoint{3.506880in}{2.118416in}}%
\pgfpathlineto{\pgfqpoint{3.507491in}{2.221526in}}%
\pgfpathlineto{\pgfqpoint{3.508101in}{2.390679in}}%
\pgfpathlineto{\pgfqpoint{3.508711in}{2.287899in}}%
\pgfpathlineto{\pgfqpoint{3.509626in}{2.018382in}}%
\pgfpathlineto{\pgfqpoint{3.510237in}{2.131751in}}%
\pgfpathlineto{\pgfqpoint{3.511152in}{2.324986in}}%
\pgfpathlineto{\pgfqpoint{3.511762in}{2.159824in}}%
\pgfpathlineto{\pgfqpoint{3.512372in}{2.036697in}}%
\pgfpathlineto{\pgfqpoint{3.512983in}{2.107178in}}%
\pgfpathlineto{\pgfqpoint{3.513898in}{2.347909in}}%
\pgfpathlineto{\pgfqpoint{3.514508in}{2.271841in}}%
\pgfpathlineto{\pgfqpoint{3.515424in}{2.122513in}}%
\pgfpathlineto{\pgfqpoint{3.516034in}{2.216950in}}%
\pgfpathlineto{\pgfqpoint{3.516644in}{2.317899in}}%
\pgfpathlineto{\pgfqpoint{3.517254in}{2.242267in}}%
\pgfpathlineto{\pgfqpoint{3.518475in}{2.045147in}}%
\pgfpathlineto{\pgfqpoint{3.518780in}{2.086863in}}%
\pgfpathlineto{\pgfqpoint{3.520000in}{2.425318in}}%
\pgfpathlineto{\pgfqpoint{3.520611in}{2.243225in}}%
\pgfpathlineto{\pgfqpoint{3.521526in}{1.959426in}}%
\pgfpathlineto{\pgfqpoint{3.522136in}{2.025491in}}%
\pgfpathlineto{\pgfqpoint{3.523357in}{2.365649in}}%
\pgfpathlineto{\pgfqpoint{3.524272in}{2.239542in}}%
\pgfpathlineto{\pgfqpoint{3.524577in}{2.235935in}}%
\pgfpathlineto{\pgfqpoint{3.525493in}{2.323103in}}%
\pgfpathlineto{\pgfqpoint{3.526103in}{2.236850in}}%
\pgfpathlineto{\pgfqpoint{3.527323in}{1.987212in}}%
\pgfpathlineto{\pgfqpoint{3.527933in}{2.076136in}}%
\pgfpathlineto{\pgfqpoint{3.528849in}{2.469610in}}%
\pgfpathlineto{\pgfqpoint{3.529459in}{2.341524in}}%
\pgfpathlineto{\pgfqpoint{3.530680in}{1.873535in}}%
\pgfpathlineto{\pgfqpoint{3.531290in}{2.036388in}}%
\pgfpathlineto{\pgfqpoint{3.532205in}{2.673104in}}%
\pgfpathlineto{\pgfqpoint{3.532815in}{2.409558in}}%
\pgfpathlineto{\pgfqpoint{3.533731in}{1.984169in}}%
\pgfpathlineto{\pgfqpoint{3.534341in}{2.095887in}}%
\pgfpathlineto{\pgfqpoint{3.535256in}{2.317452in}}%
\pgfpathlineto{\pgfqpoint{3.535867in}{2.202945in}}%
\pgfpathlineto{\pgfqpoint{3.536782in}{2.039634in}}%
\pgfpathlineto{\pgfqpoint{3.537392in}{2.152694in}}%
\pgfpathlineto{\pgfqpoint{3.538307in}{2.408823in}}%
\pgfpathlineto{\pgfqpoint{3.538918in}{2.265743in}}%
\pgfpathlineto{\pgfqpoint{3.539833in}{2.049552in}}%
\pgfpathlineto{\pgfqpoint{3.540443in}{2.146702in}}%
\pgfpathlineto{\pgfqpoint{3.541054in}{2.258230in}}%
\pgfpathlineto{\pgfqpoint{3.541664in}{2.193995in}}%
\pgfpathlineto{\pgfqpoint{3.542274in}{2.105508in}}%
\pgfpathlineto{\pgfqpoint{3.542884in}{2.203903in}}%
\pgfpathlineto{\pgfqpoint{3.543800in}{2.404907in}}%
\pgfpathlineto{\pgfqpoint{3.544410in}{2.218088in}}%
\pgfpathlineto{\pgfqpoint{3.545325in}{1.995577in}}%
\pgfpathlineto{\pgfqpoint{3.545935in}{2.099633in}}%
\pgfpathlineto{\pgfqpoint{3.546851in}{2.406546in}}%
\pgfpathlineto{\pgfqpoint{3.547461in}{2.275948in}}%
\pgfpathlineto{\pgfqpoint{3.548376in}{2.015200in}}%
\pgfpathlineto{\pgfqpoint{3.548987in}{2.112893in}}%
\pgfpathlineto{\pgfqpoint{3.549902in}{2.434896in}}%
\pgfpathlineto{\pgfqpoint{3.550512in}{2.293795in}}%
\pgfpathlineto{\pgfqpoint{3.551428in}{2.021628in}}%
\pgfpathlineto{\pgfqpoint{3.552038in}{2.103645in}}%
\pgfpathlineto{\pgfqpoint{3.552953in}{2.350548in}}%
\pgfpathlineto{\pgfqpoint{3.553563in}{2.233530in}}%
\pgfpathlineto{\pgfqpoint{3.554479in}{2.011305in}}%
\pgfpathlineto{\pgfqpoint{3.555089in}{2.105880in}}%
\pgfpathlineto{\pgfqpoint{3.556004in}{2.436907in}}%
\pgfpathlineto{\pgfqpoint{3.556615in}{2.338033in}}%
\pgfpathlineto{\pgfqpoint{3.557835in}{2.031163in}}%
\pgfpathlineto{\pgfqpoint{3.558445in}{2.184173in}}%
\pgfpathlineto{\pgfqpoint{3.559361in}{2.426095in}}%
\pgfpathlineto{\pgfqpoint{3.559971in}{2.241149in}}%
\pgfpathlineto{\pgfqpoint{3.560886in}{2.035612in}}%
\pgfpathlineto{\pgfqpoint{3.561496in}{2.111733in}}%
\pgfpathlineto{\pgfqpoint{3.562412in}{2.268755in}}%
\pgfpathlineto{\pgfqpoint{3.563022in}{2.184364in}}%
\pgfpathlineto{\pgfqpoint{3.563937in}{2.043827in}}%
\pgfpathlineto{\pgfqpoint{3.564548in}{2.147149in}}%
\pgfpathlineto{\pgfqpoint{3.565463in}{2.417528in}}%
\pgfpathlineto{\pgfqpoint{3.566073in}{2.310375in}}%
\pgfpathlineto{\pgfqpoint{3.566989in}{2.125153in}}%
\pgfpathlineto{\pgfqpoint{3.567599in}{2.193133in}}%
\pgfpathlineto{\pgfqpoint{3.568209in}{2.274331in}}%
\pgfpathlineto{\pgfqpoint{3.568819in}{2.221643in}}%
\pgfpathlineto{\pgfqpoint{3.569735in}{2.098282in}}%
\pgfpathlineto{\pgfqpoint{3.570345in}{2.157632in}}%
\pgfpathlineto{\pgfqpoint{3.571260in}{2.247375in}}%
\pgfpathlineto{\pgfqpoint{3.571870in}{2.158909in}}%
\pgfpathlineto{\pgfqpoint{3.572481in}{2.111158in}}%
\pgfpathlineto{\pgfqpoint{3.572786in}{2.144861in}}%
\pgfpathlineto{\pgfqpoint{3.573701in}{2.334202in}}%
\pgfpathlineto{\pgfqpoint{3.574311in}{2.270542in}}%
\pgfpathlineto{\pgfqpoint{3.575227in}{2.072922in}}%
\pgfpathlineto{\pgfqpoint{3.575837in}{2.124471in}}%
\pgfpathlineto{\pgfqpoint{3.576752in}{2.342279in}}%
\pgfpathlineto{\pgfqpoint{3.577363in}{2.241405in}}%
\pgfpathlineto{\pgfqpoint{3.578278in}{2.002356in}}%
\pgfpathlineto{\pgfqpoint{3.578888in}{2.128015in}}%
\pgfpathlineto{\pgfqpoint{3.579804in}{2.583638in}}%
\pgfpathlineto{\pgfqpoint{3.580414in}{2.331648in}}%
\pgfpathlineto{\pgfqpoint{3.581634in}{1.877824in}}%
\pgfpathlineto{\pgfqpoint{3.582244in}{2.017542in}}%
\pgfpathlineto{\pgfqpoint{3.583465in}{2.538357in}}%
\pgfpathlineto{\pgfqpoint{3.584075in}{2.287569in}}%
\pgfpathlineto{\pgfqpoint{3.584991in}{2.081829in}}%
\pgfpathlineto{\pgfqpoint{3.585906in}{2.100155in}}%
\pgfpathlineto{\pgfqpoint{3.586821in}{2.121747in}}%
\pgfpathlineto{\pgfqpoint{3.588347in}{2.378824in}}%
\pgfpathlineto{\pgfqpoint{3.588957in}{2.296679in}}%
\pgfpathlineto{\pgfqpoint{3.590178in}{1.982530in}}%
\pgfpathlineto{\pgfqpoint{3.590788in}{2.073646in}}%
\pgfpathlineto{\pgfqpoint{3.592008in}{2.543401in}}%
\pgfpathlineto{\pgfqpoint{3.592619in}{2.246375in}}%
\pgfpathlineto{\pgfqpoint{3.593534in}{2.032930in}}%
\pgfpathlineto{\pgfqpoint{3.594144in}{2.137891in}}%
\pgfpathlineto{\pgfqpoint{3.594754in}{2.225367in}}%
\pgfpathlineto{\pgfqpoint{3.595365in}{2.158536in}}%
\pgfpathlineto{\pgfqpoint{3.596280in}{2.044498in}}%
\pgfpathlineto{\pgfqpoint{3.596585in}{2.096920in}}%
\pgfpathlineto{\pgfqpoint{3.597806in}{2.515583in}}%
\pgfpathlineto{\pgfqpoint{3.598416in}{2.229954in}}%
\pgfpathlineto{\pgfqpoint{3.599331in}{1.923925in}}%
\pgfpathlineto{\pgfqpoint{3.599941in}{2.016435in}}%
\pgfpathlineto{\pgfqpoint{3.601162in}{2.467162in}}%
\pgfpathlineto{\pgfqpoint{3.601772in}{2.279960in}}%
\pgfpathlineto{\pgfqpoint{3.602687in}{2.127100in}}%
\pgfpathlineto{\pgfqpoint{3.603298in}{2.189419in}}%
\pgfpathlineto{\pgfqpoint{3.603603in}{2.207489in}}%
\pgfpathlineto{\pgfqpoint{3.604213in}{2.185269in}}%
\pgfpathlineto{\pgfqpoint{3.604823in}{2.157866in}}%
\pgfpathlineto{\pgfqpoint{3.605433in}{2.201168in}}%
\pgfpathlineto{\pgfqpoint{3.606349in}{2.269925in}}%
\pgfpathlineto{\pgfqpoint{3.606654in}{2.239883in}}%
\pgfpathlineto{\pgfqpoint{3.607874in}{2.042869in}}%
\pgfpathlineto{\pgfqpoint{3.608485in}{2.122088in}}%
\pgfpathlineto{\pgfqpoint{3.609705in}{2.478294in}}%
\pgfpathlineto{\pgfqpoint{3.610010in}{2.378611in}}%
\pgfpathlineto{\pgfqpoint{3.611536in}{1.950881in}}%
\pgfpathlineto{\pgfqpoint{3.612146in}{2.078083in}}%
\pgfpathlineto{\pgfqpoint{3.613367in}{2.438897in}}%
\pgfpathlineto{\pgfqpoint{3.613977in}{2.220483in}}%
\pgfpathlineto{\pgfqpoint{3.614587in}{2.087959in}}%
\pgfpathlineto{\pgfqpoint{3.615197in}{2.199667in}}%
\pgfpathlineto{\pgfqpoint{3.615807in}{2.333223in}}%
\pgfpathlineto{\pgfqpoint{3.616418in}{2.241043in}}%
\pgfpathlineto{\pgfqpoint{3.617333in}{2.063217in}}%
\pgfpathlineto{\pgfqpoint{3.617943in}{2.156131in}}%
\pgfpathlineto{\pgfqpoint{3.618859in}{2.290773in}}%
\pgfpathlineto{\pgfqpoint{3.619469in}{2.171519in}}%
\pgfpathlineto{\pgfqpoint{3.620079in}{2.088587in}}%
\pgfpathlineto{\pgfqpoint{3.620689in}{2.142520in}}%
\pgfpathlineto{\pgfqpoint{3.621605in}{2.243076in}}%
\pgfpathlineto{\pgfqpoint{3.622215in}{2.194846in}}%
\pgfpathlineto{\pgfqpoint{3.622825in}{2.165932in}}%
\pgfpathlineto{\pgfqpoint{3.623130in}{2.185269in}}%
\pgfpathlineto{\pgfqpoint{3.624351in}{2.369225in}}%
\pgfpathlineto{\pgfqpoint{3.624961in}{2.261880in}}%
\pgfpathlineto{\pgfqpoint{3.626181in}{2.011050in}}%
\pgfpathlineto{\pgfqpoint{3.626792in}{2.071571in}}%
\pgfpathlineto{\pgfqpoint{3.628012in}{2.415751in}}%
\pgfpathlineto{\pgfqpoint{3.628622in}{2.308438in}}%
\pgfpathlineto{\pgfqpoint{3.629843in}{2.126855in}}%
\pgfpathlineto{\pgfqpoint{3.630453in}{2.147139in}}%
\pgfpathlineto{\pgfqpoint{3.631063in}{2.128452in}}%
\pgfpathlineto{\pgfqpoint{3.631979in}{2.088087in}}%
\pgfpathlineto{\pgfqpoint{3.632284in}{2.114872in}}%
\pgfpathlineto{\pgfqpoint{3.633809in}{2.433630in}}%
\pgfpathlineto{\pgfqpoint{3.634420in}{2.302628in}}%
\pgfpathlineto{\pgfqpoint{3.635945in}{2.030429in}}%
\pgfpathlineto{\pgfqpoint{3.636556in}{2.062493in}}%
\pgfpathlineto{\pgfqpoint{3.638081in}{2.360041in}}%
\pgfpathlineto{\pgfqpoint{3.638996in}{2.259507in}}%
\pgfpathlineto{\pgfqpoint{3.641132in}{2.084171in}}%
\pgfpathlineto{\pgfqpoint{3.641437in}{2.092365in}}%
\pgfpathlineto{\pgfqpoint{3.642963in}{2.396500in}}%
\pgfpathlineto{\pgfqpoint{3.643573in}{2.244842in}}%
\pgfpathlineto{\pgfqpoint{3.644794in}{1.982551in}}%
\pgfpathlineto{\pgfqpoint{3.645404in}{2.112733in}}%
\pgfpathlineto{\pgfqpoint{3.646319in}{2.423499in}}%
\pgfpathlineto{\pgfqpoint{3.646930in}{2.315526in}}%
\pgfpathlineto{\pgfqpoint{3.647845in}{2.112084in}}%
\pgfpathlineto{\pgfqpoint{3.648760in}{2.209862in}}%
\pgfpathlineto{\pgfqpoint{3.649065in}{2.237244in}}%
\pgfpathlineto{\pgfqpoint{3.649676in}{2.204009in}}%
\pgfpathlineto{\pgfqpoint{3.650591in}{2.108040in}}%
\pgfpathlineto{\pgfqpoint{3.651201in}{2.149991in}}%
\pgfpathlineto{\pgfqpoint{3.652422in}{2.309023in}}%
\pgfpathlineto{\pgfqpoint{3.653032in}{2.270734in}}%
\pgfpathlineto{\pgfqpoint{3.655168in}{2.159845in}}%
\pgfpathlineto{\pgfqpoint{3.655473in}{2.160814in}}%
\pgfpathlineto{\pgfqpoint{3.657304in}{2.190632in}}%
\pgfpathlineto{\pgfqpoint{3.658524in}{2.295700in}}%
\pgfpathlineto{\pgfqpoint{3.658829in}{2.263231in}}%
\pgfpathlineto{\pgfqpoint{3.660050in}{2.109041in}}%
\pgfpathlineto{\pgfqpoint{3.660660in}{2.191782in}}%
\pgfpathlineto{\pgfqpoint{3.661270in}{2.258996in}}%
\pgfpathlineto{\pgfqpoint{3.661880in}{2.212863in}}%
\pgfpathlineto{\pgfqpoint{3.662796in}{2.134070in}}%
\pgfpathlineto{\pgfqpoint{3.663406in}{2.174127in}}%
\pgfpathlineto{\pgfqpoint{3.666152in}{2.316164in}}%
\pgfpathlineto{\pgfqpoint{3.667067in}{2.189749in}}%
\pgfpathlineto{\pgfqpoint{3.668288in}{2.059886in}}%
\pgfpathlineto{\pgfqpoint{3.668898in}{2.115905in}}%
\pgfpathlineto{\pgfqpoint{3.670119in}{2.259028in}}%
\pgfpathlineto{\pgfqpoint{3.670729in}{2.211427in}}%
\pgfpathlineto{\pgfqpoint{3.671339in}{2.156674in}}%
\pgfpathlineto{\pgfqpoint{3.671949in}{2.200359in}}%
\pgfpathlineto{\pgfqpoint{3.672865in}{2.319761in}}%
\pgfpathlineto{\pgfqpoint{3.673475in}{2.248854in}}%
\pgfpathlineto{\pgfqpoint{3.674390in}{2.132198in}}%
\pgfpathlineto{\pgfqpoint{3.675306in}{2.187418in}}%
\pgfpathlineto{\pgfqpoint{3.675916in}{2.211959in}}%
\pgfpathlineto{\pgfqpoint{3.676526in}{2.192569in}}%
\pgfpathlineto{\pgfqpoint{3.677441in}{2.169317in}}%
\pgfpathlineto{\pgfqpoint{3.677746in}{2.180895in}}%
\pgfpathlineto{\pgfqpoint{3.679272in}{2.287250in}}%
\pgfpathlineto{\pgfqpoint{3.679577in}{2.263285in}}%
\pgfpathlineto{\pgfqpoint{3.681103in}{2.071241in}}%
\pgfpathlineto{\pgfqpoint{3.681713in}{2.162261in}}%
\pgfpathlineto{\pgfqpoint{3.682628in}{2.307044in}}%
\pgfpathlineto{\pgfqpoint{3.683239in}{2.209788in}}%
\pgfpathlineto{\pgfqpoint{3.684154in}{2.111371in}}%
\pgfpathlineto{\pgfqpoint{3.684764in}{2.181725in}}%
\pgfpathlineto{\pgfqpoint{3.687815in}{2.370502in}}%
\pgfpathlineto{\pgfqpoint{3.689646in}{1.998450in}}%
\pgfpathlineto{\pgfqpoint{3.690561in}{2.106199in}}%
\pgfpathlineto{\pgfqpoint{3.692087in}{2.372173in}}%
\pgfpathlineto{\pgfqpoint{3.692697in}{2.311482in}}%
\pgfpathlineto{\pgfqpoint{3.694833in}{1.945890in}}%
\pgfpathlineto{\pgfqpoint{3.695443in}{2.050446in}}%
\pgfpathlineto{\pgfqpoint{3.696664in}{2.669561in}}%
\pgfpathlineto{\pgfqpoint{3.697274in}{2.295913in}}%
\pgfpathlineto{\pgfqpoint{3.698189in}{1.918050in}}%
\pgfpathlineto{\pgfqpoint{3.698800in}{2.041688in}}%
\pgfpathlineto{\pgfqpoint{3.699715in}{2.416102in}}%
\pgfpathlineto{\pgfqpoint{3.700630in}{2.176787in}}%
\pgfpathlineto{\pgfqpoint{3.701241in}{2.044551in}}%
\pgfpathlineto{\pgfqpoint{3.701851in}{2.148267in}}%
\pgfpathlineto{\pgfqpoint{3.702766in}{2.412090in}}%
\pgfpathlineto{\pgfqpoint{3.703376in}{2.245108in}}%
\pgfpathlineto{\pgfqpoint{3.704597in}{1.936237in}}%
\pgfpathlineto{\pgfqpoint{3.705207in}{2.106529in}}%
\pgfpathlineto{\pgfqpoint{3.706122in}{2.525310in}}%
\pgfpathlineto{\pgfqpoint{3.706733in}{2.359573in}}%
\pgfpathlineto{\pgfqpoint{3.707953in}{2.029035in}}%
\pgfpathlineto{\pgfqpoint{3.708869in}{2.075636in}}%
\pgfpathlineto{\pgfqpoint{3.710394in}{2.386954in}}%
\pgfpathlineto{\pgfqpoint{3.711004in}{2.493139in}}%
\pgfpathlineto{\pgfqpoint{3.711615in}{2.335171in}}%
\pgfpathlineto{\pgfqpoint{3.713140in}{1.974750in}}%
\pgfpathlineto{\pgfqpoint{3.713750in}{2.046222in}}%
\pgfpathlineto{\pgfqpoint{3.715276in}{2.376014in}}%
\pgfpathlineto{\pgfqpoint{3.715581in}{2.337863in}}%
\pgfusepath{stroke}%
\end{pgfscope}%
\begin{pgfscope}%
\pgfpathrectangle{\pgfqpoint{0.664400in}{1.756587in}}{\pgfqpoint{3.051181in}{1.188976in}}%
\pgfusepath{clip}%
\pgfsetrectcap%
\pgfsetroundjoin%
\pgfsetlinewidth{2.007500pt}%
\definecolor{currentstroke}{rgb}{0.000000,0.000000,0.000000}%
\pgfsetstrokecolor{currentstroke}%
\pgfsetdash{}{0pt}%
\pgfpathmoveto{\pgfqpoint{0.816959in}{2.199292in}}%
\pgfpathlineto{\pgfqpoint{0.818180in}{2.198092in}}%
\pgfpathlineto{\pgfqpoint{0.818485in}{2.198604in}}%
\pgfpathlineto{\pgfqpoint{0.819705in}{2.201773in}}%
\pgfpathlineto{\pgfqpoint{0.820315in}{2.200534in}}%
\pgfpathlineto{\pgfqpoint{0.821841in}{2.198534in}}%
\pgfpathlineto{\pgfqpoint{0.822146in}{2.198786in}}%
\pgfpathlineto{\pgfqpoint{0.825197in}{2.199915in}}%
\pgfpathlineto{\pgfqpoint{0.827333in}{2.198381in}}%
\pgfpathlineto{\pgfqpoint{0.827638in}{2.198751in}}%
\pgfpathlineto{\pgfqpoint{0.828859in}{2.199887in}}%
\pgfpathlineto{\pgfqpoint{0.829469in}{2.199244in}}%
\pgfpathlineto{\pgfqpoint{0.830384in}{2.199013in}}%
\pgfpathlineto{\pgfqpoint{0.830994in}{2.199834in}}%
\pgfpathlineto{\pgfqpoint{0.832215in}{2.200704in}}%
\pgfpathlineto{\pgfqpoint{0.832520in}{2.200232in}}%
\pgfpathlineto{\pgfqpoint{0.834046in}{2.198234in}}%
\pgfpathlineto{\pgfqpoint{0.834656in}{2.198802in}}%
\pgfpathlineto{\pgfqpoint{0.837707in}{2.199223in}}%
\pgfpathlineto{\pgfqpoint{0.844420in}{2.198416in}}%
\pgfpathlineto{\pgfqpoint{0.845030in}{2.198128in}}%
\pgfpathlineto{\pgfqpoint{0.845335in}{2.198677in}}%
\pgfpathlineto{\pgfqpoint{0.846556in}{2.201031in}}%
\pgfpathlineto{\pgfqpoint{0.847166in}{2.200286in}}%
\pgfpathlineto{\pgfqpoint{0.848386in}{2.197258in}}%
\pgfpathlineto{\pgfqpoint{0.848996in}{2.198027in}}%
\pgfpathlineto{\pgfqpoint{0.850522in}{2.200523in}}%
\pgfpathlineto{\pgfqpoint{0.851132in}{2.199823in}}%
\pgfpathlineto{\pgfqpoint{0.852963in}{2.199750in}}%
\pgfpathlineto{\pgfqpoint{0.856624in}{2.199622in}}%
\pgfpathlineto{\pgfqpoint{0.858150in}{2.197378in}}%
\pgfpathlineto{\pgfqpoint{0.858760in}{2.198341in}}%
\pgfpathlineto{\pgfqpoint{0.859981in}{2.199671in}}%
\pgfpathlineto{\pgfqpoint{0.860591in}{2.199219in}}%
\pgfpathlineto{\pgfqpoint{0.864252in}{2.198263in}}%
\pgfpathlineto{\pgfqpoint{0.865778in}{2.201047in}}%
\pgfpathlineto{\pgfqpoint{0.866693in}{2.199617in}}%
\pgfpathlineto{\pgfqpoint{0.867609in}{2.198722in}}%
\pgfpathlineto{\pgfqpoint{0.868219in}{2.199363in}}%
\pgfpathlineto{\pgfqpoint{0.869134in}{2.200433in}}%
\pgfpathlineto{\pgfqpoint{0.869744in}{2.199775in}}%
\pgfpathlineto{\pgfqpoint{0.871270in}{2.198703in}}%
\pgfpathlineto{\pgfqpoint{0.871575in}{2.198952in}}%
\pgfpathlineto{\pgfqpoint{0.873406in}{2.198699in}}%
\pgfpathlineto{\pgfqpoint{0.874931in}{2.199174in}}%
\pgfpathlineto{\pgfqpoint{0.876457in}{2.200751in}}%
\pgfpathlineto{\pgfqpoint{0.877067in}{2.199924in}}%
\pgfpathlineto{\pgfqpoint{0.878288in}{2.198185in}}%
\pgfpathlineto{\pgfqpoint{0.878898in}{2.198877in}}%
\pgfpathlineto{\pgfqpoint{0.880424in}{2.200808in}}%
\pgfpathlineto{\pgfqpoint{0.881034in}{2.199919in}}%
\pgfpathlineto{\pgfqpoint{0.882559in}{2.198501in}}%
\pgfpathlineto{\pgfqpoint{0.883170in}{2.198987in}}%
\pgfpathlineto{\pgfqpoint{0.885611in}{2.198694in}}%
\pgfpathlineto{\pgfqpoint{0.891713in}{2.199456in}}%
\pgfpathlineto{\pgfqpoint{0.892933in}{2.198362in}}%
\pgfpathlineto{\pgfqpoint{0.893544in}{2.199232in}}%
\pgfpathlineto{\pgfqpoint{0.894764in}{2.201291in}}%
\pgfpathlineto{\pgfqpoint{0.895374in}{2.200554in}}%
\pgfpathlineto{\pgfqpoint{0.896900in}{2.197221in}}%
\pgfpathlineto{\pgfqpoint{0.897510in}{2.198617in}}%
\pgfpathlineto{\pgfqpoint{0.898426in}{2.200273in}}%
\pgfpathlineto{\pgfqpoint{0.899036in}{2.199542in}}%
\pgfpathlineto{\pgfqpoint{0.899951in}{2.198530in}}%
\pgfpathlineto{\pgfqpoint{0.900561in}{2.199314in}}%
\pgfpathlineto{\pgfqpoint{0.901782in}{2.200978in}}%
\pgfpathlineto{\pgfqpoint{0.902392in}{2.199767in}}%
\pgfpathlineto{\pgfqpoint{0.903613in}{2.197897in}}%
\pgfpathlineto{\pgfqpoint{0.904223in}{2.198665in}}%
\pgfpathlineto{\pgfqpoint{0.905748in}{2.200773in}}%
\pgfpathlineto{\pgfqpoint{0.906359in}{2.199689in}}%
\pgfpathlineto{\pgfqpoint{0.907884in}{2.198205in}}%
\pgfpathlineto{\pgfqpoint{0.908189in}{2.198429in}}%
\pgfpathlineto{\pgfqpoint{0.910935in}{2.198979in}}%
\pgfpathlineto{\pgfqpoint{0.913071in}{2.199387in}}%
\pgfpathlineto{\pgfqpoint{0.915207in}{2.199116in}}%
\pgfpathlineto{\pgfqpoint{0.921615in}{2.198249in}}%
\pgfpathlineto{\pgfqpoint{0.922530in}{2.197807in}}%
\pgfpathlineto{\pgfqpoint{0.923140in}{2.198549in}}%
\pgfpathlineto{\pgfqpoint{0.924666in}{2.200055in}}%
\pgfpathlineto{\pgfqpoint{0.925276in}{2.199436in}}%
\pgfpathlineto{\pgfqpoint{0.926802in}{2.199076in}}%
\pgfpathlineto{\pgfqpoint{0.927107in}{2.199386in}}%
\pgfpathlineto{\pgfqpoint{0.929243in}{2.199817in}}%
\pgfpathlineto{\pgfqpoint{0.934430in}{2.198262in}}%
\pgfpathlineto{\pgfqpoint{0.935955in}{2.199787in}}%
\pgfpathlineto{\pgfqpoint{0.936565in}{2.199205in}}%
\pgfpathlineto{\pgfqpoint{0.937786in}{2.198180in}}%
\pgfpathlineto{\pgfqpoint{0.938091in}{2.198660in}}%
\pgfpathlineto{\pgfqpoint{0.939617in}{2.201058in}}%
\pgfpathlineto{\pgfqpoint{0.940227in}{2.200224in}}%
\pgfpathlineto{\pgfqpoint{0.941447in}{2.198701in}}%
\pgfpathlineto{\pgfqpoint{0.942057in}{2.199212in}}%
\pgfpathlineto{\pgfqpoint{0.944498in}{2.199846in}}%
\pgfpathlineto{\pgfqpoint{0.949380in}{2.198793in}}%
\pgfpathlineto{\pgfqpoint{0.951211in}{2.200784in}}%
\pgfpathlineto{\pgfqpoint{0.951821in}{2.199700in}}%
\pgfpathlineto{\pgfqpoint{0.952737in}{2.198735in}}%
\pgfpathlineto{\pgfqpoint{0.953347in}{2.199363in}}%
\pgfpathlineto{\pgfqpoint{0.954567in}{2.199882in}}%
\pgfpathlineto{\pgfqpoint{0.955178in}{2.199330in}}%
\pgfpathlineto{\pgfqpoint{0.958534in}{2.198032in}}%
\pgfpathlineto{\pgfqpoint{0.968298in}{2.199452in}}%
\pgfpathlineto{\pgfqpoint{0.972569in}{2.198779in}}%
\pgfpathlineto{\pgfqpoint{0.974400in}{2.199703in}}%
\pgfpathlineto{\pgfqpoint{0.975620in}{2.200654in}}%
\pgfpathlineto{\pgfqpoint{0.976231in}{2.199873in}}%
\pgfpathlineto{\pgfqpoint{0.977756in}{2.198298in}}%
\pgfpathlineto{\pgfqpoint{0.978367in}{2.198968in}}%
\pgfpathlineto{\pgfqpoint{0.979892in}{2.199716in}}%
\pgfpathlineto{\pgfqpoint{0.980197in}{2.199378in}}%
\pgfpathlineto{\pgfqpoint{0.981418in}{2.198452in}}%
\pgfpathlineto{\pgfqpoint{0.982028in}{2.199059in}}%
\pgfpathlineto{\pgfqpoint{0.983554in}{2.198956in}}%
\pgfpathlineto{\pgfqpoint{0.984774in}{2.198275in}}%
\pgfpathlineto{\pgfqpoint{0.985384in}{2.199068in}}%
\pgfpathlineto{\pgfqpoint{0.986605in}{2.200619in}}%
\pgfpathlineto{\pgfqpoint{0.987215in}{2.200101in}}%
\pgfpathlineto{\pgfqpoint{0.989961in}{2.199026in}}%
\pgfpathlineto{\pgfqpoint{1.003386in}{2.199271in}}%
\pgfpathlineto{\pgfqpoint{1.008268in}{2.198589in}}%
\pgfpathlineto{\pgfqpoint{1.010099in}{2.200431in}}%
\pgfpathlineto{\pgfqpoint{1.010404in}{2.200150in}}%
\pgfpathlineto{\pgfqpoint{1.011930in}{2.198521in}}%
\pgfpathlineto{\pgfqpoint{1.012540in}{2.199529in}}%
\pgfpathlineto{\pgfqpoint{1.013455in}{2.200318in}}%
\pgfpathlineto{\pgfqpoint{1.014065in}{2.199739in}}%
\pgfpathlineto{\pgfqpoint{1.015286in}{2.198601in}}%
\pgfpathlineto{\pgfqpoint{1.015896in}{2.199374in}}%
\pgfpathlineto{\pgfqpoint{1.016811in}{2.200269in}}%
\pgfpathlineto{\pgfqpoint{1.017422in}{2.199438in}}%
\pgfpathlineto{\pgfqpoint{1.018642in}{2.197647in}}%
\pgfpathlineto{\pgfqpoint{1.019252in}{2.198289in}}%
\pgfpathlineto{\pgfqpoint{1.020778in}{2.200872in}}%
\pgfpathlineto{\pgfqpoint{1.021388in}{2.199782in}}%
\pgfpathlineto{\pgfqpoint{1.022609in}{2.198301in}}%
\pgfpathlineto{\pgfqpoint{1.023219in}{2.199411in}}%
\pgfpathlineto{\pgfqpoint{1.024134in}{2.200862in}}%
\pgfpathlineto{\pgfqpoint{1.024744in}{2.200199in}}%
\pgfpathlineto{\pgfqpoint{1.026880in}{2.199189in}}%
\pgfpathlineto{\pgfqpoint{1.030237in}{2.199074in}}%
\pgfpathlineto{\pgfqpoint{1.034508in}{2.199237in}}%
\pgfpathlineto{\pgfqpoint{1.036949in}{2.199288in}}%
\pgfpathlineto{\pgfqpoint{1.038170in}{2.198666in}}%
\pgfpathlineto{\pgfqpoint{1.038780in}{2.199304in}}%
\pgfpathlineto{\pgfqpoint{1.040000in}{2.199395in}}%
\pgfpathlineto{\pgfqpoint{1.040306in}{2.199068in}}%
\pgfpathlineto{\pgfqpoint{1.042136in}{2.199286in}}%
\pgfpathlineto{\pgfqpoint{1.044272in}{2.199978in}}%
\pgfpathlineto{\pgfqpoint{1.045187in}{2.200259in}}%
\pgfpathlineto{\pgfqpoint{1.045798in}{2.199565in}}%
\pgfpathlineto{\pgfqpoint{1.047628in}{2.199386in}}%
\pgfpathlineto{\pgfqpoint{1.050985in}{2.199601in}}%
\pgfpathlineto{\pgfqpoint{1.054341in}{2.198589in}}%
\pgfpathlineto{\pgfqpoint{1.061359in}{2.198703in}}%
\pgfpathlineto{\pgfqpoint{1.063189in}{2.199077in}}%
\pgfpathlineto{\pgfqpoint{1.068987in}{2.198962in}}%
\pgfpathlineto{\pgfqpoint{1.070207in}{2.198615in}}%
\pgfpathlineto{\pgfqpoint{1.070512in}{2.199186in}}%
\pgfpathlineto{\pgfqpoint{1.071733in}{2.200824in}}%
\pgfpathlineto{\pgfqpoint{1.072343in}{2.199842in}}%
\pgfpathlineto{\pgfqpoint{1.073258in}{2.198445in}}%
\pgfpathlineto{\pgfqpoint{1.073869in}{2.199202in}}%
\pgfpathlineto{\pgfqpoint{1.075089in}{2.200691in}}%
\pgfpathlineto{\pgfqpoint{1.075699in}{2.199865in}}%
\pgfpathlineto{\pgfqpoint{1.076920in}{2.198281in}}%
\pgfpathlineto{\pgfqpoint{1.077530in}{2.199027in}}%
\pgfpathlineto{\pgfqpoint{1.078750in}{2.199533in}}%
\pgfpathlineto{\pgfqpoint{1.079361in}{2.198946in}}%
\pgfpathlineto{\pgfqpoint{1.081496in}{2.198592in}}%
\pgfpathlineto{\pgfqpoint{1.084243in}{2.199204in}}%
\pgfpathlineto{\pgfqpoint{1.085768in}{2.200087in}}%
\pgfpathlineto{\pgfqpoint{1.086683in}{2.200382in}}%
\pgfpathlineto{\pgfqpoint{1.086989in}{2.199948in}}%
\pgfpathlineto{\pgfqpoint{1.088209in}{2.198572in}}%
\pgfpathlineto{\pgfqpoint{1.088819in}{2.199235in}}%
\pgfpathlineto{\pgfqpoint{1.090040in}{2.199880in}}%
\pgfpathlineto{\pgfqpoint{1.090345in}{2.199408in}}%
\pgfpathlineto{\pgfqpoint{1.091565in}{2.197697in}}%
\pgfpathlineto{\pgfqpoint{1.092176in}{2.198696in}}%
\pgfpathlineto{\pgfqpoint{1.093701in}{2.199441in}}%
\pgfpathlineto{\pgfqpoint{1.094006in}{2.199206in}}%
\pgfpathlineto{\pgfqpoint{1.096142in}{2.199519in}}%
\pgfpathlineto{\pgfqpoint{1.098888in}{2.199839in}}%
\pgfpathlineto{\pgfqpoint{1.100109in}{2.198865in}}%
\pgfpathlineto{\pgfqpoint{1.100719in}{2.199753in}}%
\pgfpathlineto{\pgfqpoint{1.101634in}{2.200943in}}%
\pgfpathlineto{\pgfqpoint{1.102244in}{2.199894in}}%
\pgfpathlineto{\pgfqpoint{1.103465in}{2.197029in}}%
\pgfpathlineto{\pgfqpoint{1.104075in}{2.197797in}}%
\pgfpathlineto{\pgfqpoint{1.105296in}{2.199678in}}%
\pgfpathlineto{\pgfqpoint{1.105906in}{2.198812in}}%
\pgfpathlineto{\pgfqpoint{1.106821in}{2.198402in}}%
\pgfpathlineto{\pgfqpoint{1.107431in}{2.199183in}}%
\pgfpathlineto{\pgfqpoint{1.109872in}{2.199965in}}%
\pgfpathlineto{\pgfqpoint{1.114449in}{2.198366in}}%
\pgfpathlineto{\pgfqpoint{1.115059in}{2.197795in}}%
\pgfpathlineto{\pgfqpoint{1.115670in}{2.198598in}}%
\pgfpathlineto{\pgfqpoint{1.116585in}{2.199958in}}%
\pgfpathlineto{\pgfqpoint{1.117500in}{2.198921in}}%
\pgfpathlineto{\pgfqpoint{1.118111in}{2.198626in}}%
\pgfpathlineto{\pgfqpoint{1.118721in}{2.199693in}}%
\pgfpathlineto{\pgfqpoint{1.119636in}{2.200778in}}%
\pgfpathlineto{\pgfqpoint{1.120246in}{2.199928in}}%
\pgfpathlineto{\pgfqpoint{1.121467in}{2.198698in}}%
\pgfpathlineto{\pgfqpoint{1.122077in}{2.199581in}}%
\pgfpathlineto{\pgfqpoint{1.123298in}{2.200763in}}%
\pgfpathlineto{\pgfqpoint{1.123603in}{2.200325in}}%
\pgfpathlineto{\pgfqpoint{1.124823in}{2.197833in}}%
\pgfpathlineto{\pgfqpoint{1.125739in}{2.199135in}}%
\pgfpathlineto{\pgfqpoint{1.126654in}{2.199494in}}%
\pgfpathlineto{\pgfqpoint{1.127264in}{2.198888in}}%
\pgfpathlineto{\pgfqpoint{1.129400in}{2.199276in}}%
\pgfpathlineto{\pgfqpoint{1.138859in}{2.197572in}}%
\pgfpathlineto{\pgfqpoint{1.140079in}{2.198639in}}%
\pgfpathlineto{\pgfqpoint{1.143130in}{2.201345in}}%
\pgfpathlineto{\pgfqpoint{1.143435in}{2.201047in}}%
\pgfpathlineto{\pgfqpoint{1.146181in}{2.199387in}}%
\pgfpathlineto{\pgfqpoint{1.148622in}{2.199908in}}%
\pgfpathlineto{\pgfqpoint{1.149538in}{2.200314in}}%
\pgfpathlineto{\pgfqpoint{1.149843in}{2.199778in}}%
\pgfpathlineto{\pgfqpoint{1.151369in}{2.197039in}}%
\pgfpathlineto{\pgfqpoint{1.151979in}{2.197834in}}%
\pgfpathlineto{\pgfqpoint{1.153504in}{2.200934in}}%
\pgfpathlineto{\pgfqpoint{1.154115in}{2.200288in}}%
\pgfpathlineto{\pgfqpoint{1.155640in}{2.198166in}}%
\pgfpathlineto{\pgfqpoint{1.156250in}{2.198799in}}%
\pgfpathlineto{\pgfqpoint{1.158996in}{2.200161in}}%
\pgfpathlineto{\pgfqpoint{1.160522in}{2.198407in}}%
\pgfpathlineto{\pgfqpoint{1.161437in}{2.197238in}}%
\pgfpathlineto{\pgfqpoint{1.162048in}{2.198156in}}%
\pgfpathlineto{\pgfqpoint{1.163878in}{2.200355in}}%
\pgfpathlineto{\pgfqpoint{1.164183in}{2.200061in}}%
\pgfpathlineto{\pgfqpoint{1.165709in}{2.198232in}}%
\pgfpathlineto{\pgfqpoint{1.166319in}{2.198829in}}%
\pgfpathlineto{\pgfqpoint{1.167845in}{2.200855in}}%
\pgfpathlineto{\pgfqpoint{1.168455in}{2.199980in}}%
\pgfpathlineto{\pgfqpoint{1.169676in}{2.198544in}}%
\pgfpathlineto{\pgfqpoint{1.170286in}{2.199437in}}%
\pgfpathlineto{\pgfqpoint{1.171201in}{2.200402in}}%
\pgfpathlineto{\pgfqpoint{1.171811in}{2.199713in}}%
\pgfpathlineto{\pgfqpoint{1.173032in}{2.197703in}}%
\pgfpathlineto{\pgfqpoint{1.173642in}{2.198628in}}%
\pgfpathlineto{\pgfqpoint{1.174863in}{2.200459in}}%
\pgfpathlineto{\pgfqpoint{1.175473in}{2.199815in}}%
\pgfpathlineto{\pgfqpoint{1.176693in}{2.198139in}}%
\pgfpathlineto{\pgfqpoint{1.177304in}{2.199069in}}%
\pgfpathlineto{\pgfqpoint{1.179134in}{2.199952in}}%
\pgfpathlineto{\pgfqpoint{1.179439in}{2.199665in}}%
\pgfpathlineto{\pgfqpoint{1.181880in}{2.199051in}}%
\pgfpathlineto{\pgfqpoint{1.196526in}{2.199561in}}%
\pgfpathlineto{\pgfqpoint{1.199272in}{2.198363in}}%
\pgfpathlineto{\pgfqpoint{1.201408in}{2.200815in}}%
\pgfpathlineto{\pgfqpoint{1.202018in}{2.199980in}}%
\pgfpathlineto{\pgfqpoint{1.203239in}{2.197349in}}%
\pgfpathlineto{\pgfqpoint{1.203849in}{2.198274in}}%
\pgfpathlineto{\pgfqpoint{1.205374in}{2.200506in}}%
\pgfpathlineto{\pgfqpoint{1.205985in}{2.199752in}}%
\pgfpathlineto{\pgfqpoint{1.207815in}{2.199458in}}%
\pgfpathlineto{\pgfqpoint{1.214528in}{2.198576in}}%
\pgfpathlineto{\pgfqpoint{1.219715in}{2.199437in}}%
\pgfpathlineto{\pgfqpoint{1.220630in}{2.199359in}}%
\pgfpathlineto{\pgfqpoint{1.220935in}{2.199836in}}%
\pgfpathlineto{\pgfqpoint{1.221851in}{2.200873in}}%
\pgfpathlineto{\pgfqpoint{1.222461in}{2.200128in}}%
\pgfpathlineto{\pgfqpoint{1.223681in}{2.197127in}}%
\pgfpathlineto{\pgfqpoint{1.224597in}{2.198823in}}%
\pgfpathlineto{\pgfqpoint{1.225512in}{2.200457in}}%
\pgfpathlineto{\pgfqpoint{1.226122in}{2.199669in}}%
\pgfpathlineto{\pgfqpoint{1.227343in}{2.197958in}}%
\pgfpathlineto{\pgfqpoint{1.227953in}{2.198858in}}%
\pgfpathlineto{\pgfqpoint{1.229174in}{2.199934in}}%
\pgfpathlineto{\pgfqpoint{1.229784in}{2.199409in}}%
\pgfpathlineto{\pgfqpoint{1.231920in}{2.199560in}}%
\pgfpathlineto{\pgfqpoint{1.234666in}{2.198693in}}%
\pgfpathlineto{\pgfqpoint{1.235581in}{2.197760in}}%
\pgfpathlineto{\pgfqpoint{1.236191in}{2.198642in}}%
\pgfpathlineto{\pgfqpoint{1.237107in}{2.199754in}}%
\pgfpathlineto{\pgfqpoint{1.238022in}{2.198877in}}%
\pgfpathlineto{\pgfqpoint{1.239243in}{2.199789in}}%
\pgfpathlineto{\pgfqpoint{1.240158in}{2.200114in}}%
\pgfpathlineto{\pgfqpoint{1.240768in}{2.199304in}}%
\pgfpathlineto{\pgfqpoint{1.241683in}{2.198656in}}%
\pgfpathlineto{\pgfqpoint{1.242294in}{2.199704in}}%
\pgfpathlineto{\pgfqpoint{1.243514in}{2.200955in}}%
\pgfpathlineto{\pgfqpoint{1.243819in}{2.200561in}}%
\pgfpathlineto{\pgfqpoint{1.246565in}{2.198586in}}%
\pgfpathlineto{\pgfqpoint{1.247786in}{2.199576in}}%
\pgfpathlineto{\pgfqpoint{1.248396in}{2.199946in}}%
\pgfpathlineto{\pgfqpoint{1.249006in}{2.199119in}}%
\pgfpathlineto{\pgfqpoint{1.250227in}{2.197392in}}%
\pgfpathlineto{\pgfqpoint{1.250837in}{2.198691in}}%
\pgfpathlineto{\pgfqpoint{1.252057in}{2.200921in}}%
\pgfpathlineto{\pgfqpoint{1.252668in}{2.200150in}}%
\pgfpathlineto{\pgfqpoint{1.253888in}{2.198415in}}%
\pgfpathlineto{\pgfqpoint{1.254498in}{2.199502in}}%
\pgfpathlineto{\pgfqpoint{1.255719in}{2.200243in}}%
\pgfpathlineto{\pgfqpoint{1.256024in}{2.199871in}}%
\pgfpathlineto{\pgfqpoint{1.257244in}{2.197922in}}%
\pgfpathlineto{\pgfqpoint{1.257855in}{2.198698in}}%
\pgfpathlineto{\pgfqpoint{1.259075in}{2.199373in}}%
\pgfpathlineto{\pgfqpoint{1.259685in}{2.198793in}}%
\pgfpathlineto{\pgfqpoint{1.261516in}{2.199422in}}%
\pgfpathlineto{\pgfqpoint{1.264567in}{2.199491in}}%
\pgfpathlineto{\pgfqpoint{1.266703in}{2.200149in}}%
\pgfpathlineto{\pgfqpoint{1.270670in}{2.199045in}}%
\pgfpathlineto{\pgfqpoint{1.271585in}{2.197384in}}%
\pgfpathlineto{\pgfqpoint{1.272195in}{2.198278in}}%
\pgfpathlineto{\pgfqpoint{1.273416in}{2.199751in}}%
\pgfpathlineto{\pgfqpoint{1.274026in}{2.199182in}}%
\pgfpathlineto{\pgfqpoint{1.274941in}{2.198768in}}%
\pgfpathlineto{\pgfqpoint{1.275552in}{2.199709in}}%
\pgfpathlineto{\pgfqpoint{1.276772in}{2.200725in}}%
\pgfpathlineto{\pgfqpoint{1.277382in}{2.199873in}}%
\pgfpathlineto{\pgfqpoint{1.278908in}{2.198208in}}%
\pgfpathlineto{\pgfqpoint{1.279518in}{2.198983in}}%
\pgfpathlineto{\pgfqpoint{1.280739in}{2.199495in}}%
\pgfpathlineto{\pgfqpoint{1.281349in}{2.198917in}}%
\pgfpathlineto{\pgfqpoint{1.283180in}{2.199195in}}%
\pgfpathlineto{\pgfqpoint{1.285315in}{2.199446in}}%
\pgfpathlineto{\pgfqpoint{1.287146in}{2.198490in}}%
\pgfpathlineto{\pgfqpoint{1.287451in}{2.198997in}}%
\pgfpathlineto{\pgfqpoint{1.288977in}{2.200834in}}%
\pgfpathlineto{\pgfqpoint{1.289587in}{2.200159in}}%
\pgfpathlineto{\pgfqpoint{1.291723in}{2.199502in}}%
\pgfpathlineto{\pgfqpoint{1.294164in}{2.199211in}}%
\pgfpathlineto{\pgfqpoint{1.295689in}{2.198315in}}%
\pgfpathlineto{\pgfqpoint{1.296300in}{2.198914in}}%
\pgfpathlineto{\pgfqpoint{1.300571in}{2.199760in}}%
\pgfpathlineto{\pgfqpoint{1.301792in}{2.197844in}}%
\pgfpathlineto{\pgfqpoint{1.302402in}{2.198573in}}%
\pgfpathlineto{\pgfqpoint{1.303928in}{2.200416in}}%
\pgfpathlineto{\pgfqpoint{1.304233in}{2.200062in}}%
\pgfpathlineto{\pgfqpoint{1.305453in}{2.197941in}}%
\pgfpathlineto{\pgfqpoint{1.306369in}{2.199104in}}%
\pgfpathlineto{\pgfqpoint{1.308504in}{2.199414in}}%
\pgfpathlineto{\pgfqpoint{1.312776in}{2.198780in}}%
\pgfpathlineto{\pgfqpoint{1.314302in}{2.201070in}}%
\pgfpathlineto{\pgfqpoint{1.314912in}{2.199822in}}%
\pgfpathlineto{\pgfqpoint{1.316437in}{2.197964in}}%
\pgfpathlineto{\pgfqpoint{1.316743in}{2.198418in}}%
\pgfpathlineto{\pgfqpoint{1.317963in}{2.200662in}}%
\pgfpathlineto{\pgfqpoint{1.318573in}{2.200066in}}%
\pgfpathlineto{\pgfqpoint{1.320099in}{2.198381in}}%
\pgfpathlineto{\pgfqpoint{1.320709in}{2.199117in}}%
\pgfpathlineto{\pgfqpoint{1.321930in}{2.199971in}}%
\pgfpathlineto{\pgfqpoint{1.322540in}{2.199108in}}%
\pgfpathlineto{\pgfqpoint{1.323455in}{2.198237in}}%
\pgfpathlineto{\pgfqpoint{1.324065in}{2.199087in}}%
\pgfpathlineto{\pgfqpoint{1.325286in}{2.200231in}}%
\pgfpathlineto{\pgfqpoint{1.325896in}{2.199415in}}%
\pgfpathlineto{\pgfqpoint{1.327117in}{2.198655in}}%
\pgfpathlineto{\pgfqpoint{1.327727in}{2.199286in}}%
\pgfpathlineto{\pgfqpoint{1.328947in}{2.199744in}}%
\pgfpathlineto{\pgfqpoint{1.329557in}{2.198980in}}%
\pgfpathlineto{\pgfqpoint{1.330778in}{2.198169in}}%
\pgfpathlineto{\pgfqpoint{1.331083in}{2.198519in}}%
\pgfpathlineto{\pgfqpoint{1.332609in}{2.200528in}}%
\pgfpathlineto{\pgfqpoint{1.333219in}{2.199759in}}%
\pgfpathlineto{\pgfqpoint{1.334744in}{2.198499in}}%
\pgfpathlineto{\pgfqpoint{1.335355in}{2.199047in}}%
\pgfpathlineto{\pgfqpoint{1.336880in}{2.200049in}}%
\pgfpathlineto{\pgfqpoint{1.337185in}{2.199689in}}%
\pgfpathlineto{\pgfqpoint{1.338711in}{2.198877in}}%
\pgfpathlineto{\pgfqpoint{1.339016in}{2.199182in}}%
\pgfpathlineto{\pgfqpoint{1.340847in}{2.199934in}}%
\pgfpathlineto{\pgfqpoint{1.341152in}{2.199652in}}%
\pgfpathlineto{\pgfqpoint{1.343593in}{2.199254in}}%
\pgfpathlineto{\pgfqpoint{1.345424in}{2.199125in}}%
\pgfpathlineto{\pgfqpoint{1.346644in}{2.198774in}}%
\pgfpathlineto{\pgfqpoint{1.346949in}{2.199176in}}%
\pgfpathlineto{\pgfqpoint{1.347865in}{2.199693in}}%
\pgfpathlineto{\pgfqpoint{1.348475in}{2.198879in}}%
\pgfpathlineto{\pgfqpoint{1.349390in}{2.198194in}}%
\pgfpathlineto{\pgfqpoint{1.350000in}{2.199205in}}%
\pgfpathlineto{\pgfqpoint{1.351221in}{2.201444in}}%
\pgfpathlineto{\pgfqpoint{1.351831in}{2.200614in}}%
\pgfpathlineto{\pgfqpoint{1.353662in}{2.197465in}}%
\pgfpathlineto{\pgfqpoint{1.354272in}{2.198226in}}%
\pgfpathlineto{\pgfqpoint{1.355798in}{2.200256in}}%
\pgfpathlineto{\pgfqpoint{1.356408in}{2.199403in}}%
\pgfpathlineto{\pgfqpoint{1.357323in}{2.198804in}}%
\pgfpathlineto{\pgfqpoint{1.357933in}{2.199429in}}%
\pgfpathlineto{\pgfqpoint{1.359154in}{2.199757in}}%
\pgfpathlineto{\pgfqpoint{1.359459in}{2.199452in}}%
\pgfpathlineto{\pgfqpoint{1.361595in}{2.199343in}}%
\pgfpathlineto{\pgfqpoint{1.363731in}{2.199353in}}%
\pgfpathlineto{\pgfqpoint{1.365561in}{2.198926in}}%
\pgfpathlineto{\pgfqpoint{1.365867in}{2.199322in}}%
\pgfpathlineto{\pgfqpoint{1.367087in}{2.200539in}}%
\pgfpathlineto{\pgfqpoint{1.367697in}{2.199598in}}%
\pgfpathlineto{\pgfqpoint{1.368613in}{2.198745in}}%
\pgfpathlineto{\pgfqpoint{1.369223in}{2.199321in}}%
\pgfpathlineto{\pgfqpoint{1.370748in}{2.198692in}}%
\pgfpathlineto{\pgfqpoint{1.371969in}{2.199336in}}%
\pgfpathlineto{\pgfqpoint{1.372884in}{2.200218in}}%
\pgfpathlineto{\pgfqpoint{1.373494in}{2.199583in}}%
\pgfpathlineto{\pgfqpoint{1.375020in}{2.198784in}}%
\pgfpathlineto{\pgfqpoint{1.375325in}{2.198983in}}%
\pgfpathlineto{\pgfqpoint{1.377766in}{2.198929in}}%
\pgfpathlineto{\pgfqpoint{1.379597in}{2.198618in}}%
\pgfpathlineto{\pgfqpoint{1.379902in}{2.199168in}}%
\pgfpathlineto{\pgfqpoint{1.381122in}{2.200837in}}%
\pgfpathlineto{\pgfqpoint{1.381733in}{2.200199in}}%
\pgfpathlineto{\pgfqpoint{1.383258in}{2.198402in}}%
\pgfpathlineto{\pgfqpoint{1.383869in}{2.199257in}}%
\pgfpathlineto{\pgfqpoint{1.384784in}{2.199936in}}%
\pgfpathlineto{\pgfqpoint{1.385394in}{2.198898in}}%
\pgfpathlineto{\pgfqpoint{1.386309in}{2.197841in}}%
\pgfpathlineto{\pgfqpoint{1.386920in}{2.199067in}}%
\pgfpathlineto{\pgfqpoint{1.387835in}{2.201217in}}%
\pgfpathlineto{\pgfqpoint{1.388445in}{2.200460in}}%
\pgfpathlineto{\pgfqpoint{1.389666in}{2.197904in}}%
\pgfpathlineto{\pgfqpoint{1.390276in}{2.198841in}}%
\pgfpathlineto{\pgfqpoint{1.391191in}{2.200256in}}%
\pgfpathlineto{\pgfqpoint{1.392107in}{2.199436in}}%
\pgfpathlineto{\pgfqpoint{1.395463in}{2.199012in}}%
\pgfpathlineto{\pgfqpoint{1.396683in}{2.199552in}}%
\pgfpathlineto{\pgfqpoint{1.397294in}{2.198970in}}%
\pgfpathlineto{\pgfqpoint{1.398209in}{2.199203in}}%
\pgfpathlineto{\pgfqpoint{1.398514in}{2.199858in}}%
\pgfpathlineto{\pgfqpoint{1.399430in}{2.200990in}}%
\pgfpathlineto{\pgfqpoint{1.400040in}{2.199548in}}%
\pgfpathlineto{\pgfqpoint{1.400955in}{2.197423in}}%
\pgfpathlineto{\pgfqpoint{1.401565in}{2.198714in}}%
\pgfpathlineto{\pgfqpoint{1.402481in}{2.201074in}}%
\pgfpathlineto{\pgfqpoint{1.403091in}{2.200113in}}%
\pgfpathlineto{\pgfqpoint{1.404311in}{2.197454in}}%
\pgfpathlineto{\pgfqpoint{1.404922in}{2.198775in}}%
\pgfpathlineto{\pgfqpoint{1.405837in}{2.200648in}}%
\pgfpathlineto{\pgfqpoint{1.406447in}{2.199744in}}%
\pgfpathlineto{\pgfqpoint{1.407668in}{2.197591in}}%
\pgfpathlineto{\pgfqpoint{1.408278in}{2.198673in}}%
\pgfpathlineto{\pgfqpoint{1.409498in}{2.200716in}}%
\pgfpathlineto{\pgfqpoint{1.410109in}{2.199992in}}%
\pgfpathlineto{\pgfqpoint{1.412244in}{2.199427in}}%
\pgfpathlineto{\pgfqpoint{1.420483in}{2.200057in}}%
\pgfpathlineto{\pgfqpoint{1.421093in}{2.200690in}}%
\pgfpathlineto{\pgfqpoint{1.421703in}{2.199598in}}%
\pgfpathlineto{\pgfqpoint{1.422619in}{2.199104in}}%
\pgfpathlineto{\pgfqpoint{1.423229in}{2.199741in}}%
\pgfpathlineto{\pgfqpoint{1.424144in}{2.200035in}}%
\pgfpathlineto{\pgfqpoint{1.424754in}{2.199130in}}%
\pgfpathlineto{\pgfqpoint{1.425670in}{2.198747in}}%
\pgfpathlineto{\pgfqpoint{1.426280in}{2.199259in}}%
\pgfpathlineto{\pgfqpoint{1.427500in}{2.198643in}}%
\pgfpathlineto{\pgfqpoint{1.428416in}{2.198011in}}%
\pgfpathlineto{\pgfqpoint{1.428721in}{2.198645in}}%
\pgfpathlineto{\pgfqpoint{1.429941in}{2.200444in}}%
\pgfpathlineto{\pgfqpoint{1.430552in}{2.199797in}}%
\pgfpathlineto{\pgfqpoint{1.431772in}{2.198683in}}%
\pgfpathlineto{\pgfqpoint{1.432382in}{2.199775in}}%
\pgfpathlineto{\pgfqpoint{1.432993in}{2.200317in}}%
\pgfpathlineto{\pgfqpoint{1.433603in}{2.199665in}}%
\pgfpathlineto{\pgfqpoint{1.435128in}{2.199108in}}%
\pgfpathlineto{\pgfqpoint{1.435433in}{2.199434in}}%
\pgfpathlineto{\pgfqpoint{1.436959in}{2.198884in}}%
\pgfpathlineto{\pgfqpoint{1.438485in}{2.198640in}}%
\pgfpathlineto{\pgfqpoint{1.438790in}{2.199033in}}%
\pgfpathlineto{\pgfqpoint{1.440010in}{2.200032in}}%
\pgfpathlineto{\pgfqpoint{1.440620in}{2.199508in}}%
\pgfpathlineto{\pgfqpoint{1.442756in}{2.199224in}}%
\pgfpathlineto{\pgfqpoint{1.443977in}{2.200444in}}%
\pgfpathlineto{\pgfqpoint{1.444587in}{2.199654in}}%
\pgfpathlineto{\pgfqpoint{1.445807in}{2.198298in}}%
\pgfpathlineto{\pgfqpoint{1.446418in}{2.199294in}}%
\pgfpathlineto{\pgfqpoint{1.447333in}{2.200647in}}%
\pgfpathlineto{\pgfqpoint{1.447943in}{2.199550in}}%
\pgfpathlineto{\pgfqpoint{1.449164in}{2.197754in}}%
\pgfpathlineto{\pgfqpoint{1.449469in}{2.198258in}}%
\pgfpathlineto{\pgfqpoint{1.450689in}{2.201203in}}%
\pgfpathlineto{\pgfqpoint{1.451300in}{2.200250in}}%
\pgfpathlineto{\pgfqpoint{1.452520in}{2.196845in}}%
\pgfpathlineto{\pgfqpoint{1.453130in}{2.197577in}}%
\pgfpathlineto{\pgfqpoint{1.454351in}{2.201445in}}%
\pgfpathlineto{\pgfqpoint{1.455266in}{2.199995in}}%
\pgfpathlineto{\pgfqpoint{1.457402in}{2.199183in}}%
\pgfpathlineto{\pgfqpoint{1.464725in}{2.199945in}}%
\pgfpathlineto{\pgfqpoint{1.465640in}{2.201107in}}%
\pgfpathlineto{\pgfqpoint{1.466250in}{2.199621in}}%
\pgfpathlineto{\pgfqpoint{1.467166in}{2.197623in}}%
\pgfpathlineto{\pgfqpoint{1.468081in}{2.198904in}}%
\pgfpathlineto{\pgfqpoint{1.468996in}{2.200942in}}%
\pgfpathlineto{\pgfqpoint{1.469607in}{2.200155in}}%
\pgfpathlineto{\pgfqpoint{1.470827in}{2.197999in}}%
\pgfpathlineto{\pgfqpoint{1.471437in}{2.198577in}}%
\pgfpathlineto{\pgfqpoint{1.474183in}{2.199944in}}%
\pgfpathlineto{\pgfqpoint{1.475709in}{2.198384in}}%
\pgfpathlineto{\pgfqpoint{1.476319in}{2.197903in}}%
\pgfpathlineto{\pgfqpoint{1.476930in}{2.199271in}}%
\pgfpathlineto{\pgfqpoint{1.477845in}{2.200838in}}%
\pgfpathlineto{\pgfqpoint{1.478455in}{2.200054in}}%
\pgfpathlineto{\pgfqpoint{1.479676in}{2.198139in}}%
\pgfpathlineto{\pgfqpoint{1.480286in}{2.198962in}}%
\pgfpathlineto{\pgfqpoint{1.481201in}{2.200037in}}%
\pgfpathlineto{\pgfqpoint{1.481811in}{2.199360in}}%
\pgfpathlineto{\pgfqpoint{1.483032in}{2.198658in}}%
\pgfpathlineto{\pgfqpoint{1.483642in}{2.199345in}}%
\pgfpathlineto{\pgfqpoint{1.485168in}{2.199050in}}%
\pgfpathlineto{\pgfqpoint{1.486998in}{2.199188in}}%
\pgfpathlineto{\pgfqpoint{1.488219in}{2.200870in}}%
\pgfpathlineto{\pgfqpoint{1.488829in}{2.200204in}}%
\pgfpathlineto{\pgfqpoint{1.490965in}{2.198359in}}%
\pgfpathlineto{\pgfqpoint{1.491270in}{2.198614in}}%
\pgfpathlineto{\pgfqpoint{1.492796in}{2.199550in}}%
\pgfpathlineto{\pgfqpoint{1.493101in}{2.199210in}}%
\pgfpathlineto{\pgfqpoint{1.494321in}{2.198367in}}%
\pgfpathlineto{\pgfqpoint{1.494626in}{2.198788in}}%
\pgfpathlineto{\pgfqpoint{1.495847in}{2.200838in}}%
\pgfpathlineto{\pgfqpoint{1.496457in}{2.199678in}}%
\pgfpathlineto{\pgfqpoint{1.497678in}{2.198552in}}%
\pgfpathlineto{\pgfqpoint{1.497983in}{2.198940in}}%
\pgfpathlineto{\pgfqpoint{1.499203in}{2.200844in}}%
\pgfpathlineto{\pgfqpoint{1.499813in}{2.199741in}}%
\pgfpathlineto{\pgfqpoint{1.501034in}{2.198447in}}%
\pgfpathlineto{\pgfqpoint{1.501644in}{2.198962in}}%
\pgfpathlineto{\pgfqpoint{1.502865in}{2.198978in}}%
\pgfpathlineto{\pgfqpoint{1.503170in}{2.198600in}}%
\pgfpathlineto{\pgfqpoint{1.504695in}{2.198459in}}%
\pgfpathlineto{\pgfqpoint{1.505000in}{2.198805in}}%
\pgfpathlineto{\pgfqpoint{1.506831in}{2.200794in}}%
\pgfpathlineto{\pgfqpoint{1.507441in}{2.200096in}}%
\pgfpathlineto{\pgfqpoint{1.509577in}{2.199300in}}%
\pgfpathlineto{\pgfqpoint{1.518731in}{2.199235in}}%
\pgfpathlineto{\pgfqpoint{1.521782in}{2.199510in}}%
\pgfpathlineto{\pgfqpoint{1.524223in}{2.198434in}}%
\pgfpathlineto{\pgfqpoint{1.526664in}{2.199111in}}%
\pgfpathlineto{\pgfqpoint{1.529715in}{2.198746in}}%
\pgfpathlineto{\pgfqpoint{1.530325in}{2.198251in}}%
\pgfpathlineto{\pgfqpoint{1.530935in}{2.199355in}}%
\pgfpathlineto{\pgfqpoint{1.531851in}{2.201349in}}%
\pgfpathlineto{\pgfqpoint{1.532461in}{2.200305in}}%
\pgfpathlineto{\pgfqpoint{1.533987in}{2.196570in}}%
\pgfpathlineto{\pgfqpoint{1.534597in}{2.197837in}}%
\pgfpathlineto{\pgfqpoint{1.535817in}{2.200851in}}%
\pgfpathlineto{\pgfqpoint{1.536428in}{2.199892in}}%
\pgfpathlineto{\pgfqpoint{1.537648in}{2.197968in}}%
\pgfpathlineto{\pgfqpoint{1.538258in}{2.198675in}}%
\pgfpathlineto{\pgfqpoint{1.539479in}{2.200603in}}%
\pgfpathlineto{\pgfqpoint{1.540394in}{2.199810in}}%
\pgfpathlineto{\pgfqpoint{1.544056in}{2.197785in}}%
\pgfpathlineto{\pgfqpoint{1.545276in}{2.198170in}}%
\pgfpathlineto{\pgfqpoint{1.547107in}{2.200749in}}%
\pgfpathlineto{\pgfqpoint{1.547717in}{2.199483in}}%
\pgfpathlineto{\pgfqpoint{1.548632in}{2.198026in}}%
\pgfpathlineto{\pgfqpoint{1.549243in}{2.198978in}}%
\pgfpathlineto{\pgfqpoint{1.550158in}{2.200785in}}%
\pgfpathlineto{\pgfqpoint{1.550768in}{2.200084in}}%
\pgfpathlineto{\pgfqpoint{1.551989in}{2.197815in}}%
\pgfpathlineto{\pgfqpoint{1.552599in}{2.198934in}}%
\pgfpathlineto{\pgfqpoint{1.553514in}{2.200815in}}%
\pgfpathlineto{\pgfqpoint{1.554124in}{2.199952in}}%
\pgfpathlineto{\pgfqpoint{1.555345in}{2.198076in}}%
\pgfpathlineto{\pgfqpoint{1.555955in}{2.198916in}}%
\pgfpathlineto{\pgfqpoint{1.557176in}{2.200021in}}%
\pgfpathlineto{\pgfqpoint{1.557786in}{2.199087in}}%
\pgfpathlineto{\pgfqpoint{1.559006in}{2.198106in}}%
\pgfpathlineto{\pgfqpoint{1.559617in}{2.198699in}}%
\pgfpathlineto{\pgfqpoint{1.561447in}{2.200749in}}%
\pgfpathlineto{\pgfqpoint{1.562057in}{2.199961in}}%
\pgfpathlineto{\pgfqpoint{1.563583in}{2.198424in}}%
\pgfpathlineto{\pgfqpoint{1.564193in}{2.199377in}}%
\pgfpathlineto{\pgfqpoint{1.565109in}{2.200470in}}%
\pgfpathlineto{\pgfqpoint{1.565719in}{2.199773in}}%
\pgfpathlineto{\pgfqpoint{1.567244in}{2.197779in}}%
\pgfpathlineto{\pgfqpoint{1.567550in}{2.198222in}}%
\pgfpathlineto{\pgfqpoint{1.568770in}{2.200190in}}%
\pgfpathlineto{\pgfqpoint{1.569380in}{2.199336in}}%
\pgfpathlineto{\pgfqpoint{1.570296in}{2.197699in}}%
\pgfpathlineto{\pgfqpoint{1.570906in}{2.198571in}}%
\pgfpathlineto{\pgfqpoint{1.572126in}{2.202189in}}%
\pgfpathlineto{\pgfqpoint{1.572737in}{2.201185in}}%
\pgfpathlineto{\pgfqpoint{1.573957in}{2.197615in}}%
\pgfpathlineto{\pgfqpoint{1.574872in}{2.198835in}}%
\pgfpathlineto{\pgfqpoint{1.576398in}{2.199667in}}%
\pgfpathlineto{\pgfqpoint{1.576703in}{2.199439in}}%
\pgfpathlineto{\pgfqpoint{1.579449in}{2.199189in}}%
\pgfpathlineto{\pgfqpoint{1.580670in}{2.199970in}}%
\pgfpathlineto{\pgfqpoint{1.580975in}{2.199527in}}%
\pgfpathlineto{\pgfqpoint{1.582195in}{2.197835in}}%
\pgfpathlineto{\pgfqpoint{1.582806in}{2.198819in}}%
\pgfpathlineto{\pgfqpoint{1.584331in}{2.201016in}}%
\pgfpathlineto{\pgfqpoint{1.584636in}{2.200359in}}%
\pgfpathlineto{\pgfqpoint{1.585552in}{2.198140in}}%
\pgfpathlineto{\pgfqpoint{1.586467in}{2.199305in}}%
\pgfpathlineto{\pgfqpoint{1.587382in}{2.200236in}}%
\pgfpathlineto{\pgfqpoint{1.587993in}{2.199200in}}%
\pgfpathlineto{\pgfqpoint{1.588908in}{2.197829in}}%
\pgfpathlineto{\pgfqpoint{1.589518in}{2.198801in}}%
\pgfpathlineto{\pgfqpoint{1.591044in}{2.201280in}}%
\pgfpathlineto{\pgfqpoint{1.591349in}{2.200782in}}%
\pgfpathlineto{\pgfqpoint{1.592874in}{2.197396in}}%
\pgfpathlineto{\pgfqpoint{1.593485in}{2.198291in}}%
\pgfpathlineto{\pgfqpoint{1.594705in}{2.200504in}}%
\pgfpathlineto{\pgfqpoint{1.595315in}{2.199686in}}%
\pgfpathlineto{\pgfqpoint{1.596536in}{2.197992in}}%
\pgfpathlineto{\pgfqpoint{1.596841in}{2.198525in}}%
\pgfpathlineto{\pgfqpoint{1.598061in}{2.201066in}}%
\pgfpathlineto{\pgfqpoint{1.598672in}{2.200466in}}%
\pgfpathlineto{\pgfqpoint{1.600197in}{2.197641in}}%
\pgfpathlineto{\pgfqpoint{1.600807in}{2.198549in}}%
\pgfpathlineto{\pgfqpoint{1.602028in}{2.199866in}}%
\pgfpathlineto{\pgfqpoint{1.602333in}{2.199421in}}%
\pgfpathlineto{\pgfqpoint{1.603554in}{2.198265in}}%
\pgfpathlineto{\pgfqpoint{1.603859in}{2.198717in}}%
\pgfpathlineto{\pgfqpoint{1.605384in}{2.201419in}}%
\pgfpathlineto{\pgfqpoint{1.605994in}{2.200129in}}%
\pgfpathlineto{\pgfqpoint{1.607215in}{2.197526in}}%
\pgfpathlineto{\pgfqpoint{1.607825in}{2.198685in}}%
\pgfpathlineto{\pgfqpoint{1.609351in}{2.200321in}}%
\pgfpathlineto{\pgfqpoint{1.609961in}{2.199747in}}%
\pgfpathlineto{\pgfqpoint{1.612097in}{2.198398in}}%
\pgfpathlineto{\pgfqpoint{1.612402in}{2.198649in}}%
\pgfpathlineto{\pgfqpoint{1.614538in}{2.200278in}}%
\pgfpathlineto{\pgfqpoint{1.615148in}{2.199449in}}%
\pgfpathlineto{\pgfqpoint{1.617894in}{2.198667in}}%
\pgfpathlineto{\pgfqpoint{1.627353in}{2.199514in}}%
\pgfpathlineto{\pgfqpoint{1.628573in}{2.200266in}}%
\pgfpathlineto{\pgfqpoint{1.629183in}{2.199326in}}%
\pgfpathlineto{\pgfqpoint{1.630099in}{2.198154in}}%
\pgfpathlineto{\pgfqpoint{1.630709in}{2.199036in}}%
\pgfpathlineto{\pgfqpoint{1.632235in}{2.200154in}}%
\pgfpathlineto{\pgfqpoint{1.632540in}{2.199710in}}%
\pgfpathlineto{\pgfqpoint{1.633760in}{2.198373in}}%
\pgfpathlineto{\pgfqpoint{1.634370in}{2.199232in}}%
\pgfpathlineto{\pgfqpoint{1.635591in}{2.200120in}}%
\pgfpathlineto{\pgfqpoint{1.636201in}{2.199261in}}%
\pgfpathlineto{\pgfqpoint{1.637422in}{2.198459in}}%
\pgfpathlineto{\pgfqpoint{1.638032in}{2.198990in}}%
\pgfpathlineto{\pgfqpoint{1.640168in}{2.200126in}}%
\pgfpathlineto{\pgfqpoint{1.640473in}{2.199784in}}%
\pgfpathlineto{\pgfqpoint{1.642609in}{2.199444in}}%
\pgfpathlineto{\pgfqpoint{1.644134in}{2.198475in}}%
\pgfpathlineto{\pgfqpoint{1.645050in}{2.197872in}}%
\pgfpathlineto{\pgfqpoint{1.645355in}{2.198346in}}%
\pgfpathlineto{\pgfqpoint{1.646880in}{2.200684in}}%
\pgfpathlineto{\pgfqpoint{1.647491in}{2.199667in}}%
\pgfpathlineto{\pgfqpoint{1.648406in}{2.198442in}}%
\pgfpathlineto{\pgfqpoint{1.649016in}{2.199317in}}%
\pgfpathlineto{\pgfqpoint{1.650237in}{2.199887in}}%
\pgfpathlineto{\pgfqpoint{1.650542in}{2.199548in}}%
\pgfpathlineto{\pgfqpoint{1.651762in}{2.198440in}}%
\pgfpathlineto{\pgfqpoint{1.652372in}{2.199032in}}%
\pgfpathlineto{\pgfqpoint{1.656034in}{2.200193in}}%
\pgfpathlineto{\pgfqpoint{1.659085in}{2.198041in}}%
\pgfpathlineto{\pgfqpoint{1.660916in}{2.198548in}}%
\pgfpathlineto{\pgfqpoint{1.662746in}{2.200905in}}%
\pgfpathlineto{\pgfqpoint{1.663357in}{2.200445in}}%
\pgfpathlineto{\pgfqpoint{1.666713in}{2.198587in}}%
\pgfpathlineto{\pgfqpoint{1.667018in}{2.199012in}}%
\pgfpathlineto{\pgfqpoint{1.668239in}{2.199861in}}%
\pgfpathlineto{\pgfqpoint{1.668849in}{2.198949in}}%
\pgfpathlineto{\pgfqpoint{1.669764in}{2.197600in}}%
\pgfpathlineto{\pgfqpoint{1.670374in}{2.198656in}}%
\pgfpathlineto{\pgfqpoint{1.671595in}{2.200716in}}%
\pgfpathlineto{\pgfqpoint{1.672205in}{2.199932in}}%
\pgfpathlineto{\pgfqpoint{1.674646in}{2.198600in}}%
\pgfpathlineto{\pgfqpoint{1.682884in}{2.199599in}}%
\pgfpathlineto{\pgfqpoint{1.684410in}{2.197836in}}%
\pgfpathlineto{\pgfqpoint{1.685020in}{2.198879in}}%
\pgfpathlineto{\pgfqpoint{1.686851in}{2.200920in}}%
\pgfpathlineto{\pgfqpoint{1.687156in}{2.200557in}}%
\pgfpathlineto{\pgfqpoint{1.688987in}{2.198528in}}%
\pgfpathlineto{\pgfqpoint{1.689292in}{2.198855in}}%
\pgfpathlineto{\pgfqpoint{1.690817in}{2.200545in}}%
\pgfpathlineto{\pgfqpoint{1.691428in}{2.199579in}}%
\pgfpathlineto{\pgfqpoint{1.692953in}{2.197418in}}%
\pgfpathlineto{\pgfqpoint{1.693563in}{2.197906in}}%
\pgfpathlineto{\pgfqpoint{1.696920in}{2.200508in}}%
\pgfpathlineto{\pgfqpoint{1.700276in}{2.199959in}}%
\pgfpathlineto{\pgfqpoint{1.701496in}{2.198733in}}%
\pgfpathlineto{\pgfqpoint{1.702412in}{2.196831in}}%
\pgfpathlineto{\pgfqpoint{1.703022in}{2.197707in}}%
\pgfpathlineto{\pgfqpoint{1.704548in}{2.201228in}}%
\pgfpathlineto{\pgfqpoint{1.705158in}{2.200085in}}%
\pgfpathlineto{\pgfqpoint{1.706378in}{2.198420in}}%
\pgfpathlineto{\pgfqpoint{1.706989in}{2.199018in}}%
\pgfpathlineto{\pgfqpoint{1.709124in}{2.199160in}}%
\pgfpathlineto{\pgfqpoint{1.717973in}{2.199113in}}%
\pgfpathlineto{\pgfqpoint{1.719498in}{2.200092in}}%
\pgfpathlineto{\pgfqpoint{1.719804in}{2.199724in}}%
\pgfpathlineto{\pgfqpoint{1.721024in}{2.198297in}}%
\pgfpathlineto{\pgfqpoint{1.721634in}{2.199268in}}%
\pgfpathlineto{\pgfqpoint{1.722855in}{2.200863in}}%
\pgfpathlineto{\pgfqpoint{1.723465in}{2.200175in}}%
\pgfpathlineto{\pgfqpoint{1.724991in}{2.197917in}}%
\pgfpathlineto{\pgfqpoint{1.725601in}{2.198555in}}%
\pgfpathlineto{\pgfqpoint{1.727737in}{2.199241in}}%
\pgfpathlineto{\pgfqpoint{1.729872in}{2.199327in}}%
\pgfpathlineto{\pgfqpoint{1.732008in}{2.199610in}}%
\pgfpathlineto{\pgfqpoint{1.735975in}{2.198955in}}%
\pgfpathlineto{\pgfqpoint{1.737806in}{2.199459in}}%
\pgfpathlineto{\pgfqpoint{1.738111in}{2.199086in}}%
\pgfpathlineto{\pgfqpoint{1.739331in}{2.198369in}}%
\pgfpathlineto{\pgfqpoint{1.739636in}{2.198775in}}%
\pgfpathlineto{\pgfqpoint{1.741162in}{2.200624in}}%
\pgfpathlineto{\pgfqpoint{1.741772in}{2.199859in}}%
\pgfpathlineto{\pgfqpoint{1.743603in}{2.197940in}}%
\pgfpathlineto{\pgfqpoint{1.743908in}{2.198317in}}%
\pgfpathlineto{\pgfqpoint{1.745739in}{2.200643in}}%
\pgfpathlineto{\pgfqpoint{1.746349in}{2.199574in}}%
\pgfpathlineto{\pgfqpoint{1.747264in}{2.198789in}}%
\pgfpathlineto{\pgfqpoint{1.747874in}{2.199553in}}%
\pgfpathlineto{\pgfqpoint{1.748790in}{2.200408in}}%
\pgfpathlineto{\pgfqpoint{1.749400in}{2.199469in}}%
\pgfpathlineto{\pgfqpoint{1.750620in}{2.197598in}}%
\pgfpathlineto{\pgfqpoint{1.751231in}{2.198414in}}%
\pgfpathlineto{\pgfqpoint{1.752756in}{2.199558in}}%
\pgfpathlineto{\pgfqpoint{1.753367in}{2.199155in}}%
\pgfpathlineto{\pgfqpoint{1.764961in}{2.199652in}}%
\pgfpathlineto{\pgfqpoint{1.767097in}{2.198927in}}%
\pgfpathlineto{\pgfqpoint{1.768012in}{2.197509in}}%
\pgfpathlineto{\pgfqpoint{1.768622in}{2.198599in}}%
\pgfpathlineto{\pgfqpoint{1.769843in}{2.201030in}}%
\pgfpathlineto{\pgfqpoint{1.770453in}{2.200158in}}%
\pgfpathlineto{\pgfqpoint{1.771674in}{2.198314in}}%
\pgfpathlineto{\pgfqpoint{1.772284in}{2.198908in}}%
\pgfpathlineto{\pgfqpoint{1.773504in}{2.200008in}}%
\pgfpathlineto{\pgfqpoint{1.774115in}{2.199339in}}%
\pgfpathlineto{\pgfqpoint{1.775945in}{2.198447in}}%
\pgfpathlineto{\pgfqpoint{1.776250in}{2.198668in}}%
\pgfpathlineto{\pgfqpoint{1.778081in}{2.200587in}}%
\pgfpathlineto{\pgfqpoint{1.778691in}{2.199556in}}%
\pgfpathlineto{\pgfqpoint{1.779912in}{2.197908in}}%
\pgfpathlineto{\pgfqpoint{1.780522in}{2.198998in}}%
\pgfpathlineto{\pgfqpoint{1.781743in}{2.201021in}}%
\pgfpathlineto{\pgfqpoint{1.782353in}{2.199650in}}%
\pgfpathlineto{\pgfqpoint{1.783268in}{2.197957in}}%
\pgfpathlineto{\pgfqpoint{1.784183in}{2.198995in}}%
\pgfpathlineto{\pgfqpoint{1.785099in}{2.199704in}}%
\pgfpathlineto{\pgfqpoint{1.785709in}{2.199100in}}%
\pgfpathlineto{\pgfqpoint{1.786624in}{2.198160in}}%
\pgfpathlineto{\pgfqpoint{1.787235in}{2.198954in}}%
\pgfpathlineto{\pgfqpoint{1.788455in}{2.200845in}}%
\pgfpathlineto{\pgfqpoint{1.789065in}{2.200145in}}%
\pgfpathlineto{\pgfqpoint{1.790286in}{2.198615in}}%
\pgfpathlineto{\pgfqpoint{1.790896in}{2.199444in}}%
\pgfpathlineto{\pgfqpoint{1.791811in}{2.200376in}}%
\pgfpathlineto{\pgfqpoint{1.792422in}{2.199170in}}%
\pgfpathlineto{\pgfqpoint{1.793642in}{2.197283in}}%
\pgfpathlineto{\pgfqpoint{1.794252in}{2.198118in}}%
\pgfpathlineto{\pgfqpoint{1.796083in}{2.200947in}}%
\pgfpathlineto{\pgfqpoint{1.796693in}{2.200397in}}%
\pgfpathlineto{\pgfqpoint{1.799439in}{2.199168in}}%
\pgfpathlineto{\pgfqpoint{1.807067in}{2.199321in}}%
\pgfpathlineto{\pgfqpoint{1.808288in}{2.198073in}}%
\pgfpathlineto{\pgfqpoint{1.808898in}{2.198853in}}%
\pgfpathlineto{\pgfqpoint{1.810424in}{2.199548in}}%
\pgfpathlineto{\pgfqpoint{1.810729in}{2.199316in}}%
\pgfpathlineto{\pgfqpoint{1.811949in}{2.199113in}}%
\pgfpathlineto{\pgfqpoint{1.812254in}{2.199505in}}%
\pgfpathlineto{\pgfqpoint{1.813475in}{2.200755in}}%
\pgfpathlineto{\pgfqpoint{1.814085in}{2.199977in}}%
\pgfpathlineto{\pgfqpoint{1.815916in}{2.198179in}}%
\pgfpathlineto{\pgfqpoint{1.816221in}{2.198368in}}%
\pgfpathlineto{\pgfqpoint{1.818357in}{2.200263in}}%
\pgfpathlineto{\pgfqpoint{1.818967in}{2.199199in}}%
\pgfpathlineto{\pgfqpoint{1.820187in}{2.197899in}}%
\pgfpathlineto{\pgfqpoint{1.820493in}{2.198310in}}%
\pgfpathlineto{\pgfqpoint{1.822018in}{2.201542in}}%
\pgfpathlineto{\pgfqpoint{1.822628in}{2.200612in}}%
\pgfpathlineto{\pgfqpoint{1.824154in}{2.197469in}}%
\pgfpathlineto{\pgfqpoint{1.824764in}{2.198590in}}%
\pgfpathlineto{\pgfqpoint{1.825985in}{2.199989in}}%
\pgfpathlineto{\pgfqpoint{1.826595in}{2.199549in}}%
\pgfpathlineto{\pgfqpoint{1.829951in}{2.198768in}}%
\pgfpathlineto{\pgfqpoint{1.832392in}{2.200313in}}%
\pgfpathlineto{\pgfqpoint{1.833002in}{2.199219in}}%
\pgfpathlineto{\pgfqpoint{1.834223in}{2.197618in}}%
\pgfpathlineto{\pgfqpoint{1.834833in}{2.198555in}}%
\pgfpathlineto{\pgfqpoint{1.836054in}{2.200723in}}%
\pgfpathlineto{\pgfqpoint{1.836664in}{2.200052in}}%
\pgfpathlineto{\pgfqpoint{1.837884in}{2.198037in}}%
\pgfpathlineto{\pgfqpoint{1.838494in}{2.198830in}}%
\pgfpathlineto{\pgfqpoint{1.839715in}{2.199974in}}%
\pgfpathlineto{\pgfqpoint{1.840325in}{2.199343in}}%
\pgfpathlineto{\pgfqpoint{1.842156in}{2.199282in}}%
\pgfpathlineto{\pgfqpoint{1.846428in}{2.200452in}}%
\pgfpathlineto{\pgfqpoint{1.849479in}{2.197702in}}%
\pgfpathlineto{\pgfqpoint{1.850089in}{2.199040in}}%
\pgfpathlineto{\pgfqpoint{1.851004in}{2.200372in}}%
\pgfpathlineto{\pgfqpoint{1.851615in}{2.199726in}}%
\pgfpathlineto{\pgfqpoint{1.852835in}{2.198229in}}%
\pgfpathlineto{\pgfqpoint{1.853445in}{2.199099in}}%
\pgfpathlineto{\pgfqpoint{1.854971in}{2.201193in}}%
\pgfpathlineto{\pgfqpoint{1.855276in}{2.200774in}}%
\pgfpathlineto{\pgfqpoint{1.856802in}{2.197890in}}%
\pgfpathlineto{\pgfqpoint{1.857717in}{2.198678in}}%
\pgfpathlineto{\pgfqpoint{1.861073in}{2.199071in}}%
\pgfpathlineto{\pgfqpoint{1.862904in}{2.199464in}}%
\pgfpathlineto{\pgfqpoint{1.866565in}{2.199461in}}%
\pgfpathlineto{\pgfqpoint{1.868091in}{2.199141in}}%
\pgfpathlineto{\pgfqpoint{1.868396in}{2.199477in}}%
\pgfpathlineto{\pgfqpoint{1.869617in}{2.199744in}}%
\pgfpathlineto{\pgfqpoint{1.869922in}{2.199314in}}%
\pgfpathlineto{\pgfqpoint{1.871142in}{2.197992in}}%
\pgfpathlineto{\pgfqpoint{1.871752in}{2.198902in}}%
\pgfpathlineto{\pgfqpoint{1.872973in}{2.200376in}}%
\pgfpathlineto{\pgfqpoint{1.873583in}{2.199490in}}%
\pgfpathlineto{\pgfqpoint{1.874804in}{2.198382in}}%
\pgfpathlineto{\pgfqpoint{1.875414in}{2.198997in}}%
\pgfpathlineto{\pgfqpoint{1.878770in}{2.199675in}}%
\pgfpathlineto{\pgfqpoint{1.908977in}{2.199791in}}%
\pgfpathlineto{\pgfqpoint{1.910807in}{2.199585in}}%
\pgfpathlineto{\pgfqpoint{1.915079in}{2.199363in}}%
\pgfpathlineto{\pgfqpoint{1.917215in}{2.199284in}}%
\pgfpathlineto{\pgfqpoint{1.918741in}{2.198309in}}%
\pgfpathlineto{\pgfqpoint{1.919046in}{2.198741in}}%
\pgfpathlineto{\pgfqpoint{1.920571in}{2.201089in}}%
\pgfpathlineto{\pgfqpoint{1.920876in}{2.200526in}}%
\pgfpathlineto{\pgfqpoint{1.922402in}{2.197383in}}%
\pgfpathlineto{\pgfqpoint{1.923012in}{2.198616in}}%
\pgfpathlineto{\pgfqpoint{1.924233in}{2.200247in}}%
\pgfpathlineto{\pgfqpoint{1.924843in}{2.199434in}}%
\pgfpathlineto{\pgfqpoint{1.926063in}{2.198400in}}%
\pgfpathlineto{\pgfqpoint{1.926674in}{2.199048in}}%
\pgfpathlineto{\pgfqpoint{1.928809in}{2.200156in}}%
\pgfpathlineto{\pgfqpoint{1.929115in}{2.199888in}}%
\pgfpathlineto{\pgfqpoint{1.930945in}{2.198019in}}%
\pgfpathlineto{\pgfqpoint{1.931556in}{2.198930in}}%
\pgfpathlineto{\pgfqpoint{1.932776in}{2.200396in}}%
\pgfpathlineto{\pgfqpoint{1.933386in}{2.199476in}}%
\pgfpathlineto{\pgfqpoint{1.934607in}{2.198479in}}%
\pgfpathlineto{\pgfqpoint{1.935217in}{2.199168in}}%
\pgfpathlineto{\pgfqpoint{1.936437in}{2.199635in}}%
\pgfpathlineto{\pgfqpoint{1.937048in}{2.199028in}}%
\pgfpathlineto{\pgfqpoint{1.938268in}{2.199726in}}%
\pgfpathlineto{\pgfqpoint{1.939183in}{2.200966in}}%
\pgfpathlineto{\pgfqpoint{1.939794in}{2.200126in}}%
\pgfpathlineto{\pgfqpoint{1.941014in}{2.197892in}}%
\pgfpathlineto{\pgfqpoint{1.941624in}{2.198505in}}%
\pgfpathlineto{\pgfqpoint{1.943150in}{2.200213in}}%
\pgfpathlineto{\pgfqpoint{1.943760in}{2.199398in}}%
\pgfpathlineto{\pgfqpoint{1.945286in}{2.198751in}}%
\pgfpathlineto{\pgfqpoint{1.945591in}{2.199076in}}%
\pgfpathlineto{\pgfqpoint{1.946811in}{2.200541in}}%
\pgfpathlineto{\pgfqpoint{1.947422in}{2.199656in}}%
\pgfpathlineto{\pgfqpoint{1.948947in}{2.197473in}}%
\pgfpathlineto{\pgfqpoint{1.949557in}{2.198478in}}%
\pgfpathlineto{\pgfqpoint{1.950778in}{2.200819in}}%
\pgfpathlineto{\pgfqpoint{1.951388in}{2.199992in}}%
\pgfpathlineto{\pgfqpoint{1.952609in}{2.199133in}}%
\pgfpathlineto{\pgfqpoint{1.953219in}{2.199802in}}%
\pgfpathlineto{\pgfqpoint{1.954134in}{2.199405in}}%
\pgfpathlineto{\pgfqpoint{1.954439in}{2.198832in}}%
\pgfpathlineto{\pgfqpoint{1.955660in}{2.197588in}}%
\pgfpathlineto{\pgfqpoint{1.955965in}{2.198004in}}%
\pgfpathlineto{\pgfqpoint{1.957185in}{2.200901in}}%
\pgfpathlineto{\pgfqpoint{1.958101in}{2.199620in}}%
\pgfpathlineto{\pgfqpoint{1.959321in}{2.198603in}}%
\pgfpathlineto{\pgfqpoint{1.959931in}{2.199458in}}%
\pgfpathlineto{\pgfqpoint{1.960847in}{2.200233in}}%
\pgfpathlineto{\pgfqpoint{1.961457in}{2.199564in}}%
\pgfpathlineto{\pgfqpoint{1.962678in}{2.199118in}}%
\pgfpathlineto{\pgfqpoint{1.962983in}{2.199467in}}%
\pgfpathlineto{\pgfqpoint{1.964508in}{2.200430in}}%
\pgfpathlineto{\pgfqpoint{1.964813in}{2.200042in}}%
\pgfpathlineto{\pgfqpoint{1.967254in}{2.198380in}}%
\pgfpathlineto{\pgfqpoint{1.973357in}{2.199580in}}%
\pgfpathlineto{\pgfqpoint{1.974882in}{2.198161in}}%
\pgfpathlineto{\pgfqpoint{1.975493in}{2.198914in}}%
\pgfpathlineto{\pgfqpoint{1.976713in}{2.200539in}}%
\pgfpathlineto{\pgfqpoint{1.977323in}{2.199664in}}%
\pgfpathlineto{\pgfqpoint{1.978544in}{2.198085in}}%
\pgfpathlineto{\pgfqpoint{1.979154in}{2.198859in}}%
\pgfpathlineto{\pgfqpoint{1.980374in}{2.199967in}}%
\pgfpathlineto{\pgfqpoint{1.980985in}{2.199380in}}%
\pgfpathlineto{\pgfqpoint{1.982815in}{2.199590in}}%
\pgfpathlineto{\pgfqpoint{1.988002in}{2.198650in}}%
\pgfpathlineto{\pgfqpoint{1.988918in}{2.198505in}}%
\pgfpathlineto{\pgfqpoint{1.989223in}{2.198939in}}%
\pgfpathlineto{\pgfqpoint{1.990443in}{2.200283in}}%
\pgfpathlineto{\pgfqpoint{1.991054in}{2.199400in}}%
\pgfpathlineto{\pgfqpoint{1.992274in}{2.198072in}}%
\pgfpathlineto{\pgfqpoint{1.992884in}{2.198613in}}%
\pgfpathlineto{\pgfqpoint{1.994715in}{2.200586in}}%
\pgfpathlineto{\pgfqpoint{1.995325in}{2.199664in}}%
\pgfpathlineto{\pgfqpoint{1.996851in}{2.198326in}}%
\pgfpathlineto{\pgfqpoint{1.997461in}{2.198750in}}%
\pgfpathlineto{\pgfqpoint{2.002038in}{2.198873in}}%
\pgfpathlineto{\pgfqpoint{2.002953in}{2.197688in}}%
\pgfpathlineto{\pgfqpoint{2.003563in}{2.198796in}}%
\pgfpathlineto{\pgfqpoint{2.004784in}{2.200118in}}%
\pgfpathlineto{\pgfqpoint{2.005394in}{2.199506in}}%
\pgfpathlineto{\pgfqpoint{2.007225in}{2.199397in}}%
\pgfpathlineto{\pgfqpoint{2.010276in}{2.199700in}}%
\pgfpathlineto{\pgfqpoint{2.014853in}{2.198655in}}%
\pgfpathlineto{\pgfqpoint{2.015768in}{2.197283in}}%
\pgfpathlineto{\pgfqpoint{2.016378in}{2.198536in}}%
\pgfpathlineto{\pgfqpoint{2.017294in}{2.199872in}}%
\pgfpathlineto{\pgfqpoint{2.018209in}{2.199183in}}%
\pgfpathlineto{\pgfqpoint{2.021260in}{2.198130in}}%
\pgfpathlineto{\pgfqpoint{2.021870in}{2.198065in}}%
\pgfpathlineto{\pgfqpoint{2.022176in}{2.198602in}}%
\pgfpathlineto{\pgfqpoint{2.023396in}{2.200467in}}%
\pgfpathlineto{\pgfqpoint{2.024006in}{2.199717in}}%
\pgfpathlineto{\pgfqpoint{2.025227in}{2.198469in}}%
\pgfpathlineto{\pgfqpoint{2.025837in}{2.199237in}}%
\pgfpathlineto{\pgfqpoint{2.026752in}{2.199533in}}%
\pgfpathlineto{\pgfqpoint{2.027363in}{2.198929in}}%
\pgfpathlineto{\pgfqpoint{2.028888in}{2.199779in}}%
\pgfpathlineto{\pgfqpoint{2.030414in}{2.198978in}}%
\pgfpathlineto{\pgfqpoint{2.032244in}{2.199282in}}%
\pgfpathlineto{\pgfqpoint{2.034991in}{2.199857in}}%
\pgfpathlineto{\pgfqpoint{2.036516in}{2.198445in}}%
\pgfpathlineto{\pgfqpoint{2.036821in}{2.198959in}}%
\pgfpathlineto{\pgfqpoint{2.038042in}{2.200912in}}%
\pgfpathlineto{\pgfqpoint{2.038652in}{2.199784in}}%
\pgfpathlineto{\pgfqpoint{2.039872in}{2.197501in}}%
\pgfpathlineto{\pgfqpoint{2.040483in}{2.198173in}}%
\pgfpathlineto{\pgfqpoint{2.042008in}{2.199698in}}%
\pgfpathlineto{\pgfqpoint{2.042619in}{2.199374in}}%
\pgfpathlineto{\pgfqpoint{2.045365in}{2.198562in}}%
\pgfpathlineto{\pgfqpoint{2.046280in}{2.197787in}}%
\pgfpathlineto{\pgfqpoint{2.046585in}{2.198435in}}%
\pgfpathlineto{\pgfqpoint{2.047806in}{2.200766in}}%
\pgfpathlineto{\pgfqpoint{2.048416in}{2.199687in}}%
\pgfpathlineto{\pgfqpoint{2.049636in}{2.197916in}}%
\pgfpathlineto{\pgfqpoint{2.050246in}{2.198751in}}%
\pgfpathlineto{\pgfqpoint{2.051162in}{2.199824in}}%
\pgfpathlineto{\pgfqpoint{2.052077in}{2.199125in}}%
\pgfpathlineto{\pgfqpoint{2.056044in}{2.199257in}}%
\pgfpathlineto{\pgfqpoint{2.058180in}{2.199759in}}%
\pgfpathlineto{\pgfqpoint{2.061841in}{2.198963in}}%
\pgfpathlineto{\pgfqpoint{2.064587in}{2.198766in}}%
\pgfpathlineto{\pgfqpoint{2.066113in}{2.199168in}}%
\pgfpathlineto{\pgfqpoint{2.067638in}{2.200645in}}%
\pgfpathlineto{\pgfqpoint{2.068248in}{2.199770in}}%
\pgfpathlineto{\pgfqpoint{2.069774in}{2.197998in}}%
\pgfpathlineto{\pgfqpoint{2.070079in}{2.198360in}}%
\pgfpathlineto{\pgfqpoint{2.071605in}{2.200060in}}%
\pgfpathlineto{\pgfqpoint{2.072215in}{2.199180in}}%
\pgfpathlineto{\pgfqpoint{2.073130in}{2.198758in}}%
\pgfpathlineto{\pgfqpoint{2.073741in}{2.199606in}}%
\pgfpathlineto{\pgfqpoint{2.074656in}{2.200073in}}%
\pgfpathlineto{\pgfqpoint{2.075266in}{2.199055in}}%
\pgfpathlineto{\pgfqpoint{2.076181in}{2.197682in}}%
\pgfpathlineto{\pgfqpoint{2.076792in}{2.198820in}}%
\pgfpathlineto{\pgfqpoint{2.078012in}{2.200752in}}%
\pgfpathlineto{\pgfqpoint{2.078622in}{2.200087in}}%
\pgfpathlineto{\pgfqpoint{2.080148in}{2.198612in}}%
\pgfpathlineto{\pgfqpoint{2.080758in}{2.199105in}}%
\pgfpathlineto{\pgfqpoint{2.083504in}{2.199419in}}%
\pgfpathlineto{\pgfqpoint{2.088386in}{2.199179in}}%
\pgfpathlineto{\pgfqpoint{2.090522in}{2.199038in}}%
\pgfpathlineto{\pgfqpoint{2.092353in}{2.199481in}}%
\pgfpathlineto{\pgfqpoint{2.095404in}{2.199013in}}%
\pgfpathlineto{\pgfqpoint{2.097540in}{2.199881in}}%
\pgfpathlineto{\pgfqpoint{2.098455in}{2.200493in}}%
\pgfpathlineto{\pgfqpoint{2.098760in}{2.199979in}}%
\pgfpathlineto{\pgfqpoint{2.099981in}{2.198304in}}%
\pgfpathlineto{\pgfqpoint{2.100591in}{2.199068in}}%
\pgfpathlineto{\pgfqpoint{2.101811in}{2.200227in}}%
\pgfpathlineto{\pgfqpoint{2.102422in}{2.199534in}}%
\pgfpathlineto{\pgfqpoint{2.104252in}{2.199599in}}%
\pgfpathlineto{\pgfqpoint{2.107304in}{2.200186in}}%
\pgfpathlineto{\pgfqpoint{2.108524in}{2.200008in}}%
\pgfpathlineto{\pgfqpoint{2.108829in}{2.199408in}}%
\pgfpathlineto{\pgfqpoint{2.109744in}{2.198008in}}%
\pgfpathlineto{\pgfqpoint{2.110660in}{2.199015in}}%
\pgfpathlineto{\pgfqpoint{2.111880in}{2.200388in}}%
\pgfpathlineto{\pgfqpoint{2.112491in}{2.199663in}}%
\pgfpathlineto{\pgfqpoint{2.113711in}{2.197792in}}%
\pgfpathlineto{\pgfqpoint{2.114321in}{2.198672in}}%
\pgfpathlineto{\pgfqpoint{2.115542in}{2.200515in}}%
\pgfpathlineto{\pgfqpoint{2.116152in}{2.199489in}}%
\pgfpathlineto{\pgfqpoint{2.117372in}{2.198358in}}%
\pgfpathlineto{\pgfqpoint{2.117678in}{2.198792in}}%
\pgfpathlineto{\pgfqpoint{2.118898in}{2.199848in}}%
\pgfpathlineto{\pgfqpoint{2.119508in}{2.199032in}}%
\pgfpathlineto{\pgfqpoint{2.120424in}{2.198607in}}%
\pgfpathlineto{\pgfqpoint{2.120729in}{2.199071in}}%
\pgfpathlineto{\pgfqpoint{2.121949in}{2.200633in}}%
\pgfpathlineto{\pgfqpoint{2.122559in}{2.199711in}}%
\pgfpathlineto{\pgfqpoint{2.124085in}{2.198037in}}%
\pgfpathlineto{\pgfqpoint{2.124695in}{2.198861in}}%
\pgfpathlineto{\pgfqpoint{2.126221in}{2.200058in}}%
\pgfpathlineto{\pgfqpoint{2.126831in}{2.199572in}}%
\pgfpathlineto{\pgfqpoint{2.128967in}{2.198502in}}%
\pgfpathlineto{\pgfqpoint{2.129272in}{2.199043in}}%
\pgfpathlineto{\pgfqpoint{2.130493in}{2.200448in}}%
\pgfpathlineto{\pgfqpoint{2.131103in}{2.199860in}}%
\pgfpathlineto{\pgfqpoint{2.132323in}{2.198915in}}%
\pgfpathlineto{\pgfqpoint{2.132933in}{2.199528in}}%
\pgfpathlineto{\pgfqpoint{2.134459in}{2.199168in}}%
\pgfpathlineto{\pgfqpoint{2.135985in}{2.199683in}}%
\pgfpathlineto{\pgfqpoint{2.137205in}{2.199102in}}%
\pgfpathlineto{\pgfqpoint{2.138426in}{2.197943in}}%
\pgfpathlineto{\pgfqpoint{2.139036in}{2.198773in}}%
\pgfpathlineto{\pgfqpoint{2.140256in}{2.200467in}}%
\pgfpathlineto{\pgfqpoint{2.140867in}{2.200028in}}%
\pgfpathlineto{\pgfqpoint{2.143307in}{2.198752in}}%
\pgfpathlineto{\pgfqpoint{2.143613in}{2.199069in}}%
\pgfpathlineto{\pgfqpoint{2.145748in}{2.199639in}}%
\pgfpathlineto{\pgfqpoint{2.147884in}{2.199299in}}%
\pgfpathlineto{\pgfqpoint{2.151546in}{2.200307in}}%
\pgfpathlineto{\pgfqpoint{2.153071in}{2.196711in}}%
\pgfpathlineto{\pgfqpoint{2.153681in}{2.198123in}}%
\pgfpathlineto{\pgfqpoint{2.154902in}{2.202345in}}%
\pgfpathlineto{\pgfqpoint{2.155512in}{2.201354in}}%
\pgfpathlineto{\pgfqpoint{2.157038in}{2.197053in}}%
\pgfpathlineto{\pgfqpoint{2.157648in}{2.198861in}}%
\pgfpathlineto{\pgfqpoint{2.158563in}{2.200915in}}%
\pgfpathlineto{\pgfqpoint{2.159174in}{2.199937in}}%
\pgfpathlineto{\pgfqpoint{2.160394in}{2.198030in}}%
\pgfpathlineto{\pgfqpoint{2.161004in}{2.198617in}}%
\pgfpathlineto{\pgfqpoint{2.163750in}{2.199707in}}%
\pgfpathlineto{\pgfqpoint{2.165276in}{2.198970in}}%
\pgfpathlineto{\pgfqpoint{2.167107in}{2.199201in}}%
\pgfpathlineto{\pgfqpoint{2.169548in}{2.199754in}}%
\pgfpathlineto{\pgfqpoint{2.172904in}{2.199762in}}%
\pgfpathlineto{\pgfqpoint{2.174430in}{2.199068in}}%
\pgfpathlineto{\pgfqpoint{2.175955in}{2.199656in}}%
\pgfpathlineto{\pgfqpoint{2.177786in}{2.199038in}}%
\pgfpathlineto{\pgfqpoint{2.179311in}{2.199735in}}%
\pgfpathlineto{\pgfqpoint{2.180227in}{2.200043in}}%
\pgfpathlineto{\pgfqpoint{2.180837in}{2.199315in}}%
\pgfpathlineto{\pgfqpoint{2.182668in}{2.198329in}}%
\pgfpathlineto{\pgfqpoint{2.182973in}{2.198724in}}%
\pgfpathlineto{\pgfqpoint{2.184193in}{2.200816in}}%
\pgfpathlineto{\pgfqpoint{2.184804in}{2.200007in}}%
\pgfpathlineto{\pgfqpoint{2.186329in}{2.198107in}}%
\pgfpathlineto{\pgfqpoint{2.186939in}{2.198599in}}%
\pgfpathlineto{\pgfqpoint{2.188770in}{2.200346in}}%
\pgfpathlineto{\pgfqpoint{2.189075in}{2.199891in}}%
\pgfpathlineto{\pgfqpoint{2.190296in}{2.197704in}}%
\pgfpathlineto{\pgfqpoint{2.190906in}{2.198622in}}%
\pgfpathlineto{\pgfqpoint{2.192126in}{2.200600in}}%
\pgfpathlineto{\pgfqpoint{2.192737in}{2.199966in}}%
\pgfpathlineto{\pgfqpoint{2.193957in}{2.198285in}}%
\pgfpathlineto{\pgfqpoint{2.194567in}{2.199305in}}%
\pgfpathlineto{\pgfqpoint{2.195483in}{2.199900in}}%
\pgfpathlineto{\pgfqpoint{2.196093in}{2.199028in}}%
\pgfpathlineto{\pgfqpoint{2.197313in}{2.198503in}}%
\pgfpathlineto{\pgfqpoint{2.197619in}{2.198857in}}%
\pgfpathlineto{\pgfqpoint{2.199144in}{2.201039in}}%
\pgfpathlineto{\pgfqpoint{2.199754in}{2.199558in}}%
\pgfpathlineto{\pgfqpoint{2.201280in}{2.198036in}}%
\pgfpathlineto{\pgfqpoint{2.201585in}{2.198333in}}%
\pgfpathlineto{\pgfqpoint{2.202806in}{2.200634in}}%
\pgfpathlineto{\pgfqpoint{2.203721in}{2.199383in}}%
\pgfpathlineto{\pgfqpoint{2.204941in}{2.198956in}}%
\pgfpathlineto{\pgfqpoint{2.205246in}{2.199273in}}%
\pgfpathlineto{\pgfqpoint{2.207077in}{2.199125in}}%
\pgfpathlineto{\pgfqpoint{2.208908in}{2.199542in}}%
\pgfpathlineto{\pgfqpoint{2.210433in}{2.198778in}}%
\pgfpathlineto{\pgfqpoint{2.211959in}{2.198843in}}%
\pgfpathlineto{\pgfqpoint{2.214095in}{2.200157in}}%
\pgfpathlineto{\pgfqpoint{2.214400in}{2.199829in}}%
\pgfpathlineto{\pgfqpoint{2.216231in}{2.198076in}}%
\pgfpathlineto{\pgfqpoint{2.216841in}{2.199100in}}%
\pgfpathlineto{\pgfqpoint{2.218061in}{2.201377in}}%
\pgfpathlineto{\pgfqpoint{2.218672in}{2.200052in}}%
\pgfpathlineto{\pgfqpoint{2.220197in}{2.197254in}}%
\pgfpathlineto{\pgfqpoint{2.220807in}{2.198523in}}%
\pgfpathlineto{\pgfqpoint{2.222028in}{2.200729in}}%
\pgfpathlineto{\pgfqpoint{2.222638in}{2.200280in}}%
\pgfpathlineto{\pgfqpoint{2.225689in}{2.198663in}}%
\pgfpathlineto{\pgfqpoint{2.229656in}{2.198327in}}%
\pgfpathlineto{\pgfqpoint{2.230571in}{2.198236in}}%
\pgfpathlineto{\pgfqpoint{2.230876in}{2.198586in}}%
\pgfpathlineto{\pgfqpoint{2.232707in}{2.200219in}}%
\pgfpathlineto{\pgfqpoint{2.233317in}{2.199524in}}%
\pgfpathlineto{\pgfqpoint{2.234843in}{2.198221in}}%
\pgfpathlineto{\pgfqpoint{2.235148in}{2.198788in}}%
\pgfpathlineto{\pgfqpoint{2.236369in}{2.200942in}}%
\pgfpathlineto{\pgfqpoint{2.236979in}{2.199733in}}%
\pgfpathlineto{\pgfqpoint{2.238199in}{2.197176in}}%
\pgfpathlineto{\pgfqpoint{2.238809in}{2.198998in}}%
\pgfpathlineto{\pgfqpoint{2.239725in}{2.201218in}}%
\pgfpathlineto{\pgfqpoint{2.240335in}{2.200341in}}%
\pgfpathlineto{\pgfqpoint{2.241556in}{2.197913in}}%
\pgfpathlineto{\pgfqpoint{2.242166in}{2.198878in}}%
\pgfpathlineto{\pgfqpoint{2.243386in}{2.200301in}}%
\pgfpathlineto{\pgfqpoint{2.243691in}{2.199859in}}%
\pgfpathlineto{\pgfqpoint{2.244912in}{2.197814in}}%
\pgfpathlineto{\pgfqpoint{2.245522in}{2.198509in}}%
\pgfpathlineto{\pgfqpoint{2.247048in}{2.200762in}}%
\pgfpathlineto{\pgfqpoint{2.247658in}{2.200079in}}%
\pgfpathlineto{\pgfqpoint{2.250099in}{2.198507in}}%
\pgfpathlineto{\pgfqpoint{2.253150in}{2.199215in}}%
\pgfpathlineto{\pgfqpoint{2.255286in}{2.198781in}}%
\pgfpathlineto{\pgfqpoint{2.256201in}{2.197903in}}%
\pgfpathlineto{\pgfqpoint{2.256811in}{2.198721in}}%
\pgfpathlineto{\pgfqpoint{2.258032in}{2.200639in}}%
\pgfpathlineto{\pgfqpoint{2.258642in}{2.199842in}}%
\pgfpathlineto{\pgfqpoint{2.259863in}{2.198463in}}%
\pgfpathlineto{\pgfqpoint{2.260473in}{2.199277in}}%
\pgfpathlineto{\pgfqpoint{2.261998in}{2.199780in}}%
\pgfpathlineto{\pgfqpoint{2.262304in}{2.199520in}}%
\pgfpathlineto{\pgfqpoint{2.264744in}{2.199073in}}%
\pgfpathlineto{\pgfqpoint{2.266270in}{2.200881in}}%
\pgfpathlineto{\pgfqpoint{2.266880in}{2.199693in}}%
\pgfpathlineto{\pgfqpoint{2.268101in}{2.197044in}}%
\pgfpathlineto{\pgfqpoint{2.268711in}{2.198581in}}%
\pgfpathlineto{\pgfqpoint{2.269626in}{2.200954in}}%
\pgfpathlineto{\pgfqpoint{2.270237in}{2.199953in}}%
\pgfpathlineto{\pgfqpoint{2.271152in}{2.198169in}}%
\pgfpathlineto{\pgfqpoint{2.271762in}{2.199595in}}%
\pgfpathlineto{\pgfqpoint{2.272678in}{2.201148in}}%
\pgfpathlineto{\pgfqpoint{2.273288in}{2.199613in}}%
\pgfpathlineto{\pgfqpoint{2.274203in}{2.196612in}}%
\pgfpathlineto{\pgfqpoint{2.274813in}{2.197885in}}%
\pgfpathlineto{\pgfqpoint{2.276034in}{2.200979in}}%
\pgfpathlineto{\pgfqpoint{2.276644in}{2.200184in}}%
\pgfpathlineto{\pgfqpoint{2.278170in}{2.198309in}}%
\pgfpathlineto{\pgfqpoint{2.278780in}{2.198919in}}%
\pgfpathlineto{\pgfqpoint{2.281221in}{2.199898in}}%
\pgfpathlineto{\pgfqpoint{2.281526in}{2.199334in}}%
\pgfpathlineto{\pgfqpoint{2.282441in}{2.197997in}}%
\pgfpathlineto{\pgfqpoint{2.283052in}{2.199131in}}%
\pgfpathlineto{\pgfqpoint{2.283967in}{2.200634in}}%
\pgfpathlineto{\pgfqpoint{2.284577in}{2.199839in}}%
\pgfpathlineto{\pgfqpoint{2.285798in}{2.198079in}}%
\pgfpathlineto{\pgfqpoint{2.286103in}{2.198654in}}%
\pgfpathlineto{\pgfqpoint{2.287323in}{2.200566in}}%
\pgfpathlineto{\pgfqpoint{2.287933in}{2.199389in}}%
\pgfpathlineto{\pgfqpoint{2.288849in}{2.197742in}}%
\pgfpathlineto{\pgfqpoint{2.289459in}{2.199005in}}%
\pgfpathlineto{\pgfqpoint{2.290680in}{2.200948in}}%
\pgfpathlineto{\pgfqpoint{2.291290in}{2.199806in}}%
\pgfpathlineto{\pgfqpoint{2.292510in}{2.197471in}}%
\pgfpathlineto{\pgfqpoint{2.293120in}{2.198783in}}%
\pgfpathlineto{\pgfqpoint{2.295256in}{2.199840in}}%
\pgfpathlineto{\pgfqpoint{2.301359in}{2.199637in}}%
\pgfpathlineto{\pgfqpoint{2.302274in}{2.200698in}}%
\pgfpathlineto{\pgfqpoint{2.302579in}{2.200091in}}%
\pgfpathlineto{\pgfqpoint{2.303494in}{2.197703in}}%
\pgfpathlineto{\pgfqpoint{2.304410in}{2.199110in}}%
\pgfpathlineto{\pgfqpoint{2.305630in}{2.201215in}}%
\pgfpathlineto{\pgfqpoint{2.306241in}{2.199842in}}%
\pgfpathlineto{\pgfqpoint{2.307156in}{2.196979in}}%
\pgfpathlineto{\pgfqpoint{2.307766in}{2.198111in}}%
\pgfpathlineto{\pgfqpoint{2.308987in}{2.201113in}}%
\pgfpathlineto{\pgfqpoint{2.309597in}{2.200416in}}%
\pgfpathlineto{\pgfqpoint{2.310817in}{2.197777in}}%
\pgfpathlineto{\pgfqpoint{2.311428in}{2.198996in}}%
\pgfpathlineto{\pgfqpoint{2.312648in}{2.200727in}}%
\pgfpathlineto{\pgfqpoint{2.313258in}{2.199940in}}%
\pgfpathlineto{\pgfqpoint{2.314479in}{2.198199in}}%
\pgfpathlineto{\pgfqpoint{2.315089in}{2.198720in}}%
\pgfpathlineto{\pgfqpoint{2.317225in}{2.199322in}}%
\pgfpathlineto{\pgfqpoint{2.318750in}{2.199062in}}%
\pgfpathlineto{\pgfqpoint{2.319056in}{2.199509in}}%
\pgfpathlineto{\pgfqpoint{2.320276in}{2.200342in}}%
\pgfpathlineto{\pgfqpoint{2.320581in}{2.199994in}}%
\pgfpathlineto{\pgfqpoint{2.322107in}{2.197780in}}%
\pgfpathlineto{\pgfqpoint{2.322717in}{2.199189in}}%
\pgfpathlineto{\pgfqpoint{2.323632in}{2.200237in}}%
\pgfpathlineto{\pgfqpoint{2.324243in}{2.199166in}}%
\pgfpathlineto{\pgfqpoint{2.325158in}{2.197304in}}%
\pgfpathlineto{\pgfqpoint{2.325768in}{2.199019in}}%
\pgfpathlineto{\pgfqpoint{2.326683in}{2.202179in}}%
\pgfpathlineto{\pgfqpoint{2.327294in}{2.201043in}}%
\pgfpathlineto{\pgfqpoint{2.328819in}{2.196828in}}%
\pgfpathlineto{\pgfqpoint{2.329430in}{2.198724in}}%
\pgfpathlineto{\pgfqpoint{2.330345in}{2.200973in}}%
\pgfpathlineto{\pgfqpoint{2.330955in}{2.200026in}}%
\pgfpathlineto{\pgfqpoint{2.332176in}{2.197892in}}%
\pgfpathlineto{\pgfqpoint{2.332786in}{2.199083in}}%
\pgfpathlineto{\pgfqpoint{2.334617in}{2.199757in}}%
\pgfpathlineto{\pgfqpoint{2.336142in}{2.200053in}}%
\pgfpathlineto{\pgfqpoint{2.336447in}{2.199385in}}%
\pgfpathlineto{\pgfqpoint{2.337668in}{2.198097in}}%
\pgfpathlineto{\pgfqpoint{2.338278in}{2.198547in}}%
\pgfpathlineto{\pgfqpoint{2.340719in}{2.199120in}}%
\pgfpathlineto{\pgfqpoint{2.342855in}{2.199783in}}%
\pgfpathlineto{\pgfqpoint{2.343160in}{2.199327in}}%
\pgfpathlineto{\pgfqpoint{2.344380in}{2.198775in}}%
\pgfpathlineto{\pgfqpoint{2.344991in}{2.199353in}}%
\pgfpathlineto{\pgfqpoint{2.346821in}{2.198889in}}%
\pgfpathlineto{\pgfqpoint{2.348652in}{2.199002in}}%
\pgfpathlineto{\pgfqpoint{2.351703in}{2.198481in}}%
\pgfpathlineto{\pgfqpoint{2.352619in}{2.198161in}}%
\pgfpathlineto{\pgfqpoint{2.352924in}{2.198650in}}%
\pgfpathlineto{\pgfqpoint{2.354144in}{2.201130in}}%
\pgfpathlineto{\pgfqpoint{2.354754in}{2.200358in}}%
\pgfpathlineto{\pgfqpoint{2.355670in}{2.198747in}}%
\pgfpathlineto{\pgfqpoint{2.356585in}{2.199808in}}%
\pgfpathlineto{\pgfqpoint{2.357500in}{2.199070in}}%
\pgfpathlineto{\pgfqpoint{2.358416in}{2.197459in}}%
\pgfpathlineto{\pgfqpoint{2.359026in}{2.199044in}}%
\pgfpathlineto{\pgfqpoint{2.359941in}{2.201664in}}%
\pgfpathlineto{\pgfqpoint{2.360552in}{2.200835in}}%
\pgfpathlineto{\pgfqpoint{2.361772in}{2.197300in}}%
\pgfpathlineto{\pgfqpoint{2.362687in}{2.199095in}}%
\pgfpathlineto{\pgfqpoint{2.363603in}{2.200486in}}%
\pgfpathlineto{\pgfqpoint{2.364213in}{2.199575in}}%
\pgfpathlineto{\pgfqpoint{2.365433in}{2.198191in}}%
\pgfpathlineto{\pgfqpoint{2.366044in}{2.199123in}}%
\pgfpathlineto{\pgfqpoint{2.368790in}{2.200506in}}%
\pgfpathlineto{\pgfqpoint{2.370010in}{2.198055in}}%
\pgfpathlineto{\pgfqpoint{2.370620in}{2.197564in}}%
\pgfpathlineto{\pgfqpoint{2.371231in}{2.198365in}}%
\pgfpathlineto{\pgfqpoint{2.372756in}{2.199374in}}%
\pgfpathlineto{\pgfqpoint{2.373367in}{2.198868in}}%
\pgfpathlineto{\pgfqpoint{2.377028in}{2.199489in}}%
\pgfpathlineto{\pgfqpoint{2.377943in}{2.201213in}}%
\pgfpathlineto{\pgfqpoint{2.378554in}{2.200084in}}%
\pgfpathlineto{\pgfqpoint{2.379774in}{2.197521in}}%
\pgfpathlineto{\pgfqpoint{2.380384in}{2.198569in}}%
\pgfpathlineto{\pgfqpoint{2.381300in}{2.200363in}}%
\pgfpathlineto{\pgfqpoint{2.382215in}{2.199289in}}%
\pgfpathlineto{\pgfqpoint{2.383130in}{2.198208in}}%
\pgfpathlineto{\pgfqpoint{2.383741in}{2.199196in}}%
\pgfpathlineto{\pgfqpoint{2.384656in}{2.200298in}}%
\pgfpathlineto{\pgfqpoint{2.385266in}{2.199529in}}%
\pgfpathlineto{\pgfqpoint{2.386487in}{2.199209in}}%
\pgfpathlineto{\pgfqpoint{2.386792in}{2.199552in}}%
\pgfpathlineto{\pgfqpoint{2.387707in}{2.199937in}}%
\pgfpathlineto{\pgfqpoint{2.388012in}{2.199429in}}%
\pgfpathlineto{\pgfqpoint{2.388928in}{2.198268in}}%
\pgfpathlineto{\pgfqpoint{2.389538in}{2.198964in}}%
\pgfpathlineto{\pgfqpoint{2.390758in}{2.200612in}}%
\pgfpathlineto{\pgfqpoint{2.391369in}{2.199221in}}%
\pgfpathlineto{\pgfqpoint{2.392284in}{2.198052in}}%
\pgfpathlineto{\pgfqpoint{2.392894in}{2.198895in}}%
\pgfpathlineto{\pgfqpoint{2.394115in}{2.199454in}}%
\pgfpathlineto{\pgfqpoint{2.394725in}{2.198876in}}%
\pgfpathlineto{\pgfqpoint{2.398691in}{2.199436in}}%
\pgfpathlineto{\pgfqpoint{2.399302in}{2.200157in}}%
\pgfpathlineto{\pgfqpoint{2.400217in}{2.199317in}}%
\pgfpathlineto{\pgfqpoint{2.401132in}{2.199251in}}%
\pgfpathlineto{\pgfqpoint{2.401437in}{2.199681in}}%
\pgfpathlineto{\pgfqpoint{2.402353in}{2.200814in}}%
\pgfpathlineto{\pgfqpoint{2.402963in}{2.199861in}}%
\pgfpathlineto{\pgfqpoint{2.404183in}{2.198509in}}%
\pgfpathlineto{\pgfqpoint{2.404794in}{2.198937in}}%
\pgfpathlineto{\pgfqpoint{2.409065in}{2.199891in}}%
\pgfpathlineto{\pgfqpoint{2.410286in}{2.198354in}}%
\pgfpathlineto{\pgfqpoint{2.410896in}{2.199450in}}%
\pgfpathlineto{\pgfqpoint{2.411811in}{2.200249in}}%
\pgfpathlineto{\pgfqpoint{2.412422in}{2.199539in}}%
\pgfpathlineto{\pgfqpoint{2.413642in}{2.198595in}}%
\pgfpathlineto{\pgfqpoint{2.414252in}{2.199121in}}%
\pgfpathlineto{\pgfqpoint{2.416083in}{2.198385in}}%
\pgfpathlineto{\pgfqpoint{2.417609in}{2.199723in}}%
\pgfpathlineto{\pgfqpoint{2.420355in}{2.199472in}}%
\pgfpathlineto{\pgfqpoint{2.424016in}{2.198296in}}%
\pgfpathlineto{\pgfqpoint{2.424626in}{2.196932in}}%
\pgfpathlineto{\pgfqpoint{2.425237in}{2.198234in}}%
\pgfpathlineto{\pgfqpoint{2.426457in}{2.202243in}}%
\pgfpathlineto{\pgfqpoint{2.427067in}{2.200936in}}%
\pgfpathlineto{\pgfqpoint{2.428288in}{2.196756in}}%
\pgfpathlineto{\pgfqpoint{2.428898in}{2.197938in}}%
\pgfpathlineto{\pgfqpoint{2.430119in}{2.200737in}}%
\pgfpathlineto{\pgfqpoint{2.430729in}{2.199917in}}%
\pgfpathlineto{\pgfqpoint{2.431644in}{2.199183in}}%
\pgfpathlineto{\pgfqpoint{2.432559in}{2.199764in}}%
\pgfpathlineto{\pgfqpoint{2.441103in}{2.200349in}}%
\pgfpathlineto{\pgfqpoint{2.441713in}{2.200166in}}%
\pgfpathlineto{\pgfqpoint{2.442018in}{2.199411in}}%
\pgfpathlineto{\pgfqpoint{2.442933in}{2.198013in}}%
\pgfpathlineto{\pgfqpoint{2.443544in}{2.199248in}}%
\pgfpathlineto{\pgfqpoint{2.444459in}{2.201434in}}%
\pgfpathlineto{\pgfqpoint{2.445069in}{2.200608in}}%
\pgfpathlineto{\pgfqpoint{2.446595in}{2.197472in}}%
\pgfpathlineto{\pgfqpoint{2.447205in}{2.198316in}}%
\pgfpathlineto{\pgfqpoint{2.448426in}{2.199914in}}%
\pgfpathlineto{\pgfqpoint{2.449036in}{2.199370in}}%
\pgfpathlineto{\pgfqpoint{2.450867in}{2.199783in}}%
\pgfpathlineto{\pgfqpoint{2.453307in}{2.199711in}}%
\pgfpathlineto{\pgfqpoint{2.457884in}{2.198458in}}%
\pgfpathlineto{\pgfqpoint{2.460020in}{2.200470in}}%
\pgfpathlineto{\pgfqpoint{2.460630in}{2.198787in}}%
\pgfpathlineto{\pgfqpoint{2.461241in}{2.197883in}}%
\pgfpathlineto{\pgfqpoint{2.461851in}{2.198559in}}%
\pgfpathlineto{\pgfqpoint{2.463071in}{2.200332in}}%
\pgfpathlineto{\pgfqpoint{2.463681in}{2.199193in}}%
\pgfpathlineto{\pgfqpoint{2.464597in}{2.197604in}}%
\pgfpathlineto{\pgfqpoint{2.465207in}{2.198244in}}%
\pgfpathlineto{\pgfqpoint{2.466733in}{2.200535in}}%
\pgfpathlineto{\pgfqpoint{2.467343in}{2.199995in}}%
\pgfpathlineto{\pgfqpoint{2.469784in}{2.199088in}}%
\pgfpathlineto{\pgfqpoint{2.474056in}{2.201197in}}%
\pgfpathlineto{\pgfqpoint{2.474361in}{2.201635in}}%
\pgfpathlineto{\pgfqpoint{2.474971in}{2.200530in}}%
\pgfpathlineto{\pgfqpoint{2.476191in}{2.197535in}}%
\pgfpathlineto{\pgfqpoint{2.476802in}{2.198338in}}%
\pgfpathlineto{\pgfqpoint{2.478022in}{2.201400in}}%
\pgfpathlineto{\pgfqpoint{2.478632in}{2.199153in}}%
\pgfpathlineto{\pgfqpoint{2.479548in}{2.197463in}}%
\pgfpathlineto{\pgfqpoint{2.480158in}{2.198612in}}%
\pgfpathlineto{\pgfqpoint{2.481378in}{2.201013in}}%
\pgfpathlineto{\pgfqpoint{2.481989in}{2.199549in}}%
\pgfpathlineto{\pgfqpoint{2.482904in}{2.197906in}}%
\pgfpathlineto{\pgfqpoint{2.483514in}{2.199232in}}%
\pgfpathlineto{\pgfqpoint{2.484430in}{2.201348in}}%
\pgfpathlineto{\pgfqpoint{2.485040in}{2.200544in}}%
\pgfpathlineto{\pgfqpoint{2.486565in}{2.198007in}}%
\pgfpathlineto{\pgfqpoint{2.487176in}{2.198302in}}%
\pgfpathlineto{\pgfqpoint{2.493888in}{2.198352in}}%
\pgfpathlineto{\pgfqpoint{2.494498in}{2.198030in}}%
\pgfpathlineto{\pgfqpoint{2.495109in}{2.198906in}}%
\pgfpathlineto{\pgfqpoint{2.496024in}{2.200156in}}%
\pgfpathlineto{\pgfqpoint{2.496634in}{2.199491in}}%
\pgfpathlineto{\pgfqpoint{2.497550in}{2.198643in}}%
\pgfpathlineto{\pgfqpoint{2.498160in}{2.199605in}}%
\pgfpathlineto{\pgfqpoint{2.499075in}{2.200930in}}%
\pgfpathlineto{\pgfqpoint{2.499685in}{2.199951in}}%
\pgfpathlineto{\pgfqpoint{2.500906in}{2.198000in}}%
\pgfpathlineto{\pgfqpoint{2.501516in}{2.198878in}}%
\pgfpathlineto{\pgfqpoint{2.502737in}{2.199968in}}%
\pgfpathlineto{\pgfqpoint{2.503347in}{2.199446in}}%
\pgfpathlineto{\pgfqpoint{2.507008in}{2.198246in}}%
\pgfpathlineto{\pgfqpoint{2.507313in}{2.199090in}}%
\pgfpathlineto{\pgfqpoint{2.508229in}{2.200545in}}%
\pgfpathlineto{\pgfqpoint{2.508839in}{2.199643in}}%
\pgfpathlineto{\pgfqpoint{2.510059in}{2.198995in}}%
\pgfpathlineto{\pgfqpoint{2.510365in}{2.199327in}}%
\pgfpathlineto{\pgfqpoint{2.511280in}{2.199613in}}%
\pgfpathlineto{\pgfqpoint{2.511585in}{2.198936in}}%
\pgfpathlineto{\pgfqpoint{2.512500in}{2.197550in}}%
\pgfpathlineto{\pgfqpoint{2.513111in}{2.198357in}}%
\pgfpathlineto{\pgfqpoint{2.514636in}{2.200146in}}%
\pgfpathlineto{\pgfqpoint{2.515246in}{2.199487in}}%
\pgfpathlineto{\pgfqpoint{2.516772in}{2.200118in}}%
\pgfpathlineto{\pgfqpoint{2.517993in}{2.200106in}}%
\pgfpathlineto{\pgfqpoint{2.518298in}{2.199593in}}%
\pgfpathlineto{\pgfqpoint{2.520433in}{2.198433in}}%
\pgfpathlineto{\pgfqpoint{2.527146in}{2.198531in}}%
\pgfpathlineto{\pgfqpoint{2.527756in}{2.198459in}}%
\pgfpathlineto{\pgfqpoint{2.528061in}{2.198927in}}%
\pgfpathlineto{\pgfqpoint{2.529282in}{2.200761in}}%
\pgfpathlineto{\pgfqpoint{2.529587in}{2.200113in}}%
\pgfpathlineto{\pgfqpoint{2.530807in}{2.197501in}}%
\pgfpathlineto{\pgfqpoint{2.531418in}{2.198654in}}%
\pgfpathlineto{\pgfqpoint{2.532333in}{2.200376in}}%
\pgfpathlineto{\pgfqpoint{2.532943in}{2.199543in}}%
\pgfpathlineto{\pgfqpoint{2.533859in}{2.198239in}}%
\pgfpathlineto{\pgfqpoint{2.534469in}{2.199069in}}%
\pgfpathlineto{\pgfqpoint{2.535689in}{2.200411in}}%
\pgfpathlineto{\pgfqpoint{2.535994in}{2.199980in}}%
\pgfpathlineto{\pgfqpoint{2.537215in}{2.197915in}}%
\pgfpathlineto{\pgfqpoint{2.537825in}{2.198522in}}%
\pgfpathlineto{\pgfqpoint{2.540571in}{2.199787in}}%
\pgfpathlineto{\pgfqpoint{2.547589in}{2.199508in}}%
\pgfpathlineto{\pgfqpoint{2.548504in}{2.197909in}}%
\pgfpathlineto{\pgfqpoint{2.549115in}{2.198948in}}%
\pgfpathlineto{\pgfqpoint{2.550335in}{2.201511in}}%
\pgfpathlineto{\pgfqpoint{2.550945in}{2.200201in}}%
\pgfpathlineto{\pgfqpoint{2.551861in}{2.198040in}}%
\pgfpathlineto{\pgfqpoint{2.552776in}{2.198963in}}%
\pgfpathlineto{\pgfqpoint{2.553691in}{2.199129in}}%
\pgfpathlineto{\pgfqpoint{2.554302in}{2.198328in}}%
\pgfpathlineto{\pgfqpoint{2.554912in}{2.198143in}}%
\pgfpathlineto{\pgfqpoint{2.555522in}{2.199229in}}%
\pgfpathlineto{\pgfqpoint{2.556743in}{2.199971in}}%
\pgfpathlineto{\pgfqpoint{2.557353in}{2.199433in}}%
\pgfpathlineto{\pgfqpoint{2.558878in}{2.200171in}}%
\pgfpathlineto{\pgfqpoint{2.559489in}{2.200097in}}%
\pgfpathlineto{\pgfqpoint{2.559794in}{2.199439in}}%
\pgfpathlineto{\pgfqpoint{2.560709in}{2.197625in}}%
\pgfpathlineto{\pgfqpoint{2.561624in}{2.198855in}}%
\pgfpathlineto{\pgfqpoint{2.562540in}{2.200028in}}%
\pgfpathlineto{\pgfqpoint{2.563150in}{2.199240in}}%
\pgfpathlineto{\pgfqpoint{2.563760in}{2.198582in}}%
\pgfpathlineto{\pgfqpoint{2.564370in}{2.199255in}}%
\pgfpathlineto{\pgfqpoint{2.565591in}{2.200207in}}%
\pgfpathlineto{\pgfqpoint{2.566201in}{2.199422in}}%
\pgfpathlineto{\pgfqpoint{2.567422in}{2.198632in}}%
\pgfpathlineto{\pgfqpoint{2.567727in}{2.198999in}}%
\pgfpathlineto{\pgfqpoint{2.569252in}{2.200011in}}%
\pgfpathlineto{\pgfqpoint{2.569863in}{2.199367in}}%
\pgfpathlineto{\pgfqpoint{2.571998in}{2.199013in}}%
\pgfpathlineto{\pgfqpoint{2.574439in}{2.199245in}}%
\pgfpathlineto{\pgfqpoint{2.575660in}{2.199126in}}%
\pgfpathlineto{\pgfqpoint{2.575965in}{2.199531in}}%
\pgfpathlineto{\pgfqpoint{2.577185in}{2.200892in}}%
\pgfpathlineto{\pgfqpoint{2.577491in}{2.200376in}}%
\pgfpathlineto{\pgfqpoint{2.578711in}{2.196716in}}%
\pgfpathlineto{\pgfqpoint{2.579321in}{2.197642in}}%
\pgfpathlineto{\pgfqpoint{2.580847in}{2.201156in}}%
\pgfpathlineto{\pgfqpoint{2.581457in}{2.200190in}}%
\pgfpathlineto{\pgfqpoint{2.582678in}{2.198344in}}%
\pgfpathlineto{\pgfqpoint{2.583288in}{2.198826in}}%
\pgfpathlineto{\pgfqpoint{2.586949in}{2.199676in}}%
\pgfpathlineto{\pgfqpoint{2.588475in}{2.198294in}}%
\pgfpathlineto{\pgfqpoint{2.589085in}{2.199375in}}%
\pgfpathlineto{\pgfqpoint{2.590306in}{2.200302in}}%
\pgfpathlineto{\pgfqpoint{2.590916in}{2.199711in}}%
\pgfpathlineto{\pgfqpoint{2.593357in}{2.199498in}}%
\pgfpathlineto{\pgfqpoint{2.597323in}{2.198699in}}%
\pgfpathlineto{\pgfqpoint{2.599764in}{2.200260in}}%
\pgfpathlineto{\pgfqpoint{2.600069in}{2.199764in}}%
\pgfpathlineto{\pgfqpoint{2.601595in}{2.198500in}}%
\pgfpathlineto{\pgfqpoint{2.601900in}{2.198754in}}%
\pgfpathlineto{\pgfqpoint{2.604036in}{2.198903in}}%
\pgfpathlineto{\pgfqpoint{2.606782in}{2.199471in}}%
\pgfpathlineto{\pgfqpoint{2.613494in}{2.199638in}}%
\pgfpathlineto{\pgfqpoint{2.615630in}{2.199162in}}%
\pgfpathlineto{\pgfqpoint{2.620207in}{2.198725in}}%
\pgfpathlineto{\pgfqpoint{2.626309in}{2.199005in}}%
\pgfpathlineto{\pgfqpoint{2.627225in}{2.198042in}}%
\pgfpathlineto{\pgfqpoint{2.627835in}{2.199058in}}%
\pgfpathlineto{\pgfqpoint{2.629361in}{2.200541in}}%
\pgfpathlineto{\pgfqpoint{2.629666in}{2.200079in}}%
\pgfpathlineto{\pgfqpoint{2.631191in}{2.197818in}}%
\pgfpathlineto{\pgfqpoint{2.631802in}{2.198459in}}%
\pgfpathlineto{\pgfqpoint{2.633327in}{2.200519in}}%
\pgfpathlineto{\pgfqpoint{2.633937in}{2.199717in}}%
\pgfpathlineto{\pgfqpoint{2.635158in}{2.198805in}}%
\pgfpathlineto{\pgfqpoint{2.635768in}{2.199372in}}%
\pgfpathlineto{\pgfqpoint{2.637294in}{2.198855in}}%
\pgfpathlineto{\pgfqpoint{2.638819in}{2.199206in}}%
\pgfpathlineto{\pgfqpoint{2.640040in}{2.200750in}}%
\pgfpathlineto{\pgfqpoint{2.640650in}{2.200007in}}%
\pgfpathlineto{\pgfqpoint{2.641870in}{2.198168in}}%
\pgfpathlineto{\pgfqpoint{2.642481in}{2.198889in}}%
\pgfpathlineto{\pgfqpoint{2.643701in}{2.200090in}}%
\pgfpathlineto{\pgfqpoint{2.644311in}{2.199471in}}%
\pgfpathlineto{\pgfqpoint{2.645532in}{2.198591in}}%
\pgfpathlineto{\pgfqpoint{2.646142in}{2.199034in}}%
\pgfpathlineto{\pgfqpoint{2.649804in}{2.199692in}}%
\pgfpathlineto{\pgfqpoint{2.651634in}{2.198461in}}%
\pgfpathlineto{\pgfqpoint{2.652244in}{2.199324in}}%
\pgfpathlineto{\pgfqpoint{2.653465in}{2.199699in}}%
\pgfpathlineto{\pgfqpoint{2.653770in}{2.199370in}}%
\pgfpathlineto{\pgfqpoint{2.654991in}{2.198696in}}%
\pgfpathlineto{\pgfqpoint{2.655601in}{2.199406in}}%
\pgfpathlineto{\pgfqpoint{2.657431in}{2.199532in}}%
\pgfpathlineto{\pgfqpoint{2.660178in}{2.197816in}}%
\pgfpathlineto{\pgfqpoint{2.661093in}{2.196966in}}%
\pgfpathlineto{\pgfqpoint{2.661398in}{2.197494in}}%
\pgfpathlineto{\pgfqpoint{2.662924in}{2.201777in}}%
\pgfpathlineto{\pgfqpoint{2.663839in}{2.200505in}}%
\pgfpathlineto{\pgfqpoint{2.665365in}{2.198675in}}%
\pgfpathlineto{\pgfqpoint{2.665975in}{2.199173in}}%
\pgfpathlineto{\pgfqpoint{2.667195in}{2.199333in}}%
\pgfpathlineto{\pgfqpoint{2.667500in}{2.198962in}}%
\pgfpathlineto{\pgfqpoint{2.668721in}{2.198124in}}%
\pgfpathlineto{\pgfqpoint{2.669331in}{2.199021in}}%
\pgfpathlineto{\pgfqpoint{2.670857in}{2.200051in}}%
\pgfpathlineto{\pgfqpoint{2.671162in}{2.199815in}}%
\pgfpathlineto{\pgfqpoint{2.674823in}{2.198343in}}%
\pgfpathlineto{\pgfqpoint{2.676959in}{2.200511in}}%
\pgfpathlineto{\pgfqpoint{2.677264in}{2.200164in}}%
\pgfpathlineto{\pgfqpoint{2.679095in}{2.198341in}}%
\pgfpathlineto{\pgfqpoint{2.679705in}{2.198803in}}%
\pgfpathlineto{\pgfqpoint{2.683977in}{2.199709in}}%
\pgfpathlineto{\pgfqpoint{2.687333in}{2.198895in}}%
\pgfpathlineto{\pgfqpoint{2.690079in}{2.199228in}}%
\pgfpathlineto{\pgfqpoint{2.694351in}{2.198718in}}%
\pgfpathlineto{\pgfqpoint{2.696487in}{2.200285in}}%
\pgfpathlineto{\pgfqpoint{2.697097in}{2.199540in}}%
\pgfpathlineto{\pgfqpoint{2.699843in}{2.198714in}}%
\pgfpathlineto{\pgfqpoint{2.701979in}{2.199808in}}%
\pgfpathlineto{\pgfqpoint{2.702284in}{2.199463in}}%
\pgfpathlineto{\pgfqpoint{2.704420in}{2.198861in}}%
\pgfpathlineto{\pgfqpoint{2.706556in}{2.200569in}}%
\pgfpathlineto{\pgfqpoint{2.706861in}{2.200242in}}%
\pgfpathlineto{\pgfqpoint{2.708386in}{2.197619in}}%
\pgfpathlineto{\pgfqpoint{2.709302in}{2.198959in}}%
\pgfpathlineto{\pgfqpoint{2.712048in}{2.199847in}}%
\pgfpathlineto{\pgfqpoint{2.715404in}{2.200035in}}%
\pgfpathlineto{\pgfqpoint{2.716014in}{2.200312in}}%
\pgfpathlineto{\pgfqpoint{2.716624in}{2.199267in}}%
\pgfpathlineto{\pgfqpoint{2.717845in}{2.198150in}}%
\pgfpathlineto{\pgfqpoint{2.718455in}{2.198964in}}%
\pgfpathlineto{\pgfqpoint{2.719981in}{2.200008in}}%
\pgfpathlineto{\pgfqpoint{2.720286in}{2.199747in}}%
\pgfpathlineto{\pgfqpoint{2.723337in}{2.198856in}}%
\pgfpathlineto{\pgfqpoint{2.725778in}{2.200041in}}%
\pgfpathlineto{\pgfqpoint{2.726083in}{2.199664in}}%
\pgfpathlineto{\pgfqpoint{2.728829in}{2.198616in}}%
\pgfpathlineto{\pgfqpoint{2.730660in}{2.199887in}}%
\pgfpathlineto{\pgfqpoint{2.731575in}{2.201093in}}%
\pgfpathlineto{\pgfqpoint{2.732185in}{2.200401in}}%
\pgfpathlineto{\pgfqpoint{2.734321in}{2.198226in}}%
\pgfpathlineto{\pgfqpoint{2.734626in}{2.198540in}}%
\pgfpathlineto{\pgfqpoint{2.735847in}{2.200561in}}%
\pgfpathlineto{\pgfqpoint{2.736762in}{2.199262in}}%
\pgfpathlineto{\pgfqpoint{2.737983in}{2.197689in}}%
\pgfpathlineto{\pgfqpoint{2.738288in}{2.198262in}}%
\pgfpathlineto{\pgfqpoint{2.739508in}{2.201070in}}%
\pgfpathlineto{\pgfqpoint{2.740424in}{2.200069in}}%
\pgfpathlineto{\pgfqpoint{2.741949in}{2.197769in}}%
\pgfpathlineto{\pgfqpoint{2.742865in}{2.198693in}}%
\pgfpathlineto{\pgfqpoint{2.746221in}{2.200025in}}%
\pgfpathlineto{\pgfqpoint{2.750187in}{2.198775in}}%
\pgfpathlineto{\pgfqpoint{2.752323in}{2.199585in}}%
\pgfpathlineto{\pgfqpoint{2.755374in}{2.199413in}}%
\pgfpathlineto{\pgfqpoint{2.756595in}{2.197508in}}%
\pgfpathlineto{\pgfqpoint{2.757205in}{2.198156in}}%
\pgfpathlineto{\pgfqpoint{2.758731in}{2.200566in}}%
\pgfpathlineto{\pgfqpoint{2.759341in}{2.200037in}}%
\pgfpathlineto{\pgfqpoint{2.760867in}{2.198577in}}%
\pgfpathlineto{\pgfqpoint{2.761477in}{2.199218in}}%
\pgfpathlineto{\pgfqpoint{2.763307in}{2.199303in}}%
\pgfpathlineto{\pgfqpoint{2.767884in}{2.198845in}}%
\pgfpathlineto{\pgfqpoint{2.769410in}{2.200095in}}%
\pgfpathlineto{\pgfqpoint{2.770020in}{2.199157in}}%
\pgfpathlineto{\pgfqpoint{2.770935in}{2.198183in}}%
\pgfpathlineto{\pgfqpoint{2.771546in}{2.199414in}}%
\pgfpathlineto{\pgfqpoint{2.772461in}{2.201223in}}%
\pgfpathlineto{\pgfqpoint{2.773071in}{2.200436in}}%
\pgfpathlineto{\pgfqpoint{2.774597in}{2.197440in}}%
\pgfpathlineto{\pgfqpoint{2.775207in}{2.198534in}}%
\pgfpathlineto{\pgfqpoint{2.776428in}{2.200275in}}%
\pgfpathlineto{\pgfqpoint{2.777038in}{2.199795in}}%
\pgfpathlineto{\pgfqpoint{2.779479in}{2.199338in}}%
\pgfpathlineto{\pgfqpoint{2.784666in}{2.199282in}}%
\pgfpathlineto{\pgfqpoint{2.785886in}{2.197225in}}%
\pgfpathlineto{\pgfqpoint{2.786496in}{2.198482in}}%
\pgfpathlineto{\pgfqpoint{2.787717in}{2.200579in}}%
\pgfpathlineto{\pgfqpoint{2.788327in}{2.199685in}}%
\pgfpathlineto{\pgfqpoint{2.789548in}{2.198587in}}%
\pgfpathlineto{\pgfqpoint{2.790158in}{2.199105in}}%
\pgfpathlineto{\pgfqpoint{2.792294in}{2.199190in}}%
\pgfpathlineto{\pgfqpoint{2.796260in}{2.199241in}}%
\pgfpathlineto{\pgfqpoint{2.798396in}{2.199361in}}%
\pgfpathlineto{\pgfqpoint{2.800532in}{2.198571in}}%
\pgfpathlineto{\pgfqpoint{2.801447in}{2.198633in}}%
\pgfpathlineto{\pgfqpoint{2.801752in}{2.199102in}}%
\pgfpathlineto{\pgfqpoint{2.802973in}{2.200059in}}%
\pgfpathlineto{\pgfqpoint{2.803583in}{2.199459in}}%
\pgfpathlineto{\pgfqpoint{2.804804in}{2.198634in}}%
\pgfpathlineto{\pgfqpoint{2.805414in}{2.199395in}}%
\pgfpathlineto{\pgfqpoint{2.807244in}{2.200321in}}%
\pgfpathlineto{\pgfqpoint{2.807550in}{2.199943in}}%
\pgfpathlineto{\pgfqpoint{2.809685in}{2.197717in}}%
\pgfpathlineto{\pgfqpoint{2.810296in}{2.198187in}}%
\pgfpathlineto{\pgfqpoint{2.813042in}{2.201116in}}%
\pgfpathlineto{\pgfqpoint{2.813652in}{2.200203in}}%
\pgfpathlineto{\pgfqpoint{2.815178in}{2.196792in}}%
\pgfpathlineto{\pgfqpoint{2.815788in}{2.198016in}}%
\pgfpathlineto{\pgfqpoint{2.817924in}{2.199459in}}%
\pgfpathlineto{\pgfqpoint{2.822195in}{2.198452in}}%
\pgfpathlineto{\pgfqpoint{2.824026in}{2.198224in}}%
\pgfpathlineto{\pgfqpoint{2.824331in}{2.198776in}}%
\pgfpathlineto{\pgfqpoint{2.826467in}{2.200652in}}%
\pgfpathlineto{\pgfqpoint{2.829518in}{2.198645in}}%
\pgfpathlineto{\pgfqpoint{2.830128in}{2.199541in}}%
\pgfpathlineto{\pgfqpoint{2.832264in}{2.199827in}}%
\pgfpathlineto{\pgfqpoint{2.838672in}{2.199928in}}%
\pgfpathlineto{\pgfqpoint{2.839587in}{2.200858in}}%
\pgfpathlineto{\pgfqpoint{2.840197in}{2.199631in}}%
\pgfpathlineto{\pgfqpoint{2.841418in}{2.197738in}}%
\pgfpathlineto{\pgfqpoint{2.842028in}{2.198905in}}%
\pgfpathlineto{\pgfqpoint{2.842943in}{2.200100in}}%
\pgfpathlineto{\pgfqpoint{2.843859in}{2.199457in}}%
\pgfpathlineto{\pgfqpoint{2.848435in}{2.198469in}}%
\pgfpathlineto{\pgfqpoint{2.851487in}{2.199906in}}%
\pgfpathlineto{\pgfqpoint{2.855758in}{2.199052in}}%
\pgfpathlineto{\pgfqpoint{2.856979in}{2.199860in}}%
\pgfpathlineto{\pgfqpoint{2.857589in}{2.199196in}}%
\pgfpathlineto{\pgfqpoint{2.859115in}{2.198708in}}%
\pgfpathlineto{\pgfqpoint{2.859420in}{2.199022in}}%
\pgfpathlineto{\pgfqpoint{2.861556in}{2.199702in}}%
\pgfpathlineto{\pgfqpoint{2.864302in}{2.197810in}}%
\pgfpathlineto{\pgfqpoint{2.864912in}{2.199419in}}%
\pgfpathlineto{\pgfqpoint{2.866132in}{2.201013in}}%
\pgfpathlineto{\pgfqpoint{2.866743in}{2.200373in}}%
\pgfpathlineto{\pgfqpoint{2.868268in}{2.197578in}}%
\pgfpathlineto{\pgfqpoint{2.868878in}{2.198847in}}%
\pgfpathlineto{\pgfqpoint{2.869794in}{2.200248in}}%
\pgfpathlineto{\pgfqpoint{2.870709in}{2.199327in}}%
\pgfpathlineto{\pgfqpoint{2.871624in}{2.198836in}}%
\pgfpathlineto{\pgfqpoint{2.872235in}{2.199414in}}%
\pgfpathlineto{\pgfqpoint{2.874370in}{2.199145in}}%
\pgfpathlineto{\pgfqpoint{2.877117in}{2.198286in}}%
\pgfpathlineto{\pgfqpoint{2.878032in}{2.197455in}}%
\pgfpathlineto{\pgfqpoint{2.878337in}{2.197950in}}%
\pgfpathlineto{\pgfqpoint{2.879863in}{2.200905in}}%
\pgfpathlineto{\pgfqpoint{2.880473in}{2.200230in}}%
\pgfpathlineto{\pgfqpoint{2.882304in}{2.198758in}}%
\pgfpathlineto{\pgfqpoint{2.882914in}{2.199232in}}%
\pgfpathlineto{\pgfqpoint{2.891457in}{2.198196in}}%
\pgfpathlineto{\pgfqpoint{2.892067in}{2.197685in}}%
\pgfpathlineto{\pgfqpoint{2.892678in}{2.198723in}}%
\pgfpathlineto{\pgfqpoint{2.894203in}{2.200438in}}%
\pgfpathlineto{\pgfqpoint{2.894813in}{2.199776in}}%
\pgfpathlineto{\pgfqpoint{2.898170in}{2.197636in}}%
\pgfpathlineto{\pgfqpoint{2.898475in}{2.198395in}}%
\pgfpathlineto{\pgfqpoint{2.899390in}{2.200822in}}%
\pgfpathlineto{\pgfqpoint{2.900306in}{2.199616in}}%
\pgfpathlineto{\pgfqpoint{2.901526in}{2.197576in}}%
\pgfpathlineto{\pgfqpoint{2.901831in}{2.198366in}}%
\pgfpathlineto{\pgfqpoint{2.902746in}{2.201328in}}%
\pgfpathlineto{\pgfqpoint{2.903662in}{2.199989in}}%
\pgfpathlineto{\pgfqpoint{2.904882in}{2.197094in}}%
\pgfpathlineto{\pgfqpoint{2.905493in}{2.197880in}}%
\pgfpathlineto{\pgfqpoint{2.906713in}{2.199860in}}%
\pgfpathlineto{\pgfqpoint{2.907323in}{2.199330in}}%
\pgfpathlineto{\pgfqpoint{2.908849in}{2.200329in}}%
\pgfpathlineto{\pgfqpoint{2.909459in}{2.200652in}}%
\pgfpathlineto{\pgfqpoint{2.910069in}{2.199536in}}%
\pgfpathlineto{\pgfqpoint{2.911290in}{2.197628in}}%
\pgfpathlineto{\pgfqpoint{2.911900in}{2.198366in}}%
\pgfpathlineto{\pgfqpoint{2.913731in}{2.200434in}}%
\pgfpathlineto{\pgfqpoint{2.914341in}{2.199878in}}%
\pgfpathlineto{\pgfqpoint{2.916782in}{2.198843in}}%
\pgfpathlineto{\pgfqpoint{2.919223in}{2.199315in}}%
\pgfpathlineto{\pgfqpoint{2.922274in}{2.199306in}}%
\pgfpathlineto{\pgfqpoint{2.923800in}{2.199927in}}%
\pgfpathlineto{\pgfqpoint{2.924105in}{2.199138in}}%
\pgfpathlineto{\pgfqpoint{2.925020in}{2.197313in}}%
\pgfpathlineto{\pgfqpoint{2.925630in}{2.198853in}}%
\pgfpathlineto{\pgfqpoint{2.926546in}{2.201558in}}%
\pgfpathlineto{\pgfqpoint{2.927156in}{2.200832in}}%
\pgfpathlineto{\pgfqpoint{2.928376in}{2.196896in}}%
\pgfpathlineto{\pgfqpoint{2.928987in}{2.198320in}}%
\pgfpathlineto{\pgfqpoint{2.929902in}{2.200629in}}%
\pgfpathlineto{\pgfqpoint{2.930512in}{2.199777in}}%
\pgfpathlineto{\pgfqpoint{2.931733in}{2.197935in}}%
\pgfpathlineto{\pgfqpoint{2.932038in}{2.198564in}}%
\pgfpathlineto{\pgfqpoint{2.932953in}{2.200551in}}%
\pgfpathlineto{\pgfqpoint{2.933869in}{2.199722in}}%
\pgfpathlineto{\pgfqpoint{2.935699in}{2.198240in}}%
\pgfpathlineto{\pgfqpoint{2.936309in}{2.198791in}}%
\pgfpathlineto{\pgfqpoint{2.938445in}{2.200446in}}%
\pgfpathlineto{\pgfqpoint{2.939056in}{2.199655in}}%
\pgfpathlineto{\pgfqpoint{2.940581in}{2.198691in}}%
\pgfpathlineto{\pgfqpoint{2.941191in}{2.199096in}}%
\pgfpathlineto{\pgfqpoint{2.945158in}{2.199382in}}%
\pgfpathlineto{\pgfqpoint{2.946073in}{2.198312in}}%
\pgfpathlineto{\pgfqpoint{2.946683in}{2.198979in}}%
\pgfpathlineto{\pgfqpoint{2.947599in}{2.200076in}}%
\pgfpathlineto{\pgfqpoint{2.948209in}{2.199491in}}%
\pgfpathlineto{\pgfqpoint{2.949430in}{2.198217in}}%
\pgfpathlineto{\pgfqpoint{2.950040in}{2.199174in}}%
\pgfpathlineto{\pgfqpoint{2.952786in}{2.200063in}}%
\pgfpathlineto{\pgfqpoint{2.954617in}{2.198386in}}%
\pgfpathlineto{\pgfqpoint{2.955227in}{2.199801in}}%
\pgfpathlineto{\pgfqpoint{2.956142in}{2.200587in}}%
\pgfpathlineto{\pgfqpoint{2.956752in}{2.199809in}}%
\pgfpathlineto{\pgfqpoint{2.957973in}{2.198221in}}%
\pgfpathlineto{\pgfqpoint{2.958583in}{2.198926in}}%
\pgfpathlineto{\pgfqpoint{2.960414in}{2.198716in}}%
\pgfpathlineto{\pgfqpoint{2.966516in}{2.200365in}}%
\pgfpathlineto{\pgfqpoint{2.967431in}{2.201163in}}%
\pgfpathlineto{\pgfqpoint{2.968042in}{2.200352in}}%
\pgfpathlineto{\pgfqpoint{2.969262in}{2.198576in}}%
\pgfpathlineto{\pgfqpoint{2.969872in}{2.199172in}}%
\pgfpathlineto{\pgfqpoint{2.970788in}{2.199871in}}%
\pgfpathlineto{\pgfqpoint{2.971398in}{2.198748in}}%
\pgfpathlineto{\pgfqpoint{2.972313in}{2.197156in}}%
\pgfpathlineto{\pgfqpoint{2.972924in}{2.198223in}}%
\pgfpathlineto{\pgfqpoint{2.974144in}{2.201102in}}%
\pgfpathlineto{\pgfqpoint{2.974754in}{2.200288in}}%
\pgfpathlineto{\pgfqpoint{2.977806in}{2.198206in}}%
\pgfpathlineto{\pgfqpoint{2.978721in}{2.198108in}}%
\pgfpathlineto{\pgfqpoint{2.979026in}{2.198529in}}%
\pgfpathlineto{\pgfqpoint{2.980246in}{2.199969in}}%
\pgfpathlineto{\pgfqpoint{2.981162in}{2.199317in}}%
\pgfpathlineto{\pgfqpoint{2.985128in}{2.199932in}}%
\pgfpathlineto{\pgfqpoint{2.985739in}{2.200618in}}%
\pgfpathlineto{\pgfqpoint{2.986349in}{2.199938in}}%
\pgfpathlineto{\pgfqpoint{2.987264in}{2.198985in}}%
\pgfpathlineto{\pgfqpoint{2.987874in}{2.199880in}}%
\pgfpathlineto{\pgfqpoint{2.988790in}{2.200851in}}%
\pgfpathlineto{\pgfqpoint{2.989400in}{2.200059in}}%
\pgfpathlineto{\pgfqpoint{2.991231in}{2.197300in}}%
\pgfpathlineto{\pgfqpoint{2.991841in}{2.198502in}}%
\pgfpathlineto{\pgfqpoint{2.993367in}{2.200347in}}%
\pgfpathlineto{\pgfqpoint{2.993977in}{2.199819in}}%
\pgfpathlineto{\pgfqpoint{2.995197in}{2.198091in}}%
\pgfpathlineto{\pgfqpoint{2.995807in}{2.199310in}}%
\pgfpathlineto{\pgfqpoint{2.996723in}{2.200619in}}%
\pgfpathlineto{\pgfqpoint{2.997333in}{2.199637in}}%
\pgfpathlineto{\pgfqpoint{2.998554in}{2.197848in}}%
\pgfpathlineto{\pgfqpoint{2.999164in}{2.199072in}}%
\pgfpathlineto{\pgfqpoint{3.000384in}{2.201014in}}%
\pgfpathlineto{\pgfqpoint{3.000994in}{2.199883in}}%
\pgfpathlineto{\pgfqpoint{3.002215in}{2.198595in}}%
\pgfpathlineto{\pgfqpoint{3.002825in}{2.199055in}}%
\pgfpathlineto{\pgfqpoint{3.004351in}{2.198564in}}%
\pgfpathlineto{\pgfqpoint{3.005571in}{2.198021in}}%
\pgfpathlineto{\pgfqpoint{3.005876in}{2.198523in}}%
\pgfpathlineto{\pgfqpoint{3.007402in}{2.200972in}}%
\pgfpathlineto{\pgfqpoint{3.008012in}{2.200505in}}%
\pgfpathlineto{\pgfqpoint{3.010758in}{2.198883in}}%
\pgfpathlineto{\pgfqpoint{3.013809in}{2.198795in}}%
\pgfpathlineto{\pgfqpoint{3.014725in}{2.198967in}}%
\pgfpathlineto{\pgfqpoint{3.015030in}{2.199539in}}%
\pgfpathlineto{\pgfqpoint{3.016250in}{2.200879in}}%
\pgfpathlineto{\pgfqpoint{3.016556in}{2.200435in}}%
\pgfpathlineto{\pgfqpoint{3.017776in}{2.197359in}}%
\pgfpathlineto{\pgfqpoint{3.018386in}{2.198481in}}%
\pgfpathlineto{\pgfqpoint{3.019607in}{2.200244in}}%
\pgfpathlineto{\pgfqpoint{3.020217in}{2.199715in}}%
\pgfpathlineto{\pgfqpoint{3.021437in}{2.199096in}}%
\pgfpathlineto{\pgfqpoint{3.022048in}{2.199753in}}%
\pgfpathlineto{\pgfqpoint{3.023573in}{2.199045in}}%
\pgfpathlineto{\pgfqpoint{3.024489in}{2.197873in}}%
\pgfpathlineto{\pgfqpoint{3.025099in}{2.198740in}}%
\pgfpathlineto{\pgfqpoint{3.026319in}{2.200194in}}%
\pgfpathlineto{\pgfqpoint{3.026930in}{2.199653in}}%
\pgfpathlineto{\pgfqpoint{3.028150in}{2.198260in}}%
\pgfpathlineto{\pgfqpoint{3.028760in}{2.198943in}}%
\pgfpathlineto{\pgfqpoint{3.030591in}{2.200189in}}%
\pgfpathlineto{\pgfqpoint{3.030896in}{2.199691in}}%
\pgfpathlineto{\pgfqpoint{3.031811in}{2.197836in}}%
\pgfpathlineto{\pgfqpoint{3.032727in}{2.199207in}}%
\pgfpathlineto{\pgfqpoint{3.033947in}{2.200026in}}%
\pgfpathlineto{\pgfqpoint{3.034252in}{2.199711in}}%
\pgfpathlineto{\pgfqpoint{3.035473in}{2.198642in}}%
\pgfpathlineto{\pgfqpoint{3.036083in}{2.199325in}}%
\pgfpathlineto{\pgfqpoint{3.039134in}{2.200079in}}%
\pgfpathlineto{\pgfqpoint{3.040660in}{2.198232in}}%
\pgfpathlineto{\pgfqpoint{3.041270in}{2.198900in}}%
\pgfpathlineto{\pgfqpoint{3.044016in}{2.200085in}}%
\pgfpathlineto{\pgfqpoint{3.046152in}{2.198293in}}%
\pgfpathlineto{\pgfqpoint{3.046762in}{2.199393in}}%
\pgfpathlineto{\pgfqpoint{3.047678in}{2.200065in}}%
\pgfpathlineto{\pgfqpoint{3.048288in}{2.199155in}}%
\pgfpathlineto{\pgfqpoint{3.049203in}{2.198506in}}%
\pgfpathlineto{\pgfqpoint{3.049813in}{2.199330in}}%
\pgfpathlineto{\pgfqpoint{3.050729in}{2.200281in}}%
\pgfpathlineto{\pgfqpoint{3.051339in}{2.199020in}}%
\pgfpathlineto{\pgfqpoint{3.051949in}{2.198432in}}%
\pgfpathlineto{\pgfqpoint{3.052865in}{2.199353in}}%
\pgfpathlineto{\pgfqpoint{3.053780in}{2.199746in}}%
\pgfpathlineto{\pgfqpoint{3.054390in}{2.198750in}}%
\pgfpathlineto{\pgfqpoint{3.055000in}{2.198185in}}%
\pgfpathlineto{\pgfqpoint{3.055611in}{2.199361in}}%
\pgfpathlineto{\pgfqpoint{3.057136in}{2.200551in}}%
\pgfpathlineto{\pgfqpoint{3.057441in}{2.200245in}}%
\pgfpathlineto{\pgfqpoint{3.058662in}{2.198024in}}%
\pgfpathlineto{\pgfqpoint{3.059577in}{2.198897in}}%
\pgfpathlineto{\pgfqpoint{3.063849in}{2.199233in}}%
\pgfpathlineto{\pgfqpoint{3.064764in}{2.198756in}}%
\pgfpathlineto{\pgfqpoint{3.065374in}{2.199636in}}%
\pgfpathlineto{\pgfqpoint{3.065985in}{2.199733in}}%
\pgfpathlineto{\pgfqpoint{3.066595in}{2.198868in}}%
\pgfpathlineto{\pgfqpoint{3.067510in}{2.198584in}}%
\pgfpathlineto{\pgfqpoint{3.068120in}{2.199323in}}%
\pgfpathlineto{\pgfqpoint{3.069036in}{2.200105in}}%
\pgfpathlineto{\pgfqpoint{3.069646in}{2.199437in}}%
\pgfpathlineto{\pgfqpoint{3.070561in}{2.198815in}}%
\pgfpathlineto{\pgfqpoint{3.071172in}{2.199539in}}%
\pgfpathlineto{\pgfqpoint{3.072392in}{2.199409in}}%
\pgfpathlineto{\pgfqpoint{3.072697in}{2.199011in}}%
\pgfpathlineto{\pgfqpoint{3.073613in}{2.198435in}}%
\pgfpathlineto{\pgfqpoint{3.073918in}{2.198916in}}%
\pgfpathlineto{\pgfqpoint{3.074833in}{2.200294in}}%
\pgfpathlineto{\pgfqpoint{3.075443in}{2.199616in}}%
\pgfpathlineto{\pgfqpoint{3.076664in}{2.198391in}}%
\pgfpathlineto{\pgfqpoint{3.076969in}{2.198896in}}%
\pgfpathlineto{\pgfqpoint{3.077884in}{2.200133in}}%
\pgfpathlineto{\pgfqpoint{3.078494in}{2.198828in}}%
\pgfpathlineto{\pgfqpoint{3.079410in}{2.197912in}}%
\pgfpathlineto{\pgfqpoint{3.079715in}{2.198478in}}%
\pgfpathlineto{\pgfqpoint{3.080935in}{2.201694in}}%
\pgfpathlineto{\pgfqpoint{3.081546in}{2.200150in}}%
\pgfpathlineto{\pgfqpoint{3.082766in}{2.197350in}}%
\pgfpathlineto{\pgfqpoint{3.083376in}{2.198355in}}%
\pgfpathlineto{\pgfqpoint{3.084597in}{2.200697in}}%
\pgfpathlineto{\pgfqpoint{3.085207in}{2.199420in}}%
\pgfpathlineto{\pgfqpoint{3.086733in}{2.199637in}}%
\pgfpathlineto{\pgfqpoint{3.089784in}{2.199289in}}%
\pgfpathlineto{\pgfqpoint{3.091004in}{2.197409in}}%
\pgfpathlineto{\pgfqpoint{3.091615in}{2.198431in}}%
\pgfpathlineto{\pgfqpoint{3.092835in}{2.200589in}}%
\pgfpathlineto{\pgfqpoint{3.093445in}{2.199734in}}%
\pgfpathlineto{\pgfqpoint{3.094361in}{2.199028in}}%
\pgfpathlineto{\pgfqpoint{3.094971in}{2.199967in}}%
\pgfpathlineto{\pgfqpoint{3.095581in}{2.200526in}}%
\pgfpathlineto{\pgfqpoint{3.096191in}{2.199395in}}%
\pgfpathlineto{\pgfqpoint{3.097412in}{2.197762in}}%
\pgfpathlineto{\pgfqpoint{3.098022in}{2.198623in}}%
\pgfpathlineto{\pgfqpoint{3.098937in}{2.199883in}}%
\pgfpathlineto{\pgfqpoint{3.099853in}{2.198880in}}%
\pgfpathlineto{\pgfqpoint{3.101073in}{2.200180in}}%
\pgfpathlineto{\pgfqpoint{3.101683in}{2.200547in}}%
\pgfpathlineto{\pgfqpoint{3.102294in}{2.199323in}}%
\pgfpathlineto{\pgfqpoint{3.103209in}{2.198252in}}%
\pgfpathlineto{\pgfqpoint{3.103819in}{2.199406in}}%
\pgfpathlineto{\pgfqpoint{3.104430in}{2.200148in}}%
\pgfpathlineto{\pgfqpoint{3.105345in}{2.199168in}}%
\pgfpathlineto{\pgfqpoint{3.106870in}{2.199659in}}%
\pgfpathlineto{\pgfqpoint{3.108091in}{2.198882in}}%
\pgfpathlineto{\pgfqpoint{3.109617in}{2.198571in}}%
\pgfpathlineto{\pgfqpoint{3.109922in}{2.199093in}}%
\pgfpathlineto{\pgfqpoint{3.110837in}{2.200809in}}%
\pgfpathlineto{\pgfqpoint{3.111752in}{2.199573in}}%
\pgfpathlineto{\pgfqpoint{3.113278in}{2.198263in}}%
\pgfpathlineto{\pgfqpoint{3.113888in}{2.198986in}}%
\pgfpathlineto{\pgfqpoint{3.115109in}{2.200667in}}%
\pgfpathlineto{\pgfqpoint{3.115719in}{2.200278in}}%
\pgfpathlineto{\pgfqpoint{3.117855in}{2.198261in}}%
\pgfpathlineto{\pgfqpoint{3.118465in}{2.199325in}}%
\pgfpathlineto{\pgfqpoint{3.119380in}{2.200250in}}%
\pgfpathlineto{\pgfqpoint{3.119991in}{2.199368in}}%
\pgfpathlineto{\pgfqpoint{3.121211in}{2.197548in}}%
\pgfpathlineto{\pgfqpoint{3.121821in}{2.198770in}}%
\pgfpathlineto{\pgfqpoint{3.122737in}{2.200217in}}%
\pgfpathlineto{\pgfqpoint{3.123347in}{2.199125in}}%
\pgfpathlineto{\pgfqpoint{3.123957in}{2.198497in}}%
\pgfpathlineto{\pgfqpoint{3.124567in}{2.199729in}}%
\pgfpathlineto{\pgfqpoint{3.125483in}{2.201577in}}%
\pgfpathlineto{\pgfqpoint{3.126093in}{2.199560in}}%
\pgfpathlineto{\pgfqpoint{3.127313in}{2.197299in}}%
\pgfpathlineto{\pgfqpoint{3.127924in}{2.198210in}}%
\pgfpathlineto{\pgfqpoint{3.129144in}{2.200846in}}%
\pgfpathlineto{\pgfqpoint{3.129754in}{2.199836in}}%
\pgfpathlineto{\pgfqpoint{3.131890in}{2.198735in}}%
\pgfpathlineto{\pgfqpoint{3.136772in}{2.200243in}}%
\pgfpathlineto{\pgfqpoint{3.138298in}{2.197718in}}%
\pgfpathlineto{\pgfqpoint{3.138908in}{2.198818in}}%
\pgfpathlineto{\pgfqpoint{3.139823in}{2.200657in}}%
\pgfpathlineto{\pgfqpoint{3.140433in}{2.199775in}}%
\pgfpathlineto{\pgfqpoint{3.141654in}{2.197088in}}%
\pgfpathlineto{\pgfqpoint{3.142264in}{2.198281in}}%
\pgfpathlineto{\pgfqpoint{3.143485in}{2.201960in}}%
\pgfpathlineto{\pgfqpoint{3.144095in}{2.200673in}}%
\pgfpathlineto{\pgfqpoint{3.145315in}{2.197404in}}%
\pgfpathlineto{\pgfqpoint{3.145926in}{2.197906in}}%
\pgfpathlineto{\pgfqpoint{3.147451in}{2.200846in}}%
\pgfpathlineto{\pgfqpoint{3.148367in}{2.199579in}}%
\pgfpathlineto{\pgfqpoint{3.150197in}{2.199542in}}%
\pgfpathlineto{\pgfqpoint{3.152333in}{2.198466in}}%
\pgfpathlineto{\pgfqpoint{3.153859in}{2.198615in}}%
\pgfpathlineto{\pgfqpoint{3.156300in}{2.199686in}}%
\pgfpathlineto{\pgfqpoint{3.161181in}{2.197828in}}%
\pgfpathlineto{\pgfqpoint{3.161487in}{2.198525in}}%
\pgfpathlineto{\pgfqpoint{3.162707in}{2.201169in}}%
\pgfpathlineto{\pgfqpoint{3.163317in}{2.200050in}}%
\pgfpathlineto{\pgfqpoint{3.164538in}{2.198119in}}%
\pgfpathlineto{\pgfqpoint{3.165148in}{2.199272in}}%
\pgfpathlineto{\pgfqpoint{3.166063in}{2.200744in}}%
\pgfpathlineto{\pgfqpoint{3.166674in}{2.199765in}}%
\pgfpathlineto{\pgfqpoint{3.167894in}{2.196453in}}%
\pgfpathlineto{\pgfqpoint{3.168504in}{2.197889in}}%
\pgfpathlineto{\pgfqpoint{3.169725in}{2.201834in}}%
\pgfpathlineto{\pgfqpoint{3.170335in}{2.200318in}}%
\pgfpathlineto{\pgfqpoint{3.171556in}{2.196902in}}%
\pgfpathlineto{\pgfqpoint{3.172166in}{2.198642in}}%
\pgfpathlineto{\pgfqpoint{3.173386in}{2.201649in}}%
\pgfpathlineto{\pgfqpoint{3.173996in}{2.199866in}}%
\pgfpathlineto{\pgfqpoint{3.175217in}{2.197484in}}%
\pgfpathlineto{\pgfqpoint{3.175827in}{2.198192in}}%
\pgfpathlineto{\pgfqpoint{3.177353in}{2.200469in}}%
\pgfpathlineto{\pgfqpoint{3.177963in}{2.199427in}}%
\pgfpathlineto{\pgfqpoint{3.178878in}{2.198958in}}%
\pgfpathlineto{\pgfqpoint{3.179489in}{2.199801in}}%
\pgfpathlineto{\pgfqpoint{3.180404in}{2.200240in}}%
\pgfpathlineto{\pgfqpoint{3.180709in}{2.199650in}}%
\pgfpathlineto{\pgfqpoint{3.181930in}{2.198170in}}%
\pgfpathlineto{\pgfqpoint{3.182540in}{2.198979in}}%
\pgfpathlineto{\pgfqpoint{3.184065in}{2.200019in}}%
\pgfpathlineto{\pgfqpoint{3.184370in}{2.199690in}}%
\pgfpathlineto{\pgfqpoint{3.185591in}{2.198593in}}%
\pgfpathlineto{\pgfqpoint{3.186201in}{2.199233in}}%
\pgfpathlineto{\pgfqpoint{3.187727in}{2.200221in}}%
\pgfpathlineto{\pgfqpoint{3.188032in}{2.199847in}}%
\pgfpathlineto{\pgfqpoint{3.191083in}{2.197518in}}%
\pgfpathlineto{\pgfqpoint{3.193219in}{2.201241in}}%
\pgfpathlineto{\pgfqpoint{3.194134in}{2.199902in}}%
\pgfpathlineto{\pgfqpoint{3.196880in}{2.198670in}}%
\pgfpathlineto{\pgfqpoint{3.200542in}{2.197973in}}%
\pgfpathlineto{\pgfqpoint{3.201457in}{2.198351in}}%
\pgfpathlineto{\pgfqpoint{3.202678in}{2.201115in}}%
\pgfpathlineto{\pgfqpoint{3.203288in}{2.200120in}}%
\pgfpathlineto{\pgfqpoint{3.204508in}{2.197340in}}%
\pgfpathlineto{\pgfqpoint{3.205119in}{2.198037in}}%
\pgfpathlineto{\pgfqpoint{3.206644in}{2.201978in}}%
\pgfpathlineto{\pgfqpoint{3.207254in}{2.200446in}}%
\pgfpathlineto{\pgfqpoint{3.208475in}{2.197084in}}%
\pgfpathlineto{\pgfqpoint{3.209085in}{2.198138in}}%
\pgfpathlineto{\pgfqpoint{3.210611in}{2.200780in}}%
\pgfpathlineto{\pgfqpoint{3.211221in}{2.199457in}}%
\pgfpathlineto{\pgfqpoint{3.212136in}{2.198158in}}%
\pgfpathlineto{\pgfqpoint{3.212746in}{2.199372in}}%
\pgfpathlineto{\pgfqpoint{3.213662in}{2.200974in}}%
\pgfpathlineto{\pgfqpoint{3.214272in}{2.199663in}}%
\pgfpathlineto{\pgfqpoint{3.215493in}{2.197599in}}%
\pgfpathlineto{\pgfqpoint{3.216103in}{2.198684in}}%
\pgfpathlineto{\pgfqpoint{3.217323in}{2.200681in}}%
\pgfpathlineto{\pgfqpoint{3.217933in}{2.199517in}}%
\pgfpathlineto{\pgfqpoint{3.218849in}{2.198309in}}%
\pgfpathlineto{\pgfqpoint{3.219764in}{2.199076in}}%
\pgfpathlineto{\pgfqpoint{3.222815in}{2.199784in}}%
\pgfpathlineto{\pgfqpoint{3.224341in}{2.199152in}}%
\pgfpathlineto{\pgfqpoint{3.224646in}{2.199471in}}%
\pgfpathlineto{\pgfqpoint{3.225561in}{2.199835in}}%
\pgfpathlineto{\pgfqpoint{3.225867in}{2.199441in}}%
\pgfpathlineto{\pgfqpoint{3.227087in}{2.197227in}}%
\pgfpathlineto{\pgfqpoint{3.227697in}{2.198617in}}%
\pgfpathlineto{\pgfqpoint{3.228918in}{2.201334in}}%
\pgfpathlineto{\pgfqpoint{3.229528in}{2.200602in}}%
\pgfpathlineto{\pgfqpoint{3.231054in}{2.198086in}}%
\pgfpathlineto{\pgfqpoint{3.231664in}{2.199061in}}%
\pgfpathlineto{\pgfqpoint{3.232579in}{2.199981in}}%
\pgfpathlineto{\pgfqpoint{3.233189in}{2.199113in}}%
\pgfpathlineto{\pgfqpoint{3.234105in}{2.198343in}}%
\pgfpathlineto{\pgfqpoint{3.234715in}{2.199133in}}%
\pgfpathlineto{\pgfqpoint{3.235935in}{2.201488in}}%
\pgfpathlineto{\pgfqpoint{3.236546in}{2.200692in}}%
\pgfpathlineto{\pgfqpoint{3.238071in}{2.197451in}}%
\pgfpathlineto{\pgfqpoint{3.238681in}{2.198311in}}%
\pgfpathlineto{\pgfqpoint{3.239902in}{2.199690in}}%
\pgfpathlineto{\pgfqpoint{3.240512in}{2.198930in}}%
\pgfpathlineto{\pgfqpoint{3.241428in}{2.197833in}}%
\pgfpathlineto{\pgfqpoint{3.242038in}{2.199048in}}%
\pgfpathlineto{\pgfqpoint{3.243563in}{2.201898in}}%
\pgfpathlineto{\pgfqpoint{3.243869in}{2.201319in}}%
\pgfpathlineto{\pgfqpoint{3.245089in}{2.197642in}}%
\pgfpathlineto{\pgfqpoint{3.246004in}{2.199131in}}%
\pgfpathlineto{\pgfqpoint{3.246920in}{2.200321in}}%
\pgfpathlineto{\pgfqpoint{3.247530in}{2.199132in}}%
\pgfpathlineto{\pgfqpoint{3.248750in}{2.197319in}}%
\pgfpathlineto{\pgfqpoint{3.249056in}{2.197855in}}%
\pgfpathlineto{\pgfqpoint{3.250276in}{2.201365in}}%
\pgfpathlineto{\pgfqpoint{3.251191in}{2.199659in}}%
\pgfpathlineto{\pgfqpoint{3.252107in}{2.197530in}}%
\pgfpathlineto{\pgfqpoint{3.252717in}{2.198252in}}%
\pgfpathlineto{\pgfqpoint{3.254243in}{2.200427in}}%
\pgfpathlineto{\pgfqpoint{3.254853in}{2.200012in}}%
\pgfpathlineto{\pgfqpoint{3.257904in}{2.199447in}}%
\pgfpathlineto{\pgfqpoint{3.259124in}{2.199778in}}%
\pgfpathlineto{\pgfqpoint{3.259430in}{2.199415in}}%
\pgfpathlineto{\pgfqpoint{3.260650in}{2.198836in}}%
\pgfpathlineto{\pgfqpoint{3.261260in}{2.199505in}}%
\pgfpathlineto{\pgfqpoint{3.262176in}{2.199928in}}%
\pgfpathlineto{\pgfqpoint{3.262481in}{2.199540in}}%
\pgfpathlineto{\pgfqpoint{3.263701in}{2.197625in}}%
\pgfpathlineto{\pgfqpoint{3.264311in}{2.198882in}}%
\pgfpathlineto{\pgfqpoint{3.265837in}{2.201096in}}%
\pgfpathlineto{\pgfqpoint{3.266447in}{2.200044in}}%
\pgfpathlineto{\pgfqpoint{3.267363in}{2.198345in}}%
\pgfpathlineto{\pgfqpoint{3.268278in}{2.199146in}}%
\pgfpathlineto{\pgfqpoint{3.270109in}{2.199170in}}%
\pgfpathlineto{\pgfqpoint{3.271634in}{2.198998in}}%
\pgfpathlineto{\pgfqpoint{3.271939in}{2.199330in}}%
\pgfpathlineto{\pgfqpoint{3.273465in}{2.200117in}}%
\pgfpathlineto{\pgfqpoint{3.273770in}{2.199783in}}%
\pgfpathlineto{\pgfqpoint{3.274991in}{2.197287in}}%
\pgfpathlineto{\pgfqpoint{3.275601in}{2.198925in}}%
\pgfpathlineto{\pgfqpoint{3.276516in}{2.201684in}}%
\pgfpathlineto{\pgfqpoint{3.277126in}{2.200674in}}%
\pgfpathlineto{\pgfqpoint{3.278347in}{2.197012in}}%
\pgfpathlineto{\pgfqpoint{3.278957in}{2.198217in}}%
\pgfpathlineto{\pgfqpoint{3.280178in}{2.201711in}}%
\pgfpathlineto{\pgfqpoint{3.280788in}{2.200506in}}%
\pgfpathlineto{\pgfqpoint{3.282008in}{2.196788in}}%
\pgfpathlineto{\pgfqpoint{3.282619in}{2.198007in}}%
\pgfpathlineto{\pgfqpoint{3.283839in}{2.201292in}}%
\pgfpathlineto{\pgfqpoint{3.284449in}{2.200154in}}%
\pgfpathlineto{\pgfqpoint{3.285365in}{2.198237in}}%
\pgfpathlineto{\pgfqpoint{3.285975in}{2.199238in}}%
\pgfpathlineto{\pgfqpoint{3.286890in}{2.200579in}}%
\pgfpathlineto{\pgfqpoint{3.287500in}{2.199672in}}%
\pgfpathlineto{\pgfqpoint{3.289026in}{2.197214in}}%
\pgfpathlineto{\pgfqpoint{3.289636in}{2.197834in}}%
\pgfpathlineto{\pgfqpoint{3.291162in}{2.201775in}}%
\pgfpathlineto{\pgfqpoint{3.292077in}{2.200429in}}%
\pgfpathlineto{\pgfqpoint{3.293603in}{2.197752in}}%
\pgfpathlineto{\pgfqpoint{3.294213in}{2.198440in}}%
\pgfpathlineto{\pgfqpoint{3.295433in}{2.200248in}}%
\pgfpathlineto{\pgfqpoint{3.296044in}{2.199155in}}%
\pgfpathlineto{\pgfqpoint{3.296959in}{2.198007in}}%
\pgfpathlineto{\pgfqpoint{3.297569in}{2.198792in}}%
\pgfpathlineto{\pgfqpoint{3.298790in}{2.201874in}}%
\pgfpathlineto{\pgfqpoint{3.299400in}{2.200371in}}%
\pgfpathlineto{\pgfqpoint{3.300620in}{2.197535in}}%
\pgfpathlineto{\pgfqpoint{3.301231in}{2.198156in}}%
\pgfpathlineto{\pgfqpoint{3.302451in}{2.199702in}}%
\pgfpathlineto{\pgfqpoint{3.303061in}{2.199220in}}%
\pgfpathlineto{\pgfqpoint{3.304282in}{2.198476in}}%
\pgfpathlineto{\pgfqpoint{3.304587in}{2.198971in}}%
\pgfpathlineto{\pgfqpoint{3.305807in}{2.201190in}}%
\pgfpathlineto{\pgfqpoint{3.306418in}{2.200426in}}%
\pgfpathlineto{\pgfqpoint{3.307943in}{2.197615in}}%
\pgfpathlineto{\pgfqpoint{3.308554in}{2.198342in}}%
\pgfpathlineto{\pgfqpoint{3.310079in}{2.199964in}}%
\pgfpathlineto{\pgfqpoint{3.310689in}{2.199362in}}%
\pgfpathlineto{\pgfqpoint{3.311605in}{2.198134in}}%
\pgfpathlineto{\pgfqpoint{3.312215in}{2.198750in}}%
\pgfpathlineto{\pgfqpoint{3.313741in}{2.201140in}}%
\pgfpathlineto{\pgfqpoint{3.314351in}{2.199769in}}%
\pgfpathlineto{\pgfqpoint{3.315266in}{2.197329in}}%
\pgfpathlineto{\pgfqpoint{3.316181in}{2.198804in}}%
\pgfpathlineto{\pgfqpoint{3.317097in}{2.201573in}}%
\pgfpathlineto{\pgfqpoint{3.317707in}{2.200385in}}%
\pgfpathlineto{\pgfqpoint{3.318928in}{2.197543in}}%
\pgfpathlineto{\pgfqpoint{3.319538in}{2.198766in}}%
\pgfpathlineto{\pgfqpoint{3.320758in}{2.201861in}}%
\pgfpathlineto{\pgfqpoint{3.321063in}{2.200952in}}%
\pgfpathlineto{\pgfqpoint{3.322284in}{2.197112in}}%
\pgfpathlineto{\pgfqpoint{3.322894in}{2.198101in}}%
\pgfpathlineto{\pgfqpoint{3.324115in}{2.201426in}}%
\pgfpathlineto{\pgfqpoint{3.324725in}{2.200097in}}%
\pgfpathlineto{\pgfqpoint{3.325945in}{2.197124in}}%
\pgfpathlineto{\pgfqpoint{3.326556in}{2.198200in}}%
\pgfpathlineto{\pgfqpoint{3.328691in}{2.200723in}}%
\pgfpathlineto{\pgfqpoint{3.328996in}{2.200533in}}%
\pgfpathlineto{\pgfqpoint{3.331743in}{2.199188in}}%
\pgfpathlineto{\pgfqpoint{3.333573in}{2.198802in}}%
\pgfpathlineto{\pgfqpoint{3.333878in}{2.199379in}}%
\pgfpathlineto{\pgfqpoint{3.334794in}{2.200491in}}%
\pgfpathlineto{\pgfqpoint{3.335404in}{2.199912in}}%
\pgfpathlineto{\pgfqpoint{3.336930in}{2.198558in}}%
\pgfpathlineto{\pgfqpoint{3.337540in}{2.199105in}}%
\pgfpathlineto{\pgfqpoint{3.339676in}{2.199485in}}%
\pgfpathlineto{\pgfqpoint{3.341506in}{2.198446in}}%
\pgfpathlineto{\pgfqpoint{3.341811in}{2.199010in}}%
\pgfpathlineto{\pgfqpoint{3.343337in}{2.200968in}}%
\pgfpathlineto{\pgfqpoint{3.343642in}{2.200583in}}%
\pgfpathlineto{\pgfqpoint{3.346388in}{2.198227in}}%
\pgfpathlineto{\pgfqpoint{3.347914in}{2.199294in}}%
\pgfpathlineto{\pgfqpoint{3.349744in}{2.200765in}}%
\pgfpathlineto{\pgfqpoint{3.350050in}{2.200546in}}%
\pgfpathlineto{\pgfqpoint{3.351880in}{2.196719in}}%
\pgfpathlineto{\pgfqpoint{3.352796in}{2.198742in}}%
\pgfpathlineto{\pgfqpoint{3.354016in}{2.201667in}}%
\pgfpathlineto{\pgfqpoint{3.354626in}{2.200014in}}%
\pgfpathlineto{\pgfqpoint{3.355847in}{2.197843in}}%
\pgfpathlineto{\pgfqpoint{3.356457in}{2.198727in}}%
\pgfpathlineto{\pgfqpoint{3.357678in}{2.200536in}}%
\pgfpathlineto{\pgfqpoint{3.358288in}{2.199877in}}%
\pgfpathlineto{\pgfqpoint{3.360424in}{2.198894in}}%
\pgfpathlineto{\pgfqpoint{3.364695in}{2.199990in}}%
\pgfpathlineto{\pgfqpoint{3.366221in}{2.197282in}}%
\pgfpathlineto{\pgfqpoint{3.366831in}{2.198370in}}%
\pgfpathlineto{\pgfqpoint{3.368052in}{2.200950in}}%
\pgfpathlineto{\pgfqpoint{3.368662in}{2.200357in}}%
\pgfpathlineto{\pgfqpoint{3.369882in}{2.198533in}}%
\pgfpathlineto{\pgfqpoint{3.370798in}{2.199417in}}%
\pgfpathlineto{\pgfqpoint{3.372933in}{2.199315in}}%
\pgfpathlineto{\pgfqpoint{3.377815in}{2.199930in}}%
\pgfpathlineto{\pgfqpoint{3.379036in}{2.201794in}}%
\pgfpathlineto{\pgfqpoint{3.379646in}{2.200733in}}%
\pgfpathlineto{\pgfqpoint{3.381172in}{2.196914in}}%
\pgfpathlineto{\pgfqpoint{3.381782in}{2.197925in}}%
\pgfpathlineto{\pgfqpoint{3.383307in}{2.200408in}}%
\pgfpathlineto{\pgfqpoint{3.383918in}{2.199960in}}%
\pgfpathlineto{\pgfqpoint{3.386054in}{2.199419in}}%
\pgfpathlineto{\pgfqpoint{3.387579in}{2.198549in}}%
\pgfpathlineto{\pgfqpoint{3.388189in}{2.198145in}}%
\pgfpathlineto{\pgfqpoint{3.388800in}{2.199002in}}%
\pgfpathlineto{\pgfqpoint{3.390020in}{2.200303in}}%
\pgfpathlineto{\pgfqpoint{3.390630in}{2.199860in}}%
\pgfpathlineto{\pgfqpoint{3.392766in}{2.199789in}}%
\pgfpathlineto{\pgfqpoint{3.394597in}{2.199244in}}%
\pgfpathlineto{\pgfqpoint{3.396733in}{2.198586in}}%
\pgfpathlineto{\pgfqpoint{3.400699in}{2.199966in}}%
\pgfpathlineto{\pgfqpoint{3.401920in}{2.198715in}}%
\pgfpathlineto{\pgfqpoint{3.402530in}{2.197957in}}%
\pgfpathlineto{\pgfqpoint{3.403140in}{2.199028in}}%
\pgfpathlineto{\pgfqpoint{3.404361in}{2.201015in}}%
\pgfpathlineto{\pgfqpoint{3.404971in}{2.200429in}}%
\pgfpathlineto{\pgfqpoint{3.406496in}{2.197596in}}%
\pgfpathlineto{\pgfqpoint{3.407107in}{2.198575in}}%
\pgfpathlineto{\pgfqpoint{3.408632in}{2.199923in}}%
\pgfpathlineto{\pgfqpoint{3.408937in}{2.199652in}}%
\pgfpathlineto{\pgfqpoint{3.410463in}{2.198800in}}%
\pgfpathlineto{\pgfqpoint{3.410768in}{2.199096in}}%
\pgfpathlineto{\pgfqpoint{3.412294in}{2.199811in}}%
\pgfpathlineto{\pgfqpoint{3.412904in}{2.199195in}}%
\pgfpathlineto{\pgfqpoint{3.414124in}{2.199117in}}%
\pgfpathlineto{\pgfqpoint{3.414430in}{2.199570in}}%
\pgfpathlineto{\pgfqpoint{3.415345in}{2.200600in}}%
\pgfpathlineto{\pgfqpoint{3.415955in}{2.199726in}}%
\pgfpathlineto{\pgfqpoint{3.416870in}{2.198115in}}%
\pgfpathlineto{\pgfqpoint{3.417481in}{2.198852in}}%
\pgfpathlineto{\pgfqpoint{3.418701in}{2.200267in}}%
\pgfpathlineto{\pgfqpoint{3.419006in}{2.199811in}}%
\pgfpathlineto{\pgfqpoint{3.420227in}{2.197526in}}%
\pgfpathlineto{\pgfqpoint{3.420837in}{2.198536in}}%
\pgfpathlineto{\pgfqpoint{3.422057in}{2.201052in}}%
\pgfpathlineto{\pgfqpoint{3.422668in}{2.200506in}}%
\pgfpathlineto{\pgfqpoint{3.424193in}{2.197269in}}%
\pgfpathlineto{\pgfqpoint{3.425109in}{2.198650in}}%
\pgfpathlineto{\pgfqpoint{3.427244in}{2.201240in}}%
\pgfpathlineto{\pgfqpoint{3.427550in}{2.200852in}}%
\pgfpathlineto{\pgfqpoint{3.429075in}{2.197270in}}%
\pgfpathlineto{\pgfqpoint{3.429991in}{2.198630in}}%
\pgfpathlineto{\pgfqpoint{3.432126in}{2.199648in}}%
\pgfpathlineto{\pgfqpoint{3.435788in}{2.199700in}}%
\pgfpathlineto{\pgfqpoint{3.437008in}{2.200079in}}%
\pgfpathlineto{\pgfqpoint{3.437313in}{2.199786in}}%
\pgfpathlineto{\pgfqpoint{3.438839in}{2.198092in}}%
\pgfpathlineto{\pgfqpoint{3.439449in}{2.198833in}}%
\pgfpathlineto{\pgfqpoint{3.442195in}{2.201164in}}%
\pgfpathlineto{\pgfqpoint{3.442500in}{2.200752in}}%
\pgfpathlineto{\pgfqpoint{3.444026in}{2.197438in}}%
\pgfpathlineto{\pgfqpoint{3.444941in}{2.198301in}}%
\pgfpathlineto{\pgfqpoint{3.447687in}{2.199865in}}%
\pgfpathlineto{\pgfqpoint{3.450433in}{2.199401in}}%
\pgfpathlineto{\pgfqpoint{3.454095in}{2.199381in}}%
\pgfpathlineto{\pgfqpoint{3.461723in}{2.199702in}}%
\pgfpathlineto{\pgfqpoint{3.462943in}{2.200876in}}%
\pgfpathlineto{\pgfqpoint{3.463554in}{2.199998in}}%
\pgfpathlineto{\pgfqpoint{3.464774in}{2.197406in}}%
\pgfpathlineto{\pgfqpoint{3.465384in}{2.198303in}}%
\pgfpathlineto{\pgfqpoint{3.466605in}{2.200362in}}%
\pgfpathlineto{\pgfqpoint{3.467215in}{2.199592in}}%
\pgfpathlineto{\pgfqpoint{3.470571in}{2.198157in}}%
\pgfpathlineto{\pgfqpoint{3.472707in}{2.199854in}}%
\pgfpathlineto{\pgfqpoint{3.473317in}{2.198613in}}%
\pgfpathlineto{\pgfqpoint{3.474233in}{2.198374in}}%
\pgfpathlineto{\pgfqpoint{3.474538in}{2.198879in}}%
\pgfpathlineto{\pgfqpoint{3.475758in}{2.200501in}}%
\pgfpathlineto{\pgfqpoint{3.476369in}{2.199322in}}%
\pgfpathlineto{\pgfqpoint{3.477284in}{2.198676in}}%
\pgfpathlineto{\pgfqpoint{3.477894in}{2.199602in}}%
\pgfpathlineto{\pgfqpoint{3.478809in}{2.200073in}}%
\pgfpathlineto{\pgfqpoint{3.479115in}{2.199575in}}%
\pgfpathlineto{\pgfqpoint{3.480335in}{2.198272in}}%
\pgfpathlineto{\pgfqpoint{3.480945in}{2.199003in}}%
\pgfpathlineto{\pgfqpoint{3.481861in}{2.199641in}}%
\pgfpathlineto{\pgfqpoint{3.482471in}{2.198989in}}%
\pgfpathlineto{\pgfqpoint{3.483386in}{2.198238in}}%
\pgfpathlineto{\pgfqpoint{3.483691in}{2.198704in}}%
\pgfpathlineto{\pgfqpoint{3.484912in}{2.201282in}}%
\pgfpathlineto{\pgfqpoint{3.485522in}{2.200594in}}%
\pgfpathlineto{\pgfqpoint{3.487048in}{2.198127in}}%
\pgfpathlineto{\pgfqpoint{3.487658in}{2.199114in}}%
\pgfpathlineto{\pgfqpoint{3.488878in}{2.199544in}}%
\pgfpathlineto{\pgfqpoint{3.489183in}{2.199291in}}%
\pgfpathlineto{\pgfqpoint{3.491319in}{2.199363in}}%
\pgfpathlineto{\pgfqpoint{3.493760in}{2.199527in}}%
\pgfpathlineto{\pgfqpoint{3.494981in}{2.197949in}}%
\pgfpathlineto{\pgfqpoint{3.495591in}{2.199123in}}%
\pgfpathlineto{\pgfqpoint{3.496506in}{2.200097in}}%
\pgfpathlineto{\pgfqpoint{3.497117in}{2.199044in}}%
\pgfpathlineto{\pgfqpoint{3.498032in}{2.198000in}}%
\pgfpathlineto{\pgfqpoint{3.498337in}{2.198617in}}%
\pgfpathlineto{\pgfqpoint{3.499557in}{2.201543in}}%
\pgfpathlineto{\pgfqpoint{3.500168in}{2.200575in}}%
\pgfpathlineto{\pgfqpoint{3.501388in}{2.196720in}}%
\pgfpathlineto{\pgfqpoint{3.501998in}{2.198028in}}%
\pgfpathlineto{\pgfqpoint{3.503219in}{2.200299in}}%
\pgfpathlineto{\pgfqpoint{3.504134in}{2.199793in}}%
\pgfpathlineto{\pgfqpoint{3.507491in}{2.198759in}}%
\pgfpathlineto{\pgfqpoint{3.507796in}{2.199395in}}%
\pgfpathlineto{\pgfqpoint{3.509016in}{2.200905in}}%
\pgfpathlineto{\pgfqpoint{3.509321in}{2.200416in}}%
\pgfpathlineto{\pgfqpoint{3.510542in}{2.198402in}}%
\pgfpathlineto{\pgfqpoint{3.511457in}{2.199162in}}%
\pgfpathlineto{\pgfqpoint{3.515119in}{2.199510in}}%
\pgfpathlineto{\pgfqpoint{3.516339in}{2.198282in}}%
\pgfpathlineto{\pgfqpoint{3.516949in}{2.199094in}}%
\pgfpathlineto{\pgfqpoint{3.518475in}{2.200278in}}%
\pgfpathlineto{\pgfqpoint{3.519085in}{2.199682in}}%
\pgfpathlineto{\pgfqpoint{3.521221in}{2.199205in}}%
\pgfpathlineto{\pgfqpoint{3.524577in}{2.199121in}}%
\pgfpathlineto{\pgfqpoint{3.526713in}{2.199995in}}%
\pgfpathlineto{\pgfqpoint{3.527018in}{2.199596in}}%
\pgfpathlineto{\pgfqpoint{3.528239in}{2.198927in}}%
\pgfpathlineto{\pgfqpoint{3.528544in}{2.199481in}}%
\pgfpathlineto{\pgfqpoint{3.529459in}{2.201131in}}%
\pgfpathlineto{\pgfqpoint{3.530069in}{2.199884in}}%
\pgfpathlineto{\pgfqpoint{3.531290in}{2.196306in}}%
\pgfpathlineto{\pgfqpoint{3.531900in}{2.197542in}}%
\pgfpathlineto{\pgfqpoint{3.533120in}{2.201411in}}%
\pgfpathlineto{\pgfqpoint{3.534036in}{2.200225in}}%
\pgfpathlineto{\pgfqpoint{3.535256in}{2.198915in}}%
\pgfpathlineto{\pgfqpoint{3.535867in}{2.199272in}}%
\pgfpathlineto{\pgfqpoint{3.541359in}{2.199063in}}%
\pgfpathlineto{\pgfqpoint{3.542579in}{2.198021in}}%
\pgfpathlineto{\pgfqpoint{3.543189in}{2.198858in}}%
\pgfpathlineto{\pgfqpoint{3.544410in}{2.200203in}}%
\pgfpathlineto{\pgfqpoint{3.545020in}{2.199491in}}%
\pgfpathlineto{\pgfqpoint{3.546546in}{2.198864in}}%
\pgfpathlineto{\pgfqpoint{3.546851in}{2.199149in}}%
\pgfpathlineto{\pgfqpoint{3.550817in}{2.200072in}}%
\pgfpathlineto{\pgfqpoint{3.552343in}{2.198404in}}%
\pgfpathlineto{\pgfqpoint{3.552953in}{2.199217in}}%
\pgfpathlineto{\pgfqpoint{3.554174in}{2.200227in}}%
\pgfpathlineto{\pgfqpoint{3.554784in}{2.199480in}}%
\pgfpathlineto{\pgfqpoint{3.556004in}{2.198105in}}%
\pgfpathlineto{\pgfqpoint{3.556615in}{2.198902in}}%
\pgfpathlineto{\pgfqpoint{3.559361in}{2.199835in}}%
\pgfpathlineto{\pgfqpoint{3.562717in}{2.199616in}}%
\pgfpathlineto{\pgfqpoint{3.565158in}{2.198402in}}%
\pgfpathlineto{\pgfqpoint{3.565768in}{2.199627in}}%
\pgfpathlineto{\pgfqpoint{3.566683in}{2.200121in}}%
\pgfpathlineto{\pgfqpoint{3.567294in}{2.199460in}}%
\pgfpathlineto{\pgfqpoint{3.568819in}{2.200035in}}%
\pgfpathlineto{\pgfqpoint{3.569735in}{2.199613in}}%
\pgfpathlineto{\pgfqpoint{3.570040in}{2.198983in}}%
\pgfpathlineto{\pgfqpoint{3.570955in}{2.198162in}}%
\pgfpathlineto{\pgfqpoint{3.571565in}{2.198977in}}%
\pgfpathlineto{\pgfqpoint{3.572786in}{2.199821in}}%
\pgfpathlineto{\pgfqpoint{3.573396in}{2.199122in}}%
\pgfpathlineto{\pgfqpoint{3.574617in}{2.198743in}}%
\pgfpathlineto{\pgfqpoint{3.575227in}{2.199256in}}%
\pgfpathlineto{\pgfqpoint{3.577057in}{2.200011in}}%
\pgfpathlineto{\pgfqpoint{3.577363in}{2.199637in}}%
\pgfpathlineto{\pgfqpoint{3.578888in}{2.198104in}}%
\pgfpathlineto{\pgfqpoint{3.579498in}{2.198893in}}%
\pgfpathlineto{\pgfqpoint{3.580719in}{2.200732in}}%
\pgfpathlineto{\pgfqpoint{3.581329in}{2.200029in}}%
\pgfpathlineto{\pgfqpoint{3.582550in}{2.198191in}}%
\pgfpathlineto{\pgfqpoint{3.583160in}{2.199035in}}%
\pgfpathlineto{\pgfqpoint{3.584075in}{2.200176in}}%
\pgfpathlineto{\pgfqpoint{3.584685in}{2.199510in}}%
\pgfpathlineto{\pgfqpoint{3.585906in}{2.198798in}}%
\pgfpathlineto{\pgfqpoint{3.586516in}{2.199347in}}%
\pgfpathlineto{\pgfqpoint{3.587737in}{2.199864in}}%
\pgfpathlineto{\pgfqpoint{3.588042in}{2.199357in}}%
\pgfpathlineto{\pgfqpoint{3.588957in}{2.198326in}}%
\pgfpathlineto{\pgfqpoint{3.589872in}{2.199134in}}%
\pgfpathlineto{\pgfqpoint{3.593839in}{2.199815in}}%
\pgfpathlineto{\pgfqpoint{3.596890in}{2.198335in}}%
\pgfpathlineto{\pgfqpoint{3.597195in}{2.198876in}}%
\pgfpathlineto{\pgfqpoint{3.598416in}{2.201358in}}%
\pgfpathlineto{\pgfqpoint{3.599026in}{2.200740in}}%
\pgfpathlineto{\pgfqpoint{3.600552in}{2.196894in}}%
\pgfpathlineto{\pgfqpoint{3.601162in}{2.198643in}}%
\pgfpathlineto{\pgfqpoint{3.602382in}{2.200947in}}%
\pgfpathlineto{\pgfqpoint{3.602993in}{2.199675in}}%
\pgfpathlineto{\pgfqpoint{3.603908in}{2.198075in}}%
\pgfpathlineto{\pgfqpoint{3.604518in}{2.198706in}}%
\pgfpathlineto{\pgfqpoint{3.606044in}{2.200413in}}%
\pgfpathlineto{\pgfqpoint{3.606654in}{2.199617in}}%
\pgfpathlineto{\pgfqpoint{3.608485in}{2.198755in}}%
\pgfpathlineto{\pgfqpoint{3.608790in}{2.199019in}}%
\pgfpathlineto{\pgfqpoint{3.611231in}{2.199538in}}%
\pgfpathlineto{\pgfqpoint{3.615502in}{2.198256in}}%
\pgfpathlineto{\pgfqpoint{3.617333in}{2.200910in}}%
\pgfpathlineto{\pgfqpoint{3.617638in}{2.200463in}}%
\pgfpathlineto{\pgfqpoint{3.618859in}{2.198079in}}%
\pgfpathlineto{\pgfqpoint{3.619774in}{2.198980in}}%
\pgfpathlineto{\pgfqpoint{3.621300in}{2.199132in}}%
\pgfpathlineto{\pgfqpoint{3.621605in}{2.198795in}}%
\pgfpathlineto{\pgfqpoint{3.623130in}{2.199218in}}%
\pgfpathlineto{\pgfqpoint{3.625571in}{2.199627in}}%
\pgfpathlineto{\pgfqpoint{3.629233in}{2.199678in}}%
\pgfpathlineto{\pgfqpoint{3.631369in}{2.198926in}}%
\pgfpathlineto{\pgfqpoint{3.633809in}{2.199533in}}%
\pgfpathlineto{\pgfqpoint{3.635945in}{2.200167in}}%
\pgfpathlineto{\pgfqpoint{3.636250in}{2.199418in}}%
\pgfpathlineto{\pgfqpoint{3.637166in}{2.197037in}}%
\pgfpathlineto{\pgfqpoint{3.638081in}{2.198649in}}%
\pgfpathlineto{\pgfqpoint{3.639302in}{2.200724in}}%
\pgfpathlineto{\pgfqpoint{3.639912in}{2.199768in}}%
\pgfpathlineto{\pgfqpoint{3.641132in}{2.198844in}}%
\pgfpathlineto{\pgfqpoint{3.641743in}{2.199351in}}%
\pgfpathlineto{\pgfqpoint{3.643878in}{2.199232in}}%
\pgfpathlineto{\pgfqpoint{3.645404in}{2.198118in}}%
\pgfpathlineto{\pgfqpoint{3.646014in}{2.199039in}}%
\pgfpathlineto{\pgfqpoint{3.647235in}{2.200670in}}%
\pgfpathlineto{\pgfqpoint{3.647845in}{2.199539in}}%
\pgfpathlineto{\pgfqpoint{3.648760in}{2.198482in}}%
\pgfpathlineto{\pgfqpoint{3.649370in}{2.199302in}}%
\pgfpathlineto{\pgfqpoint{3.650286in}{2.200085in}}%
\pgfpathlineto{\pgfqpoint{3.650896in}{2.199083in}}%
\pgfpathlineto{\pgfqpoint{3.651811in}{2.197737in}}%
\pgfpathlineto{\pgfqpoint{3.652422in}{2.198591in}}%
\pgfpathlineto{\pgfqpoint{3.653947in}{2.200707in}}%
\pgfpathlineto{\pgfqpoint{3.654252in}{2.200197in}}%
\pgfpathlineto{\pgfqpoint{3.655473in}{2.198307in}}%
\pgfpathlineto{\pgfqpoint{3.656083in}{2.198978in}}%
\pgfpathlineto{\pgfqpoint{3.658219in}{2.199547in}}%
\pgfpathlineto{\pgfqpoint{3.665237in}{2.199764in}}%
\pgfpathlineto{\pgfqpoint{3.667067in}{2.199920in}}%
\pgfpathlineto{\pgfqpoint{3.673170in}{2.198809in}}%
\pgfpathlineto{\pgfqpoint{3.676221in}{2.199542in}}%
\pgfpathlineto{\pgfqpoint{3.679272in}{2.199031in}}%
\pgfpathlineto{\pgfqpoint{3.680493in}{2.200354in}}%
\pgfpathlineto{\pgfqpoint{3.681103in}{2.199522in}}%
\pgfpathlineto{\pgfqpoint{3.682018in}{2.197966in}}%
\pgfpathlineto{\pgfqpoint{3.682628in}{2.198823in}}%
\pgfpathlineto{\pgfqpoint{3.683849in}{2.200851in}}%
\pgfpathlineto{\pgfqpoint{3.684459in}{2.199834in}}%
\pgfpathlineto{\pgfqpoint{3.685374in}{2.197690in}}%
\pgfpathlineto{\pgfqpoint{3.686290in}{2.198582in}}%
\pgfpathlineto{\pgfqpoint{3.689036in}{2.200090in}}%
\pgfpathlineto{\pgfqpoint{3.690561in}{2.198659in}}%
\pgfpathlineto{\pgfqpoint{3.691477in}{2.198168in}}%
\pgfpathlineto{\pgfqpoint{3.691782in}{2.198609in}}%
\pgfpathlineto{\pgfqpoint{3.693307in}{2.200461in}}%
\pgfpathlineto{\pgfqpoint{3.693918in}{2.199826in}}%
\pgfpathlineto{\pgfqpoint{3.696054in}{2.197978in}}%
\pgfpathlineto{\pgfqpoint{3.696359in}{2.198395in}}%
\pgfpathlineto{\pgfqpoint{3.697884in}{2.200131in}}%
\pgfpathlineto{\pgfqpoint{3.698494in}{2.199623in}}%
\pgfpathlineto{\pgfqpoint{3.700020in}{2.199120in}}%
\pgfpathlineto{\pgfqpoint{3.700325in}{2.199418in}}%
\pgfpathlineto{\pgfqpoint{3.702461in}{2.199205in}}%
\pgfpathlineto{\pgfqpoint{3.708869in}{2.198175in}}%
\pgfpathlineto{\pgfqpoint{3.709479in}{2.197601in}}%
\pgfpathlineto{\pgfqpoint{3.710089in}{2.198336in}}%
\pgfpathlineto{\pgfqpoint{3.711309in}{2.200703in}}%
\pgfpathlineto{\pgfqpoint{3.712225in}{2.199608in}}%
\pgfpathlineto{\pgfqpoint{3.714666in}{2.198781in}}%
\pgfpathlineto{\pgfqpoint{3.715581in}{2.199535in}}%
\pgfpathlineto{\pgfqpoint{3.715581in}{2.199535in}}%
\pgfusepath{stroke}%
\end{pgfscope}%
\begin{pgfscope}%
\pgfsetrectcap%
\pgfsetmiterjoin%
\pgfsetlinewidth{0.803000pt}%
\definecolor{currentstroke}{rgb}{0.000000,0.000000,0.000000}%
\pgfsetstrokecolor{currentstroke}%
\pgfsetdash{}{0pt}%
\pgfpathmoveto{\pgfqpoint{0.664400in}{1.756587in}}%
\pgfpathlineto{\pgfqpoint{0.664400in}{2.945563in}}%
\pgfusepath{stroke}%
\end{pgfscope}%
\begin{pgfscope}%
\pgfsetrectcap%
\pgfsetmiterjoin%
\pgfsetlinewidth{0.803000pt}%
\definecolor{currentstroke}{rgb}{0.000000,0.000000,0.000000}%
\pgfsetstrokecolor{currentstroke}%
\pgfsetdash{}{0pt}%
\pgfpathmoveto{\pgfqpoint{3.715581in}{1.756587in}}%
\pgfpathlineto{\pgfqpoint{3.715581in}{2.945563in}}%
\pgfusepath{stroke}%
\end{pgfscope}%
\begin{pgfscope}%
\pgfsetrectcap%
\pgfsetmiterjoin%
\pgfsetlinewidth{0.803000pt}%
\definecolor{currentstroke}{rgb}{0.000000,0.000000,0.000000}%
\pgfsetstrokecolor{currentstroke}%
\pgfsetdash{}{0pt}%
\pgfpathmoveto{\pgfqpoint{0.664400in}{1.756587in}}%
\pgfpathlineto{\pgfqpoint{3.715581in}{1.756587in}}%
\pgfusepath{stroke}%
\end{pgfscope}%
\begin{pgfscope}%
\pgfsetrectcap%
\pgfsetmiterjoin%
\pgfsetlinewidth{0.803000pt}%
\definecolor{currentstroke}{rgb}{0.000000,0.000000,0.000000}%
\pgfsetstrokecolor{currentstroke}%
\pgfsetdash{}{0pt}%
\pgfpathmoveto{\pgfqpoint{0.664400in}{2.945563in}}%
\pgfpathlineto{\pgfqpoint{3.715581in}{2.945563in}}%
\pgfusepath{stroke}%
\end{pgfscope}%
\begin{pgfscope}%
\pgfsetbuttcap%
\pgfsetmiterjoin%
\definecolor{currentfill}{rgb}{1.000000,1.000000,1.000000}%
\pgfsetfillcolor{currentfill}%
\pgfsetfillopacity{0.800000}%
\pgfsetlinewidth{1.003750pt}%
\definecolor{currentstroke}{rgb}{0.800000,0.800000,0.800000}%
\pgfsetstrokecolor{currentstroke}%
\pgfsetstrokeopacity{0.800000}%
\pgfsetdash{}{0pt}%
\pgfpathmoveto{\pgfqpoint{0.781067in}{2.577564in}}%
\pgfpathlineto{\pgfqpoint{2.149234in}{2.577564in}}%
\pgfpathquadraticcurveto{\pgfqpoint{2.182567in}{2.577564in}}{\pgfqpoint{2.182567in}{2.610897in}}%
\pgfpathlineto{\pgfqpoint{2.182567in}{2.828897in}}%
\pgfpathquadraticcurveto{\pgfqpoint{2.182567in}{2.862230in}}{\pgfqpoint{2.149234in}{2.862230in}}%
\pgfpathlineto{\pgfqpoint{0.781067in}{2.862230in}}%
\pgfpathquadraticcurveto{\pgfqpoint{0.747733in}{2.862230in}}{\pgfqpoint{0.747733in}{2.828897in}}%
\pgfpathlineto{\pgfqpoint{0.747733in}{2.610897in}}%
\pgfpathquadraticcurveto{\pgfqpoint{0.747733in}{2.577564in}}{\pgfqpoint{0.781067in}{2.577564in}}%
\pgfpathlineto{\pgfqpoint{0.781067in}{2.577564in}}%
\pgfpathclose%
\pgfusepath{stroke,fill}%
\end{pgfscope}%
\begin{pgfscope}%
\pgfsetrectcap%
\pgfsetroundjoin%
\pgfsetlinewidth{2.007500pt}%
\definecolor{currentstroke}{rgb}{0.000000,0.000000,0.000000}%
\pgfsetstrokecolor{currentstroke}%
\pgfsetdash{}{0pt}%
\pgfpathmoveto{\pgfqpoint{0.814400in}{2.734897in}}%
\pgfpathlineto{\pgfqpoint{0.981067in}{2.734897in}}%
\pgfpathlineto{\pgfqpoint{1.147733in}{2.734897in}}%
\pgfusepath{stroke}%
\end{pgfscope}%
\begin{pgfscope}%
\definecolor{textcolor}{rgb}{0.000000,0.000000,0.000000}%
\pgfsetstrokecolor{textcolor}%
\pgfsetfillcolor{textcolor}%
\pgftext[x=1.281067in,y=2.676563in,left,base]{\color{textcolor}\rmfamily\fontsize{12.000000}{14.400000}\selectfont SMA 0.5 ps}%
\end{pgfscope}%
\begin{pgfscope}%
\pgfsetbuttcap%
\pgfsetmiterjoin%
\definecolor{currentfill}{rgb}{1.000000,1.000000,1.000000}%
\pgfsetfillcolor{currentfill}%
\pgfsetlinewidth{0.000000pt}%
\definecolor{currentstroke}{rgb}{0.000000,0.000000,0.000000}%
\pgfsetstrokecolor{currentstroke}%
\pgfsetstrokeopacity{0.000000}%
\pgfsetdash{}{0pt}%
\pgfpathmoveto{\pgfqpoint{0.664400in}{0.567611in}}%
\pgfpathlineto{\pgfqpoint{3.715581in}{0.567611in}}%
\pgfpathlineto{\pgfqpoint{3.715581in}{1.756587in}}%
\pgfpathlineto{\pgfqpoint{0.664400in}{1.756587in}}%
\pgfpathlineto{\pgfqpoint{0.664400in}{0.567611in}}%
\pgfpathclose%
\pgfusepath{fill}%
\end{pgfscope}%
\begin{pgfscope}%
\pgfsetbuttcap%
\pgfsetroundjoin%
\definecolor{currentfill}{rgb}{0.000000,0.000000,0.000000}%
\pgfsetfillcolor{currentfill}%
\pgfsetlinewidth{0.803000pt}%
\definecolor{currentstroke}{rgb}{0.000000,0.000000,0.000000}%
\pgfsetstrokecolor{currentstroke}%
\pgfsetdash{}{0pt}%
\pgfsys@defobject{currentmarker}{\pgfqpoint{0.000000in}{-0.048611in}}{\pgfqpoint{0.000000in}{0.000000in}}{%
\pgfpathmoveto{\pgfqpoint{0.000000in}{0.000000in}}%
\pgfpathlineto{\pgfqpoint{0.000000in}{-0.048611in}}%
\pgfusepath{stroke,fill}%
}%
\begin{pgfscope}%
\pgfsys@transformshift{0.664400in}{0.567611in}%
\pgfsys@useobject{currentmarker}{}%
\end{pgfscope}%
\end{pgfscope}%
\begin{pgfscope}%
\definecolor{textcolor}{rgb}{0.000000,0.000000,0.000000}%
\pgfsetstrokecolor{textcolor}%
\pgfsetfillcolor{textcolor}%
\pgftext[x=0.664400in,y=0.470388in,,top]{\color{textcolor}\rmfamily\fontsize{12.000000}{14.400000}\selectfont \(\displaystyle {0}\)}%
\end{pgfscope}%
\begin{pgfscope}%
\pgfsetbuttcap%
\pgfsetroundjoin%
\definecolor{currentfill}{rgb}{0.000000,0.000000,0.000000}%
\pgfsetfillcolor{currentfill}%
\pgfsetlinewidth{0.803000pt}%
\definecolor{currentstroke}{rgb}{0.000000,0.000000,0.000000}%
\pgfsetstrokecolor{currentstroke}%
\pgfsetdash{}{0pt}%
\pgfsys@defobject{currentmarker}{\pgfqpoint{0.000000in}{-0.048611in}}{\pgfqpoint{0.000000in}{0.000000in}}{%
\pgfpathmoveto{\pgfqpoint{0.000000in}{0.000000in}}%
\pgfpathlineto{\pgfqpoint{0.000000in}{-0.048611in}}%
\pgfusepath{stroke,fill}%
}%
\begin{pgfscope}%
\pgfsys@transformshift{1.274636in}{0.567611in}%
\pgfsys@useobject{currentmarker}{}%
\end{pgfscope}%
\end{pgfscope}%
\begin{pgfscope}%
\definecolor{textcolor}{rgb}{0.000000,0.000000,0.000000}%
\pgfsetstrokecolor{textcolor}%
\pgfsetfillcolor{textcolor}%
\pgftext[x=1.274636in,y=0.470388in,,top]{\color{textcolor}\rmfamily\fontsize{12.000000}{14.400000}\selectfont \(\displaystyle {2}\)}%
\end{pgfscope}%
\begin{pgfscope}%
\pgfsetbuttcap%
\pgfsetroundjoin%
\definecolor{currentfill}{rgb}{0.000000,0.000000,0.000000}%
\pgfsetfillcolor{currentfill}%
\pgfsetlinewidth{0.803000pt}%
\definecolor{currentstroke}{rgb}{0.000000,0.000000,0.000000}%
\pgfsetstrokecolor{currentstroke}%
\pgfsetdash{}{0pt}%
\pgfsys@defobject{currentmarker}{\pgfqpoint{0.000000in}{-0.048611in}}{\pgfqpoint{0.000000in}{0.000000in}}{%
\pgfpathmoveto{\pgfqpoint{0.000000in}{0.000000in}}%
\pgfpathlineto{\pgfqpoint{0.000000in}{-0.048611in}}%
\pgfusepath{stroke,fill}%
}%
\begin{pgfscope}%
\pgfsys@transformshift{1.884872in}{0.567611in}%
\pgfsys@useobject{currentmarker}{}%
\end{pgfscope}%
\end{pgfscope}%
\begin{pgfscope}%
\definecolor{textcolor}{rgb}{0.000000,0.000000,0.000000}%
\pgfsetstrokecolor{textcolor}%
\pgfsetfillcolor{textcolor}%
\pgftext[x=1.884872in,y=0.470388in,,top]{\color{textcolor}\rmfamily\fontsize{12.000000}{14.400000}\selectfont \(\displaystyle {4}\)}%
\end{pgfscope}%
\begin{pgfscope}%
\pgfsetbuttcap%
\pgfsetroundjoin%
\definecolor{currentfill}{rgb}{0.000000,0.000000,0.000000}%
\pgfsetfillcolor{currentfill}%
\pgfsetlinewidth{0.803000pt}%
\definecolor{currentstroke}{rgb}{0.000000,0.000000,0.000000}%
\pgfsetstrokecolor{currentstroke}%
\pgfsetdash{}{0pt}%
\pgfsys@defobject{currentmarker}{\pgfqpoint{0.000000in}{-0.048611in}}{\pgfqpoint{0.000000in}{0.000000in}}{%
\pgfpathmoveto{\pgfqpoint{0.000000in}{0.000000in}}%
\pgfpathlineto{\pgfqpoint{0.000000in}{-0.048611in}}%
\pgfusepath{stroke,fill}%
}%
\begin{pgfscope}%
\pgfsys@transformshift{2.495109in}{0.567611in}%
\pgfsys@useobject{currentmarker}{}%
\end{pgfscope}%
\end{pgfscope}%
\begin{pgfscope}%
\definecolor{textcolor}{rgb}{0.000000,0.000000,0.000000}%
\pgfsetstrokecolor{textcolor}%
\pgfsetfillcolor{textcolor}%
\pgftext[x=2.495109in,y=0.470388in,,top]{\color{textcolor}\rmfamily\fontsize{12.000000}{14.400000}\selectfont \(\displaystyle {6}\)}%
\end{pgfscope}%
\begin{pgfscope}%
\pgfsetbuttcap%
\pgfsetroundjoin%
\definecolor{currentfill}{rgb}{0.000000,0.000000,0.000000}%
\pgfsetfillcolor{currentfill}%
\pgfsetlinewidth{0.803000pt}%
\definecolor{currentstroke}{rgb}{0.000000,0.000000,0.000000}%
\pgfsetstrokecolor{currentstroke}%
\pgfsetdash{}{0pt}%
\pgfsys@defobject{currentmarker}{\pgfqpoint{0.000000in}{-0.048611in}}{\pgfqpoint{0.000000in}{0.000000in}}{%
\pgfpathmoveto{\pgfqpoint{0.000000in}{0.000000in}}%
\pgfpathlineto{\pgfqpoint{0.000000in}{-0.048611in}}%
\pgfusepath{stroke,fill}%
}%
\begin{pgfscope}%
\pgfsys@transformshift{3.105345in}{0.567611in}%
\pgfsys@useobject{currentmarker}{}%
\end{pgfscope}%
\end{pgfscope}%
\begin{pgfscope}%
\definecolor{textcolor}{rgb}{0.000000,0.000000,0.000000}%
\pgfsetstrokecolor{textcolor}%
\pgfsetfillcolor{textcolor}%
\pgftext[x=3.105345in,y=0.470388in,,top]{\color{textcolor}\rmfamily\fontsize{12.000000}{14.400000}\selectfont \(\displaystyle {8}\)}%
\end{pgfscope}%
\begin{pgfscope}%
\pgfsetbuttcap%
\pgfsetroundjoin%
\definecolor{currentfill}{rgb}{0.000000,0.000000,0.000000}%
\pgfsetfillcolor{currentfill}%
\pgfsetlinewidth{0.803000pt}%
\definecolor{currentstroke}{rgb}{0.000000,0.000000,0.000000}%
\pgfsetstrokecolor{currentstroke}%
\pgfsetdash{}{0pt}%
\pgfsys@defobject{currentmarker}{\pgfqpoint{0.000000in}{-0.048611in}}{\pgfqpoint{0.000000in}{0.000000in}}{%
\pgfpathmoveto{\pgfqpoint{0.000000in}{0.000000in}}%
\pgfpathlineto{\pgfqpoint{0.000000in}{-0.048611in}}%
\pgfusepath{stroke,fill}%
}%
\begin{pgfscope}%
\pgfsys@transformshift{3.715581in}{0.567611in}%
\pgfsys@useobject{currentmarker}{}%
\end{pgfscope}%
\end{pgfscope}%
\begin{pgfscope}%
\definecolor{textcolor}{rgb}{0.000000,0.000000,0.000000}%
\pgfsetstrokecolor{textcolor}%
\pgfsetfillcolor{textcolor}%
\pgftext[x=3.715581in,y=0.470388in,,top]{\color{textcolor}\rmfamily\fontsize{12.000000}{14.400000}\selectfont \(\displaystyle {10}\)}%
\end{pgfscope}%
\begin{pgfscope}%
\definecolor{textcolor}{rgb}{0.000000,0.000000,0.000000}%
\pgfsetstrokecolor{textcolor}%
\pgfsetfillcolor{textcolor}%
\pgftext[x=2.189991in,y=0.266833in,,top]{\color{textcolor}\rmfamily\fontsize{12.000000}{14.400000}\selectfont Time (ps)}%
\end{pgfscope}%
\begin{pgfscope}%
\pgfsetbuttcap%
\pgfsetroundjoin%
\definecolor{currentfill}{rgb}{0.000000,0.000000,0.000000}%
\pgfsetfillcolor{currentfill}%
\pgfsetlinewidth{0.803000pt}%
\definecolor{currentstroke}{rgb}{0.000000,0.000000,0.000000}%
\pgfsetstrokecolor{currentstroke}%
\pgfsetdash{}{0pt}%
\pgfsys@defobject{currentmarker}{\pgfqpoint{-0.048611in}{0.000000in}}{\pgfqpoint{-0.000000in}{0.000000in}}{%
\pgfpathmoveto{\pgfqpoint{-0.000000in}{0.000000in}}%
\pgfpathlineto{\pgfqpoint{-0.048611in}{0.000000in}}%
\pgfusepath{stroke,fill}%
}%
\begin{pgfscope}%
\pgfsys@transformshift{0.664400in}{0.790240in}%
\pgfsys@useobject{currentmarker}{}%
\end{pgfscope}%
\end{pgfscope}%
\begin{pgfscope}%
\definecolor{textcolor}{rgb}{0.000000,0.000000,0.000000}%
\pgfsetstrokecolor{textcolor}%
\pgfsetfillcolor{textcolor}%
\pgftext[x=0.358654in, y=0.732406in, left, base]{\color{textcolor}\rmfamily\fontsize{12.000000}{14.400000}\selectfont \(\displaystyle {0.5}\)}%
\end{pgfscope}%
\begin{pgfscope}%
\pgfsetbuttcap%
\pgfsetroundjoin%
\definecolor{currentfill}{rgb}{0.000000,0.000000,0.000000}%
\pgfsetfillcolor{currentfill}%
\pgfsetlinewidth{0.803000pt}%
\definecolor{currentstroke}{rgb}{0.000000,0.000000,0.000000}%
\pgfsetstrokecolor{currentstroke}%
\pgfsetdash{}{0pt}%
\pgfsys@defobject{currentmarker}{\pgfqpoint{-0.048611in}{0.000000in}}{\pgfqpoint{-0.000000in}{0.000000in}}{%
\pgfpathmoveto{\pgfqpoint{-0.000000in}{0.000000in}}%
\pgfpathlineto{\pgfqpoint{-0.048611in}{0.000000in}}%
\pgfusepath{stroke,fill}%
}%
\begin{pgfscope}%
\pgfsys@transformshift{0.664400in}{1.182723in}%
\pgfsys@useobject{currentmarker}{}%
\end{pgfscope}%
\end{pgfscope}%
\begin{pgfscope}%
\definecolor{textcolor}{rgb}{0.000000,0.000000,0.000000}%
\pgfsetstrokecolor{textcolor}%
\pgfsetfillcolor{textcolor}%
\pgftext[x=0.358654in, y=1.124889in, left, base]{\color{textcolor}\rmfamily\fontsize{12.000000}{14.400000}\selectfont \(\displaystyle {1.0}\)}%
\end{pgfscope}%
\begin{pgfscope}%
\pgfsetbuttcap%
\pgfsetroundjoin%
\definecolor{currentfill}{rgb}{0.000000,0.000000,0.000000}%
\pgfsetfillcolor{currentfill}%
\pgfsetlinewidth{0.803000pt}%
\definecolor{currentstroke}{rgb}{0.000000,0.000000,0.000000}%
\pgfsetstrokecolor{currentstroke}%
\pgfsetdash{}{0pt}%
\pgfsys@defobject{currentmarker}{\pgfqpoint{-0.048611in}{0.000000in}}{\pgfqpoint{-0.000000in}{0.000000in}}{%
\pgfpathmoveto{\pgfqpoint{-0.000000in}{0.000000in}}%
\pgfpathlineto{\pgfqpoint{-0.048611in}{0.000000in}}%
\pgfusepath{stroke,fill}%
}%
\begin{pgfscope}%
\pgfsys@transformshift{0.664400in}{1.575206in}%
\pgfsys@useobject{currentmarker}{}%
\end{pgfscope}%
\end{pgfscope}%
\begin{pgfscope}%
\definecolor{textcolor}{rgb}{0.000000,0.000000,0.000000}%
\pgfsetstrokecolor{textcolor}%
\pgfsetfillcolor{textcolor}%
\pgftext[x=0.358654in, y=1.517372in, left, base]{\color{textcolor}\rmfamily\fontsize{12.000000}{14.400000}\selectfont \(\displaystyle {1.5}\)}%
\end{pgfscope}%
\begin{pgfscope}%
\definecolor{textcolor}{rgb}{0.000000,0.000000,0.000000}%
\pgfsetstrokecolor{textcolor}%
\pgfsetfillcolor{textcolor}%
\pgftext[x=0.303098in,y=1.162099in,,bottom,rotate=90.000000]{\color{textcolor}\rmfamily\fontsize{12.000000}{14.400000}\selectfont \(\displaystyle \Delta\)Free E (eV)}%
\end{pgfscope}%
\begin{pgfscope}%
\pgfpathrectangle{\pgfqpoint{0.664400in}{0.567611in}}{\pgfqpoint{3.051181in}{1.188976in}}%
\pgfusepath{clip}%
\pgfsetrectcap%
\pgfsetroundjoin%
\pgfsetlinewidth{1.505625pt}%
\definecolor{currentstroke}{rgb}{0.768627,0.768627,0.768627}%
\pgfsetstrokecolor{currentstroke}%
\pgfsetdash{}{0pt}%
\pgfpathmoveto{\pgfqpoint{0.665609in}{0.553722in}}%
\pgfpathlineto{\pgfqpoint{0.669587in}{1.272591in}}%
\pgfpathlineto{\pgfqpoint{0.669892in}{1.269865in}}%
\pgfpathlineto{\pgfqpoint{0.670197in}{1.272684in}}%
\pgfpathlineto{\pgfqpoint{0.671113in}{1.384465in}}%
\pgfpathlineto{\pgfqpoint{0.672028in}{1.528472in}}%
\pgfpathlineto{\pgfqpoint{0.672638in}{1.486829in}}%
\pgfpathlineto{\pgfqpoint{0.675689in}{1.151801in}}%
\pgfpathlineto{\pgfqpoint{0.676910in}{1.065799in}}%
\pgfpathlineto{\pgfqpoint{0.677520in}{1.098687in}}%
\pgfpathlineto{\pgfqpoint{0.681181in}{1.491899in}}%
\pgfpathlineto{\pgfqpoint{0.681487in}{1.461683in}}%
\pgfpathlineto{\pgfqpoint{0.685758in}{0.818226in}}%
\pgfpathlineto{\pgfqpoint{0.686063in}{0.812094in}}%
\pgfpathlineto{\pgfqpoint{0.686674in}{0.828566in}}%
\pgfpathlineto{\pgfqpoint{0.686979in}{0.831330in}}%
\pgfpathlineto{\pgfqpoint{0.687284in}{0.820141in}}%
\pgfpathlineto{\pgfqpoint{0.688809in}{0.695023in}}%
\pgfpathlineto{\pgfqpoint{0.689115in}{0.721368in}}%
\pgfpathlineto{\pgfqpoint{0.692776in}{1.105762in}}%
\pgfpathlineto{\pgfqpoint{0.693386in}{1.151954in}}%
\pgfpathlineto{\pgfqpoint{0.693996in}{1.120783in}}%
\pgfpathlineto{\pgfqpoint{0.695217in}{0.989319in}}%
\pgfpathlineto{\pgfqpoint{0.695827in}{1.043059in}}%
\pgfpathlineto{\pgfqpoint{0.696743in}{1.109727in}}%
\pgfpathlineto{\pgfqpoint{0.697353in}{1.069016in}}%
\pgfpathlineto{\pgfqpoint{0.697963in}{1.027558in}}%
\pgfpathlineto{\pgfqpoint{0.698573in}{1.074662in}}%
\pgfpathlineto{\pgfqpoint{0.702235in}{1.487866in}}%
\pgfpathlineto{\pgfqpoint{0.703150in}{1.601565in}}%
\pgfpathlineto{\pgfqpoint{0.703760in}{1.565287in}}%
\pgfpathlineto{\pgfqpoint{0.708032in}{0.853647in}}%
\pgfpathlineto{\pgfqpoint{0.708642in}{0.881112in}}%
\pgfpathlineto{\pgfqpoint{0.708947in}{0.893670in}}%
\pgfpathlineto{\pgfqpoint{0.709557in}{0.877917in}}%
\pgfpathlineto{\pgfqpoint{0.710778in}{0.783218in}}%
\pgfpathlineto{\pgfqpoint{0.711693in}{0.822947in}}%
\pgfpathlineto{\pgfqpoint{0.712304in}{0.840971in}}%
\pgfpathlineto{\pgfqpoint{0.712914in}{0.825639in}}%
\pgfpathlineto{\pgfqpoint{0.713524in}{0.811100in}}%
\pgfpathlineto{\pgfqpoint{0.713829in}{0.821431in}}%
\pgfpathlineto{\pgfqpoint{0.718406in}{1.271363in}}%
\pgfpathlineto{\pgfqpoint{0.719321in}{1.200281in}}%
\pgfpathlineto{\pgfqpoint{0.720237in}{1.129211in}}%
\pgfpathlineto{\pgfqpoint{0.721152in}{1.150216in}}%
\pgfpathlineto{\pgfqpoint{0.723288in}{1.240360in}}%
\pgfpathlineto{\pgfqpoint{0.724203in}{1.321616in}}%
\pgfpathlineto{\pgfqpoint{0.724813in}{1.263227in}}%
\pgfpathlineto{\pgfqpoint{0.726339in}{1.106447in}}%
\pgfpathlineto{\pgfqpoint{0.726949in}{1.112936in}}%
\pgfpathlineto{\pgfqpoint{0.728170in}{1.112617in}}%
\pgfpathlineto{\pgfqpoint{0.728780in}{1.153097in}}%
\pgfpathlineto{\pgfqpoint{0.730306in}{1.304958in}}%
\pgfpathlineto{\pgfqpoint{0.730916in}{1.254081in}}%
\pgfpathlineto{\pgfqpoint{0.735798in}{0.780126in}}%
\pgfpathlineto{\pgfqpoint{0.737628in}{0.708644in}}%
\pgfpathlineto{\pgfqpoint{0.738239in}{0.752699in}}%
\pgfpathlineto{\pgfqpoint{0.739154in}{0.846840in}}%
\pgfpathlineto{\pgfqpoint{0.740374in}{0.834407in}}%
\pgfpathlineto{\pgfqpoint{0.742815in}{1.229937in}}%
\pgfpathlineto{\pgfqpoint{0.744646in}{1.671544in}}%
\pgfpathlineto{\pgfqpoint{0.745561in}{1.546811in}}%
\pgfpathlineto{\pgfqpoint{0.747392in}{1.256562in}}%
\pgfpathlineto{\pgfqpoint{0.748002in}{1.266062in}}%
\pgfpathlineto{\pgfqpoint{0.748307in}{1.267405in}}%
\pgfpathlineto{\pgfqpoint{0.749223in}{1.185640in}}%
\pgfpathlineto{\pgfqpoint{0.751054in}{0.959397in}}%
\pgfpathlineto{\pgfqpoint{0.751664in}{0.976109in}}%
\pgfpathlineto{\pgfqpoint{0.753189in}{1.103469in}}%
\pgfpathlineto{\pgfqpoint{0.755020in}{1.464160in}}%
\pgfpathlineto{\pgfqpoint{0.755630in}{1.384902in}}%
\pgfpathlineto{\pgfqpoint{0.760207in}{0.719675in}}%
\pgfpathlineto{\pgfqpoint{0.761733in}{0.619559in}}%
\pgfpathlineto{\pgfqpoint{0.762343in}{0.640615in}}%
\pgfpathlineto{\pgfqpoint{0.764784in}{0.975361in}}%
\pgfpathlineto{\pgfqpoint{0.767835in}{1.656716in}}%
\pgfpathlineto{\pgfqpoint{0.768750in}{1.561355in}}%
\pgfpathlineto{\pgfqpoint{0.772107in}{1.104957in}}%
\pgfpathlineto{\pgfqpoint{0.773022in}{1.162816in}}%
\pgfpathlineto{\pgfqpoint{0.773632in}{1.184438in}}%
\pgfpathlineto{\pgfqpoint{0.774548in}{1.175340in}}%
\pgfpathlineto{\pgfqpoint{0.774853in}{1.176411in}}%
\pgfpathlineto{\pgfqpoint{0.775768in}{1.218333in}}%
\pgfpathlineto{\pgfqpoint{0.777904in}{1.370793in}}%
\pgfpathlineto{\pgfqpoint{0.778514in}{1.349580in}}%
\pgfpathlineto{\pgfqpoint{0.780040in}{1.200202in}}%
\pgfpathlineto{\pgfqpoint{0.784311in}{0.572291in}}%
\pgfpathlineto{\pgfqpoint{0.784617in}{0.575571in}}%
\pgfpathlineto{\pgfqpoint{0.785837in}{0.761329in}}%
\pgfpathlineto{\pgfqpoint{0.789498in}{1.292726in}}%
\pgfpathlineto{\pgfqpoint{0.790719in}{1.261038in}}%
\pgfpathlineto{\pgfqpoint{0.791329in}{1.282796in}}%
\pgfpathlineto{\pgfqpoint{0.792244in}{1.325598in}}%
\pgfpathlineto{\pgfqpoint{0.792855in}{1.313601in}}%
\pgfpathlineto{\pgfqpoint{0.795906in}{1.080437in}}%
\pgfpathlineto{\pgfqpoint{0.796516in}{1.113267in}}%
\pgfpathlineto{\pgfqpoint{0.799872in}{1.535276in}}%
\pgfpathlineto{\pgfqpoint{0.800483in}{1.505490in}}%
\pgfpathlineto{\pgfqpoint{0.807806in}{0.699606in}}%
\pgfpathlineto{\pgfqpoint{0.808111in}{0.699669in}}%
\pgfpathlineto{\pgfqpoint{0.808721in}{0.706353in}}%
\pgfpathlineto{\pgfqpoint{0.809941in}{0.807129in}}%
\pgfpathlineto{\pgfqpoint{0.814213in}{1.178121in}}%
\pgfpathlineto{\pgfqpoint{0.814823in}{1.144952in}}%
\pgfpathlineto{\pgfqpoint{0.816044in}{1.030745in}}%
\pgfpathlineto{\pgfqpoint{0.816349in}{1.057868in}}%
\pgfpathlineto{\pgfqpoint{0.820010in}{1.436964in}}%
\pgfpathlineto{\pgfqpoint{0.820926in}{1.371082in}}%
\pgfpathlineto{\pgfqpoint{0.821841in}{1.296981in}}%
\pgfpathlineto{\pgfqpoint{0.822451in}{1.327025in}}%
\pgfpathlineto{\pgfqpoint{0.823367in}{1.450306in}}%
\pgfpathlineto{\pgfqpoint{0.823977in}{1.403924in}}%
\pgfpathlineto{\pgfqpoint{0.827028in}{0.924165in}}%
\pgfpathlineto{\pgfqpoint{0.827638in}{0.946369in}}%
\pgfpathlineto{\pgfqpoint{0.829469in}{1.128878in}}%
\pgfpathlineto{\pgfqpoint{0.830079in}{1.076548in}}%
\pgfpathlineto{\pgfqpoint{0.832215in}{0.885733in}}%
\pgfpathlineto{\pgfqpoint{0.832825in}{0.894453in}}%
\pgfpathlineto{\pgfqpoint{0.833130in}{0.895419in}}%
\pgfpathlineto{\pgfqpoint{0.834046in}{0.851867in}}%
\pgfpathlineto{\pgfqpoint{0.834961in}{0.804881in}}%
\pgfpathlineto{\pgfqpoint{0.835571in}{0.825100in}}%
\pgfpathlineto{\pgfqpoint{0.838622in}{1.034232in}}%
\pgfpathlineto{\pgfqpoint{0.841063in}{1.170130in}}%
\pgfpathlineto{\pgfqpoint{0.844420in}{1.423237in}}%
\pgfpathlineto{\pgfqpoint{0.845640in}{1.256396in}}%
\pgfpathlineto{\pgfqpoint{0.847166in}{1.162774in}}%
\pgfpathlineto{\pgfqpoint{0.847776in}{1.168288in}}%
\pgfpathlineto{\pgfqpoint{0.848691in}{1.215236in}}%
\pgfpathlineto{\pgfqpoint{0.850522in}{1.443559in}}%
\pgfpathlineto{\pgfqpoint{0.851132in}{1.389906in}}%
\pgfpathlineto{\pgfqpoint{0.856930in}{0.633759in}}%
\pgfpathlineto{\pgfqpoint{0.857235in}{0.635340in}}%
\pgfpathlineto{\pgfqpoint{0.858150in}{0.670949in}}%
\pgfpathlineto{\pgfqpoint{0.863032in}{1.180304in}}%
\pgfpathlineto{\pgfqpoint{0.864252in}{1.442640in}}%
\pgfpathlineto{\pgfqpoint{0.864863in}{1.405423in}}%
\pgfpathlineto{\pgfqpoint{0.868829in}{1.013940in}}%
\pgfpathlineto{\pgfqpoint{0.873101in}{1.330489in}}%
\pgfpathlineto{\pgfqpoint{0.874626in}{1.500390in}}%
\pgfpathlineto{\pgfqpoint{0.875237in}{1.471333in}}%
\pgfpathlineto{\pgfqpoint{0.881949in}{0.652199in}}%
\pgfpathlineto{\pgfqpoint{0.882865in}{0.731743in}}%
\pgfpathlineto{\pgfqpoint{0.887746in}{1.247573in}}%
\pgfpathlineto{\pgfqpoint{0.888662in}{1.304198in}}%
\pgfpathlineto{\pgfqpoint{0.889272in}{1.275148in}}%
\pgfpathlineto{\pgfqpoint{0.891408in}{1.063289in}}%
\pgfpathlineto{\pgfqpoint{0.892323in}{1.069141in}}%
\pgfpathlineto{\pgfqpoint{0.893239in}{1.048732in}}%
\pgfpathlineto{\pgfqpoint{0.894154in}{1.022083in}}%
\pgfpathlineto{\pgfqpoint{0.894764in}{1.047702in}}%
\pgfpathlineto{\pgfqpoint{0.897510in}{1.441023in}}%
\pgfpathlineto{\pgfqpoint{0.898731in}{1.605262in}}%
\pgfpathlineto{\pgfqpoint{0.899341in}{1.576007in}}%
\pgfpathlineto{\pgfqpoint{0.903918in}{0.695395in}}%
\pgfpathlineto{\pgfqpoint{0.905138in}{0.799788in}}%
\pgfpathlineto{\pgfqpoint{0.909105in}{1.157872in}}%
\pgfpathlineto{\pgfqpoint{0.909715in}{1.147400in}}%
\pgfpathlineto{\pgfqpoint{0.913376in}{1.056320in}}%
\pgfpathlineto{\pgfqpoint{0.916428in}{1.045548in}}%
\pgfpathlineto{\pgfqpoint{0.917038in}{1.064216in}}%
\pgfpathlineto{\pgfqpoint{0.919174in}{1.226893in}}%
\pgfpathlineto{\pgfqpoint{0.920394in}{1.205869in}}%
\pgfpathlineto{\pgfqpoint{0.921615in}{1.238987in}}%
\pgfpathlineto{\pgfqpoint{0.922225in}{1.250802in}}%
\pgfpathlineto{\pgfqpoint{0.922835in}{1.240244in}}%
\pgfpathlineto{\pgfqpoint{0.923750in}{1.204783in}}%
\pgfpathlineto{\pgfqpoint{0.924666in}{1.226923in}}%
\pgfpathlineto{\pgfqpoint{0.925276in}{1.240557in}}%
\pgfpathlineto{\pgfqpoint{0.925886in}{1.221264in}}%
\pgfpathlineto{\pgfqpoint{0.930158in}{0.832491in}}%
\pgfpathlineto{\pgfqpoint{0.931378in}{0.763405in}}%
\pgfpathlineto{\pgfqpoint{0.931989in}{0.781836in}}%
\pgfpathlineto{\pgfqpoint{0.939617in}{1.399910in}}%
\pgfpathlineto{\pgfqpoint{0.940227in}{1.362379in}}%
\pgfpathlineto{\pgfqpoint{0.945109in}{0.873149in}}%
\pgfpathlineto{\pgfqpoint{0.947244in}{1.015928in}}%
\pgfpathlineto{\pgfqpoint{0.950906in}{1.546637in}}%
\pgfpathlineto{\pgfqpoint{0.951516in}{1.520112in}}%
\pgfpathlineto{\pgfqpoint{0.953042in}{1.112544in}}%
\pgfpathlineto{\pgfqpoint{0.955788in}{0.663484in}}%
\pgfpathlineto{\pgfqpoint{0.956093in}{0.647876in}}%
\pgfpathlineto{\pgfqpoint{0.956703in}{0.673409in}}%
\pgfpathlineto{\pgfqpoint{0.961280in}{1.279257in}}%
\pgfpathlineto{\pgfqpoint{0.961890in}{1.251089in}}%
\pgfpathlineto{\pgfqpoint{0.962195in}{1.243849in}}%
\pgfpathlineto{\pgfqpoint{0.962806in}{1.263503in}}%
\pgfpathlineto{\pgfqpoint{0.963416in}{1.285633in}}%
\pgfpathlineto{\pgfqpoint{0.964026in}{1.252725in}}%
\pgfpathlineto{\pgfqpoint{0.968298in}{0.860011in}}%
\pgfpathlineto{\pgfqpoint{0.968603in}{0.855304in}}%
\pgfpathlineto{\pgfqpoint{0.968908in}{0.870374in}}%
\pgfpathlineto{\pgfqpoint{0.971654in}{1.274148in}}%
\pgfpathlineto{\pgfqpoint{0.973180in}{1.565742in}}%
\pgfpathlineto{\pgfqpoint{0.973790in}{1.497678in}}%
\pgfpathlineto{\pgfqpoint{0.978672in}{0.706321in}}%
\pgfpathlineto{\pgfqpoint{0.978977in}{0.716166in}}%
\pgfpathlineto{\pgfqpoint{0.982638in}{1.142831in}}%
\pgfpathlineto{\pgfqpoint{0.985384in}{1.337954in}}%
\pgfpathlineto{\pgfqpoint{0.985689in}{1.324728in}}%
\pgfpathlineto{\pgfqpoint{0.987215in}{1.027944in}}%
\pgfpathlineto{\pgfqpoint{0.988741in}{0.816119in}}%
\pgfpathlineto{\pgfqpoint{0.989351in}{0.847398in}}%
\pgfpathlineto{\pgfqpoint{0.996369in}{1.530997in}}%
\pgfpathlineto{\pgfqpoint{0.996979in}{1.497577in}}%
\pgfpathlineto{\pgfqpoint{0.998809in}{1.233427in}}%
\pgfpathlineto{\pgfqpoint{1.001556in}{0.848432in}}%
\pgfpathlineto{\pgfqpoint{1.002776in}{0.792195in}}%
\pgfpathlineto{\pgfqpoint{1.003081in}{0.807708in}}%
\pgfpathlineto{\pgfqpoint{1.006437in}{1.216293in}}%
\pgfpathlineto{\pgfqpoint{1.007353in}{1.371297in}}%
\pgfpathlineto{\pgfqpoint{1.007963in}{1.318366in}}%
\pgfpathlineto{\pgfqpoint{1.010709in}{0.988017in}}%
\pgfpathlineto{\pgfqpoint{1.012845in}{0.917991in}}%
\pgfpathlineto{\pgfqpoint{1.013455in}{0.930601in}}%
\pgfpathlineto{\pgfqpoint{1.014065in}{0.941802in}}%
\pgfpathlineto{\pgfqpoint{1.014676in}{0.931549in}}%
\pgfpathlineto{\pgfqpoint{1.014981in}{0.927976in}}%
\pgfpathlineto{\pgfqpoint{1.015286in}{0.939803in}}%
\pgfpathlineto{\pgfqpoint{1.018947in}{1.445362in}}%
\pgfpathlineto{\pgfqpoint{1.019863in}{1.351930in}}%
\pgfpathlineto{\pgfqpoint{1.021998in}{1.197263in}}%
\pgfpathlineto{\pgfqpoint{1.023219in}{0.996391in}}%
\pgfpathlineto{\pgfqpoint{1.025355in}{0.858288in}}%
\pgfpathlineto{\pgfqpoint{1.025660in}{0.847111in}}%
\pgfpathlineto{\pgfqpoint{1.026270in}{0.861906in}}%
\pgfpathlineto{\pgfqpoint{1.027796in}{1.059759in}}%
\pgfpathlineto{\pgfqpoint{1.028711in}{1.135653in}}%
\pgfpathlineto{\pgfqpoint{1.029321in}{1.081267in}}%
\pgfpathlineto{\pgfqpoint{1.029626in}{1.060177in}}%
\pgfpathlineto{\pgfqpoint{1.030237in}{1.094733in}}%
\pgfpathlineto{\pgfqpoint{1.031152in}{1.183118in}}%
\pgfpathlineto{\pgfqpoint{1.031762in}{1.095310in}}%
\pgfpathlineto{\pgfqpoint{1.032372in}{1.030096in}}%
\pgfpathlineto{\pgfqpoint{1.033288in}{1.091306in}}%
\pgfpathlineto{\pgfqpoint{1.033898in}{1.037660in}}%
\pgfpathlineto{\pgfqpoint{1.035424in}{0.823632in}}%
\pgfpathlineto{\pgfqpoint{1.036034in}{0.854710in}}%
\pgfpathlineto{\pgfqpoint{1.039390in}{1.176846in}}%
\pgfpathlineto{\pgfqpoint{1.040611in}{1.402896in}}%
\pgfpathlineto{\pgfqpoint{1.041526in}{1.373169in}}%
\pgfpathlineto{\pgfqpoint{1.042746in}{1.545071in}}%
\pgfpathlineto{\pgfqpoint{1.043357in}{1.466795in}}%
\pgfpathlineto{\pgfqpoint{1.046408in}{0.989587in}}%
\pgfpathlineto{\pgfqpoint{1.047018in}{0.909567in}}%
\pgfpathlineto{\pgfqpoint{1.047933in}{0.989206in}}%
\pgfpathlineto{\pgfqpoint{1.048849in}{1.108269in}}%
\pgfpathlineto{\pgfqpoint{1.049459in}{1.074386in}}%
\pgfpathlineto{\pgfqpoint{1.050680in}{0.912755in}}%
\pgfpathlineto{\pgfqpoint{1.051595in}{0.959547in}}%
\pgfpathlineto{\pgfqpoint{1.051900in}{0.962121in}}%
\pgfpathlineto{\pgfqpoint{1.053120in}{0.870474in}}%
\pgfpathlineto{\pgfqpoint{1.053731in}{0.946673in}}%
\pgfpathlineto{\pgfqpoint{1.054646in}{1.181050in}}%
\pgfpathlineto{\pgfqpoint{1.055256in}{1.116154in}}%
\pgfpathlineto{\pgfqpoint{1.056172in}{0.926135in}}%
\pgfpathlineto{\pgfqpoint{1.057087in}{1.039157in}}%
\pgfpathlineto{\pgfqpoint{1.058002in}{1.120982in}}%
\pgfpathlineto{\pgfqpoint{1.058613in}{1.043076in}}%
\pgfpathlineto{\pgfqpoint{1.059223in}{0.974332in}}%
\pgfpathlineto{\pgfqpoint{1.060138in}{1.027328in}}%
\pgfpathlineto{\pgfqpoint{1.060443in}{1.025644in}}%
\pgfpathlineto{\pgfqpoint{1.061969in}{0.883072in}}%
\pgfpathlineto{\pgfqpoint{1.062579in}{0.965752in}}%
\pgfpathlineto{\pgfqpoint{1.063800in}{1.171384in}}%
\pgfpathlineto{\pgfqpoint{1.064410in}{1.130810in}}%
\pgfpathlineto{\pgfqpoint{1.065325in}{1.071216in}}%
\pgfpathlineto{\pgfqpoint{1.065935in}{1.114905in}}%
\pgfpathlineto{\pgfqpoint{1.066546in}{1.133118in}}%
\pgfpathlineto{\pgfqpoint{1.067156in}{1.102173in}}%
\pgfpathlineto{\pgfqpoint{1.067461in}{1.094639in}}%
\pgfpathlineto{\pgfqpoint{1.067766in}{1.110197in}}%
\pgfpathlineto{\pgfqpoint{1.069902in}{1.515350in}}%
\pgfpathlineto{\pgfqpoint{1.071122in}{1.416904in}}%
\pgfpathlineto{\pgfqpoint{1.072038in}{1.418926in}}%
\pgfpathlineto{\pgfqpoint{1.072343in}{1.405499in}}%
\pgfpathlineto{\pgfqpoint{1.073563in}{1.112889in}}%
\pgfpathlineto{\pgfqpoint{1.075699in}{0.887973in}}%
\pgfpathlineto{\pgfqpoint{1.077225in}{0.713406in}}%
\pgfpathlineto{\pgfqpoint{1.077835in}{0.795637in}}%
\pgfpathlineto{\pgfqpoint{1.081496in}{1.289569in}}%
\pgfpathlineto{\pgfqpoint{1.081802in}{1.302240in}}%
\pgfpathlineto{\pgfqpoint{1.082412in}{1.267562in}}%
\pgfpathlineto{\pgfqpoint{1.088514in}{0.785658in}}%
\pgfpathlineto{\pgfqpoint{1.088819in}{0.797564in}}%
\pgfpathlineto{\pgfqpoint{1.091565in}{1.368863in}}%
\pgfpathlineto{\pgfqpoint{1.093396in}{1.738872in}}%
\pgfpathlineto{\pgfqpoint{1.094311in}{1.675377in}}%
\pgfpathlineto{\pgfqpoint{1.095837in}{1.624853in}}%
\pgfpathlineto{\pgfqpoint{1.097363in}{1.088647in}}%
\pgfpathlineto{\pgfqpoint{1.099498in}{0.772525in}}%
\pgfpathlineto{\pgfqpoint{1.100719in}{0.710085in}}%
\pgfpathlineto{\pgfqpoint{1.101024in}{0.727078in}}%
\pgfpathlineto{\pgfqpoint{1.105601in}{1.416548in}}%
\pgfpathlineto{\pgfqpoint{1.106516in}{1.294021in}}%
\pgfpathlineto{\pgfqpoint{1.110178in}{0.784493in}}%
\pgfpathlineto{\pgfqpoint{1.111093in}{0.828697in}}%
\pgfpathlineto{\pgfqpoint{1.112924in}{0.908297in}}%
\pgfpathlineto{\pgfqpoint{1.116939in}{1.770476in}}%
\pgfpathmoveto{\pgfqpoint{1.117535in}{1.770476in}}%
\pgfpathlineto{\pgfqpoint{1.122382in}{0.744505in}}%
\pgfpathlineto{\pgfqpoint{1.123298in}{0.865439in}}%
\pgfpathlineto{\pgfqpoint{1.126654in}{1.360800in}}%
\pgfpathlineto{\pgfqpoint{1.127569in}{1.278565in}}%
\pgfpathlineto{\pgfqpoint{1.131536in}{0.849399in}}%
\pgfpathlineto{\pgfqpoint{1.132146in}{0.853439in}}%
\pgfpathlineto{\pgfqpoint{1.132756in}{0.829694in}}%
\pgfpathlineto{\pgfqpoint{1.133977in}{0.705125in}}%
\pgfpathlineto{\pgfqpoint{1.134587in}{0.745289in}}%
\pgfpathlineto{\pgfqpoint{1.137943in}{1.507165in}}%
\pgfpathlineto{\pgfqpoint{1.140145in}{1.770476in}}%
\pgfpathmoveto{\pgfqpoint{1.141985in}{1.770476in}}%
\pgfpathlineto{\pgfqpoint{1.145571in}{0.738722in}}%
\pgfpathlineto{\pgfqpoint{1.145876in}{0.737394in}}%
\pgfpathlineto{\pgfqpoint{1.147097in}{0.975834in}}%
\pgfpathlineto{\pgfqpoint{1.148012in}{1.089111in}}%
\pgfpathlineto{\pgfqpoint{1.148622in}{1.047299in}}%
\pgfpathlineto{\pgfqpoint{1.149233in}{1.010334in}}%
\pgfpathlineto{\pgfqpoint{1.149843in}{1.044251in}}%
\pgfpathlineto{\pgfqpoint{1.151369in}{1.192078in}}%
\pgfpathlineto{\pgfqpoint{1.151979in}{1.169705in}}%
\pgfpathlineto{\pgfqpoint{1.155640in}{0.799059in}}%
\pgfpathlineto{\pgfqpoint{1.156250in}{0.869356in}}%
\pgfpathlineto{\pgfqpoint{1.157166in}{1.024131in}}%
\pgfpathlineto{\pgfqpoint{1.158081in}{0.943064in}}%
\pgfpathlineto{\pgfqpoint{1.158996in}{0.868162in}}%
\pgfpathlineto{\pgfqpoint{1.159607in}{0.906168in}}%
\pgfpathlineto{\pgfqpoint{1.163268in}{1.356547in}}%
\pgfpathlineto{\pgfqpoint{1.164489in}{1.250395in}}%
\pgfpathlineto{\pgfqpoint{1.165099in}{1.229468in}}%
\pgfpathlineto{\pgfqpoint{1.165709in}{1.254058in}}%
\pgfpathlineto{\pgfqpoint{1.166014in}{1.267086in}}%
\pgfpathlineto{\pgfqpoint{1.166624in}{1.240651in}}%
\pgfpathlineto{\pgfqpoint{1.167845in}{1.104758in}}%
\pgfpathlineto{\pgfqpoint{1.168455in}{1.160613in}}%
\pgfpathlineto{\pgfqpoint{1.169065in}{1.199621in}}%
\pgfpathlineto{\pgfqpoint{1.169676in}{1.136582in}}%
\pgfpathlineto{\pgfqpoint{1.170591in}{1.047301in}}%
\pgfpathlineto{\pgfqpoint{1.171201in}{1.121014in}}%
\pgfpathlineto{\pgfqpoint{1.172117in}{1.198376in}}%
\pgfpathlineto{\pgfqpoint{1.172727in}{1.143817in}}%
\pgfpathlineto{\pgfqpoint{1.173642in}{1.057934in}}%
\pgfpathlineto{\pgfqpoint{1.174557in}{1.092093in}}%
\pgfpathlineto{\pgfqpoint{1.174863in}{1.102082in}}%
\pgfpathlineto{\pgfqpoint{1.175168in}{1.095908in}}%
\pgfpathlineto{\pgfqpoint{1.179439in}{0.787653in}}%
\pgfpathlineto{\pgfqpoint{1.180050in}{0.809115in}}%
\pgfpathlineto{\pgfqpoint{1.184321in}{1.232895in}}%
\pgfpathlineto{\pgfqpoint{1.185542in}{1.178174in}}%
\pgfpathlineto{\pgfqpoint{1.187678in}{1.052336in}}%
\pgfpathlineto{\pgfqpoint{1.187983in}{1.070270in}}%
\pgfpathlineto{\pgfqpoint{1.189813in}{1.308977in}}%
\pgfpathlineto{\pgfqpoint{1.190729in}{1.263793in}}%
\pgfpathlineto{\pgfqpoint{1.193170in}{1.202448in}}%
\pgfpathlineto{\pgfqpoint{1.193780in}{1.214121in}}%
\pgfpathlineto{\pgfqpoint{1.194390in}{1.231698in}}%
\pgfpathlineto{\pgfqpoint{1.194695in}{1.217888in}}%
\pgfpathlineto{\pgfqpoint{1.197136in}{0.845570in}}%
\pgfpathlineto{\pgfqpoint{1.198052in}{0.924237in}}%
\pgfpathlineto{\pgfqpoint{1.199882in}{1.111826in}}%
\pgfpathlineto{\pgfqpoint{1.200493in}{1.090665in}}%
\pgfpathlineto{\pgfqpoint{1.202933in}{0.906292in}}%
\pgfpathlineto{\pgfqpoint{1.203544in}{0.929514in}}%
\pgfpathlineto{\pgfqpoint{1.205069in}{1.107875in}}%
\pgfpathlineto{\pgfqpoint{1.205985in}{1.047937in}}%
\pgfpathlineto{\pgfqpoint{1.208731in}{0.902642in}}%
\pgfpathlineto{\pgfqpoint{1.209036in}{0.903165in}}%
\pgfpathlineto{\pgfqpoint{1.210256in}{1.065296in}}%
\pgfpathlineto{\pgfqpoint{1.210867in}{1.121693in}}%
\pgfpathlineto{\pgfqpoint{1.211782in}{1.071528in}}%
\pgfpathlineto{\pgfqpoint{1.212087in}{1.067738in}}%
\pgfpathlineto{\pgfqpoint{1.215138in}{1.299799in}}%
\pgfpathlineto{\pgfqpoint{1.216359in}{1.517255in}}%
\pgfpathlineto{\pgfqpoint{1.216969in}{1.470541in}}%
\pgfpathlineto{\pgfqpoint{1.218494in}{1.109866in}}%
\pgfpathlineto{\pgfqpoint{1.219410in}{1.172871in}}%
\pgfpathlineto{\pgfqpoint{1.219715in}{1.170791in}}%
\pgfpathlineto{\pgfqpoint{1.221241in}{0.838276in}}%
\pgfpathlineto{\pgfqpoint{1.222461in}{0.962719in}}%
\pgfpathlineto{\pgfqpoint{1.225817in}{1.143631in}}%
\pgfpathlineto{\pgfqpoint{1.226122in}{1.120012in}}%
\pgfpathlineto{\pgfqpoint{1.227648in}{0.938153in}}%
\pgfpathlineto{\pgfqpoint{1.228563in}{0.996019in}}%
\pgfpathlineto{\pgfqpoint{1.228869in}{1.016889in}}%
\pgfpathlineto{\pgfqpoint{1.229479in}{0.988119in}}%
\pgfpathlineto{\pgfqpoint{1.230394in}{0.903715in}}%
\pgfpathlineto{\pgfqpoint{1.231004in}{0.947789in}}%
\pgfpathlineto{\pgfqpoint{1.231615in}{0.999326in}}%
\pgfpathlineto{\pgfqpoint{1.232225in}{0.970001in}}%
\pgfpathlineto{\pgfqpoint{1.233140in}{0.900389in}}%
\pgfpathlineto{\pgfqpoint{1.233445in}{0.925145in}}%
\pgfpathlineto{\pgfqpoint{1.237717in}{1.534260in}}%
\pgfpathlineto{\pgfqpoint{1.238022in}{1.482894in}}%
\pgfpathlineto{\pgfqpoint{1.239548in}{1.055794in}}%
\pgfpathlineto{\pgfqpoint{1.240768in}{1.070988in}}%
\pgfpathlineto{\pgfqpoint{1.242294in}{0.905575in}}%
\pgfpathlineto{\pgfqpoint{1.242904in}{0.982122in}}%
\pgfpathlineto{\pgfqpoint{1.243514in}{1.039679in}}%
\pgfpathlineto{\pgfqpoint{1.244124in}{0.994406in}}%
\pgfpathlineto{\pgfqpoint{1.244735in}{0.925664in}}%
\pgfpathlineto{\pgfqpoint{1.245345in}{0.966390in}}%
\pgfpathlineto{\pgfqpoint{1.246870in}{1.295136in}}%
\pgfpathlineto{\pgfqpoint{1.247786in}{1.231115in}}%
\pgfpathlineto{\pgfqpoint{1.249311in}{1.368456in}}%
\pgfpathlineto{\pgfqpoint{1.249922in}{1.289390in}}%
\pgfpathlineto{\pgfqpoint{1.251447in}{0.942268in}}%
\pgfpathlineto{\pgfqpoint{1.252668in}{0.955549in}}%
\pgfpathlineto{\pgfqpoint{1.254498in}{0.638697in}}%
\pgfpathlineto{\pgfqpoint{1.255109in}{0.736832in}}%
\pgfpathlineto{\pgfqpoint{1.259075in}{1.497216in}}%
\pgfpathlineto{\pgfqpoint{1.259991in}{1.442271in}}%
\pgfpathlineto{\pgfqpoint{1.261821in}{1.328267in}}%
\pgfpathlineto{\pgfqpoint{1.265483in}{0.915664in}}%
\pgfpathlineto{\pgfqpoint{1.265788in}{0.918133in}}%
\pgfpathlineto{\pgfqpoint{1.267619in}{1.182150in}}%
\pgfpathlineto{\pgfqpoint{1.268839in}{1.055764in}}%
\pgfpathlineto{\pgfqpoint{1.269449in}{1.103838in}}%
\pgfpathlineto{\pgfqpoint{1.270365in}{1.207142in}}%
\pgfpathlineto{\pgfqpoint{1.270975in}{1.177277in}}%
\pgfpathlineto{\pgfqpoint{1.272195in}{1.053504in}}%
\pgfpathlineto{\pgfqpoint{1.272806in}{1.107335in}}%
\pgfpathlineto{\pgfqpoint{1.273416in}{1.160291in}}%
\pgfpathlineto{\pgfqpoint{1.274026in}{1.128421in}}%
\pgfpathlineto{\pgfqpoint{1.277382in}{0.768443in}}%
\pgfpathlineto{\pgfqpoint{1.277687in}{0.770069in}}%
\pgfpathlineto{\pgfqpoint{1.278603in}{0.888175in}}%
\pgfpathlineto{\pgfqpoint{1.281654in}{1.328806in}}%
\pgfpathlineto{\pgfqpoint{1.281959in}{1.324257in}}%
\pgfpathlineto{\pgfqpoint{1.285926in}{0.972812in}}%
\pgfpathlineto{\pgfqpoint{1.286841in}{1.035048in}}%
\pgfpathlineto{\pgfqpoint{1.288367in}{1.183998in}}%
\pgfpathlineto{\pgfqpoint{1.288977in}{1.159022in}}%
\pgfpathlineto{\pgfqpoint{1.289892in}{1.113566in}}%
\pgfpathlineto{\pgfqpoint{1.290502in}{1.156774in}}%
\pgfpathlineto{\pgfqpoint{1.291418in}{1.224338in}}%
\pgfpathlineto{\pgfqpoint{1.292028in}{1.197939in}}%
\pgfpathlineto{\pgfqpoint{1.292333in}{1.187655in}}%
\pgfpathlineto{\pgfqpoint{1.292638in}{1.194922in}}%
\pgfpathlineto{\pgfqpoint{1.293859in}{1.289090in}}%
\pgfpathlineto{\pgfqpoint{1.294469in}{1.240520in}}%
\pgfpathlineto{\pgfqpoint{1.297520in}{0.946603in}}%
\pgfpathlineto{\pgfqpoint{1.299046in}{0.814502in}}%
\pgfpathlineto{\pgfqpoint{1.299656in}{0.872363in}}%
\pgfpathlineto{\pgfqpoint{1.300876in}{0.969268in}}%
\pgfpathlineto{\pgfqpoint{1.301487in}{0.939218in}}%
\pgfpathlineto{\pgfqpoint{1.301792in}{0.925676in}}%
\pgfpathlineto{\pgfqpoint{1.302402in}{0.954993in}}%
\pgfpathlineto{\pgfqpoint{1.304233in}{1.255052in}}%
\pgfpathlineto{\pgfqpoint{1.305148in}{1.187769in}}%
\pgfpathlineto{\pgfqpoint{1.308504in}{0.939851in}}%
\pgfpathlineto{\pgfqpoint{1.308809in}{0.941543in}}%
\pgfpathlineto{\pgfqpoint{1.312166in}{1.176820in}}%
\pgfpathlineto{\pgfqpoint{1.312776in}{1.223345in}}%
\pgfpathlineto{\pgfqpoint{1.313386in}{1.176140in}}%
\pgfpathlineto{\pgfqpoint{1.313996in}{1.136012in}}%
\pgfpathlineto{\pgfqpoint{1.314607in}{1.169889in}}%
\pgfpathlineto{\pgfqpoint{1.315522in}{1.223091in}}%
\pgfpathlineto{\pgfqpoint{1.316132in}{1.197520in}}%
\pgfpathlineto{\pgfqpoint{1.317353in}{1.070058in}}%
\pgfpathlineto{\pgfqpoint{1.317963in}{1.110362in}}%
\pgfpathlineto{\pgfqpoint{1.318573in}{1.145516in}}%
\pgfpathlineto{\pgfqpoint{1.319183in}{1.088302in}}%
\pgfpathlineto{\pgfqpoint{1.320404in}{0.879009in}}%
\pgfpathlineto{\pgfqpoint{1.321014in}{0.929969in}}%
\pgfpathlineto{\pgfqpoint{1.322235in}{1.066435in}}%
\pgfpathlineto{\pgfqpoint{1.322845in}{1.039774in}}%
\pgfpathlineto{\pgfqpoint{1.323455in}{1.022645in}}%
\pgfpathlineto{\pgfqpoint{1.323760in}{1.035534in}}%
\pgfpathlineto{\pgfqpoint{1.324981in}{1.125614in}}%
\pgfpathlineto{\pgfqpoint{1.325591in}{1.108568in}}%
\pgfpathlineto{\pgfqpoint{1.329252in}{0.898253in}}%
\pgfpathlineto{\pgfqpoint{1.333219in}{1.067888in}}%
\pgfpathlineto{\pgfqpoint{1.334134in}{1.082481in}}%
\pgfpathlineto{\pgfqpoint{1.336880in}{1.347740in}}%
\pgfpathlineto{\pgfqpoint{1.337796in}{1.290162in}}%
\pgfpathlineto{\pgfqpoint{1.340542in}{0.925974in}}%
\pgfpathlineto{\pgfqpoint{1.341457in}{0.970920in}}%
\pgfpathlineto{\pgfqpoint{1.344813in}{1.104506in}}%
\pgfpathlineto{\pgfqpoint{1.346034in}{1.148718in}}%
\pgfpathlineto{\pgfqpoint{1.346644in}{1.123789in}}%
\pgfpathlineto{\pgfqpoint{1.346949in}{1.110592in}}%
\pgfpathlineto{\pgfqpoint{1.347559in}{1.133675in}}%
\pgfpathlineto{\pgfqpoint{1.348475in}{1.219938in}}%
\pgfpathlineto{\pgfqpoint{1.349085in}{1.179574in}}%
\pgfpathlineto{\pgfqpoint{1.352746in}{0.813943in}}%
\pgfpathlineto{\pgfqpoint{1.354272in}{1.009924in}}%
\pgfpathlineto{\pgfqpoint{1.356408in}{1.281783in}}%
\pgfpathlineto{\pgfqpoint{1.357323in}{1.210733in}}%
\pgfpathlineto{\pgfqpoint{1.360374in}{0.987692in}}%
\pgfpathlineto{\pgfqpoint{1.361290in}{0.945486in}}%
\pgfpathlineto{\pgfqpoint{1.361595in}{0.958595in}}%
\pgfpathlineto{\pgfqpoint{1.365867in}{1.243186in}}%
\pgfpathlineto{\pgfqpoint{1.366477in}{1.230916in}}%
\pgfpathlineto{\pgfqpoint{1.366782in}{1.239380in}}%
\pgfpathlineto{\pgfqpoint{1.367697in}{1.289171in}}%
\pgfpathlineto{\pgfqpoint{1.368307in}{1.244321in}}%
\pgfpathlineto{\pgfqpoint{1.371359in}{1.054384in}}%
\pgfpathlineto{\pgfqpoint{1.373189in}{0.865823in}}%
\pgfpathlineto{\pgfqpoint{1.374105in}{0.749564in}}%
\pgfpathlineto{\pgfqpoint{1.374715in}{0.780585in}}%
\pgfpathlineto{\pgfqpoint{1.378987in}{1.419494in}}%
\pgfpathlineto{\pgfqpoint{1.379902in}{1.274406in}}%
\pgfpathlineto{\pgfqpoint{1.383258in}{0.780351in}}%
\pgfpathlineto{\pgfqpoint{1.383563in}{0.760019in}}%
\pgfpathlineto{\pgfqpoint{1.384174in}{0.815495in}}%
\pgfpathlineto{\pgfqpoint{1.387225in}{1.166637in}}%
\pgfpathlineto{\pgfqpoint{1.389056in}{1.372746in}}%
\pgfpathlineto{\pgfqpoint{1.389971in}{1.343079in}}%
\pgfpathlineto{\pgfqpoint{1.391191in}{1.230033in}}%
\pgfpathlineto{\pgfqpoint{1.393327in}{0.864004in}}%
\pgfpathlineto{\pgfqpoint{1.393937in}{0.873241in}}%
\pgfpathlineto{\pgfqpoint{1.395158in}{0.935219in}}%
\pgfpathlineto{\pgfqpoint{1.396989in}{1.168203in}}%
\pgfpathlineto{\pgfqpoint{1.397904in}{1.092037in}}%
\pgfpathlineto{\pgfqpoint{1.398514in}{1.022424in}}%
\pgfpathlineto{\pgfqpoint{1.399124in}{1.056847in}}%
\pgfpathlineto{\pgfqpoint{1.400345in}{1.205924in}}%
\pgfpathlineto{\pgfqpoint{1.400955in}{1.138291in}}%
\pgfpathlineto{\pgfqpoint{1.401870in}{0.970360in}}%
\pgfpathlineto{\pgfqpoint{1.402786in}{1.024911in}}%
\pgfpathlineto{\pgfqpoint{1.403091in}{1.037002in}}%
\pgfpathlineto{\pgfqpoint{1.403396in}{1.026230in}}%
\pgfpathlineto{\pgfqpoint{1.405227in}{0.791150in}}%
\pgfpathlineto{\pgfqpoint{1.406142in}{0.861491in}}%
\pgfpathlineto{\pgfqpoint{1.408888in}{1.214959in}}%
\pgfpathlineto{\pgfqpoint{1.409804in}{1.404645in}}%
\pgfpathlineto{\pgfqpoint{1.410719in}{1.373410in}}%
\pgfpathlineto{\pgfqpoint{1.411024in}{1.369638in}}%
\pgfpathlineto{\pgfqpoint{1.411329in}{1.375189in}}%
\pgfpathlineto{\pgfqpoint{1.411939in}{1.387464in}}%
\pgfpathlineto{\pgfqpoint{1.412244in}{1.381383in}}%
\pgfpathlineto{\pgfqpoint{1.413770in}{1.162493in}}%
\pgfpathlineto{\pgfqpoint{1.417431in}{0.858296in}}%
\pgfpathlineto{\pgfqpoint{1.421398in}{1.326853in}}%
\pgfpathlineto{\pgfqpoint{1.422313in}{1.200472in}}%
\pgfpathlineto{\pgfqpoint{1.425059in}{0.902400in}}%
\pgfpathlineto{\pgfqpoint{1.425975in}{0.876537in}}%
\pgfpathlineto{\pgfqpoint{1.426280in}{0.886399in}}%
\pgfpathlineto{\pgfqpoint{1.427806in}{1.077774in}}%
\pgfpathlineto{\pgfqpoint{1.428721in}{0.973218in}}%
\pgfpathlineto{\pgfqpoint{1.429331in}{0.908855in}}%
\pgfpathlineto{\pgfqpoint{1.429941in}{0.963046in}}%
\pgfpathlineto{\pgfqpoint{1.430857in}{1.046899in}}%
\pgfpathlineto{\pgfqpoint{1.431467in}{1.013311in}}%
\pgfpathlineto{\pgfqpoint{1.432077in}{0.944157in}}%
\pgfpathlineto{\pgfqpoint{1.432687in}{0.993505in}}%
\pgfpathlineto{\pgfqpoint{1.433908in}{1.221018in}}%
\pgfpathlineto{\pgfqpoint{1.434518in}{1.134934in}}%
\pgfpathlineto{\pgfqpoint{1.435739in}{0.875436in}}%
\pgfpathlineto{\pgfqpoint{1.436654in}{0.959122in}}%
\pgfpathlineto{\pgfqpoint{1.440315in}{1.312325in}}%
\pgfpathlineto{\pgfqpoint{1.441536in}{1.257645in}}%
\pgfpathlineto{\pgfqpoint{1.443672in}{1.156370in}}%
\pgfpathlineto{\pgfqpoint{1.445197in}{1.181205in}}%
\pgfpathlineto{\pgfqpoint{1.446113in}{1.120751in}}%
\pgfpathlineto{\pgfqpoint{1.447333in}{0.986079in}}%
\pgfpathlineto{\pgfqpoint{1.448248in}{1.029194in}}%
\pgfpathlineto{\pgfqpoint{1.448554in}{1.025942in}}%
\pgfpathlineto{\pgfqpoint{1.449774in}{0.838217in}}%
\pgfpathlineto{\pgfqpoint{1.450384in}{0.756398in}}%
\pgfpathlineto{\pgfqpoint{1.451300in}{0.829979in}}%
\pgfpathlineto{\pgfqpoint{1.454656in}{1.217204in}}%
\pgfpathlineto{\pgfqpoint{1.454961in}{1.192013in}}%
\pgfpathlineto{\pgfqpoint{1.456487in}{0.985727in}}%
\pgfpathlineto{\pgfqpoint{1.457097in}{1.020542in}}%
\pgfpathlineto{\pgfqpoint{1.458012in}{1.069090in}}%
\pgfpathlineto{\pgfqpoint{1.458317in}{1.047014in}}%
\pgfpathlineto{\pgfqpoint{1.459538in}{0.898245in}}%
\pgfpathlineto{\pgfqpoint{1.460453in}{0.958232in}}%
\pgfpathlineto{\pgfqpoint{1.463809in}{1.553943in}}%
\pgfpathlineto{\pgfqpoint{1.464420in}{1.412269in}}%
\pgfpathlineto{\pgfqpoint{1.465640in}{1.020271in}}%
\pgfpathlineto{\pgfqpoint{1.466556in}{1.080696in}}%
\pgfpathlineto{\pgfqpoint{1.466861in}{1.081091in}}%
\pgfpathlineto{\pgfqpoint{1.468081in}{1.007614in}}%
\pgfpathlineto{\pgfqpoint{1.468691in}{1.039772in}}%
\pgfpathlineto{\pgfqpoint{1.469607in}{1.092523in}}%
\pgfpathlineto{\pgfqpoint{1.470522in}{1.065836in}}%
\pgfpathlineto{\pgfqpoint{1.471132in}{1.045346in}}%
\pgfpathlineto{\pgfqpoint{1.471743in}{1.064836in}}%
\pgfpathlineto{\pgfqpoint{1.472353in}{1.108303in}}%
\pgfpathlineto{\pgfqpoint{1.472963in}{1.077401in}}%
\pgfpathlineto{\pgfqpoint{1.474183in}{0.914121in}}%
\pgfpathlineto{\pgfqpoint{1.474794in}{0.968659in}}%
\pgfpathlineto{\pgfqpoint{1.475709in}{1.057591in}}%
\pgfpathlineto{\pgfqpoint{1.476319in}{1.029655in}}%
\pgfpathlineto{\pgfqpoint{1.477235in}{0.960831in}}%
\pgfpathlineto{\pgfqpoint{1.477845in}{1.007475in}}%
\pgfpathlineto{\pgfqpoint{1.478455in}{1.030055in}}%
\pgfpathlineto{\pgfqpoint{1.478760in}{1.010347in}}%
\pgfpathlineto{\pgfqpoint{1.480286in}{0.741103in}}%
\pgfpathlineto{\pgfqpoint{1.481201in}{0.853201in}}%
\pgfpathlineto{\pgfqpoint{1.482422in}{0.970743in}}%
\pgfpathlineto{\pgfqpoint{1.483032in}{0.952564in}}%
\pgfpathlineto{\pgfqpoint{1.483337in}{0.956788in}}%
\pgfpathlineto{\pgfqpoint{1.486998in}{1.326557in}}%
\pgfpathlineto{\pgfqpoint{1.488219in}{1.245258in}}%
\pgfpathlineto{\pgfqpoint{1.491880in}{1.003527in}}%
\pgfpathlineto{\pgfqpoint{1.492185in}{1.004950in}}%
\pgfpathlineto{\pgfqpoint{1.494016in}{1.224057in}}%
\pgfpathlineto{\pgfqpoint{1.494626in}{1.151546in}}%
\pgfpathlineto{\pgfqpoint{1.495237in}{1.067373in}}%
\pgfpathlineto{\pgfqpoint{1.496152in}{1.136923in}}%
\pgfpathlineto{\pgfqpoint{1.496457in}{1.152400in}}%
\pgfpathlineto{\pgfqpoint{1.497067in}{1.115278in}}%
\pgfpathlineto{\pgfqpoint{1.500424in}{0.822972in}}%
\pgfpathlineto{\pgfqpoint{1.501644in}{0.752522in}}%
\pgfpathlineto{\pgfqpoint{1.501949in}{0.769068in}}%
\pgfpathlineto{\pgfqpoint{1.505916in}{1.286415in}}%
\pgfpathlineto{\pgfqpoint{1.506526in}{1.246517in}}%
\pgfpathlineto{\pgfqpoint{1.509577in}{0.977445in}}%
\pgfpathlineto{\pgfqpoint{1.510187in}{0.988661in}}%
\pgfpathlineto{\pgfqpoint{1.511408in}{1.166928in}}%
\pgfpathlineto{\pgfqpoint{1.512323in}{1.279781in}}%
\pgfpathlineto{\pgfqpoint{1.512933in}{1.220392in}}%
\pgfpathlineto{\pgfqpoint{1.513849in}{1.130504in}}%
\pgfpathlineto{\pgfqpoint{1.514764in}{1.160412in}}%
\pgfpathlineto{\pgfqpoint{1.515374in}{1.139155in}}%
\pgfpathlineto{\pgfqpoint{1.516595in}{1.061414in}}%
\pgfpathlineto{\pgfqpoint{1.516900in}{1.086156in}}%
\pgfpathlineto{\pgfqpoint{1.518120in}{1.246281in}}%
\pgfpathlineto{\pgfqpoint{1.519036in}{1.185854in}}%
\pgfpathlineto{\pgfqpoint{1.522697in}{0.867385in}}%
\pgfpathlineto{\pgfqpoint{1.523613in}{0.896264in}}%
\pgfpathlineto{\pgfqpoint{1.526054in}{1.112015in}}%
\pgfpathlineto{\pgfqpoint{1.526664in}{1.138878in}}%
\pgfpathlineto{\pgfqpoint{1.526969in}{1.118524in}}%
\pgfpathlineto{\pgfqpoint{1.528494in}{0.960212in}}%
\pgfpathlineto{\pgfqpoint{1.529105in}{0.997873in}}%
\pgfpathlineto{\pgfqpoint{1.530020in}{1.053557in}}%
\pgfpathlineto{\pgfqpoint{1.530630in}{1.006693in}}%
\pgfpathlineto{\pgfqpoint{1.532156in}{0.860954in}}%
\pgfpathlineto{\pgfqpoint{1.532766in}{0.884681in}}%
\pgfpathlineto{\pgfqpoint{1.535207in}{1.108586in}}%
\pgfpathlineto{\pgfqpoint{1.539479in}{1.637975in}}%
\pgfpathlineto{\pgfqpoint{1.539784in}{1.605959in}}%
\pgfpathlineto{\pgfqpoint{1.543140in}{0.842042in}}%
\pgfpathlineto{\pgfqpoint{1.544056in}{0.905174in}}%
\pgfpathlineto{\pgfqpoint{1.545886in}{1.032814in}}%
\pgfpathlineto{\pgfqpoint{1.546496in}{1.031629in}}%
\pgfpathlineto{\pgfqpoint{1.548022in}{1.120651in}}%
\pgfpathlineto{\pgfqpoint{1.548937in}{1.064832in}}%
\pgfpathlineto{\pgfqpoint{1.549853in}{0.984915in}}%
\pgfpathlineto{\pgfqpoint{1.550463in}{1.024955in}}%
\pgfpathlineto{\pgfqpoint{1.551378in}{1.071089in}}%
\pgfpathlineto{\pgfqpoint{1.551989in}{1.030623in}}%
\pgfpathlineto{\pgfqpoint{1.552599in}{0.991470in}}%
\pgfpathlineto{\pgfqpoint{1.553514in}{1.028292in}}%
\pgfpathlineto{\pgfqpoint{1.553819in}{1.039067in}}%
\pgfpathlineto{\pgfqpoint{1.554430in}{1.024029in}}%
\pgfpathlineto{\pgfqpoint{1.556565in}{0.891819in}}%
\pgfpathlineto{\pgfqpoint{1.557176in}{0.906827in}}%
\pgfpathlineto{\pgfqpoint{1.558701in}{1.072572in}}%
\pgfpathlineto{\pgfqpoint{1.561447in}{1.288720in}}%
\pgfpathlineto{\pgfqpoint{1.561752in}{1.290878in}}%
\pgfpathlineto{\pgfqpoint{1.562057in}{1.287773in}}%
\pgfpathlineto{\pgfqpoint{1.563278in}{1.192368in}}%
\pgfpathlineto{\pgfqpoint{1.566024in}{1.015356in}}%
\pgfpathlineto{\pgfqpoint{1.567550in}{0.986380in}}%
\pgfpathlineto{\pgfqpoint{1.567855in}{0.995205in}}%
\pgfpathlineto{\pgfqpoint{1.570296in}{1.310573in}}%
\pgfpathlineto{\pgfqpoint{1.571516in}{1.212021in}}%
\pgfpathlineto{\pgfqpoint{1.573042in}{1.133293in}}%
\pgfpathlineto{\pgfqpoint{1.575178in}{0.843887in}}%
\pgfpathlineto{\pgfqpoint{1.576093in}{0.849764in}}%
\pgfpathlineto{\pgfqpoint{1.577619in}{0.766980in}}%
\pgfpathlineto{\pgfqpoint{1.578229in}{0.824100in}}%
\pgfpathlineto{\pgfqpoint{1.583111in}{1.341885in}}%
\pgfpathlineto{\pgfqpoint{1.583416in}{1.344258in}}%
\pgfpathlineto{\pgfqpoint{1.583721in}{1.336220in}}%
\pgfpathlineto{\pgfqpoint{1.586467in}{1.102085in}}%
\pgfpathlineto{\pgfqpoint{1.587382in}{1.068696in}}%
\pgfpathlineto{\pgfqpoint{1.587993in}{1.090113in}}%
\pgfpathlineto{\pgfqpoint{1.588908in}{1.127368in}}%
\pgfpathlineto{\pgfqpoint{1.589518in}{1.100052in}}%
\pgfpathlineto{\pgfqpoint{1.591044in}{0.932044in}}%
\pgfpathlineto{\pgfqpoint{1.591959in}{0.981798in}}%
\pgfpathlineto{\pgfqpoint{1.595620in}{1.310516in}}%
\pgfpathlineto{\pgfqpoint{1.596231in}{1.250658in}}%
\pgfpathlineto{\pgfqpoint{1.599892in}{0.799826in}}%
\pgfpathlineto{\pgfqpoint{1.600197in}{0.793636in}}%
\pgfpathlineto{\pgfqpoint{1.600502in}{0.805861in}}%
\pgfpathlineto{\pgfqpoint{1.602638in}{1.139373in}}%
\pgfpathlineto{\pgfqpoint{1.604469in}{1.086769in}}%
\pgfpathlineto{\pgfqpoint{1.605079in}{1.070785in}}%
\pgfpathlineto{\pgfqpoint{1.605689in}{1.087714in}}%
\pgfpathlineto{\pgfqpoint{1.606605in}{1.125455in}}%
\pgfpathlineto{\pgfqpoint{1.607215in}{1.101903in}}%
\pgfpathlineto{\pgfqpoint{1.610266in}{0.952294in}}%
\pgfpathlineto{\pgfqpoint{1.610876in}{0.934072in}}%
\pgfpathlineto{\pgfqpoint{1.611181in}{0.943447in}}%
\pgfpathlineto{\pgfqpoint{1.612707in}{1.195280in}}%
\pgfpathlineto{\pgfqpoint{1.614538in}{1.371655in}}%
\pgfpathlineto{\pgfqpoint{1.615148in}{1.359926in}}%
\pgfpathlineto{\pgfqpoint{1.616063in}{1.375347in}}%
\pgfpathlineto{\pgfqpoint{1.616369in}{1.371211in}}%
\pgfpathlineto{\pgfqpoint{1.617589in}{1.187816in}}%
\pgfpathlineto{\pgfqpoint{1.618809in}{1.016377in}}%
\pgfpathlineto{\pgfqpoint{1.619725in}{1.027951in}}%
\pgfpathlineto{\pgfqpoint{1.620640in}{0.959337in}}%
\pgfpathlineto{\pgfqpoint{1.621861in}{0.852735in}}%
\pgfpathlineto{\pgfqpoint{1.622471in}{0.875073in}}%
\pgfpathlineto{\pgfqpoint{1.623691in}{0.869811in}}%
\pgfpathlineto{\pgfqpoint{1.624302in}{0.908635in}}%
\pgfpathlineto{\pgfqpoint{1.627658in}{1.170476in}}%
\pgfpathlineto{\pgfqpoint{1.628878in}{1.217118in}}%
\pgfpathlineto{\pgfqpoint{1.629183in}{1.206270in}}%
\pgfpathlineto{\pgfqpoint{1.633150in}{0.884301in}}%
\pgfpathlineto{\pgfqpoint{1.634370in}{0.939418in}}%
\pgfpathlineto{\pgfqpoint{1.638032in}{1.411888in}}%
\pgfpathlineto{\pgfqpoint{1.638947in}{1.513570in}}%
\pgfpathlineto{\pgfqpoint{1.639557in}{1.413120in}}%
\pgfpathlineto{\pgfqpoint{1.642609in}{0.756485in}}%
\pgfpathlineto{\pgfqpoint{1.643219in}{0.790927in}}%
\pgfpathlineto{\pgfqpoint{1.646880in}{1.463660in}}%
\pgfpathlineto{\pgfqpoint{1.647796in}{1.334360in}}%
\pgfpathlineto{\pgfqpoint{1.650237in}{1.030658in}}%
\pgfpathlineto{\pgfqpoint{1.651762in}{0.872197in}}%
\pgfpathlineto{\pgfqpoint{1.652372in}{0.924490in}}%
\pgfpathlineto{\pgfqpoint{1.653288in}{1.034834in}}%
\pgfpathlineto{\pgfqpoint{1.654203in}{0.975921in}}%
\pgfpathlineto{\pgfqpoint{1.655119in}{0.899399in}}%
\pgfpathlineto{\pgfqpoint{1.656034in}{0.922986in}}%
\pgfpathlineto{\pgfqpoint{1.656644in}{0.907909in}}%
\pgfpathlineto{\pgfqpoint{1.657559in}{0.871872in}}%
\pgfpathlineto{\pgfqpoint{1.657865in}{0.883876in}}%
\pgfpathlineto{\pgfqpoint{1.659085in}{1.180445in}}%
\pgfpathlineto{\pgfqpoint{1.661221in}{1.698037in}}%
\pgfpathlineto{\pgfqpoint{1.661526in}{1.679653in}}%
\pgfpathlineto{\pgfqpoint{1.663357in}{1.140526in}}%
\pgfpathlineto{\pgfqpoint{1.665798in}{0.868793in}}%
\pgfpathlineto{\pgfqpoint{1.666713in}{0.860491in}}%
\pgfpathlineto{\pgfqpoint{1.667018in}{0.862325in}}%
\pgfpathlineto{\pgfqpoint{1.667933in}{0.921480in}}%
\pgfpathlineto{\pgfqpoint{1.671290in}{1.323582in}}%
\pgfpathlineto{\pgfqpoint{1.671900in}{1.299467in}}%
\pgfpathlineto{\pgfqpoint{1.673731in}{1.040802in}}%
\pgfpathlineto{\pgfqpoint{1.674646in}{1.108331in}}%
\pgfpathlineto{\pgfqpoint{1.674951in}{1.108430in}}%
\pgfpathlineto{\pgfqpoint{1.678002in}{0.897981in}}%
\pgfpathlineto{\pgfqpoint{1.678613in}{0.913435in}}%
\pgfpathlineto{\pgfqpoint{1.680443in}{1.130159in}}%
\pgfpathlineto{\pgfqpoint{1.681664in}{1.048456in}}%
\pgfpathlineto{\pgfqpoint{1.682274in}{1.029657in}}%
\pgfpathlineto{\pgfqpoint{1.682884in}{1.065534in}}%
\pgfpathlineto{\pgfqpoint{1.685630in}{1.188308in}}%
\pgfpathlineto{\pgfqpoint{1.686241in}{1.158118in}}%
\pgfpathlineto{\pgfqpoint{1.688071in}{0.901487in}}%
\pgfpathlineto{\pgfqpoint{1.688987in}{0.965219in}}%
\pgfpathlineto{\pgfqpoint{1.694174in}{1.502466in}}%
\pgfpathlineto{\pgfqpoint{1.694784in}{1.532350in}}%
\pgfpathlineto{\pgfqpoint{1.695089in}{1.491808in}}%
\pgfpathlineto{\pgfqpoint{1.699361in}{0.709129in}}%
\pgfpathlineto{\pgfqpoint{1.699666in}{0.703694in}}%
\pgfpathlineto{\pgfqpoint{1.699971in}{0.717235in}}%
\pgfpathlineto{\pgfqpoint{1.701802in}{0.970245in}}%
\pgfpathlineto{\pgfqpoint{1.704243in}{1.386222in}}%
\pgfpathlineto{\pgfqpoint{1.704853in}{1.333847in}}%
\pgfpathlineto{\pgfqpoint{1.706989in}{1.019387in}}%
\pgfpathlineto{\pgfqpoint{1.707599in}{1.057274in}}%
\pgfpathlineto{\pgfqpoint{1.708209in}{1.087850in}}%
\pgfpathlineto{\pgfqpoint{1.708819in}{1.052100in}}%
\pgfpathlineto{\pgfqpoint{1.710040in}{0.961827in}}%
\pgfpathlineto{\pgfqpoint{1.710650in}{0.978832in}}%
\pgfpathlineto{\pgfqpoint{1.713701in}{1.126662in}}%
\pgfpathlineto{\pgfqpoint{1.716752in}{1.286100in}}%
\pgfpathlineto{\pgfqpoint{1.717363in}{1.249420in}}%
\pgfpathlineto{\pgfqpoint{1.718888in}{1.048587in}}%
\pgfpathlineto{\pgfqpoint{1.719804in}{1.079979in}}%
\pgfpathlineto{\pgfqpoint{1.722244in}{1.135295in}}%
\pgfpathlineto{\pgfqpoint{1.721024in}{1.073019in}}%
\pgfpathlineto{\pgfqpoint{1.722550in}{1.127629in}}%
\pgfpathlineto{\pgfqpoint{1.724380in}{0.928395in}}%
\pgfpathlineto{\pgfqpoint{1.725296in}{0.969558in}}%
\pgfpathlineto{\pgfqpoint{1.726516in}{1.007149in}}%
\pgfpathlineto{\pgfqpoint{1.727126in}{0.994595in}}%
\pgfpathlineto{\pgfqpoint{1.727431in}{0.991302in}}%
\pgfpathlineto{\pgfqpoint{1.727737in}{0.996238in}}%
\pgfpathlineto{\pgfqpoint{1.728957in}{1.041076in}}%
\pgfpathlineto{\pgfqpoint{1.729262in}{1.027143in}}%
\pgfpathlineto{\pgfqpoint{1.730483in}{0.951771in}}%
\pgfpathlineto{\pgfqpoint{1.731093in}{0.988482in}}%
\pgfpathlineto{\pgfqpoint{1.731703in}{1.003710in}}%
\pgfpathlineto{\pgfqpoint{1.732008in}{0.989736in}}%
\pgfpathlineto{\pgfqpoint{1.732924in}{0.948092in}}%
\pgfpathlineto{\pgfqpoint{1.733229in}{0.964080in}}%
\pgfpathlineto{\pgfqpoint{1.735670in}{1.281650in}}%
\pgfpathlineto{\pgfqpoint{1.736890in}{1.223378in}}%
\pgfpathlineto{\pgfqpoint{1.737195in}{1.212730in}}%
\pgfpathlineto{\pgfqpoint{1.737806in}{1.233294in}}%
\pgfpathlineto{\pgfqpoint{1.738416in}{1.260032in}}%
\pgfpathlineto{\pgfqpoint{1.739026in}{1.233733in}}%
\pgfpathlineto{\pgfqpoint{1.742687in}{0.942073in}}%
\pgfpathlineto{\pgfqpoint{1.742993in}{0.952319in}}%
\pgfpathlineto{\pgfqpoint{1.745433in}{1.237505in}}%
\pgfpathlineto{\pgfqpoint{1.746349in}{1.185043in}}%
\pgfpathlineto{\pgfqpoint{1.749095in}{0.895231in}}%
\pgfpathlineto{\pgfqpoint{1.750010in}{0.903308in}}%
\pgfpathlineto{\pgfqpoint{1.750620in}{0.913079in}}%
\pgfpathlineto{\pgfqpoint{1.751536in}{0.991804in}}%
\pgfpathlineto{\pgfqpoint{1.753061in}{1.240873in}}%
\pgfpathlineto{\pgfqpoint{1.753977in}{1.192798in}}%
\pgfpathlineto{\pgfqpoint{1.757028in}{0.904698in}}%
\pgfpathlineto{\pgfqpoint{1.757943in}{0.969061in}}%
\pgfpathlineto{\pgfqpoint{1.759164in}{1.096780in}}%
\pgfpathlineto{\pgfqpoint{1.759774in}{1.070064in}}%
\pgfpathlineto{\pgfqpoint{1.760689in}{1.026363in}}%
\pgfpathlineto{\pgfqpoint{1.761605in}{1.045853in}}%
\pgfpathlineto{\pgfqpoint{1.763130in}{1.085078in}}%
\pgfpathlineto{\pgfqpoint{1.764656in}{1.207358in}}%
\pgfpathlineto{\pgfqpoint{1.765571in}{1.171247in}}%
\pgfpathlineto{\pgfqpoint{1.765876in}{1.163672in}}%
\pgfpathlineto{\pgfqpoint{1.766181in}{1.173099in}}%
\pgfpathlineto{\pgfqpoint{1.768012in}{1.443520in}}%
\pgfpathlineto{\pgfqpoint{1.768928in}{1.368039in}}%
\pgfpathlineto{\pgfqpoint{1.772589in}{0.795034in}}%
\pgfpathlineto{\pgfqpoint{1.773809in}{0.798020in}}%
\pgfpathlineto{\pgfqpoint{1.774115in}{0.796419in}}%
\pgfpathlineto{\pgfqpoint{1.774420in}{0.803037in}}%
\pgfpathlineto{\pgfqpoint{1.775640in}{0.949589in}}%
\pgfpathlineto{\pgfqpoint{1.778386in}{1.167888in}}%
\pgfpathlineto{\pgfqpoint{1.779302in}{1.220863in}}%
\pgfpathlineto{\pgfqpoint{1.779912in}{1.198668in}}%
\pgfpathlineto{\pgfqpoint{1.781132in}{1.107459in}}%
\pgfpathlineto{\pgfqpoint{1.781743in}{1.140342in}}%
\pgfpathlineto{\pgfqpoint{1.782048in}{1.147316in}}%
\pgfpathlineto{\pgfqpoint{1.782353in}{1.133498in}}%
\pgfpathlineto{\pgfqpoint{1.784183in}{0.889620in}}%
\pgfpathlineto{\pgfqpoint{1.784794in}{0.949874in}}%
\pgfpathlineto{\pgfqpoint{1.786624in}{1.188772in}}%
\pgfpathlineto{\pgfqpoint{1.787235in}{1.159723in}}%
\pgfpathlineto{\pgfqpoint{1.787845in}{1.140165in}}%
\pgfpathlineto{\pgfqpoint{1.788455in}{1.163530in}}%
\pgfpathlineto{\pgfqpoint{1.789065in}{1.176423in}}%
\pgfpathlineto{\pgfqpoint{1.789370in}{1.168141in}}%
\pgfpathlineto{\pgfqpoint{1.790896in}{1.063453in}}%
\pgfpathlineto{\pgfqpoint{1.791811in}{1.111584in}}%
\pgfpathlineto{\pgfqpoint{1.793032in}{1.159672in}}%
\pgfpathlineto{\pgfqpoint{1.793642in}{1.139613in}}%
\pgfpathlineto{\pgfqpoint{1.797609in}{0.826836in}}%
\pgfpathlineto{\pgfqpoint{1.798524in}{0.919159in}}%
\pgfpathlineto{\pgfqpoint{1.800660in}{1.312660in}}%
\pgfpathlineto{\pgfqpoint{1.801270in}{1.256933in}}%
\pgfpathlineto{\pgfqpoint{1.805542in}{0.809324in}}%
\pgfpathlineto{\pgfqpoint{1.805847in}{0.824960in}}%
\pgfpathlineto{\pgfqpoint{1.811339in}{1.433303in}}%
\pgfpathlineto{\pgfqpoint{1.812865in}{1.683145in}}%
\pgfpathlineto{\pgfqpoint{1.813475in}{1.642488in}}%
\pgfpathlineto{\pgfqpoint{1.818357in}{0.827939in}}%
\pgfpathlineto{\pgfqpoint{1.818662in}{0.837587in}}%
\pgfpathlineto{\pgfqpoint{1.822933in}{1.023611in}}%
\pgfpathlineto{\pgfqpoint{1.823544in}{1.015848in}}%
\pgfpathlineto{\pgfqpoint{1.824459in}{0.995777in}}%
\pgfpathlineto{\pgfqpoint{1.824764in}{1.005211in}}%
\pgfpathlineto{\pgfqpoint{1.826290in}{1.098508in}}%
\pgfpathlineto{\pgfqpoint{1.826900in}{1.065074in}}%
\pgfpathlineto{\pgfqpoint{1.828120in}{1.001678in}}%
\pgfpathlineto{\pgfqpoint{1.828426in}{1.017416in}}%
\pgfpathlineto{\pgfqpoint{1.830256in}{1.199288in}}%
\pgfpathlineto{\pgfqpoint{1.830867in}{1.134018in}}%
\pgfpathlineto{\pgfqpoint{1.832392in}{0.948256in}}%
\pgfpathlineto{\pgfqpoint{1.833002in}{0.974807in}}%
\pgfpathlineto{\pgfqpoint{1.836054in}{1.237142in}}%
\pgfpathlineto{\pgfqpoint{1.836664in}{1.296160in}}%
\pgfpathlineto{\pgfqpoint{1.837579in}{1.241488in}}%
\pgfpathlineto{\pgfqpoint{1.840935in}{1.026425in}}%
\pgfpathlineto{\pgfqpoint{1.841241in}{1.015987in}}%
\pgfpathlineto{\pgfqpoint{1.841546in}{1.024592in}}%
\pgfpathlineto{\pgfqpoint{1.843987in}{1.325759in}}%
\pgfpathlineto{\pgfqpoint{1.844902in}{1.255142in}}%
\pgfpathlineto{\pgfqpoint{1.847953in}{0.776194in}}%
\pgfpathlineto{\pgfqpoint{1.848563in}{0.801510in}}%
\pgfpathlineto{\pgfqpoint{1.851615in}{1.038871in}}%
\pgfpathlineto{\pgfqpoint{1.852530in}{1.014395in}}%
\pgfpathlineto{\pgfqpoint{1.853445in}{0.989235in}}%
\pgfpathlineto{\pgfqpoint{1.853750in}{0.999936in}}%
\pgfpathlineto{\pgfqpoint{1.855276in}{1.113170in}}%
\pgfpathlineto{\pgfqpoint{1.856191in}{1.089367in}}%
\pgfpathlineto{\pgfqpoint{1.856496in}{1.089452in}}%
\pgfpathlineto{\pgfqpoint{1.857412in}{1.187429in}}%
\pgfpathlineto{\pgfqpoint{1.858327in}{1.302831in}}%
\pgfpathlineto{\pgfqpoint{1.858937in}{1.257484in}}%
\pgfpathlineto{\pgfqpoint{1.859548in}{1.180081in}}%
\pgfpathlineto{\pgfqpoint{1.860158in}{1.218828in}}%
\pgfpathlineto{\pgfqpoint{1.861073in}{1.304364in}}%
\pgfpathlineto{\pgfqpoint{1.861683in}{1.235293in}}%
\pgfpathlineto{\pgfqpoint{1.862294in}{1.158671in}}%
\pgfpathlineto{\pgfqpoint{1.862904in}{1.202802in}}%
\pgfpathlineto{\pgfqpoint{1.863819in}{1.299413in}}%
\pgfpathlineto{\pgfqpoint{1.864430in}{1.211582in}}%
\pgfpathlineto{\pgfqpoint{1.865955in}{0.980717in}}%
\pgfpathlineto{\pgfqpoint{1.866565in}{1.010038in}}%
\pgfpathlineto{\pgfqpoint{1.866870in}{1.017377in}}%
\pgfpathlineto{\pgfqpoint{1.867176in}{1.011230in}}%
\pgfpathlineto{\pgfqpoint{1.868701in}{0.919981in}}%
\pgfpathlineto{\pgfqpoint{1.869311in}{0.962241in}}%
\pgfpathlineto{\pgfqpoint{1.870227in}{1.026245in}}%
\pgfpathlineto{\pgfqpoint{1.870837in}{1.004842in}}%
\pgfpathlineto{\pgfqpoint{1.871752in}{0.931536in}}%
\pgfpathlineto{\pgfqpoint{1.872363in}{0.985233in}}%
\pgfpathlineto{\pgfqpoint{1.873583in}{1.124423in}}%
\pgfpathlineto{\pgfqpoint{1.874193in}{1.085846in}}%
\pgfpathlineto{\pgfqpoint{1.874498in}{1.072674in}}%
\pgfpathlineto{\pgfqpoint{1.875109in}{1.105979in}}%
\pgfpathlineto{\pgfqpoint{1.876329in}{1.186264in}}%
\pgfpathlineto{\pgfqpoint{1.876939in}{1.142828in}}%
\pgfpathlineto{\pgfqpoint{1.880601in}{0.845117in}}%
\pgfpathlineto{\pgfqpoint{1.881211in}{0.871404in}}%
\pgfpathlineto{\pgfqpoint{1.883347in}{1.094443in}}%
\pgfpathlineto{\pgfqpoint{1.885178in}{1.379113in}}%
\pgfpathlineto{\pgfqpoint{1.885788in}{1.363390in}}%
\pgfpathlineto{\pgfqpoint{1.889449in}{1.169849in}}%
\pgfpathlineto{\pgfqpoint{1.891585in}{1.058109in}}%
\pgfpathlineto{\pgfqpoint{1.893111in}{0.932300in}}%
\pgfpathlineto{\pgfqpoint{1.893721in}{0.983294in}}%
\pgfpathlineto{\pgfqpoint{1.894636in}{1.060645in}}%
\pgfpathlineto{\pgfqpoint{1.895246in}{1.021765in}}%
\pgfpathlineto{\pgfqpoint{1.896162in}{0.959052in}}%
\pgfpathlineto{\pgfqpoint{1.896772in}{0.990723in}}%
\pgfpathlineto{\pgfqpoint{1.897382in}{1.024737in}}%
\pgfpathlineto{\pgfqpoint{1.897993in}{1.003112in}}%
\pgfpathlineto{\pgfqpoint{1.899213in}{0.931411in}}%
\pgfpathlineto{\pgfqpoint{1.899823in}{0.970353in}}%
\pgfpathlineto{\pgfqpoint{1.900739in}{1.006081in}}%
\pgfpathlineto{\pgfqpoint{1.901654in}{0.997848in}}%
\pgfpathlineto{\pgfqpoint{1.902264in}{0.994009in}}%
\pgfpathlineto{\pgfqpoint{1.902569in}{0.998992in}}%
\pgfpathlineto{\pgfqpoint{1.906231in}{1.170249in}}%
\pgfpathlineto{\pgfqpoint{1.907756in}{1.340305in}}%
\pgfpathlineto{\pgfqpoint{1.908367in}{1.329627in}}%
\pgfpathlineto{\pgfqpoint{1.910807in}{1.174324in}}%
\pgfpathlineto{\pgfqpoint{1.912943in}{0.980156in}}%
\pgfpathlineto{\pgfqpoint{1.913554in}{0.999293in}}%
\pgfpathlineto{\pgfqpoint{1.914469in}{1.024853in}}%
\pgfpathlineto{\pgfqpoint{1.915079in}{1.007222in}}%
\pgfpathlineto{\pgfqpoint{1.916300in}{0.949994in}}%
\pgfpathlineto{\pgfqpoint{1.916910in}{0.978113in}}%
\pgfpathlineto{\pgfqpoint{1.920876in}{1.191742in}}%
\pgfpathlineto{\pgfqpoint{1.921181in}{1.191403in}}%
\pgfpathlineto{\pgfqpoint{1.922097in}{1.125954in}}%
\pgfpathlineto{\pgfqpoint{1.923622in}{0.910731in}}%
\pgfpathlineto{\pgfqpoint{1.924538in}{0.940828in}}%
\pgfpathlineto{\pgfqpoint{1.925148in}{0.921497in}}%
\pgfpathlineto{\pgfqpoint{1.926063in}{0.874003in}}%
\pgfpathlineto{\pgfqpoint{1.926674in}{0.903410in}}%
\pgfpathlineto{\pgfqpoint{1.927894in}{0.981691in}}%
\pgfpathlineto{\pgfqpoint{1.928504in}{0.959076in}}%
\pgfpathlineto{\pgfqpoint{1.929115in}{0.947342in}}%
\pgfpathlineto{\pgfqpoint{1.929420in}{0.961250in}}%
\pgfpathlineto{\pgfqpoint{1.932776in}{1.361465in}}%
\pgfpathlineto{\pgfqpoint{1.934302in}{1.311386in}}%
\pgfpathlineto{\pgfqpoint{1.935827in}{1.184539in}}%
\pgfpathlineto{\pgfqpoint{1.937048in}{1.227277in}}%
\pgfpathlineto{\pgfqpoint{1.937353in}{1.226356in}}%
\pgfpathlineto{\pgfqpoint{1.938573in}{1.101238in}}%
\pgfpathlineto{\pgfqpoint{1.941014in}{0.962966in}}%
\pgfpathlineto{\pgfqpoint{1.941319in}{0.954046in}}%
\pgfpathlineto{\pgfqpoint{1.941930in}{0.966680in}}%
\pgfpathlineto{\pgfqpoint{1.943150in}{1.020240in}}%
\pgfpathlineto{\pgfqpoint{1.943760in}{1.004126in}}%
\pgfpathlineto{\pgfqpoint{1.944065in}{1.002459in}}%
\pgfpathlineto{\pgfqpoint{1.945286in}{1.045056in}}%
\pgfpathlineto{\pgfqpoint{1.945591in}{1.029308in}}%
\pgfpathlineto{\pgfqpoint{1.947117in}{0.841068in}}%
\pgfpathlineto{\pgfqpoint{1.947727in}{0.891350in}}%
\pgfpathlineto{\pgfqpoint{1.948642in}{0.978396in}}%
\pgfpathlineto{\pgfqpoint{1.949557in}{0.945511in}}%
\pgfpathlineto{\pgfqpoint{1.950168in}{0.993164in}}%
\pgfpathlineto{\pgfqpoint{1.951388in}{1.226234in}}%
\pgfpathlineto{\pgfqpoint{1.952304in}{1.144982in}}%
\pgfpathlineto{\pgfqpoint{1.953219in}{1.027596in}}%
\pgfpathlineto{\pgfqpoint{1.953829in}{1.064450in}}%
\pgfpathlineto{\pgfqpoint{1.955355in}{1.274285in}}%
\pgfpathlineto{\pgfqpoint{1.955965in}{1.227833in}}%
\pgfpathlineto{\pgfqpoint{1.958101in}{1.108251in}}%
\pgfpathlineto{\pgfqpoint{1.959931in}{0.958363in}}%
\pgfpathlineto{\pgfqpoint{1.960542in}{1.021481in}}%
\pgfpathlineto{\pgfqpoint{1.961762in}{1.156338in}}%
\pgfpathlineto{\pgfqpoint{1.962372in}{1.107121in}}%
\pgfpathlineto{\pgfqpoint{1.963288in}{1.027438in}}%
\pgfpathlineto{\pgfqpoint{1.964203in}{1.069288in}}%
\pgfpathlineto{\pgfqpoint{1.964813in}{1.088393in}}%
\pgfpathlineto{\pgfqpoint{1.965424in}{1.063384in}}%
\pgfpathlineto{\pgfqpoint{1.966034in}{1.031810in}}%
\pgfpathlineto{\pgfqpoint{1.966644in}{1.049176in}}%
\pgfpathlineto{\pgfqpoint{1.968170in}{1.148794in}}%
\pgfpathlineto{\pgfqpoint{1.969085in}{1.139346in}}%
\pgfpathlineto{\pgfqpoint{1.969390in}{1.140893in}}%
\pgfpathlineto{\pgfqpoint{1.969695in}{1.137958in}}%
\pgfpathlineto{\pgfqpoint{1.970916in}{1.015063in}}%
\pgfpathlineto{\pgfqpoint{1.973357in}{0.878523in}}%
\pgfpathlineto{\pgfqpoint{1.974272in}{0.973704in}}%
\pgfpathlineto{\pgfqpoint{1.975187in}{1.071217in}}%
\pgfpathlineto{\pgfqpoint{1.976103in}{1.028540in}}%
\pgfpathlineto{\pgfqpoint{1.976713in}{1.051478in}}%
\pgfpathlineto{\pgfqpoint{1.978239in}{1.142908in}}%
\pgfpathlineto{\pgfqpoint{1.978849in}{1.128880in}}%
\pgfpathlineto{\pgfqpoint{1.979154in}{1.127262in}}%
\pgfpathlineto{\pgfqpoint{1.980069in}{1.195520in}}%
\pgfpathlineto{\pgfqpoint{1.980985in}{1.256441in}}%
\pgfpathlineto{\pgfqpoint{1.981595in}{1.234199in}}%
\pgfpathlineto{\pgfqpoint{1.982205in}{1.216770in}}%
\pgfpathlineto{\pgfqpoint{1.982510in}{1.230493in}}%
\pgfpathlineto{\pgfqpoint{1.983731in}{1.316723in}}%
\pgfpathlineto{\pgfqpoint{1.984341in}{1.259192in}}%
\pgfpathlineto{\pgfqpoint{1.987392in}{0.946219in}}%
\pgfpathlineto{\pgfqpoint{1.988002in}{0.973804in}}%
\pgfpathlineto{\pgfqpoint{1.988918in}{1.009576in}}%
\pgfpathlineto{\pgfqpoint{1.989833in}{0.998900in}}%
\pgfpathlineto{\pgfqpoint{1.990443in}{0.994185in}}%
\pgfpathlineto{\pgfqpoint{1.990748in}{0.998568in}}%
\pgfpathlineto{\pgfqpoint{1.992579in}{1.053754in}}%
\pgfpathlineto{\pgfqpoint{1.993189in}{1.035996in}}%
\pgfpathlineto{\pgfqpoint{1.994715in}{0.927178in}}%
\pgfpathlineto{\pgfqpoint{1.995325in}{0.951649in}}%
\pgfpathlineto{\pgfqpoint{1.997156in}{1.049921in}}%
\pgfpathlineto{\pgfqpoint{1.998681in}{1.374969in}}%
\pgfpathlineto{\pgfqpoint{1.999292in}{1.223151in}}%
\pgfpathlineto{\pgfqpoint{2.000817in}{0.864997in}}%
\pgfpathlineto{\pgfqpoint{2.001428in}{0.878208in}}%
\pgfpathlineto{\pgfqpoint{2.003869in}{1.144047in}}%
\pgfpathlineto{\pgfqpoint{2.005089in}{1.314595in}}%
\pgfpathlineto{\pgfqpoint{2.005699in}{1.245284in}}%
\pgfpathlineto{\pgfqpoint{2.006615in}{1.151613in}}%
\pgfpathlineto{\pgfqpoint{2.007225in}{1.203878in}}%
\pgfpathlineto{\pgfqpoint{2.007835in}{1.242941in}}%
\pgfpathlineto{\pgfqpoint{2.008445in}{1.220051in}}%
\pgfpathlineto{\pgfqpoint{2.009971in}{1.038612in}}%
\pgfpathlineto{\pgfqpoint{2.010581in}{1.086103in}}%
\pgfpathlineto{\pgfqpoint{2.011496in}{1.188703in}}%
\pgfpathlineto{\pgfqpoint{2.012107in}{1.109426in}}%
\pgfpathlineto{\pgfqpoint{2.013632in}{0.845194in}}%
\pgfpathlineto{\pgfqpoint{2.014243in}{0.863199in}}%
\pgfpathlineto{\pgfqpoint{2.015768in}{0.899210in}}%
\pgfpathlineto{\pgfqpoint{2.017599in}{1.344612in}}%
\pgfpathlineto{\pgfqpoint{2.018514in}{1.107119in}}%
\pgfpathlineto{\pgfqpoint{2.019735in}{0.880249in}}%
\pgfpathlineto{\pgfqpoint{2.020650in}{0.911370in}}%
\pgfpathlineto{\pgfqpoint{2.023701in}{1.075174in}}%
\pgfpathlineto{\pgfqpoint{2.024311in}{1.055205in}}%
\pgfpathlineto{\pgfqpoint{2.025227in}{0.996537in}}%
\pgfpathlineto{\pgfqpoint{2.025532in}{1.018344in}}%
\pgfpathlineto{\pgfqpoint{2.027363in}{1.337844in}}%
\pgfpathlineto{\pgfqpoint{2.028278in}{1.286893in}}%
\pgfpathlineto{\pgfqpoint{2.028888in}{1.272075in}}%
\pgfpathlineto{\pgfqpoint{2.029498in}{1.285477in}}%
\pgfpathlineto{\pgfqpoint{2.029804in}{1.289538in}}%
\pgfpathlineto{\pgfqpoint{2.030414in}{1.278451in}}%
\pgfpathlineto{\pgfqpoint{2.033160in}{1.061939in}}%
\pgfpathlineto{\pgfqpoint{2.034685in}{0.838361in}}%
\pgfpathlineto{\pgfqpoint{2.035296in}{0.882043in}}%
\pgfpathlineto{\pgfqpoint{2.036516in}{1.012630in}}%
\pgfpathlineto{\pgfqpoint{2.037126in}{0.975660in}}%
\pgfpathlineto{\pgfqpoint{2.037431in}{0.961223in}}%
\pgfpathlineto{\pgfqpoint{2.038042in}{0.978488in}}%
\pgfpathlineto{\pgfqpoint{2.041093in}{1.251019in}}%
\pgfpathlineto{\pgfqpoint{2.042008in}{1.389293in}}%
\pgfpathlineto{\pgfqpoint{2.042313in}{1.332434in}}%
\pgfpathlineto{\pgfqpoint{2.044449in}{0.805351in}}%
\pgfpathlineto{\pgfqpoint{2.045059in}{0.840300in}}%
\pgfpathlineto{\pgfqpoint{2.048416in}{1.047795in}}%
\pgfpathlineto{\pgfqpoint{2.048721in}{1.045825in}}%
\pgfpathlineto{\pgfqpoint{2.049636in}{0.997928in}}%
\pgfpathlineto{\pgfqpoint{2.050246in}{1.029766in}}%
\pgfpathlineto{\pgfqpoint{2.051772in}{1.335349in}}%
\pgfpathlineto{\pgfqpoint{2.052687in}{1.242613in}}%
\pgfpathlineto{\pgfqpoint{2.053298in}{1.199229in}}%
\pgfpathlineto{\pgfqpoint{2.053908in}{1.227098in}}%
\pgfpathlineto{\pgfqpoint{2.054823in}{1.279207in}}%
\pgfpathlineto{\pgfqpoint{2.055433in}{1.257382in}}%
\pgfpathlineto{\pgfqpoint{2.058180in}{0.937107in}}%
\pgfpathlineto{\pgfqpoint{2.059095in}{0.868670in}}%
\pgfpathlineto{\pgfqpoint{2.059705in}{0.917637in}}%
\pgfpathlineto{\pgfqpoint{2.062451in}{1.082924in}}%
\pgfpathlineto{\pgfqpoint{2.062756in}{1.082595in}}%
\pgfpathlineto{\pgfqpoint{2.063977in}{1.043618in}}%
\pgfpathlineto{\pgfqpoint{2.064587in}{1.068381in}}%
\pgfpathlineto{\pgfqpoint{2.065807in}{1.158565in}}%
\pgfpathlineto{\pgfqpoint{2.066418in}{1.132666in}}%
\pgfpathlineto{\pgfqpoint{2.068554in}{0.834382in}}%
\pgfpathlineto{\pgfqpoint{2.069469in}{0.877091in}}%
\pgfpathlineto{\pgfqpoint{2.070689in}{1.075539in}}%
\pgfpathlineto{\pgfqpoint{2.072215in}{1.388447in}}%
\pgfpathlineto{\pgfqpoint{2.072825in}{1.336404in}}%
\pgfpathlineto{\pgfqpoint{2.074046in}{1.134425in}}%
\pgfpathlineto{\pgfqpoint{2.074961in}{1.185439in}}%
\pgfpathlineto{\pgfqpoint{2.075571in}{1.204503in}}%
\pgfpathlineto{\pgfqpoint{2.075876in}{1.186429in}}%
\pgfpathlineto{\pgfqpoint{2.079843in}{0.908328in}}%
\pgfpathlineto{\pgfqpoint{2.080453in}{0.930958in}}%
\pgfpathlineto{\pgfqpoint{2.084420in}{1.185426in}}%
\pgfpathlineto{\pgfqpoint{2.084725in}{1.183172in}}%
\pgfpathlineto{\pgfqpoint{2.088691in}{0.977222in}}%
\pgfpathlineto{\pgfqpoint{2.089302in}{1.021962in}}%
\pgfpathlineto{\pgfqpoint{2.090217in}{1.073295in}}%
\pgfpathlineto{\pgfqpoint{2.090522in}{1.054963in}}%
\pgfpathlineto{\pgfqpoint{2.092048in}{0.886688in}}%
\pgfpathlineto{\pgfqpoint{2.092963in}{0.936427in}}%
\pgfpathlineto{\pgfqpoint{2.095404in}{1.119067in}}%
\pgfpathlineto{\pgfqpoint{2.096624in}{1.403384in}}%
\pgfpathlineto{\pgfqpoint{2.097235in}{1.301210in}}%
\pgfpathlineto{\pgfqpoint{2.100286in}{1.001993in}}%
\pgfpathlineto{\pgfqpoint{2.100896in}{1.051462in}}%
\pgfpathlineto{\pgfqpoint{2.101811in}{1.170062in}}%
\pgfpathlineto{\pgfqpoint{2.102422in}{1.122548in}}%
\pgfpathlineto{\pgfqpoint{2.103337in}{1.008698in}}%
\pgfpathlineto{\pgfqpoint{2.103947in}{1.047552in}}%
\pgfpathlineto{\pgfqpoint{2.104863in}{1.093080in}}%
\pgfpathlineto{\pgfqpoint{2.105473in}{1.064705in}}%
\pgfpathlineto{\pgfqpoint{2.106083in}{1.032202in}}%
\pgfpathlineto{\pgfqpoint{2.106693in}{1.056638in}}%
\pgfpathlineto{\pgfqpoint{2.107609in}{1.128927in}}%
\pgfpathlineto{\pgfqpoint{2.108219in}{1.094062in}}%
\pgfpathlineto{\pgfqpoint{2.109439in}{0.958756in}}%
\pgfpathlineto{\pgfqpoint{2.110355in}{0.987367in}}%
\pgfpathlineto{\pgfqpoint{2.111270in}{0.900211in}}%
\pgfpathlineto{\pgfqpoint{2.111880in}{0.848777in}}%
\pgfpathlineto{\pgfqpoint{2.112491in}{0.888277in}}%
\pgfpathlineto{\pgfqpoint{2.114016in}{1.084483in}}%
\pgfpathlineto{\pgfqpoint{2.114931in}{1.044945in}}%
\pgfpathlineto{\pgfqpoint{2.115237in}{1.049621in}}%
\pgfpathlineto{\pgfqpoint{2.116457in}{1.160703in}}%
\pgfpathlineto{\pgfqpoint{2.117067in}{1.102631in}}%
\pgfpathlineto{\pgfqpoint{2.117983in}{1.006049in}}%
\pgfpathlineto{\pgfqpoint{2.118288in}{1.043749in}}%
\pgfpathlineto{\pgfqpoint{2.119813in}{1.341262in}}%
\pgfpathlineto{\pgfqpoint{2.120424in}{1.261369in}}%
\pgfpathlineto{\pgfqpoint{2.122865in}{1.012906in}}%
\pgfpathlineto{\pgfqpoint{2.124085in}{0.910035in}}%
\pgfpathlineto{\pgfqpoint{2.124695in}{0.944822in}}%
\pgfpathlineto{\pgfqpoint{2.125916in}{1.008270in}}%
\pgfpathlineto{\pgfqpoint{2.126526in}{0.997871in}}%
\pgfpathlineto{\pgfqpoint{2.126831in}{0.999169in}}%
\pgfpathlineto{\pgfqpoint{2.127746in}{1.073449in}}%
\pgfpathlineto{\pgfqpoint{2.130187in}{1.217994in}}%
\pgfpathlineto{\pgfqpoint{2.130798in}{1.247251in}}%
\pgfpathlineto{\pgfqpoint{2.131103in}{1.226351in}}%
\pgfpathlineto{\pgfqpoint{2.132628in}{0.956065in}}%
\pgfpathlineto{\pgfqpoint{2.133544in}{1.041422in}}%
\pgfpathlineto{\pgfqpoint{2.134154in}{1.086746in}}%
\pgfpathlineto{\pgfqpoint{2.134764in}{1.048242in}}%
\pgfpathlineto{\pgfqpoint{2.135680in}{0.957832in}}%
\pgfpathlineto{\pgfqpoint{2.136290in}{0.988677in}}%
\pgfpathlineto{\pgfqpoint{2.138120in}{1.130691in}}%
\pgfpathlineto{\pgfqpoint{2.138731in}{1.109233in}}%
\pgfpathlineto{\pgfqpoint{2.141782in}{0.938762in}}%
\pgfpathlineto{\pgfqpoint{2.142087in}{0.929605in}}%
\pgfpathlineto{\pgfqpoint{2.142392in}{0.939098in}}%
\pgfpathlineto{\pgfqpoint{2.144223in}{1.186414in}}%
\pgfpathlineto{\pgfqpoint{2.145748in}{1.117089in}}%
\pgfpathlineto{\pgfqpoint{2.147274in}{0.963954in}}%
\pgfpathlineto{\pgfqpoint{2.147884in}{0.995936in}}%
\pgfpathlineto{\pgfqpoint{2.149410in}{1.230372in}}%
\pgfpathlineto{\pgfqpoint{2.150020in}{1.150972in}}%
\pgfpathlineto{\pgfqpoint{2.150935in}{1.038995in}}%
\pgfpathlineto{\pgfqpoint{2.151851in}{1.087400in}}%
\pgfpathlineto{\pgfqpoint{2.152156in}{1.088153in}}%
\pgfpathlineto{\pgfqpoint{2.153071in}{1.026484in}}%
\pgfpathlineto{\pgfqpoint{2.154292in}{0.914899in}}%
\pgfpathlineto{\pgfqpoint{2.155207in}{0.949108in}}%
\pgfpathlineto{\pgfqpoint{2.155512in}{0.957825in}}%
\pgfpathlineto{\pgfqpoint{2.156122in}{0.947649in}}%
\pgfpathlineto{\pgfqpoint{2.157343in}{0.863695in}}%
\pgfpathlineto{\pgfqpoint{2.157953in}{0.908690in}}%
\pgfpathlineto{\pgfqpoint{2.159479in}{1.093376in}}%
\pgfpathlineto{\pgfqpoint{2.160089in}{1.052403in}}%
\pgfpathlineto{\pgfqpoint{2.160699in}{1.029267in}}%
\pgfpathlineto{\pgfqpoint{2.161004in}{1.059446in}}%
\pgfpathlineto{\pgfqpoint{2.162225in}{1.257703in}}%
\pgfpathlineto{\pgfqpoint{2.163140in}{1.209065in}}%
\pgfpathlineto{\pgfqpoint{2.163445in}{1.208696in}}%
\pgfpathlineto{\pgfqpoint{2.163750in}{1.211392in}}%
\pgfpathlineto{\pgfqpoint{2.164056in}{1.203771in}}%
\pgfpathlineto{\pgfqpoint{2.165886in}{0.996274in}}%
\pgfpathlineto{\pgfqpoint{2.167107in}{1.043241in}}%
\pgfpathlineto{\pgfqpoint{2.170158in}{1.108095in}}%
\pgfpathlineto{\pgfqpoint{2.170463in}{1.108189in}}%
\pgfpathlineto{\pgfqpoint{2.171378in}{1.002225in}}%
\pgfpathlineto{\pgfqpoint{2.172294in}{0.883005in}}%
\pgfpathlineto{\pgfqpoint{2.172904in}{0.947288in}}%
\pgfpathlineto{\pgfqpoint{2.174124in}{1.116214in}}%
\pgfpathlineto{\pgfqpoint{2.174735in}{1.052275in}}%
\pgfpathlineto{\pgfqpoint{2.175345in}{1.007008in}}%
\pgfpathlineto{\pgfqpoint{2.176260in}{1.022641in}}%
\pgfpathlineto{\pgfqpoint{2.176565in}{1.024937in}}%
\pgfpathlineto{\pgfqpoint{2.176870in}{1.021849in}}%
\pgfpathlineto{\pgfqpoint{2.178091in}{0.968385in}}%
\pgfpathlineto{\pgfqpoint{2.178701in}{0.996527in}}%
\pgfpathlineto{\pgfqpoint{2.180227in}{1.143244in}}%
\pgfpathlineto{\pgfqpoint{2.180837in}{1.107685in}}%
\pgfpathlineto{\pgfqpoint{2.181142in}{1.102549in}}%
\pgfpathlineto{\pgfqpoint{2.182363in}{1.295549in}}%
\pgfpathlineto{\pgfqpoint{2.182668in}{1.325921in}}%
\pgfpathlineto{\pgfqpoint{2.183278in}{1.273787in}}%
\pgfpathlineto{\pgfqpoint{2.186634in}{0.900779in}}%
\pgfpathlineto{\pgfqpoint{2.186939in}{0.888602in}}%
\pgfpathlineto{\pgfqpoint{2.187550in}{0.902994in}}%
\pgfpathlineto{\pgfqpoint{2.189685in}{1.070122in}}%
\pgfpathlineto{\pgfqpoint{2.190906in}{1.046640in}}%
\pgfpathlineto{\pgfqpoint{2.192126in}{1.165311in}}%
\pgfpathlineto{\pgfqpoint{2.193652in}{1.273935in}}%
\pgfpathlineto{\pgfqpoint{2.194262in}{1.255969in}}%
\pgfpathlineto{\pgfqpoint{2.199754in}{0.831348in}}%
\pgfpathlineto{\pgfqpoint{2.200365in}{0.868275in}}%
\pgfpathlineto{\pgfqpoint{2.203111in}{1.152978in}}%
\pgfpathlineto{\pgfqpoint{2.203721in}{1.285814in}}%
\pgfpathlineto{\pgfqpoint{2.204636in}{1.156534in}}%
\pgfpathlineto{\pgfqpoint{2.205552in}{0.997817in}}%
\pgfpathlineto{\pgfqpoint{2.206162in}{1.052295in}}%
\pgfpathlineto{\pgfqpoint{2.207382in}{1.161616in}}%
\pgfpathlineto{\pgfqpoint{2.207993in}{1.112806in}}%
\pgfpathlineto{\pgfqpoint{2.208908in}{1.021786in}}%
\pgfpathlineto{\pgfqpoint{2.209823in}{1.061660in}}%
\pgfpathlineto{\pgfqpoint{2.210128in}{1.070489in}}%
\pgfpathlineto{\pgfqpoint{2.210739in}{1.051017in}}%
\pgfpathlineto{\pgfqpoint{2.211349in}{1.028977in}}%
\pgfpathlineto{\pgfqpoint{2.211654in}{1.046420in}}%
\pgfpathlineto{\pgfqpoint{2.212874in}{1.274062in}}%
\pgfpathlineto{\pgfqpoint{2.213790in}{1.144058in}}%
\pgfpathlineto{\pgfqpoint{2.214400in}{1.049908in}}%
\pgfpathlineto{\pgfqpoint{2.215315in}{1.111609in}}%
\pgfpathlineto{\pgfqpoint{2.215926in}{1.159339in}}%
\pgfpathlineto{\pgfqpoint{2.216536in}{1.125333in}}%
\pgfpathlineto{\pgfqpoint{2.219892in}{0.821689in}}%
\pgfpathlineto{\pgfqpoint{2.220502in}{0.776971in}}%
\pgfpathlineto{\pgfqpoint{2.221113in}{0.846852in}}%
\pgfpathlineto{\pgfqpoint{2.222333in}{1.033502in}}%
\pgfpathlineto{\pgfqpoint{2.222943in}{0.995980in}}%
\pgfpathlineto{\pgfqpoint{2.223248in}{0.978844in}}%
\pgfpathlineto{\pgfqpoint{2.223859in}{1.018489in}}%
\pgfpathlineto{\pgfqpoint{2.225994in}{1.393440in}}%
\pgfpathlineto{\pgfqpoint{2.226605in}{1.362700in}}%
\pgfpathlineto{\pgfqpoint{2.229656in}{1.017240in}}%
\pgfpathlineto{\pgfqpoint{2.230571in}{1.073040in}}%
\pgfpathlineto{\pgfqpoint{2.231181in}{1.098818in}}%
\pgfpathlineto{\pgfqpoint{2.231487in}{1.080800in}}%
\pgfpathlineto{\pgfqpoint{2.232707in}{0.929549in}}%
\pgfpathlineto{\pgfqpoint{2.233317in}{0.967595in}}%
\pgfpathlineto{\pgfqpoint{2.234538in}{1.068493in}}%
\pgfpathlineto{\pgfqpoint{2.235148in}{1.014589in}}%
\pgfpathlineto{\pgfqpoint{2.235758in}{0.973282in}}%
\pgfpathlineto{\pgfqpoint{2.236063in}{1.010900in}}%
\pgfpathlineto{\pgfqpoint{2.237284in}{1.249757in}}%
\pgfpathlineto{\pgfqpoint{2.237894in}{1.205567in}}%
\pgfpathlineto{\pgfqpoint{2.239115in}{1.001332in}}%
\pgfpathlineto{\pgfqpoint{2.239725in}{1.099847in}}%
\pgfpathlineto{\pgfqpoint{2.240030in}{1.133225in}}%
\pgfpathlineto{\pgfqpoint{2.240640in}{1.098769in}}%
\pgfpathlineto{\pgfqpoint{2.242166in}{0.849045in}}%
\pgfpathlineto{\pgfqpoint{2.242776in}{0.913602in}}%
\pgfpathlineto{\pgfqpoint{2.243691in}{1.001941in}}%
\pgfpathlineto{\pgfqpoint{2.244607in}{0.983483in}}%
\pgfpathlineto{\pgfqpoint{2.244912in}{0.982569in}}%
\pgfpathlineto{\pgfqpoint{2.246132in}{1.050133in}}%
\pgfpathlineto{\pgfqpoint{2.247048in}{1.007941in}}%
\pgfpathlineto{\pgfqpoint{2.247658in}{0.975356in}}%
\pgfpathlineto{\pgfqpoint{2.248268in}{0.996520in}}%
\pgfpathlineto{\pgfqpoint{2.251319in}{1.184286in}}%
\pgfpathlineto{\pgfqpoint{2.251624in}{1.181307in}}%
\pgfpathlineto{\pgfqpoint{2.253455in}{0.933472in}}%
\pgfpathlineto{\pgfqpoint{2.254676in}{1.034808in}}%
\pgfpathlineto{\pgfqpoint{2.258337in}{1.306701in}}%
\pgfpathlineto{\pgfqpoint{2.258642in}{1.309206in}}%
\pgfpathlineto{\pgfqpoint{2.258947in}{1.300652in}}%
\pgfpathlineto{\pgfqpoint{2.265965in}{0.848212in}}%
\pgfpathlineto{\pgfqpoint{2.266270in}{0.861534in}}%
\pgfpathlineto{\pgfqpoint{2.269626in}{1.198927in}}%
\pgfpathlineto{\pgfqpoint{2.270237in}{1.338592in}}%
\pgfpathlineto{\pgfqpoint{2.270847in}{1.267566in}}%
\pgfpathlineto{\pgfqpoint{2.272067in}{0.930267in}}%
\pgfpathlineto{\pgfqpoint{2.272678in}{1.021059in}}%
\pgfpathlineto{\pgfqpoint{2.273593in}{1.115922in}}%
\pgfpathlineto{\pgfqpoint{2.274203in}{1.033667in}}%
\pgfpathlineto{\pgfqpoint{2.275119in}{0.919086in}}%
\pgfpathlineto{\pgfqpoint{2.275729in}{1.022435in}}%
\pgfpathlineto{\pgfqpoint{2.276339in}{1.099056in}}%
\pgfpathlineto{\pgfqpoint{2.276949in}{1.052744in}}%
\pgfpathlineto{\pgfqpoint{2.277865in}{0.944981in}}%
\pgfpathlineto{\pgfqpoint{2.278475in}{1.022982in}}%
\pgfpathlineto{\pgfqpoint{2.279695in}{1.201461in}}%
\pgfpathlineto{\pgfqpoint{2.280611in}{1.168885in}}%
\pgfpathlineto{\pgfqpoint{2.282136in}{1.343135in}}%
\pgfpathlineto{\pgfqpoint{2.282746in}{1.237425in}}%
\pgfpathlineto{\pgfqpoint{2.283662in}{1.116333in}}%
\pgfpathlineto{\pgfqpoint{2.284577in}{1.136807in}}%
\pgfpathlineto{\pgfqpoint{2.285493in}{1.028829in}}%
\pgfpathlineto{\pgfqpoint{2.286408in}{0.907248in}}%
\pgfpathlineto{\pgfqpoint{2.287018in}{0.969013in}}%
\pgfpathlineto{\pgfqpoint{2.287933in}{1.099974in}}%
\pgfpathlineto{\pgfqpoint{2.288544in}{1.068327in}}%
\pgfpathlineto{\pgfqpoint{2.289764in}{0.900098in}}%
\pgfpathlineto{\pgfqpoint{2.290374in}{0.972836in}}%
\pgfpathlineto{\pgfqpoint{2.291290in}{1.048667in}}%
\pgfpathlineto{\pgfqpoint{2.291900in}{0.986234in}}%
\pgfpathlineto{\pgfqpoint{2.292510in}{0.908358in}}%
\pgfpathlineto{\pgfqpoint{2.293120in}{0.958284in}}%
\pgfpathlineto{\pgfqpoint{2.294036in}{1.127232in}}%
\pgfpathlineto{\pgfqpoint{2.294646in}{1.084708in}}%
\pgfpathlineto{\pgfqpoint{2.295867in}{0.888384in}}%
\pgfpathlineto{\pgfqpoint{2.296782in}{0.963440in}}%
\pgfpathlineto{\pgfqpoint{2.297392in}{0.984635in}}%
\pgfpathlineto{\pgfqpoint{2.298002in}{0.942330in}}%
\pgfpathlineto{\pgfqpoint{2.298307in}{0.923846in}}%
\pgfpathlineto{\pgfqpoint{2.298918in}{0.958557in}}%
\pgfpathlineto{\pgfqpoint{2.302884in}{1.475681in}}%
\pgfpathlineto{\pgfqpoint{2.303800in}{1.357986in}}%
\pgfpathlineto{\pgfqpoint{2.305630in}{1.112459in}}%
\pgfpathlineto{\pgfqpoint{2.306241in}{1.147362in}}%
\pgfpathlineto{\pgfqpoint{2.306546in}{1.161216in}}%
\pgfpathlineto{\pgfqpoint{2.307156in}{1.121155in}}%
\pgfpathlineto{\pgfqpoint{2.308071in}{0.971134in}}%
\pgfpathlineto{\pgfqpoint{2.308987in}{1.052706in}}%
\pgfpathlineto{\pgfqpoint{2.309597in}{1.086963in}}%
\pgfpathlineto{\pgfqpoint{2.310207in}{1.026474in}}%
\pgfpathlineto{\pgfqpoint{2.311122in}{0.919867in}}%
\pgfpathlineto{\pgfqpoint{2.312038in}{0.937053in}}%
\pgfpathlineto{\pgfqpoint{2.313563in}{0.907719in}}%
\pgfpathlineto{\pgfqpoint{2.314479in}{0.924427in}}%
\pgfpathlineto{\pgfqpoint{2.316309in}{1.168361in}}%
\pgfpathlineto{\pgfqpoint{2.317835in}{1.107197in}}%
\pgfpathlineto{\pgfqpoint{2.319666in}{0.897768in}}%
\pgfpathlineto{\pgfqpoint{2.320581in}{0.953272in}}%
\pgfpathlineto{\pgfqpoint{2.321496in}{1.020574in}}%
\pgfpathlineto{\pgfqpoint{2.322412in}{0.985247in}}%
\pgfpathlineto{\pgfqpoint{2.323022in}{1.030797in}}%
\pgfpathlineto{\pgfqpoint{2.324548in}{1.283141in}}%
\pgfpathlineto{\pgfqpoint{2.325158in}{1.227475in}}%
\pgfpathlineto{\pgfqpoint{2.326378in}{0.959239in}}%
\pgfpathlineto{\pgfqpoint{2.327294in}{1.034567in}}%
\pgfpathlineto{\pgfqpoint{2.327599in}{1.039534in}}%
\pgfpathlineto{\pgfqpoint{2.327904in}{1.023775in}}%
\pgfpathlineto{\pgfqpoint{2.329124in}{0.856635in}}%
\pgfpathlineto{\pgfqpoint{2.329735in}{0.929139in}}%
\pgfpathlineto{\pgfqpoint{2.331260in}{1.298493in}}%
\pgfpathlineto{\pgfqpoint{2.331870in}{1.240966in}}%
\pgfpathlineto{\pgfqpoint{2.332481in}{1.163285in}}%
\pgfpathlineto{\pgfqpoint{2.333091in}{1.225483in}}%
\pgfpathlineto{\pgfqpoint{2.334006in}{1.402580in}}%
\pgfpathlineto{\pgfqpoint{2.334617in}{1.320730in}}%
\pgfpathlineto{\pgfqpoint{2.337363in}{0.864932in}}%
\pgfpathlineto{\pgfqpoint{2.338278in}{0.831592in}}%
\pgfpathlineto{\pgfqpoint{2.338583in}{0.844580in}}%
\pgfpathlineto{\pgfqpoint{2.340109in}{1.136046in}}%
\pgfpathlineto{\pgfqpoint{2.341024in}{1.035259in}}%
\pgfpathlineto{\pgfqpoint{2.343770in}{0.843200in}}%
\pgfpathlineto{\pgfqpoint{2.344075in}{0.837907in}}%
\pgfpathlineto{\pgfqpoint{2.344380in}{0.856170in}}%
\pgfpathlineto{\pgfqpoint{2.348957in}{1.279051in}}%
\pgfpathlineto{\pgfqpoint{2.349872in}{1.202651in}}%
\pgfpathlineto{\pgfqpoint{2.350788in}{1.128815in}}%
\pgfpathlineto{\pgfqpoint{2.351093in}{1.158619in}}%
\pgfpathlineto{\pgfqpoint{2.352313in}{1.285931in}}%
\pgfpathlineto{\pgfqpoint{2.352924in}{1.239619in}}%
\pgfpathlineto{\pgfqpoint{2.353534in}{1.205475in}}%
\pgfpathlineto{\pgfqpoint{2.354144in}{1.243028in}}%
\pgfpathlineto{\pgfqpoint{2.354754in}{1.269230in}}%
\pgfpathlineto{\pgfqpoint{2.355059in}{1.235529in}}%
\pgfpathlineto{\pgfqpoint{2.356280in}{0.966786in}}%
\pgfpathlineto{\pgfqpoint{2.357195in}{1.057871in}}%
\pgfpathlineto{\pgfqpoint{2.358111in}{1.154278in}}%
\pgfpathlineto{\pgfqpoint{2.358416in}{1.120351in}}%
\pgfpathlineto{\pgfqpoint{2.359636in}{0.914219in}}%
\pgfpathlineto{\pgfqpoint{2.360552in}{0.965153in}}%
\pgfpathlineto{\pgfqpoint{2.361162in}{0.922223in}}%
\pgfpathlineto{\pgfqpoint{2.362382in}{0.719493in}}%
\pgfpathlineto{\pgfqpoint{2.362993in}{0.804208in}}%
\pgfpathlineto{\pgfqpoint{2.364823in}{1.141357in}}%
\pgfpathlineto{\pgfqpoint{2.365433in}{1.133301in}}%
\pgfpathlineto{\pgfqpoint{2.366654in}{1.217698in}}%
\pgfpathlineto{\pgfqpoint{2.367264in}{1.155804in}}%
\pgfpathlineto{\pgfqpoint{2.369400in}{0.827972in}}%
\pgfpathlineto{\pgfqpoint{2.370315in}{0.839506in}}%
\pgfpathlineto{\pgfqpoint{2.371231in}{0.913990in}}%
\pgfpathlineto{\pgfqpoint{2.373367in}{1.240457in}}%
\pgfpathlineto{\pgfqpoint{2.374282in}{1.227832in}}%
\pgfpathlineto{\pgfqpoint{2.375807in}{1.350253in}}%
\pgfpathlineto{\pgfqpoint{2.376418in}{1.297744in}}%
\pgfpathlineto{\pgfqpoint{2.377333in}{1.114411in}}%
\pgfpathlineto{\pgfqpoint{2.378248in}{1.227970in}}%
\pgfpathlineto{\pgfqpoint{2.378859in}{1.285991in}}%
\pgfpathlineto{\pgfqpoint{2.379469in}{1.214334in}}%
\pgfpathlineto{\pgfqpoint{2.380689in}{0.951318in}}%
\pgfpathlineto{\pgfqpoint{2.381605in}{1.066175in}}%
\pgfpathlineto{\pgfqpoint{2.382520in}{1.146235in}}%
\pgfpathlineto{\pgfqpoint{2.383130in}{1.070908in}}%
\pgfpathlineto{\pgfqpoint{2.384046in}{0.949593in}}%
\pgfpathlineto{\pgfqpoint{2.384961in}{0.986662in}}%
\pgfpathlineto{\pgfqpoint{2.385876in}{0.909741in}}%
\pgfpathlineto{\pgfqpoint{2.387097in}{0.797313in}}%
\pgfpathlineto{\pgfqpoint{2.387707in}{0.846178in}}%
\pgfpathlineto{\pgfqpoint{2.389233in}{1.001186in}}%
\pgfpathlineto{\pgfqpoint{2.390148in}{0.994002in}}%
\pgfpathlineto{\pgfqpoint{2.390453in}{0.992380in}}%
\pgfpathlineto{\pgfqpoint{2.391063in}{0.996674in}}%
\pgfpathlineto{\pgfqpoint{2.391369in}{0.999275in}}%
\pgfpathlineto{\pgfqpoint{2.391674in}{0.997051in}}%
\pgfpathlineto{\pgfqpoint{2.392589in}{0.966396in}}%
\pgfpathlineto{\pgfqpoint{2.393199in}{0.983081in}}%
\pgfpathlineto{\pgfqpoint{2.394115in}{1.048675in}}%
\pgfpathlineto{\pgfqpoint{2.394725in}{1.019234in}}%
\pgfpathlineto{\pgfqpoint{2.395335in}{0.963623in}}%
\pgfpathlineto{\pgfqpoint{2.395945in}{1.012465in}}%
\pgfpathlineto{\pgfqpoint{2.397471in}{1.276003in}}%
\pgfpathlineto{\pgfqpoint{2.398386in}{1.230965in}}%
\pgfpathlineto{\pgfqpoint{2.399302in}{1.371362in}}%
\pgfpathlineto{\pgfqpoint{2.399912in}{1.430394in}}%
\pgfpathlineto{\pgfqpoint{2.400522in}{1.361576in}}%
\pgfpathlineto{\pgfqpoint{2.403573in}{1.022200in}}%
\pgfpathlineto{\pgfqpoint{2.404794in}{0.904324in}}%
\pgfpathlineto{\pgfqpoint{2.405099in}{0.936118in}}%
\pgfpathlineto{\pgfqpoint{2.406624in}{1.245942in}}%
\pgfpathlineto{\pgfqpoint{2.407540in}{1.122582in}}%
\pgfpathlineto{\pgfqpoint{2.407845in}{1.111120in}}%
\pgfpathlineto{\pgfqpoint{2.408150in}{1.130339in}}%
\pgfpathlineto{\pgfqpoint{2.408760in}{1.180762in}}%
\pgfpathlineto{\pgfqpoint{2.409370in}{1.154801in}}%
\pgfpathlineto{\pgfqpoint{2.410896in}{0.892264in}}%
\pgfpathlineto{\pgfqpoint{2.412117in}{0.942577in}}%
\pgfpathlineto{\pgfqpoint{2.413032in}{0.843104in}}%
\pgfpathlineto{\pgfqpoint{2.413947in}{0.722666in}}%
\pgfpathlineto{\pgfqpoint{2.414557in}{0.774525in}}%
\pgfpathlineto{\pgfqpoint{2.417609in}{1.107485in}}%
\pgfpathlineto{\pgfqpoint{2.417914in}{1.112421in}}%
\pgfpathlineto{\pgfqpoint{2.418219in}{1.104081in}}%
\pgfpathlineto{\pgfqpoint{2.419744in}{1.019625in}}%
\pgfpathlineto{\pgfqpoint{2.420660in}{1.038973in}}%
\pgfpathlineto{\pgfqpoint{2.423101in}{1.213445in}}%
\pgfpathlineto{\pgfqpoint{2.424626in}{1.334017in}}%
\pgfpathlineto{\pgfqpoint{2.424931in}{1.312579in}}%
\pgfpathlineto{\pgfqpoint{2.428593in}{0.955046in}}%
\pgfpathlineto{\pgfqpoint{2.429203in}{1.018473in}}%
\pgfpathlineto{\pgfqpoint{2.430424in}{1.262471in}}%
\pgfpathlineto{\pgfqpoint{2.431034in}{1.191067in}}%
\pgfpathlineto{\pgfqpoint{2.432559in}{0.956750in}}%
\pgfpathlineto{\pgfqpoint{2.433170in}{1.006751in}}%
\pgfpathlineto{\pgfqpoint{2.433780in}{1.044551in}}%
\pgfpathlineto{\pgfqpoint{2.434390in}{1.007408in}}%
\pgfpathlineto{\pgfqpoint{2.437136in}{0.854222in}}%
\pgfpathlineto{\pgfqpoint{2.437746in}{0.798624in}}%
\pgfpathlineto{\pgfqpoint{2.438357in}{0.828406in}}%
\pgfpathlineto{\pgfqpoint{2.442323in}{1.363208in}}%
\pgfpathlineto{\pgfqpoint{2.442628in}{1.355251in}}%
\pgfpathlineto{\pgfqpoint{2.444764in}{0.915221in}}%
\pgfpathlineto{\pgfqpoint{2.446290in}{0.980482in}}%
\pgfpathlineto{\pgfqpoint{2.448120in}{1.229617in}}%
\pgfpathlineto{\pgfqpoint{2.449036in}{1.372363in}}%
\pgfpathlineto{\pgfqpoint{2.449646in}{1.335037in}}%
\pgfpathlineto{\pgfqpoint{2.453307in}{0.976287in}}%
\pgfpathlineto{\pgfqpoint{2.453613in}{0.959986in}}%
\pgfpathlineto{\pgfqpoint{2.454223in}{0.986491in}}%
\pgfpathlineto{\pgfqpoint{2.455443in}{1.156859in}}%
\pgfpathlineto{\pgfqpoint{2.456664in}{1.144446in}}%
\pgfpathlineto{\pgfqpoint{2.457579in}{1.206953in}}%
\pgfpathlineto{\pgfqpoint{2.458189in}{1.161423in}}%
\pgfpathlineto{\pgfqpoint{2.461241in}{0.870702in}}%
\pgfpathlineto{\pgfqpoint{2.462156in}{0.819571in}}%
\pgfpathlineto{\pgfqpoint{2.462766in}{0.869446in}}%
\pgfpathlineto{\pgfqpoint{2.466428in}{1.224835in}}%
\pgfpathlineto{\pgfqpoint{2.466733in}{1.240138in}}%
\pgfpathlineto{\pgfqpoint{2.467038in}{1.226523in}}%
\pgfpathlineto{\pgfqpoint{2.468258in}{1.065605in}}%
\pgfpathlineto{\pgfqpoint{2.469174in}{1.110403in}}%
\pgfpathlineto{\pgfqpoint{2.469479in}{1.126658in}}%
\pgfpathlineto{\pgfqpoint{2.470089in}{1.107860in}}%
\pgfpathlineto{\pgfqpoint{2.471004in}{1.048396in}}%
\pgfpathlineto{\pgfqpoint{2.471309in}{1.067401in}}%
\pgfpathlineto{\pgfqpoint{2.472835in}{1.269943in}}%
\pgfpathlineto{\pgfqpoint{2.473445in}{1.178865in}}%
\pgfpathlineto{\pgfqpoint{2.474361in}{1.000598in}}%
\pgfpathlineto{\pgfqpoint{2.475276in}{1.044580in}}%
\pgfpathlineto{\pgfqpoint{2.475886in}{1.067328in}}%
\pgfpathlineto{\pgfqpoint{2.476496in}{1.043947in}}%
\pgfpathlineto{\pgfqpoint{2.477107in}{1.011219in}}%
\pgfpathlineto{\pgfqpoint{2.477717in}{1.047953in}}%
\pgfpathlineto{\pgfqpoint{2.478632in}{1.176231in}}%
\pgfpathlineto{\pgfqpoint{2.479548in}{1.095311in}}%
\pgfpathlineto{\pgfqpoint{2.480768in}{0.936115in}}%
\pgfpathlineto{\pgfqpoint{2.481378in}{0.994801in}}%
\pgfpathlineto{\pgfqpoint{2.482599in}{1.114537in}}%
\pgfpathlineto{\pgfqpoint{2.483209in}{1.058090in}}%
\pgfpathlineto{\pgfqpoint{2.485955in}{0.859493in}}%
\pgfpathlineto{\pgfqpoint{2.486870in}{0.973643in}}%
\pgfpathlineto{\pgfqpoint{2.488091in}{1.179655in}}%
\pgfpathlineto{\pgfqpoint{2.489006in}{1.147726in}}%
\pgfpathlineto{\pgfqpoint{2.489311in}{1.145621in}}%
\pgfpathlineto{\pgfqpoint{2.489617in}{1.149274in}}%
\pgfpathlineto{\pgfqpoint{2.489922in}{1.153208in}}%
\pgfpathlineto{\pgfqpoint{2.490227in}{1.150132in}}%
\pgfpathlineto{\pgfqpoint{2.491447in}{1.020727in}}%
\pgfpathlineto{\pgfqpoint{2.492057in}{0.974840in}}%
\pgfpathlineto{\pgfqpoint{2.492668in}{1.023952in}}%
\pgfpathlineto{\pgfqpoint{2.493888in}{1.108686in}}%
\pgfpathlineto{\pgfqpoint{2.494498in}{1.065134in}}%
\pgfpathlineto{\pgfqpoint{2.495109in}{1.010251in}}%
\pgfpathlineto{\pgfqpoint{2.495719in}{1.046754in}}%
\pgfpathlineto{\pgfqpoint{2.496939in}{1.231963in}}%
\pgfpathlineto{\pgfqpoint{2.497550in}{1.173541in}}%
\pgfpathlineto{\pgfqpoint{2.498465in}{1.058212in}}%
\pgfpathlineto{\pgfqpoint{2.499075in}{1.105593in}}%
\pgfpathlineto{\pgfqpoint{2.499685in}{1.153550in}}%
\pgfpathlineto{\pgfqpoint{2.500601in}{1.104139in}}%
\pgfpathlineto{\pgfqpoint{2.501211in}{1.078296in}}%
\pgfpathlineto{\pgfqpoint{2.501821in}{1.122729in}}%
\pgfpathlineto{\pgfqpoint{2.502737in}{1.186976in}}%
\pgfpathlineto{\pgfqpoint{2.503042in}{1.162950in}}%
\pgfpathlineto{\pgfqpoint{2.504567in}{0.901619in}}%
\pgfpathlineto{\pgfqpoint{2.505483in}{0.989242in}}%
\pgfpathlineto{\pgfqpoint{2.506398in}{1.097011in}}%
\pgfpathlineto{\pgfqpoint{2.507008in}{1.058529in}}%
\pgfpathlineto{\pgfqpoint{2.510059in}{0.813342in}}%
\pgfpathlineto{\pgfqpoint{2.510365in}{0.806796in}}%
\pgfpathlineto{\pgfqpoint{2.510670in}{0.828095in}}%
\pgfpathlineto{\pgfqpoint{2.514026in}{1.311987in}}%
\pgfpathlineto{\pgfqpoint{2.514636in}{1.294387in}}%
\pgfpathlineto{\pgfqpoint{2.518603in}{0.951681in}}%
\pgfpathlineto{\pgfqpoint{2.518908in}{0.952652in}}%
\pgfpathlineto{\pgfqpoint{2.520739in}{1.100771in}}%
\pgfpathlineto{\pgfqpoint{2.521654in}{1.029051in}}%
\pgfpathlineto{\pgfqpoint{2.522264in}{0.990709in}}%
\pgfpathlineto{\pgfqpoint{2.522874in}{1.030955in}}%
\pgfpathlineto{\pgfqpoint{2.524400in}{1.162770in}}%
\pgfpathlineto{\pgfqpoint{2.525010in}{1.149146in}}%
\pgfpathlineto{\pgfqpoint{2.525315in}{1.141448in}}%
\pgfpathlineto{\pgfqpoint{2.525926in}{1.160118in}}%
\pgfpathlineto{\pgfqpoint{2.526536in}{1.193510in}}%
\pgfpathlineto{\pgfqpoint{2.527146in}{1.160547in}}%
\pgfpathlineto{\pgfqpoint{2.528977in}{0.860310in}}%
\pgfpathlineto{\pgfqpoint{2.529587in}{0.923419in}}%
\pgfpathlineto{\pgfqpoint{2.530502in}{1.011053in}}%
\pgfpathlineto{\pgfqpoint{2.531418in}{0.980791in}}%
\pgfpathlineto{\pgfqpoint{2.532943in}{1.147482in}}%
\pgfpathlineto{\pgfqpoint{2.533859in}{1.081064in}}%
\pgfpathlineto{\pgfqpoint{2.534469in}{1.042939in}}%
\pgfpathlineto{\pgfqpoint{2.535079in}{1.067748in}}%
\pgfpathlineto{\pgfqpoint{2.535689in}{1.094837in}}%
\pgfpathlineto{\pgfqpoint{2.536605in}{1.077078in}}%
\pgfpathlineto{\pgfqpoint{2.536910in}{1.076045in}}%
\pgfpathlineto{\pgfqpoint{2.538130in}{1.164904in}}%
\pgfpathlineto{\pgfqpoint{2.538741in}{1.201493in}}%
\pgfpathlineto{\pgfqpoint{2.539351in}{1.175868in}}%
\pgfpathlineto{\pgfqpoint{2.543012in}{0.914021in}}%
\pgfpathlineto{\pgfqpoint{2.543622in}{0.965722in}}%
\pgfpathlineto{\pgfqpoint{2.545148in}{1.282637in}}%
\pgfpathlineto{\pgfqpoint{2.546063in}{1.171379in}}%
\pgfpathlineto{\pgfqpoint{2.546674in}{1.109756in}}%
\pgfpathlineto{\pgfqpoint{2.547589in}{1.154203in}}%
\pgfpathlineto{\pgfqpoint{2.548199in}{1.123901in}}%
\pgfpathlineto{\pgfqpoint{2.549115in}{1.037509in}}%
\pgfpathlineto{\pgfqpoint{2.549725in}{1.068336in}}%
\pgfpathlineto{\pgfqpoint{2.550335in}{1.112489in}}%
\pgfpathlineto{\pgfqpoint{2.550945in}{1.067436in}}%
\pgfpathlineto{\pgfqpoint{2.552166in}{0.889178in}}%
\pgfpathlineto{\pgfqpoint{2.553081in}{0.959639in}}%
\pgfpathlineto{\pgfqpoint{2.554607in}{1.161603in}}%
\pgfpathlineto{\pgfqpoint{2.555522in}{1.130863in}}%
\pgfpathlineto{\pgfqpoint{2.556743in}{1.094355in}}%
\pgfpathlineto{\pgfqpoint{2.558573in}{0.930314in}}%
\pgfpathlineto{\pgfqpoint{2.559183in}{0.964514in}}%
\pgfpathlineto{\pgfqpoint{2.563150in}{1.345618in}}%
\pgfpathlineto{\pgfqpoint{2.567117in}{0.819988in}}%
\pgfpathlineto{\pgfqpoint{2.567727in}{0.906080in}}%
\pgfpathlineto{\pgfqpoint{2.569252in}{1.160532in}}%
\pgfpathlineto{\pgfqpoint{2.569863in}{1.117925in}}%
\pgfpathlineto{\pgfqpoint{2.570473in}{1.099844in}}%
\pgfpathlineto{\pgfqpoint{2.571083in}{1.121542in}}%
\pgfpathlineto{\pgfqpoint{2.571693in}{1.136451in}}%
\pgfpathlineto{\pgfqpoint{2.572609in}{1.124371in}}%
\pgfpathlineto{\pgfqpoint{2.572914in}{1.120875in}}%
\pgfpathlineto{\pgfqpoint{2.573524in}{1.127783in}}%
\pgfpathlineto{\pgfqpoint{2.574744in}{1.173689in}}%
\pgfpathlineto{\pgfqpoint{2.575355in}{1.144782in}}%
\pgfpathlineto{\pgfqpoint{2.576880in}{0.955454in}}%
\pgfpathlineto{\pgfqpoint{2.577491in}{0.996221in}}%
\pgfpathlineto{\pgfqpoint{2.578711in}{1.099355in}}%
\pgfpathlineto{\pgfqpoint{2.579321in}{1.065708in}}%
\pgfpathlineto{\pgfqpoint{2.579626in}{1.051961in}}%
\pgfpathlineto{\pgfqpoint{2.580542in}{1.079011in}}%
\pgfpathlineto{\pgfqpoint{2.580847in}{1.088509in}}%
\pgfpathlineto{\pgfqpoint{2.581457in}{1.067819in}}%
\pgfpathlineto{\pgfqpoint{2.582372in}{0.996675in}}%
\pgfpathlineto{\pgfqpoint{2.582983in}{1.022846in}}%
\pgfpathlineto{\pgfqpoint{2.584203in}{1.130115in}}%
\pgfpathlineto{\pgfqpoint{2.585119in}{1.121097in}}%
\pgfpathlineto{\pgfqpoint{2.586034in}{1.108244in}}%
\pgfpathlineto{\pgfqpoint{2.586339in}{1.111894in}}%
\pgfpathlineto{\pgfqpoint{2.588170in}{1.222823in}}%
\pgfpathlineto{\pgfqpoint{2.588780in}{1.161711in}}%
\pgfpathlineto{\pgfqpoint{2.591526in}{0.915297in}}%
\pgfpathlineto{\pgfqpoint{2.592136in}{0.929088in}}%
\pgfpathlineto{\pgfqpoint{2.595187in}{1.146926in}}%
\pgfpathlineto{\pgfqpoint{2.596408in}{1.121049in}}%
\pgfpathlineto{\pgfqpoint{2.600069in}{0.942917in}}%
\pgfpathlineto{\pgfqpoint{2.600374in}{0.960021in}}%
\pgfpathlineto{\pgfqpoint{2.603426in}{1.147321in}}%
\pgfpathlineto{\pgfqpoint{2.604036in}{1.166829in}}%
\pgfpathlineto{\pgfqpoint{2.604646in}{1.142230in}}%
\pgfpathlineto{\pgfqpoint{2.605561in}{1.091639in}}%
\pgfpathlineto{\pgfqpoint{2.606477in}{1.116584in}}%
\pgfpathlineto{\pgfqpoint{2.608002in}{1.172871in}}%
\pgfpathlineto{\pgfqpoint{2.608613in}{1.151585in}}%
\pgfpathlineto{\pgfqpoint{2.611054in}{1.082072in}}%
\pgfpathlineto{\pgfqpoint{2.611969in}{1.039311in}}%
\pgfpathlineto{\pgfqpoint{2.612884in}{1.064942in}}%
\pgfpathlineto{\pgfqpoint{2.613189in}{1.070597in}}%
\pgfpathlineto{\pgfqpoint{2.613494in}{1.061416in}}%
\pgfpathlineto{\pgfqpoint{2.615020in}{0.909135in}}%
\pgfpathlineto{\pgfqpoint{2.615630in}{0.959422in}}%
\pgfpathlineto{\pgfqpoint{2.616546in}{1.035023in}}%
\pgfpathlineto{\pgfqpoint{2.617461in}{1.012617in}}%
\pgfpathlineto{\pgfqpoint{2.621122in}{1.191557in}}%
\pgfpathlineto{\pgfqpoint{2.621428in}{1.173773in}}%
\pgfpathlineto{\pgfqpoint{2.624479in}{0.935786in}}%
\pgfpathlineto{\pgfqpoint{2.624784in}{0.948316in}}%
\pgfpathlineto{\pgfqpoint{2.626615in}{1.144774in}}%
\pgfpathlineto{\pgfqpoint{2.627530in}{1.069189in}}%
\pgfpathlineto{\pgfqpoint{2.629971in}{1.009337in}}%
\pgfpathlineto{\pgfqpoint{2.631191in}{1.217431in}}%
\pgfpathlineto{\pgfqpoint{2.631802in}{1.288863in}}%
\pgfpathlineto{\pgfqpoint{2.632412in}{1.191539in}}%
\pgfpathlineto{\pgfqpoint{2.633327in}{1.077885in}}%
\pgfpathlineto{\pgfqpoint{2.633937in}{1.119996in}}%
\pgfpathlineto{\pgfqpoint{2.634243in}{1.129413in}}%
\pgfpathlineto{\pgfqpoint{2.634548in}{1.115844in}}%
\pgfpathlineto{\pgfqpoint{2.636073in}{0.926431in}}%
\pgfpathlineto{\pgfqpoint{2.636683in}{1.000878in}}%
\pgfpathlineto{\pgfqpoint{2.637599in}{1.065480in}}%
\pgfpathlineto{\pgfqpoint{2.638209in}{1.033883in}}%
\pgfpathlineto{\pgfqpoint{2.638819in}{1.003433in}}%
\pgfpathlineto{\pgfqpoint{2.639430in}{1.043912in}}%
\pgfpathlineto{\pgfqpoint{2.640650in}{1.167073in}}%
\pgfpathlineto{\pgfqpoint{2.641260in}{1.119052in}}%
\pgfpathlineto{\pgfqpoint{2.642481in}{1.002937in}}%
\pgfpathlineto{\pgfqpoint{2.643396in}{1.017710in}}%
\pgfpathlineto{\pgfqpoint{2.643701in}{1.016876in}}%
\pgfpathlineto{\pgfqpoint{2.644922in}{0.974159in}}%
\pgfpathlineto{\pgfqpoint{2.645227in}{0.983583in}}%
\pgfpathlineto{\pgfqpoint{2.647057in}{1.186589in}}%
\pgfpathlineto{\pgfqpoint{2.647668in}{1.120622in}}%
\pgfpathlineto{\pgfqpoint{2.648888in}{1.036177in}}%
\pgfpathlineto{\pgfqpoint{2.649498in}{1.045406in}}%
\pgfpathlineto{\pgfqpoint{2.650414in}{1.018000in}}%
\pgfpathlineto{\pgfqpoint{2.650719in}{1.008984in}}%
\pgfpathlineto{\pgfqpoint{2.651329in}{1.035420in}}%
\pgfpathlineto{\pgfqpoint{2.654685in}{1.303150in}}%
\pgfpathlineto{\pgfqpoint{2.654991in}{1.301276in}}%
\pgfpathlineto{\pgfqpoint{2.655906in}{1.192018in}}%
\pgfpathlineto{\pgfqpoint{2.658652in}{0.796087in}}%
\pgfpathlineto{\pgfqpoint{2.658957in}{0.797631in}}%
\pgfpathlineto{\pgfqpoint{2.659872in}{0.963554in}}%
\pgfpathlineto{\pgfqpoint{2.661398in}{1.394756in}}%
\pgfpathlineto{\pgfqpoint{2.662008in}{1.321888in}}%
\pgfpathlineto{\pgfqpoint{2.665365in}{0.860066in}}%
\pgfpathlineto{\pgfqpoint{2.665670in}{0.865790in}}%
\pgfpathlineto{\pgfqpoint{2.669941in}{1.226601in}}%
\pgfpathlineto{\pgfqpoint{2.670552in}{1.185356in}}%
\pgfpathlineto{\pgfqpoint{2.672382in}{0.876404in}}%
\pgfpathlineto{\pgfqpoint{2.672993in}{0.938289in}}%
\pgfpathlineto{\pgfqpoint{2.674518in}{1.115313in}}%
\pgfpathlineto{\pgfqpoint{2.675433in}{1.098242in}}%
\pgfpathlineto{\pgfqpoint{2.676349in}{1.122018in}}%
\pgfpathlineto{\pgfqpoint{2.676654in}{1.109109in}}%
\pgfpathlineto{\pgfqpoint{2.678180in}{0.923459in}}%
\pgfpathlineto{\pgfqpoint{2.678790in}{0.986259in}}%
\pgfpathlineto{\pgfqpoint{2.680010in}{1.180836in}}%
\pgfpathlineto{\pgfqpoint{2.680926in}{1.162884in}}%
\pgfpathlineto{\pgfqpoint{2.681231in}{1.161839in}}%
\pgfpathlineto{\pgfqpoint{2.681536in}{1.164725in}}%
\pgfpathlineto{\pgfqpoint{2.681841in}{1.166215in}}%
\pgfpathlineto{\pgfqpoint{2.682146in}{1.163570in}}%
\pgfpathlineto{\pgfqpoint{2.683367in}{1.120891in}}%
\pgfpathlineto{\pgfqpoint{2.685197in}{0.969090in}}%
\pgfpathlineto{\pgfqpoint{2.686113in}{1.026599in}}%
\pgfpathlineto{\pgfqpoint{2.686418in}{1.036743in}}%
\pgfpathlineto{\pgfqpoint{2.687028in}{1.008656in}}%
\pgfpathlineto{\pgfqpoint{2.687638in}{0.982679in}}%
\pgfpathlineto{\pgfqpoint{2.688248in}{1.013913in}}%
\pgfpathlineto{\pgfqpoint{2.688554in}{1.027771in}}%
\pgfpathlineto{\pgfqpoint{2.689164in}{1.003183in}}%
\pgfpathlineto{\pgfqpoint{2.689774in}{0.963401in}}%
\pgfpathlineto{\pgfqpoint{2.690384in}{0.999622in}}%
\pgfpathlineto{\pgfqpoint{2.693741in}{1.264859in}}%
\pgfpathlineto{\pgfqpoint{2.694046in}{1.267592in}}%
\pgfpathlineto{\pgfqpoint{2.695266in}{1.080268in}}%
\pgfpathlineto{\pgfqpoint{2.695876in}{1.012358in}}%
\pgfpathlineto{\pgfqpoint{2.696792in}{1.044547in}}%
\pgfpathlineto{\pgfqpoint{2.697097in}{1.045881in}}%
\pgfpathlineto{\pgfqpoint{2.698012in}{0.991459in}}%
\pgfpathlineto{\pgfqpoint{2.698622in}{1.022051in}}%
\pgfpathlineto{\pgfqpoint{2.700148in}{1.266978in}}%
\pgfpathlineto{\pgfqpoint{2.701063in}{1.185470in}}%
\pgfpathlineto{\pgfqpoint{2.701979in}{1.108692in}}%
\pgfpathlineto{\pgfqpoint{2.702894in}{1.141651in}}%
\pgfpathlineto{\pgfqpoint{2.703199in}{1.146821in}}%
\pgfpathlineto{\pgfqpoint{2.703504in}{1.141613in}}%
\pgfpathlineto{\pgfqpoint{2.705030in}{1.009070in}}%
\pgfpathlineto{\pgfqpoint{2.706556in}{0.909829in}}%
\pgfpathlineto{\pgfqpoint{2.707166in}{0.923500in}}%
\pgfpathlineto{\pgfqpoint{2.708081in}{0.920551in}}%
\pgfpathlineto{\pgfqpoint{2.708386in}{0.928856in}}%
\pgfpathlineto{\pgfqpoint{2.709912in}{1.248467in}}%
\pgfpathlineto{\pgfqpoint{2.710827in}{1.118871in}}%
\pgfpathlineto{\pgfqpoint{2.712048in}{0.960630in}}%
\pgfpathlineto{\pgfqpoint{2.712658in}{0.977773in}}%
\pgfpathlineto{\pgfqpoint{2.716319in}{1.187908in}}%
\pgfpathlineto{\pgfqpoint{2.716624in}{1.166526in}}%
\pgfpathlineto{\pgfqpoint{2.719370in}{0.986036in}}%
\pgfpathlineto{\pgfqpoint{2.719676in}{0.986309in}}%
\pgfpathlineto{\pgfqpoint{2.720896in}{1.075404in}}%
\pgfpathlineto{\pgfqpoint{2.723337in}{1.329815in}}%
\pgfpathlineto{\pgfqpoint{2.724252in}{1.242045in}}%
\pgfpathlineto{\pgfqpoint{2.726693in}{0.915436in}}%
\pgfpathlineto{\pgfqpoint{2.727304in}{0.946863in}}%
\pgfpathlineto{\pgfqpoint{2.730050in}{1.403130in}}%
\pgfpathlineto{\pgfqpoint{2.730965in}{1.267990in}}%
\pgfpathlineto{\pgfqpoint{2.733101in}{0.837117in}}%
\pgfpathlineto{\pgfqpoint{2.733711in}{0.863643in}}%
\pgfpathlineto{\pgfqpoint{2.734321in}{0.890456in}}%
\pgfpathlineto{\pgfqpoint{2.734931in}{0.857270in}}%
\pgfpathlineto{\pgfqpoint{2.735237in}{0.844678in}}%
\pgfpathlineto{\pgfqpoint{2.735542in}{0.865717in}}%
\pgfpathlineto{\pgfqpoint{2.737372in}{1.143074in}}%
\pgfpathlineto{\pgfqpoint{2.737983in}{1.088883in}}%
\pgfpathlineto{\pgfqpoint{2.738898in}{1.028123in}}%
\pgfpathlineto{\pgfqpoint{2.739508in}{1.064259in}}%
\pgfpathlineto{\pgfqpoint{2.739813in}{1.072391in}}%
\pgfpathlineto{\pgfqpoint{2.740119in}{1.061547in}}%
\pgfpathlineto{\pgfqpoint{2.741644in}{0.914273in}}%
\pgfpathlineto{\pgfqpoint{2.742254in}{0.972223in}}%
\pgfpathlineto{\pgfqpoint{2.744085in}{1.331657in}}%
\pgfpathlineto{\pgfqpoint{2.745000in}{1.235059in}}%
\pgfpathlineto{\pgfqpoint{2.747746in}{1.024738in}}%
\pgfpathlineto{\pgfqpoint{2.748357in}{0.995022in}}%
\pgfpathlineto{\pgfqpoint{2.748967in}{1.014561in}}%
\pgfpathlineto{\pgfqpoint{2.752628in}{1.200918in}}%
\pgfpathlineto{\pgfqpoint{2.753239in}{1.141692in}}%
\pgfpathlineto{\pgfqpoint{2.754764in}{0.810656in}}%
\pgfpathlineto{\pgfqpoint{2.755680in}{0.871338in}}%
\pgfpathlineto{\pgfqpoint{2.758120in}{1.169953in}}%
\pgfpathlineto{\pgfqpoint{2.759341in}{1.136201in}}%
\pgfpathlineto{\pgfqpoint{2.760256in}{1.086916in}}%
\pgfpathlineto{\pgfqpoint{2.761477in}{0.964892in}}%
\pgfpathlineto{\pgfqpoint{2.762087in}{0.999695in}}%
\pgfpathlineto{\pgfqpoint{2.762697in}{1.041462in}}%
\pgfpathlineto{\pgfqpoint{2.763613in}{1.000610in}}%
\pgfpathlineto{\pgfqpoint{2.764223in}{0.977685in}}%
\pgfpathlineto{\pgfqpoint{2.764833in}{1.020166in}}%
\pgfpathlineto{\pgfqpoint{2.767274in}{1.151289in}}%
\pgfpathlineto{\pgfqpoint{2.768494in}{1.088668in}}%
\pgfpathlineto{\pgfqpoint{2.769105in}{1.122370in}}%
\pgfpathlineto{\pgfqpoint{2.770630in}{1.304464in}}%
\pgfpathlineto{\pgfqpoint{2.771241in}{1.222717in}}%
\pgfpathlineto{\pgfqpoint{2.772461in}{1.005086in}}%
\pgfpathlineto{\pgfqpoint{2.773376in}{1.038358in}}%
\pgfpathlineto{\pgfqpoint{2.773987in}{1.013128in}}%
\pgfpathlineto{\pgfqpoint{2.774902in}{0.920943in}}%
\pgfpathlineto{\pgfqpoint{2.775512in}{0.965129in}}%
\pgfpathlineto{\pgfqpoint{2.776428in}{1.066077in}}%
\pgfpathlineto{\pgfqpoint{2.777038in}{1.027168in}}%
\pgfpathlineto{\pgfqpoint{2.777953in}{0.959791in}}%
\pgfpathlineto{\pgfqpoint{2.778563in}{1.007934in}}%
\pgfpathlineto{\pgfqpoint{2.781309in}{1.201662in}}%
\pgfpathlineto{\pgfqpoint{2.781615in}{1.181989in}}%
\pgfpathlineto{\pgfqpoint{2.784666in}{0.907881in}}%
\pgfpathlineto{\pgfqpoint{2.784971in}{0.906400in}}%
\pgfpathlineto{\pgfqpoint{2.785581in}{0.932664in}}%
\pgfpathlineto{\pgfqpoint{2.787717in}{1.206236in}}%
\pgfpathlineto{\pgfqpoint{2.788632in}{1.114022in}}%
\pgfpathlineto{\pgfqpoint{2.789243in}{1.069140in}}%
\pgfpathlineto{\pgfqpoint{2.789853in}{1.099332in}}%
\pgfpathlineto{\pgfqpoint{2.792599in}{1.243298in}}%
\pgfpathlineto{\pgfqpoint{2.793514in}{1.146631in}}%
\pgfpathlineto{\pgfqpoint{2.794124in}{1.099923in}}%
\pgfpathlineto{\pgfqpoint{2.795040in}{1.141127in}}%
\pgfpathlineto{\pgfqpoint{2.795345in}{1.141552in}}%
\pgfpathlineto{\pgfqpoint{2.799006in}{0.976123in}}%
\pgfpathlineto{\pgfqpoint{2.800532in}{1.095109in}}%
\pgfpathlineto{\pgfqpoint{2.801142in}{1.047227in}}%
\pgfpathlineto{\pgfqpoint{2.801752in}{1.014204in}}%
\pgfpathlineto{\pgfqpoint{2.802668in}{1.046814in}}%
\pgfpathlineto{\pgfqpoint{2.805414in}{1.143353in}}%
\pgfpathlineto{\pgfqpoint{2.805719in}{1.134142in}}%
\pgfpathlineto{\pgfqpoint{2.808465in}{0.842068in}}%
\pgfpathlineto{\pgfqpoint{2.809075in}{0.883731in}}%
\pgfpathlineto{\pgfqpoint{2.810906in}{1.319587in}}%
\pgfpathlineto{\pgfqpoint{2.811821in}{1.168774in}}%
\pgfpathlineto{\pgfqpoint{2.813347in}{0.961761in}}%
\pgfpathlineto{\pgfqpoint{2.813957in}{0.972557in}}%
\pgfpathlineto{\pgfqpoint{2.815178in}{0.986672in}}%
\pgfpathlineto{\pgfqpoint{2.816093in}{1.213472in}}%
\pgfpathlineto{\pgfqpoint{2.817008in}{1.434642in}}%
\pgfpathlineto{\pgfqpoint{2.817924in}{1.321439in}}%
\pgfpathlineto{\pgfqpoint{2.821280in}{0.861169in}}%
\pgfpathlineto{\pgfqpoint{2.821585in}{0.848596in}}%
\pgfpathlineto{\pgfqpoint{2.821890in}{0.862116in}}%
\pgfpathlineto{\pgfqpoint{2.823721in}{1.308652in}}%
\pgfpathlineto{\pgfqpoint{2.825246in}{1.181391in}}%
\pgfpathlineto{\pgfqpoint{2.825857in}{1.148736in}}%
\pgfpathlineto{\pgfqpoint{2.827687in}{0.851726in}}%
\pgfpathlineto{\pgfqpoint{2.828298in}{0.930322in}}%
\pgfpathlineto{\pgfqpoint{2.829518in}{1.097384in}}%
\pgfpathlineto{\pgfqpoint{2.830433in}{1.078075in}}%
\pgfpathlineto{\pgfqpoint{2.831654in}{1.006574in}}%
\pgfpathlineto{\pgfqpoint{2.833180in}{0.854929in}}%
\pgfpathlineto{\pgfqpoint{2.833790in}{0.883878in}}%
\pgfpathlineto{\pgfqpoint{2.837451in}{1.457578in}}%
\pgfpathlineto{\pgfqpoint{2.837756in}{1.414968in}}%
\pgfpathlineto{\pgfqpoint{2.839282in}{0.965289in}}%
\pgfpathlineto{\pgfqpoint{2.840197in}{1.028322in}}%
\pgfpathlineto{\pgfqpoint{2.840502in}{1.040912in}}%
\pgfpathlineto{\pgfqpoint{2.840807in}{1.030884in}}%
\pgfpathlineto{\pgfqpoint{2.841723in}{0.916896in}}%
\pgfpathlineto{\pgfqpoint{2.842333in}{0.994511in}}%
\pgfpathlineto{\pgfqpoint{2.843554in}{1.318230in}}%
\pgfpathlineto{\pgfqpoint{2.844164in}{1.228427in}}%
\pgfpathlineto{\pgfqpoint{2.845384in}{0.982988in}}%
\pgfpathlineto{\pgfqpoint{2.845994in}{1.031913in}}%
\pgfpathlineto{\pgfqpoint{2.846910in}{1.093091in}}%
\pgfpathlineto{\pgfqpoint{2.847520in}{1.041741in}}%
\pgfpathlineto{\pgfqpoint{2.848130in}{1.006071in}}%
\pgfpathlineto{\pgfqpoint{2.848435in}{1.038849in}}%
\pgfpathlineto{\pgfqpoint{2.849656in}{1.253358in}}%
\pgfpathlineto{\pgfqpoint{2.850266in}{1.137842in}}%
\pgfpathlineto{\pgfqpoint{2.852707in}{0.862902in}}%
\pgfpathlineto{\pgfqpoint{2.853317in}{0.834013in}}%
\pgfpathlineto{\pgfqpoint{2.853928in}{0.850199in}}%
\pgfpathlineto{\pgfqpoint{2.857284in}{1.129505in}}%
\pgfpathlineto{\pgfqpoint{2.857894in}{1.119993in}}%
\pgfpathlineto{\pgfqpoint{2.858809in}{1.097868in}}%
\pgfpathlineto{\pgfqpoint{2.859420in}{1.107925in}}%
\pgfpathlineto{\pgfqpoint{2.860335in}{1.133808in}}%
\pgfpathlineto{\pgfqpoint{2.860945in}{1.108842in}}%
\pgfpathlineto{\pgfqpoint{2.861861in}{1.057315in}}%
\pgfpathlineto{\pgfqpoint{2.862471in}{1.090906in}}%
\pgfpathlineto{\pgfqpoint{2.863996in}{1.232095in}}%
\pgfpathlineto{\pgfqpoint{2.864607in}{1.171776in}}%
\pgfpathlineto{\pgfqpoint{2.867353in}{1.050027in}}%
\pgfpathlineto{\pgfqpoint{2.867658in}{1.051140in}}%
\pgfpathlineto{\pgfqpoint{2.870404in}{1.181619in}}%
\pgfpathlineto{\pgfqpoint{2.870709in}{1.155579in}}%
\pgfpathlineto{\pgfqpoint{2.872235in}{0.911110in}}%
\pgfpathlineto{\pgfqpoint{2.872845in}{0.983193in}}%
\pgfpathlineto{\pgfqpoint{2.873760in}{1.081016in}}%
\pgfpathlineto{\pgfqpoint{2.874370in}{1.054522in}}%
\pgfpathlineto{\pgfqpoint{2.875591in}{0.895053in}}%
\pgfpathlineto{\pgfqpoint{2.876201in}{0.938871in}}%
\pgfpathlineto{\pgfqpoint{2.877422in}{1.041174in}}%
\pgfpathlineto{\pgfqpoint{2.878032in}{0.985972in}}%
\pgfpathlineto{\pgfqpoint{2.878947in}{0.892267in}}%
\pgfpathlineto{\pgfqpoint{2.879863in}{0.933779in}}%
\pgfpathlineto{\pgfqpoint{2.880168in}{0.932545in}}%
\pgfpathlineto{\pgfqpoint{2.881388in}{0.882023in}}%
\pgfpathlineto{\pgfqpoint{2.881998in}{0.909727in}}%
\pgfpathlineto{\pgfqpoint{2.883524in}{1.236746in}}%
\pgfpathlineto{\pgfqpoint{2.884744in}{1.146141in}}%
\pgfpathlineto{\pgfqpoint{2.887185in}{1.087111in}}%
\pgfpathlineto{\pgfqpoint{2.885965in}{1.150935in}}%
\pgfpathlineto{\pgfqpoint{2.887491in}{1.096508in}}%
\pgfpathlineto{\pgfqpoint{2.888406in}{1.188026in}}%
\pgfpathlineto{\pgfqpoint{2.889321in}{1.146971in}}%
\pgfpathlineto{\pgfqpoint{2.889626in}{1.137984in}}%
\pgfpathlineto{\pgfqpoint{2.889931in}{1.150183in}}%
\pgfpathlineto{\pgfqpoint{2.891762in}{1.338181in}}%
\pgfpathlineto{\pgfqpoint{2.892678in}{1.277953in}}%
\pgfpathlineto{\pgfqpoint{2.896034in}{0.858205in}}%
\pgfpathlineto{\pgfqpoint{2.896644in}{0.969130in}}%
\pgfpathlineto{\pgfqpoint{2.897559in}{1.113078in}}%
\pgfpathlineto{\pgfqpoint{2.898170in}{0.981564in}}%
\pgfpathlineto{\pgfqpoint{2.899085in}{0.784660in}}%
\pgfpathlineto{\pgfqpoint{2.899695in}{0.885677in}}%
\pgfpathlineto{\pgfqpoint{2.900611in}{0.976606in}}%
\pgfpathlineto{\pgfqpoint{2.901221in}{0.889883in}}%
\pgfpathlineto{\pgfqpoint{2.902136in}{0.666419in}}%
\pgfpathlineto{\pgfqpoint{2.902746in}{0.762397in}}%
\pgfpathlineto{\pgfqpoint{2.903967in}{0.956759in}}%
\pgfpathlineto{\pgfqpoint{2.904577in}{0.911844in}}%
\pgfpathlineto{\pgfqpoint{2.904882in}{0.893508in}}%
\pgfpathlineto{\pgfqpoint{2.905493in}{0.944431in}}%
\pgfpathlineto{\pgfqpoint{2.906713in}{1.233132in}}%
\pgfpathlineto{\pgfqpoint{2.907628in}{1.146172in}}%
\pgfpathlineto{\pgfqpoint{2.908849in}{0.922187in}}%
\pgfpathlineto{\pgfqpoint{2.909459in}{0.962774in}}%
\pgfpathlineto{\pgfqpoint{2.912815in}{1.407908in}}%
\pgfpathlineto{\pgfqpoint{2.916172in}{1.010961in}}%
\pgfpathlineto{\pgfqpoint{2.916782in}{1.028848in}}%
\pgfpathlineto{\pgfqpoint{2.918918in}{1.192755in}}%
\pgfpathlineto{\pgfqpoint{2.919528in}{1.178554in}}%
\pgfpathlineto{\pgfqpoint{2.920748in}{1.032395in}}%
\pgfpathlineto{\pgfqpoint{2.921359in}{1.100295in}}%
\pgfpathlineto{\pgfqpoint{2.921969in}{1.182631in}}%
\pgfpathlineto{\pgfqpoint{2.922579in}{1.103188in}}%
\pgfpathlineto{\pgfqpoint{2.923189in}{0.997053in}}%
\pgfpathlineto{\pgfqpoint{2.924105in}{1.103154in}}%
\pgfpathlineto{\pgfqpoint{2.924715in}{1.156112in}}%
\pgfpathlineto{\pgfqpoint{2.925020in}{1.102470in}}%
\pgfpathlineto{\pgfqpoint{2.926241in}{0.740184in}}%
\pgfpathlineto{\pgfqpoint{2.927156in}{0.858754in}}%
\pgfpathlineto{\pgfqpoint{2.927461in}{0.890007in}}%
\pgfpathlineto{\pgfqpoint{2.928071in}{0.854792in}}%
\pgfpathlineto{\pgfqpoint{2.928987in}{0.680965in}}%
\pgfpathlineto{\pgfqpoint{2.929597in}{0.804645in}}%
\pgfpathlineto{\pgfqpoint{2.930817in}{1.161974in}}%
\pgfpathlineto{\pgfqpoint{2.931428in}{1.088227in}}%
\pgfpathlineto{\pgfqpoint{2.932343in}{0.901090in}}%
\pgfpathlineto{\pgfqpoint{2.932953in}{1.070652in}}%
\pgfpathlineto{\pgfqpoint{2.933563in}{1.217796in}}%
\pgfpathlineto{\pgfqpoint{2.934479in}{1.109759in}}%
\pgfpathlineto{\pgfqpoint{2.935089in}{1.014459in}}%
\pgfpathlineto{\pgfqpoint{2.935699in}{1.096238in}}%
\pgfpathlineto{\pgfqpoint{2.936615in}{1.306071in}}%
\pgfpathlineto{\pgfqpoint{2.937530in}{1.206359in}}%
\pgfpathlineto{\pgfqpoint{2.938750in}{1.118330in}}%
\pgfpathlineto{\pgfqpoint{2.939361in}{1.126625in}}%
\pgfpathlineto{\pgfqpoint{2.941496in}{1.187110in}}%
\pgfpathlineto{\pgfqpoint{2.942412in}{1.358100in}}%
\pgfpathlineto{\pgfqpoint{2.943022in}{1.296989in}}%
\pgfpathlineto{\pgfqpoint{2.944243in}{0.994826in}}%
\pgfpathlineto{\pgfqpoint{2.945158in}{1.052957in}}%
\pgfpathlineto{\pgfqpoint{2.946073in}{0.885324in}}%
\pgfpathlineto{\pgfqpoint{2.946683in}{0.743766in}}%
\pgfpathlineto{\pgfqpoint{2.947599in}{0.880996in}}%
\pgfpathlineto{\pgfqpoint{2.948514in}{1.010802in}}%
\pgfpathlineto{\pgfqpoint{2.949124in}{0.931903in}}%
\pgfpathlineto{\pgfqpoint{2.949430in}{0.893137in}}%
\pgfpathlineto{\pgfqpoint{2.950040in}{0.968201in}}%
\pgfpathlineto{\pgfqpoint{2.950955in}{1.218490in}}%
\pgfpathlineto{\pgfqpoint{2.951565in}{1.121549in}}%
\pgfpathlineto{\pgfqpoint{2.952481in}{0.970909in}}%
\pgfpathlineto{\pgfqpoint{2.953091in}{1.091594in}}%
\pgfpathlineto{\pgfqpoint{2.953701in}{1.176208in}}%
\pgfpathlineto{\pgfqpoint{2.954311in}{1.087351in}}%
\pgfpathlineto{\pgfqpoint{2.955227in}{0.923228in}}%
\pgfpathlineto{\pgfqpoint{2.956142in}{0.996651in}}%
\pgfpathlineto{\pgfqpoint{2.956752in}{1.021004in}}%
\pgfpathlineto{\pgfqpoint{2.957363in}{0.979580in}}%
\pgfpathlineto{\pgfqpoint{2.958278in}{0.900268in}}%
\pgfpathlineto{\pgfqpoint{2.958888in}{0.972780in}}%
\pgfpathlineto{\pgfqpoint{2.960109in}{1.112505in}}%
\pgfpathlineto{\pgfqpoint{2.960719in}{1.067825in}}%
\pgfpathlineto{\pgfqpoint{2.961329in}{1.046047in}}%
\pgfpathlineto{\pgfqpoint{2.961634in}{1.074780in}}%
\pgfpathlineto{\pgfqpoint{2.964075in}{1.409099in}}%
\pgfpathlineto{\pgfqpoint{2.964685in}{1.400195in}}%
\pgfpathlineto{\pgfqpoint{2.966516in}{1.343021in}}%
\pgfpathlineto{\pgfqpoint{2.970178in}{0.875743in}}%
\pgfpathlineto{\pgfqpoint{2.970483in}{0.856559in}}%
\pgfpathlineto{\pgfqpoint{2.970788in}{0.883486in}}%
\pgfpathlineto{\pgfqpoint{2.972313in}{1.174499in}}%
\pgfpathlineto{\pgfqpoint{2.972924in}{1.050524in}}%
\pgfpathlineto{\pgfqpoint{2.973534in}{0.955816in}}%
\pgfpathlineto{\pgfqpoint{2.974449in}{1.040307in}}%
\pgfpathlineto{\pgfqpoint{2.974754in}{1.042676in}}%
\pgfpathlineto{\pgfqpoint{2.976585in}{0.793202in}}%
\pgfpathlineto{\pgfqpoint{2.977195in}{0.885920in}}%
\pgfpathlineto{\pgfqpoint{2.978111in}{1.038792in}}%
\pgfpathlineto{\pgfqpoint{2.979026in}{0.953901in}}%
\pgfpathlineto{\pgfqpoint{2.979636in}{0.898251in}}%
\pgfpathlineto{\pgfqpoint{2.980246in}{0.983528in}}%
\pgfpathlineto{\pgfqpoint{2.981162in}{1.084173in}}%
\pgfpathlineto{\pgfqpoint{2.981772in}{1.010160in}}%
\pgfpathlineto{\pgfqpoint{2.982077in}{0.988103in}}%
\pgfpathlineto{\pgfqpoint{2.982382in}{1.011821in}}%
\pgfpathlineto{\pgfqpoint{2.983908in}{1.419950in}}%
\pgfpathlineto{\pgfqpoint{2.984823in}{1.266291in}}%
\pgfpathlineto{\pgfqpoint{2.985128in}{1.231243in}}%
\pgfpathlineto{\pgfqpoint{2.985739in}{1.292227in}}%
\pgfpathlineto{\pgfqpoint{2.986654in}{1.398078in}}%
\pgfpathlineto{\pgfqpoint{2.987264in}{1.290787in}}%
\pgfpathlineto{\pgfqpoint{2.988790in}{0.899753in}}%
\pgfpathlineto{\pgfqpoint{2.989705in}{0.971229in}}%
\pgfpathlineto{\pgfqpoint{2.990010in}{0.979844in}}%
\pgfpathlineto{\pgfqpoint{2.990315in}{0.969313in}}%
\pgfpathlineto{\pgfqpoint{2.991536in}{0.825882in}}%
\pgfpathlineto{\pgfqpoint{2.992146in}{0.935551in}}%
\pgfpathlineto{\pgfqpoint{2.993061in}{1.101993in}}%
\pgfpathlineto{\pgfqpoint{2.993672in}{1.048938in}}%
\pgfpathlineto{\pgfqpoint{2.994282in}{0.999206in}}%
\pgfpathlineto{\pgfqpoint{2.994892in}{1.041967in}}%
\pgfpathlineto{\pgfqpoint{2.996113in}{1.151816in}}%
\pgfpathlineto{\pgfqpoint{2.997028in}{1.133383in}}%
\pgfpathlineto{\pgfqpoint{3.001910in}{0.899684in}}%
\pgfpathlineto{\pgfqpoint{3.002825in}{0.795066in}}%
\pgfpathlineto{\pgfqpoint{3.003435in}{0.828663in}}%
\pgfpathlineto{\pgfqpoint{3.006792in}{1.223933in}}%
\pgfpathlineto{\pgfqpoint{3.007707in}{1.140667in}}%
\pgfpathlineto{\pgfqpoint{3.009538in}{0.951089in}}%
\pgfpathlineto{\pgfqpoint{3.010148in}{1.010363in}}%
\pgfpathlineto{\pgfqpoint{3.012894in}{1.272645in}}%
\pgfpathlineto{\pgfqpoint{3.013809in}{1.345311in}}%
\pgfpathlineto{\pgfqpoint{3.014420in}{1.272115in}}%
\pgfpathlineto{\pgfqpoint{3.015640in}{1.069333in}}%
\pgfpathlineto{\pgfqpoint{3.016250in}{1.127223in}}%
\pgfpathlineto{\pgfqpoint{3.016861in}{1.162595in}}%
\pgfpathlineto{\pgfqpoint{3.017471in}{1.106592in}}%
\pgfpathlineto{\pgfqpoint{3.018386in}{0.972783in}}%
\pgfpathlineto{\pgfqpoint{3.018996in}{1.016334in}}%
\pgfpathlineto{\pgfqpoint{3.019912in}{1.136947in}}%
\pgfpathlineto{\pgfqpoint{3.020522in}{1.096074in}}%
\pgfpathlineto{\pgfqpoint{3.022048in}{0.789694in}}%
\pgfpathlineto{\pgfqpoint{3.022658in}{0.860896in}}%
\pgfpathlineto{\pgfqpoint{3.023268in}{0.913065in}}%
\pgfpathlineto{\pgfqpoint{3.023878in}{0.879676in}}%
\pgfpathlineto{\pgfqpoint{3.024794in}{0.788283in}}%
\pgfpathlineto{\pgfqpoint{3.025099in}{0.817094in}}%
\pgfpathlineto{\pgfqpoint{3.026624in}{1.147215in}}%
\pgfpathlineto{\pgfqpoint{3.027540in}{1.034850in}}%
\pgfpathlineto{\pgfqpoint{3.028150in}{0.960661in}}%
\pgfpathlineto{\pgfqpoint{3.028760in}{1.028196in}}%
\pgfpathlineto{\pgfqpoint{3.029370in}{1.110870in}}%
\pgfpathlineto{\pgfqpoint{3.030286in}{1.067936in}}%
\pgfpathlineto{\pgfqpoint{3.032727in}{0.982397in}}%
\pgfpathlineto{\pgfqpoint{3.031506in}{1.070133in}}%
\pgfpathlineto{\pgfqpoint{3.033032in}{0.992752in}}%
\pgfpathlineto{\pgfqpoint{3.036693in}{1.388541in}}%
\pgfpathlineto{\pgfqpoint{3.037304in}{1.324157in}}%
\pgfpathlineto{\pgfqpoint{3.038829in}{0.955875in}}%
\pgfpathlineto{\pgfqpoint{3.039744in}{1.004320in}}%
\pgfpathlineto{\pgfqpoint{3.040355in}{0.994608in}}%
\pgfpathlineto{\pgfqpoint{3.040660in}{0.997642in}}%
\pgfpathlineto{\pgfqpoint{3.042185in}{1.245244in}}%
\pgfpathlineto{\pgfqpoint{3.043101in}{1.136056in}}%
\pgfpathlineto{\pgfqpoint{3.044016in}{1.062595in}}%
\pgfpathlineto{\pgfqpoint{3.044626in}{1.087086in}}%
\pgfpathlineto{\pgfqpoint{3.044931in}{1.093647in}}%
\pgfpathlineto{\pgfqpoint{3.045542in}{1.082569in}}%
\pgfpathlineto{\pgfqpoint{3.048898in}{0.977387in}}%
\pgfpathlineto{\pgfqpoint{3.049508in}{0.988304in}}%
\pgfpathlineto{\pgfqpoint{3.050729in}{1.068390in}}%
\pgfpathlineto{\pgfqpoint{3.051339in}{1.022342in}}%
\pgfpathlineto{\pgfqpoint{3.052254in}{0.925541in}}%
\pgfpathlineto{\pgfqpoint{3.052865in}{0.959931in}}%
\pgfpathlineto{\pgfqpoint{3.054085in}{1.058890in}}%
\pgfpathlineto{\pgfqpoint{3.054695in}{0.997774in}}%
\pgfpathlineto{\pgfqpoint{3.055306in}{0.946198in}}%
\pgfpathlineto{\pgfqpoint{3.056221in}{0.981577in}}%
\pgfpathlineto{\pgfqpoint{3.056526in}{0.981430in}}%
\pgfpathlineto{\pgfqpoint{3.058052in}{0.899107in}}%
\pgfpathlineto{\pgfqpoint{3.058662in}{0.945292in}}%
\pgfpathlineto{\pgfqpoint{3.060493in}{1.398014in}}%
\pgfpathlineto{\pgfqpoint{3.061103in}{1.288507in}}%
\pgfpathlineto{\pgfqpoint{3.062018in}{1.081515in}}%
\pgfpathlineto{\pgfqpoint{3.062933in}{1.152969in}}%
\pgfpathlineto{\pgfqpoint{3.063239in}{1.170408in}}%
\pgfpathlineto{\pgfqpoint{3.063849in}{1.150328in}}%
\pgfpathlineto{\pgfqpoint{3.065069in}{1.044613in}}%
\pgfpathlineto{\pgfqpoint{3.065680in}{1.091240in}}%
\pgfpathlineto{\pgfqpoint{3.066290in}{1.123885in}}%
\pgfpathlineto{\pgfqpoint{3.066900in}{1.077526in}}%
\pgfpathlineto{\pgfqpoint{3.068120in}{0.927308in}}%
\pgfpathlineto{\pgfqpoint{3.068731in}{0.968450in}}%
\pgfpathlineto{\pgfqpoint{3.069951in}{1.136285in}}%
\pgfpathlineto{\pgfqpoint{3.070256in}{1.090476in}}%
\pgfpathlineto{\pgfqpoint{3.071172in}{0.951895in}}%
\pgfpathlineto{\pgfqpoint{3.071782in}{1.021025in}}%
\pgfpathlineto{\pgfqpoint{3.072392in}{1.077352in}}%
\pgfpathlineto{\pgfqpoint{3.073002in}{1.040616in}}%
\pgfpathlineto{\pgfqpoint{3.073918in}{0.941179in}}%
\pgfpathlineto{\pgfqpoint{3.074528in}{0.983651in}}%
\pgfpathlineto{\pgfqpoint{3.075748in}{1.081231in}}%
\pgfpathlineto{\pgfqpoint{3.076359in}{1.041531in}}%
\pgfpathlineto{\pgfqpoint{3.077274in}{0.967383in}}%
\pgfpathlineto{\pgfqpoint{3.077884in}{1.028333in}}%
\pgfpathlineto{\pgfqpoint{3.078800in}{1.169503in}}%
\pgfpathlineto{\pgfqpoint{3.079410in}{1.128745in}}%
\pgfpathlineto{\pgfqpoint{3.080935in}{0.861210in}}%
\pgfpathlineto{\pgfqpoint{3.081851in}{0.934349in}}%
\pgfpathlineto{\pgfqpoint{3.084902in}{1.327966in}}%
\pgfpathlineto{\pgfqpoint{3.085817in}{1.248054in}}%
\pgfpathlineto{\pgfqpoint{3.088869in}{0.866348in}}%
\pgfpathlineto{\pgfqpoint{3.089479in}{0.938475in}}%
\pgfpathlineto{\pgfqpoint{3.093140in}{1.364894in}}%
\pgfpathlineto{\pgfqpoint{3.093750in}{1.299916in}}%
\pgfpathlineto{\pgfqpoint{3.096191in}{0.916736in}}%
\pgfpathlineto{\pgfqpoint{3.096802in}{0.928194in}}%
\pgfpathlineto{\pgfqpoint{3.098632in}{0.983365in}}%
\pgfpathlineto{\pgfqpoint{3.099853in}{1.200784in}}%
\pgfpathlineto{\pgfqpoint{3.100463in}{1.122258in}}%
\pgfpathlineto{\pgfqpoint{3.101378in}{0.971861in}}%
\pgfpathlineto{\pgfqpoint{3.101989in}{1.034047in}}%
\pgfpathlineto{\pgfqpoint{3.102904in}{1.112470in}}%
\pgfpathlineto{\pgfqpoint{3.103514in}{1.041139in}}%
\pgfpathlineto{\pgfqpoint{3.104124in}{0.973313in}}%
\pgfpathlineto{\pgfqpoint{3.104735in}{1.012508in}}%
\pgfpathlineto{\pgfqpoint{3.105650in}{1.090749in}}%
\pgfpathlineto{\pgfqpoint{3.106260in}{1.029906in}}%
\pgfpathlineto{\pgfqpoint{3.107176in}{0.888454in}}%
\pgfpathlineto{\pgfqpoint{3.107786in}{0.943044in}}%
\pgfpathlineto{\pgfqpoint{3.109311in}{1.179724in}}%
\pgfpathlineto{\pgfqpoint{3.109922in}{1.102461in}}%
\pgfpathlineto{\pgfqpoint{3.110837in}{0.960856in}}%
\pgfpathlineto{\pgfqpoint{3.111447in}{1.012708in}}%
\pgfpathlineto{\pgfqpoint{3.112363in}{1.084886in}}%
\pgfpathlineto{\pgfqpoint{3.112973in}{1.061960in}}%
\pgfpathlineto{\pgfqpoint{3.113888in}{1.021589in}}%
\pgfpathlineto{\pgfqpoint{3.114193in}{1.044010in}}%
\pgfpathlineto{\pgfqpoint{3.115109in}{1.122038in}}%
\pgfpathlineto{\pgfqpoint{3.116024in}{1.082748in}}%
\pgfpathlineto{\pgfqpoint{3.116939in}{1.040452in}}%
\pgfpathlineto{\pgfqpoint{3.117855in}{1.060601in}}%
\pgfpathlineto{\pgfqpoint{3.118465in}{1.031980in}}%
\pgfpathlineto{\pgfqpoint{3.119380in}{0.953745in}}%
\pgfpathlineto{\pgfqpoint{3.120296in}{0.976815in}}%
\pgfpathlineto{\pgfqpoint{3.120601in}{0.980212in}}%
\pgfpathlineto{\pgfqpoint{3.120906in}{0.971638in}}%
\pgfpathlineto{\pgfqpoint{3.121821in}{0.923331in}}%
\pgfpathlineto{\pgfqpoint{3.122126in}{0.961404in}}%
\pgfpathlineto{\pgfqpoint{3.123652in}{1.401040in}}%
\pgfpathlineto{\pgfqpoint{3.124262in}{1.249015in}}%
\pgfpathlineto{\pgfqpoint{3.125483in}{0.839886in}}%
\pgfpathlineto{\pgfqpoint{3.126398in}{0.929908in}}%
\pgfpathlineto{\pgfqpoint{3.126703in}{0.948463in}}%
\pgfpathlineto{\pgfqpoint{3.127313in}{0.923886in}}%
\pgfpathlineto{\pgfqpoint{3.128229in}{0.814516in}}%
\pgfpathlineto{\pgfqpoint{3.128839in}{0.856496in}}%
\pgfpathlineto{\pgfqpoint{3.130059in}{1.006035in}}%
\pgfpathlineto{\pgfqpoint{3.130670in}{0.984425in}}%
\pgfpathlineto{\pgfqpoint{3.131280in}{0.947235in}}%
\pgfpathlineto{\pgfqpoint{3.131890in}{0.976364in}}%
\pgfpathlineto{\pgfqpoint{3.133416in}{1.321379in}}%
\pgfpathlineto{\pgfqpoint{3.134331in}{1.221214in}}%
\pgfpathlineto{\pgfqpoint{3.136467in}{1.030848in}}%
\pgfpathlineto{\pgfqpoint{3.137382in}{0.987096in}}%
\pgfpathlineto{\pgfqpoint{3.137993in}{1.002802in}}%
\pgfpathlineto{\pgfqpoint{3.141654in}{1.217002in}}%
\pgfpathlineto{\pgfqpoint{3.142264in}{1.192188in}}%
\pgfpathlineto{\pgfqpoint{3.144095in}{1.045977in}}%
\pgfpathlineto{\pgfqpoint{3.145010in}{1.078117in}}%
\pgfpathlineto{\pgfqpoint{3.147756in}{1.296299in}}%
\pgfpathlineto{\pgfqpoint{3.148367in}{1.209951in}}%
\pgfpathlineto{\pgfqpoint{3.150807in}{0.858610in}}%
\pgfpathlineto{\pgfqpoint{3.151113in}{0.856715in}}%
\pgfpathlineto{\pgfqpoint{3.151418in}{0.862674in}}%
\pgfpathlineto{\pgfqpoint{3.152638in}{1.035650in}}%
\pgfpathlineto{\pgfqpoint{3.154164in}{1.260057in}}%
\pgfpathlineto{\pgfqpoint{3.154774in}{1.229901in}}%
\pgfpathlineto{\pgfqpoint{3.156300in}{1.072093in}}%
\pgfpathlineto{\pgfqpoint{3.156910in}{1.100416in}}%
\pgfpathlineto{\pgfqpoint{3.157215in}{1.109978in}}%
\pgfpathlineto{\pgfqpoint{3.157520in}{1.100131in}}%
\pgfpathlineto{\pgfqpoint{3.158741in}{0.885071in}}%
\pgfpathlineto{\pgfqpoint{3.159656in}{0.998436in}}%
\pgfpathlineto{\pgfqpoint{3.160571in}{1.104024in}}%
\pgfpathlineto{\pgfqpoint{3.161181in}{1.030783in}}%
\pgfpathlineto{\pgfqpoint{3.162097in}{0.900391in}}%
\pgfpathlineto{\pgfqpoint{3.162707in}{1.013153in}}%
\pgfpathlineto{\pgfqpoint{3.163622in}{1.166209in}}%
\pgfpathlineto{\pgfqpoint{3.164233in}{1.096660in}}%
\pgfpathlineto{\pgfqpoint{3.165453in}{0.808952in}}%
\pgfpathlineto{\pgfqpoint{3.166063in}{0.919674in}}%
\pgfpathlineto{\pgfqpoint{3.167284in}{1.115487in}}%
\pgfpathlineto{\pgfqpoint{3.167894in}{1.058392in}}%
\pgfpathlineto{\pgfqpoint{3.168504in}{0.968817in}}%
\pgfpathlineto{\pgfqpoint{3.169115in}{1.035366in}}%
\pgfpathlineto{\pgfqpoint{3.170335in}{1.205028in}}%
\pgfpathlineto{\pgfqpoint{3.170945in}{1.143703in}}%
\pgfpathlineto{\pgfqpoint{3.172166in}{1.010739in}}%
\pgfpathlineto{\pgfqpoint{3.173081in}{1.041428in}}%
\pgfpathlineto{\pgfqpoint{3.174912in}{0.806058in}}%
\pgfpathlineto{\pgfqpoint{3.175522in}{0.881744in}}%
\pgfpathlineto{\pgfqpoint{3.178268in}{1.267048in}}%
\pgfpathlineto{\pgfqpoint{3.178573in}{1.254549in}}%
\pgfpathlineto{\pgfqpoint{3.180099in}{0.953261in}}%
\pgfpathlineto{\pgfqpoint{3.181014in}{1.048983in}}%
\pgfpathlineto{\pgfqpoint{3.181624in}{1.084708in}}%
\pgfpathlineto{\pgfqpoint{3.182540in}{1.064069in}}%
\pgfpathlineto{\pgfqpoint{3.183760in}{1.128777in}}%
\pgfpathlineto{\pgfqpoint{3.184370in}{1.109228in}}%
\pgfpathlineto{\pgfqpoint{3.186201in}{0.973028in}}%
\pgfpathlineto{\pgfqpoint{3.186811in}{1.005490in}}%
\pgfpathlineto{\pgfqpoint{3.187422in}{1.036317in}}%
\pgfpathlineto{\pgfqpoint{3.188032in}{1.001078in}}%
\pgfpathlineto{\pgfqpoint{3.188642in}{0.969530in}}%
\pgfpathlineto{\pgfqpoint{3.188947in}{0.997932in}}%
\pgfpathlineto{\pgfqpoint{3.191998in}{1.387450in}}%
\pgfpathlineto{\pgfqpoint{3.192304in}{1.372041in}}%
\pgfpathlineto{\pgfqpoint{3.195660in}{0.824462in}}%
\pgfpathlineto{\pgfqpoint{3.196270in}{0.901765in}}%
\pgfpathlineto{\pgfqpoint{3.197491in}{1.173637in}}%
\pgfpathlineto{\pgfqpoint{3.198101in}{1.093255in}}%
\pgfpathlineto{\pgfqpoint{3.199016in}{0.927865in}}%
\pgfpathlineto{\pgfqpoint{3.199931in}{1.002350in}}%
\pgfpathlineto{\pgfqpoint{3.200542in}{1.036889in}}%
\pgfpathlineto{\pgfqpoint{3.201152in}{1.006450in}}%
\pgfpathlineto{\pgfqpoint{3.201762in}{0.957608in}}%
\pgfpathlineto{\pgfqpoint{3.202372in}{1.001938in}}%
\pgfpathlineto{\pgfqpoint{3.203593in}{1.177621in}}%
\pgfpathlineto{\pgfqpoint{3.204508in}{1.129856in}}%
\pgfpathlineto{\pgfqpoint{3.208475in}{0.906140in}}%
\pgfpathlineto{\pgfqpoint{3.208780in}{0.906787in}}%
\pgfpathlineto{\pgfqpoint{3.210306in}{1.050097in}}%
\pgfpathlineto{\pgfqpoint{3.211831in}{1.206941in}}%
\pgfpathlineto{\pgfqpoint{3.212441in}{1.182896in}}%
\pgfpathlineto{\pgfqpoint{3.213967in}{1.075247in}}%
\pgfpathlineto{\pgfqpoint{3.214882in}{1.091466in}}%
\pgfpathlineto{\pgfqpoint{3.216103in}{1.091255in}}%
\pgfpathlineto{\pgfqpoint{3.217018in}{1.137275in}}%
\pgfpathlineto{\pgfqpoint{3.217933in}{1.214757in}}%
\pgfpathlineto{\pgfqpoint{3.218544in}{1.178040in}}%
\pgfpathlineto{\pgfqpoint{3.219459in}{1.077023in}}%
\pgfpathlineto{\pgfqpoint{3.220374in}{1.126061in}}%
\pgfpathlineto{\pgfqpoint{3.220680in}{1.139193in}}%
\pgfpathlineto{\pgfqpoint{3.221595in}{1.118369in}}%
\pgfpathlineto{\pgfqpoint{3.222510in}{1.085348in}}%
\pgfpathlineto{\pgfqpoint{3.224646in}{0.808865in}}%
\pgfpathlineto{\pgfqpoint{3.225561in}{0.886101in}}%
\pgfpathlineto{\pgfqpoint{3.228002in}{1.262632in}}%
\pgfpathlineto{\pgfqpoint{3.228613in}{1.215714in}}%
\pgfpathlineto{\pgfqpoint{3.231054in}{0.981796in}}%
\pgfpathlineto{\pgfqpoint{3.231359in}{0.986534in}}%
\pgfpathlineto{\pgfqpoint{3.232884in}{1.071913in}}%
\pgfpathlineto{\pgfqpoint{3.233800in}{1.109925in}}%
\pgfpathlineto{\pgfqpoint{3.234410in}{1.078335in}}%
\pgfpathlineto{\pgfqpoint{3.235630in}{1.004422in}}%
\pgfpathlineto{\pgfqpoint{3.236241in}{1.029782in}}%
\pgfpathlineto{\pgfqpoint{3.236851in}{1.048154in}}%
\pgfpathlineto{\pgfqpoint{3.237766in}{1.030311in}}%
\pgfpathlineto{\pgfqpoint{3.238071in}{1.027781in}}%
\pgfpathlineto{\pgfqpoint{3.238376in}{1.035608in}}%
\pgfpathlineto{\pgfqpoint{3.239597in}{1.206081in}}%
\pgfpathlineto{\pgfqpoint{3.241122in}{1.442109in}}%
\pgfpathlineto{\pgfqpoint{3.241733in}{1.414350in}}%
\pgfpathlineto{\pgfqpoint{3.246309in}{0.822663in}}%
\pgfpathlineto{\pgfqpoint{3.247530in}{0.927601in}}%
\pgfpathlineto{\pgfqpoint{3.248445in}{1.006374in}}%
\pgfpathlineto{\pgfqpoint{3.249361in}{0.973423in}}%
\pgfpathlineto{\pgfqpoint{3.249666in}{0.972174in}}%
\pgfpathlineto{\pgfqpoint{3.251496in}{1.215622in}}%
\pgfpathlineto{\pgfqpoint{3.252717in}{1.142912in}}%
\pgfpathlineto{\pgfqpoint{3.256378in}{0.880554in}}%
\pgfpathlineto{\pgfqpoint{3.256989in}{0.919206in}}%
\pgfpathlineto{\pgfqpoint{3.258209in}{1.072878in}}%
\pgfpathlineto{\pgfqpoint{3.259124in}{1.027016in}}%
\pgfpathlineto{\pgfqpoint{3.260040in}{0.954437in}}%
\pgfpathlineto{\pgfqpoint{3.260650in}{0.983568in}}%
\pgfpathlineto{\pgfqpoint{3.264311in}{1.378803in}}%
\pgfpathlineto{\pgfqpoint{3.264922in}{1.366558in}}%
\pgfpathlineto{\pgfqpoint{3.266142in}{1.238778in}}%
\pgfpathlineto{\pgfqpoint{3.268278in}{0.925023in}}%
\pgfpathlineto{\pgfqpoint{3.268888in}{0.948008in}}%
\pgfpathlineto{\pgfqpoint{3.271024in}{1.065271in}}%
\pgfpathlineto{\pgfqpoint{3.271329in}{1.055194in}}%
\pgfpathlineto{\pgfqpoint{3.273160in}{0.904468in}}%
\pgfpathlineto{\pgfqpoint{3.273770in}{0.941111in}}%
\pgfpathlineto{\pgfqpoint{3.275906in}{1.220585in}}%
\pgfpathlineto{\pgfqpoint{3.276821in}{1.148771in}}%
\pgfpathlineto{\pgfqpoint{3.279262in}{0.879070in}}%
\pgfpathlineto{\pgfqpoint{3.280483in}{0.911531in}}%
\pgfpathlineto{\pgfqpoint{3.284449in}{1.329215in}}%
\pgfpathlineto{\pgfqpoint{3.285059in}{1.274265in}}%
\pgfpathlineto{\pgfqpoint{3.287195in}{0.951547in}}%
\pgfpathlineto{\pgfqpoint{3.287806in}{0.966657in}}%
\pgfpathlineto{\pgfqpoint{3.289026in}{1.128749in}}%
\pgfpathlineto{\pgfqpoint{3.290246in}{1.367288in}}%
\pgfpathlineto{\pgfqpoint{3.290857in}{1.248057in}}%
\pgfpathlineto{\pgfqpoint{3.292993in}{0.869255in}}%
\pgfpathlineto{\pgfqpoint{3.293298in}{0.873299in}}%
\pgfpathlineto{\pgfqpoint{3.294213in}{0.950195in}}%
\pgfpathlineto{\pgfqpoint{3.296654in}{1.305422in}}%
\pgfpathlineto{\pgfqpoint{3.297264in}{1.233046in}}%
\pgfpathlineto{\pgfqpoint{3.298790in}{0.933958in}}%
\pgfpathlineto{\pgfqpoint{3.299705in}{0.996462in}}%
\pgfpathlineto{\pgfqpoint{3.300315in}{0.953089in}}%
\pgfpathlineto{\pgfqpoint{3.301536in}{0.785180in}}%
\pgfpathlineto{\pgfqpoint{3.302146in}{0.871901in}}%
\pgfpathlineto{\pgfqpoint{3.303672in}{1.106739in}}%
\pgfpathlineto{\pgfqpoint{3.304282in}{1.070261in}}%
\pgfpathlineto{\pgfqpoint{3.305197in}{0.990996in}}%
\pgfpathlineto{\pgfqpoint{3.306113in}{1.032895in}}%
\pgfpathlineto{\pgfqpoint{3.306418in}{1.038330in}}%
\pgfpathlineto{\pgfqpoint{3.306723in}{1.029621in}}%
\pgfpathlineto{\pgfqpoint{3.308248in}{0.909018in}}%
\pgfpathlineto{\pgfqpoint{3.308554in}{0.943335in}}%
\pgfpathlineto{\pgfqpoint{3.310079in}{1.249537in}}%
\pgfpathlineto{\pgfqpoint{3.310994in}{1.195657in}}%
\pgfpathlineto{\pgfqpoint{3.311300in}{1.187649in}}%
\pgfpathlineto{\pgfqpoint{3.311605in}{1.198590in}}%
\pgfpathlineto{\pgfqpoint{3.312520in}{1.251702in}}%
\pgfpathlineto{\pgfqpoint{3.313130in}{1.200574in}}%
\pgfpathlineto{\pgfqpoint{3.314351in}{1.054181in}}%
\pgfpathlineto{\pgfqpoint{3.314961in}{1.114318in}}%
\pgfpathlineto{\pgfqpoint{3.315876in}{1.193832in}}%
\pgfpathlineto{\pgfqpoint{3.316181in}{1.160753in}}%
\pgfpathlineto{\pgfqpoint{3.317402in}{1.003886in}}%
\pgfpathlineto{\pgfqpoint{3.318012in}{1.047339in}}%
\pgfpathlineto{\pgfqpoint{3.318317in}{1.061368in}}%
\pgfpathlineto{\pgfqpoint{3.318928in}{1.042284in}}%
\pgfpathlineto{\pgfqpoint{3.320148in}{0.945321in}}%
\pgfpathlineto{\pgfqpoint{3.320758in}{0.984070in}}%
\pgfpathlineto{\pgfqpoint{3.321979in}{1.108620in}}%
\pgfpathlineto{\pgfqpoint{3.322589in}{1.072140in}}%
\pgfpathlineto{\pgfqpoint{3.323809in}{0.916958in}}%
\pgfpathlineto{\pgfqpoint{3.324420in}{1.007448in}}%
\pgfpathlineto{\pgfqpoint{3.325335in}{1.113880in}}%
\pgfpathlineto{\pgfqpoint{3.325945in}{1.042853in}}%
\pgfpathlineto{\pgfqpoint{3.326556in}{0.978922in}}%
\pgfpathlineto{\pgfqpoint{3.327166in}{1.050678in}}%
\pgfpathlineto{\pgfqpoint{3.327776in}{1.127877in}}%
\pgfpathlineto{\pgfqpoint{3.328386in}{1.084144in}}%
\pgfpathlineto{\pgfqpoint{3.330827in}{0.928608in}}%
\pgfpathlineto{\pgfqpoint{3.331743in}{0.995645in}}%
\pgfpathlineto{\pgfqpoint{3.333878in}{1.200729in}}%
\pgfpathlineto{\pgfqpoint{3.334489in}{1.195865in}}%
\pgfpathlineto{\pgfqpoint{3.335404in}{1.196063in}}%
\pgfpathlineto{\pgfqpoint{3.336319in}{1.166362in}}%
\pgfpathlineto{\pgfqpoint{3.339370in}{1.022356in}}%
\pgfpathlineto{\pgfqpoint{3.339981in}{1.042472in}}%
\pgfpathlineto{\pgfqpoint{3.343337in}{1.239268in}}%
\pgfpathlineto{\pgfqpoint{3.343642in}{1.211703in}}%
\pgfpathlineto{\pgfqpoint{3.345168in}{0.907769in}}%
\pgfpathlineto{\pgfqpoint{3.345778in}{0.985798in}}%
\pgfpathlineto{\pgfqpoint{3.346998in}{1.153573in}}%
\pgfpathlineto{\pgfqpoint{3.347609in}{1.119252in}}%
\pgfpathlineto{\pgfqpoint{3.350660in}{0.753638in}}%
\pgfpathlineto{\pgfqpoint{3.351270in}{0.787703in}}%
\pgfpathlineto{\pgfqpoint{3.353711in}{1.194290in}}%
\pgfpathlineto{\pgfqpoint{3.354626in}{1.167340in}}%
\pgfpathlineto{\pgfqpoint{3.355847in}{1.105958in}}%
\pgfpathlineto{\pgfqpoint{3.356457in}{1.120313in}}%
\pgfpathlineto{\pgfqpoint{3.357372in}{1.166698in}}%
\pgfpathlineto{\pgfqpoint{3.357983in}{1.135572in}}%
\pgfpathlineto{\pgfqpoint{3.360119in}{0.956899in}}%
\pgfpathlineto{\pgfqpoint{3.360729in}{0.991232in}}%
\pgfpathlineto{\pgfqpoint{3.362254in}{1.250672in}}%
\pgfpathlineto{\pgfqpoint{3.363170in}{1.148758in}}%
\pgfpathlineto{\pgfqpoint{3.364390in}{1.014935in}}%
\pgfpathlineto{\pgfqpoint{3.365306in}{1.047028in}}%
\pgfpathlineto{\pgfqpoint{3.368357in}{1.256685in}}%
\pgfpathlineto{\pgfqpoint{3.368967in}{1.197131in}}%
\pgfpathlineto{\pgfqpoint{3.371713in}{0.978460in}}%
\pgfpathlineto{\pgfqpoint{3.374154in}{0.848915in}}%
\pgfpathlineto{\pgfqpoint{3.374459in}{0.855382in}}%
\pgfpathlineto{\pgfqpoint{3.375680in}{1.025209in}}%
\pgfpathlineto{\pgfqpoint{3.377205in}{1.236708in}}%
\pgfpathlineto{\pgfqpoint{3.377815in}{1.201592in}}%
\pgfpathlineto{\pgfqpoint{3.380256in}{0.974252in}}%
\pgfpathlineto{\pgfqpoint{3.380867in}{0.990780in}}%
\pgfpathlineto{\pgfqpoint{3.382392in}{1.158840in}}%
\pgfpathlineto{\pgfqpoint{3.383002in}{1.225839in}}%
\pgfpathlineto{\pgfqpoint{3.383918in}{1.170538in}}%
\pgfpathlineto{\pgfqpoint{3.385748in}{0.995946in}}%
\pgfpathlineto{\pgfqpoint{3.386359in}{1.042764in}}%
\pgfpathlineto{\pgfqpoint{3.387884in}{1.223546in}}%
\pgfpathlineto{\pgfqpoint{3.389105in}{1.219449in}}%
\pgfpathlineto{\pgfqpoint{3.389715in}{1.254800in}}%
\pgfpathlineto{\pgfqpoint{3.390325in}{1.228372in}}%
\pgfpathlineto{\pgfqpoint{3.391851in}{1.012342in}}%
\pgfpathlineto{\pgfqpoint{3.392766in}{1.069785in}}%
\pgfpathlineto{\pgfqpoint{3.393071in}{1.086579in}}%
\pgfpathlineto{\pgfqpoint{3.393681in}{1.063690in}}%
\pgfpathlineto{\pgfqpoint{3.395207in}{0.918825in}}%
\pgfpathlineto{\pgfqpoint{3.396122in}{0.935665in}}%
\pgfpathlineto{\pgfqpoint{3.397648in}{0.993483in}}%
\pgfpathlineto{\pgfqpoint{3.398563in}{1.053429in}}%
\pgfpathlineto{\pgfqpoint{3.399174in}{1.032921in}}%
\pgfpathlineto{\pgfqpoint{3.400089in}{0.975402in}}%
\pgfpathlineto{\pgfqpoint{3.400699in}{1.003332in}}%
\pgfpathlineto{\pgfqpoint{3.402225in}{1.164307in}}%
\pgfpathlineto{\pgfqpoint{3.403140in}{1.129896in}}%
\pgfpathlineto{\pgfqpoint{3.404666in}{1.081115in}}%
\pgfpathlineto{\pgfqpoint{3.406191in}{0.956983in}}%
\pgfpathlineto{\pgfqpoint{3.406802in}{0.993347in}}%
\pgfpathlineto{\pgfqpoint{3.409548in}{1.125630in}}%
\pgfpathlineto{\pgfqpoint{3.410768in}{1.205838in}}%
\pgfpathlineto{\pgfqpoint{3.411378in}{1.183176in}}%
\pgfpathlineto{\pgfqpoint{3.412599in}{1.108447in}}%
\pgfpathlineto{\pgfqpoint{3.413514in}{1.131977in}}%
\pgfpathlineto{\pgfqpoint{3.414124in}{1.112898in}}%
\pgfpathlineto{\pgfqpoint{3.414735in}{1.077849in}}%
\pgfpathlineto{\pgfqpoint{3.415345in}{1.128560in}}%
\pgfpathlineto{\pgfqpoint{3.416260in}{1.278805in}}%
\pgfpathlineto{\pgfqpoint{3.416870in}{1.216731in}}%
\pgfpathlineto{\pgfqpoint{3.418091in}{0.954344in}}%
\pgfpathlineto{\pgfqpoint{3.419006in}{1.018717in}}%
\pgfpathlineto{\pgfqpoint{3.419311in}{1.030446in}}%
\pgfpathlineto{\pgfqpoint{3.419617in}{1.021499in}}%
\pgfpathlineto{\pgfqpoint{3.421142in}{0.830320in}}%
\pgfpathlineto{\pgfqpoint{3.422057in}{0.902052in}}%
\pgfpathlineto{\pgfqpoint{3.422973in}{0.942799in}}%
\pgfpathlineto{\pgfqpoint{3.423583in}{0.921176in}}%
\pgfpathlineto{\pgfqpoint{3.423888in}{0.922121in}}%
\pgfpathlineto{\pgfqpoint{3.424804in}{1.073424in}}%
\pgfpathlineto{\pgfqpoint{3.425719in}{1.229911in}}%
\pgfpathlineto{\pgfqpoint{3.426329in}{1.164272in}}%
\pgfpathlineto{\pgfqpoint{3.427550in}{1.004410in}}%
\pgfpathlineto{\pgfqpoint{3.428160in}{1.048227in}}%
\pgfpathlineto{\pgfqpoint{3.430906in}{1.241312in}}%
\pgfpathlineto{\pgfqpoint{3.431516in}{1.193190in}}%
\pgfpathlineto{\pgfqpoint{3.433042in}{0.968458in}}%
\pgfpathlineto{\pgfqpoint{3.433652in}{1.030793in}}%
\pgfpathlineto{\pgfqpoint{3.434567in}{1.143389in}}%
\pgfpathlineto{\pgfqpoint{3.435483in}{1.075196in}}%
\pgfpathlineto{\pgfqpoint{3.435788in}{1.053041in}}%
\pgfpathlineto{\pgfqpoint{3.436398in}{1.112928in}}%
\pgfpathlineto{\pgfqpoint{3.437313in}{1.206978in}}%
\pgfpathlineto{\pgfqpoint{3.437924in}{1.133313in}}%
\pgfpathlineto{\pgfqpoint{3.438839in}{0.969936in}}%
\pgfpathlineto{\pgfqpoint{3.439449in}{1.057583in}}%
\pgfpathlineto{\pgfqpoint{3.440365in}{1.221576in}}%
\pgfpathlineto{\pgfqpoint{3.440975in}{1.149406in}}%
\pgfpathlineto{\pgfqpoint{3.444026in}{0.844504in}}%
\pgfpathlineto{\pgfqpoint{3.444636in}{0.824430in}}%
\pgfpathlineto{\pgfqpoint{3.444941in}{0.841681in}}%
\pgfpathlineto{\pgfqpoint{3.448908in}{1.203027in}}%
\pgfpathlineto{\pgfqpoint{3.449213in}{1.196405in}}%
\pgfpathlineto{\pgfqpoint{3.450739in}{0.913854in}}%
\pgfpathlineto{\pgfqpoint{3.451654in}{1.028333in}}%
\pgfpathlineto{\pgfqpoint{3.452264in}{1.061027in}}%
\pgfpathlineto{\pgfqpoint{3.452874in}{1.009999in}}%
\pgfpathlineto{\pgfqpoint{3.453790in}{0.920433in}}%
\pgfpathlineto{\pgfqpoint{3.454400in}{0.987808in}}%
\pgfpathlineto{\pgfqpoint{3.455620in}{1.157770in}}%
\pgfpathlineto{\pgfqpoint{3.456231in}{1.091106in}}%
\pgfpathlineto{\pgfqpoint{3.457146in}{0.983420in}}%
\pgfpathlineto{\pgfqpoint{3.457756in}{1.047747in}}%
\pgfpathlineto{\pgfqpoint{3.461113in}{1.436649in}}%
\pgfpathlineto{\pgfqpoint{3.465689in}{0.928838in}}%
\pgfpathlineto{\pgfqpoint{3.466910in}{1.074121in}}%
\pgfpathlineto{\pgfqpoint{3.467825in}{0.982500in}}%
\pgfpathlineto{\pgfqpoint{3.468741in}{0.802851in}}%
\pgfpathlineto{\pgfqpoint{3.469351in}{0.855625in}}%
\pgfpathlineto{\pgfqpoint{3.470571in}{1.049388in}}%
\pgfpathlineto{\pgfqpoint{3.471181in}{1.007989in}}%
\pgfpathlineto{\pgfqpoint{3.471792in}{0.985163in}}%
\pgfpathlineto{\pgfqpoint{3.472097in}{1.009893in}}%
\pgfpathlineto{\pgfqpoint{3.473928in}{1.221172in}}%
\pgfpathlineto{\pgfqpoint{3.474538in}{1.174028in}}%
\pgfpathlineto{\pgfqpoint{3.475453in}{1.092722in}}%
\pgfpathlineto{\pgfqpoint{3.476369in}{1.119998in}}%
\pgfpathlineto{\pgfqpoint{3.477284in}{1.015908in}}%
\pgfpathlineto{\pgfqpoint{3.478199in}{0.922384in}}%
\pgfpathlineto{\pgfqpoint{3.478809in}{0.972937in}}%
\pgfpathlineto{\pgfqpoint{3.480030in}{1.098653in}}%
\pgfpathlineto{\pgfqpoint{3.480945in}{1.077127in}}%
\pgfpathlineto{\pgfqpoint{3.481556in}{1.116413in}}%
\pgfpathlineto{\pgfqpoint{3.482776in}{1.366315in}}%
\pgfpathlineto{\pgfqpoint{3.483386in}{1.313942in}}%
\pgfpathlineto{\pgfqpoint{3.484607in}{1.072882in}}%
\pgfpathlineto{\pgfqpoint{3.485217in}{1.136338in}}%
\pgfpathlineto{\pgfqpoint{3.485522in}{1.152143in}}%
\pgfpathlineto{\pgfqpoint{3.486132in}{1.104276in}}%
\pgfpathlineto{\pgfqpoint{3.487353in}{0.890000in}}%
\pgfpathlineto{\pgfqpoint{3.487963in}{0.990842in}}%
\pgfpathlineto{\pgfqpoint{3.488878in}{1.159885in}}%
\pgfpathlineto{\pgfqpoint{3.489794in}{1.070695in}}%
\pgfpathlineto{\pgfqpoint{3.492845in}{0.879640in}}%
\pgfpathlineto{\pgfqpoint{3.491014in}{1.074963in}}%
\pgfpathlineto{\pgfqpoint{3.493150in}{0.882372in}}%
\pgfpathlineto{\pgfqpoint{3.497117in}{1.223048in}}%
\pgfpathlineto{\pgfqpoint{3.497727in}{1.148982in}}%
\pgfpathlineto{\pgfqpoint{3.498947in}{0.960909in}}%
\pgfpathlineto{\pgfqpoint{3.499863in}{1.014929in}}%
\pgfpathlineto{\pgfqpoint{3.500473in}{1.047320in}}%
\pgfpathlineto{\pgfqpoint{3.501083in}{0.997329in}}%
\pgfpathlineto{\pgfqpoint{3.501998in}{0.864922in}}%
\pgfpathlineto{\pgfqpoint{3.502609in}{0.923793in}}%
\pgfpathlineto{\pgfqpoint{3.503524in}{1.041496in}}%
\pgfpathlineto{\pgfqpoint{3.504439in}{1.013004in}}%
\pgfpathlineto{\pgfqpoint{3.504744in}{1.008536in}}%
\pgfpathlineto{\pgfqpoint{3.505355in}{1.079639in}}%
\pgfpathlineto{\pgfqpoint{3.506880in}{1.482607in}}%
\pgfpathlineto{\pgfqpoint{3.507796in}{1.342331in}}%
\pgfpathlineto{\pgfqpoint{3.510847in}{0.906347in}}%
\pgfpathlineto{\pgfqpoint{3.511152in}{0.884125in}}%
\pgfpathlineto{\pgfqpoint{3.511457in}{0.907919in}}%
\pgfpathlineto{\pgfqpoint{3.512678in}{1.039686in}}%
\pgfpathlineto{\pgfqpoint{3.513288in}{1.004626in}}%
\pgfpathlineto{\pgfqpoint{3.513593in}{0.993045in}}%
\pgfpathlineto{\pgfqpoint{3.513898in}{1.006618in}}%
\pgfpathlineto{\pgfqpoint{3.515424in}{1.207604in}}%
\pgfpathlineto{\pgfqpoint{3.516034in}{1.127733in}}%
\pgfpathlineto{\pgfqpoint{3.517254in}{0.987851in}}%
\pgfpathlineto{\pgfqpoint{3.517865in}{1.020291in}}%
\pgfpathlineto{\pgfqpoint{3.518475in}{1.041586in}}%
\pgfpathlineto{\pgfqpoint{3.518780in}{1.029388in}}%
\pgfpathlineto{\pgfqpoint{3.520306in}{0.861092in}}%
\pgfpathlineto{\pgfqpoint{3.520916in}{0.931833in}}%
\pgfpathlineto{\pgfqpoint{3.524272in}{1.456894in}}%
\pgfpathlineto{\pgfqpoint{3.524882in}{1.404738in}}%
\pgfpathlineto{\pgfqpoint{3.526713in}{1.035171in}}%
\pgfpathlineto{\pgfqpoint{3.527628in}{1.039272in}}%
\pgfpathlineto{\pgfqpoint{3.529154in}{0.849038in}}%
\pgfpathlineto{\pgfqpoint{3.529764in}{0.960254in}}%
\pgfpathlineto{\pgfqpoint{3.530985in}{1.174999in}}%
\pgfpathlineto{\pgfqpoint{3.531595in}{1.121360in}}%
\pgfpathlineto{\pgfqpoint{3.531900in}{1.094353in}}%
\pgfpathlineto{\pgfqpoint{3.532510in}{1.174685in}}%
\pgfpathlineto{\pgfqpoint{3.533426in}{1.367343in}}%
\pgfpathlineto{\pgfqpoint{3.534036in}{1.267682in}}%
\pgfpathlineto{\pgfqpoint{3.535561in}{0.946859in}}%
\pgfpathlineto{\pgfqpoint{3.536172in}{1.014094in}}%
\pgfpathlineto{\pgfqpoint{3.536782in}{1.068867in}}%
\pgfpathlineto{\pgfqpoint{3.537392in}{1.041392in}}%
\pgfpathlineto{\pgfqpoint{3.538307in}{0.958061in}}%
\pgfpathlineto{\pgfqpoint{3.538918in}{1.000173in}}%
\pgfpathlineto{\pgfqpoint{3.539528in}{1.058429in}}%
\pgfpathlineto{\pgfqpoint{3.540138in}{1.011646in}}%
\pgfpathlineto{\pgfqpoint{3.541054in}{0.911721in}}%
\pgfpathlineto{\pgfqpoint{3.541359in}{0.947665in}}%
\pgfpathlineto{\pgfqpoint{3.542579in}{1.168056in}}%
\pgfpathlineto{\pgfqpoint{3.543189in}{1.104768in}}%
\pgfpathlineto{\pgfqpoint{3.544105in}{1.023633in}}%
\pgfpathlineto{\pgfqpoint{3.544715in}{1.071867in}}%
\pgfpathlineto{\pgfqpoint{3.545325in}{1.102122in}}%
\pgfpathlineto{\pgfqpoint{3.545935in}{1.053416in}}%
\pgfpathlineto{\pgfqpoint{3.546851in}{0.954712in}}%
\pgfpathlineto{\pgfqpoint{3.547461in}{1.038058in}}%
\pgfpathlineto{\pgfqpoint{3.548376in}{1.186874in}}%
\pgfpathlineto{\pgfqpoint{3.549292in}{1.103173in}}%
\pgfpathlineto{\pgfqpoint{3.549902in}{1.022973in}}%
\pgfpathlineto{\pgfqpoint{3.550817in}{1.097157in}}%
\pgfpathlineto{\pgfqpoint{3.551428in}{1.129623in}}%
\pgfpathlineto{\pgfqpoint{3.551733in}{1.105091in}}%
\pgfpathlineto{\pgfqpoint{3.552953in}{0.924671in}}%
\pgfpathlineto{\pgfqpoint{3.553563in}{0.995551in}}%
\pgfpathlineto{\pgfqpoint{3.554784in}{1.186339in}}%
\pgfpathlineto{\pgfqpoint{3.555699in}{1.158414in}}%
\pgfpathlineto{\pgfqpoint{3.556309in}{1.193181in}}%
\pgfpathlineto{\pgfqpoint{3.557530in}{1.358603in}}%
\pgfpathlineto{\pgfqpoint{3.558140in}{1.318459in}}%
\pgfpathlineto{\pgfqpoint{3.562412in}{0.753256in}}%
\pgfpathlineto{\pgfqpoint{3.563632in}{0.975618in}}%
\pgfpathlineto{\pgfqpoint{3.566683in}{1.359568in}}%
\pgfpathlineto{\pgfqpoint{3.566989in}{1.340739in}}%
\pgfpathlineto{\pgfqpoint{3.570955in}{0.902117in}}%
\pgfpathlineto{\pgfqpoint{3.571565in}{0.946760in}}%
\pgfpathlineto{\pgfqpoint{3.572786in}{1.080109in}}%
\pgfpathlineto{\pgfqpoint{3.573396in}{1.042777in}}%
\pgfpathlineto{\pgfqpoint{3.573701in}{1.029611in}}%
\pgfpathlineto{\pgfqpoint{3.574311in}{1.062422in}}%
\pgfpathlineto{\pgfqpoint{3.575532in}{1.160570in}}%
\pgfpathlineto{\pgfqpoint{3.576142in}{1.102766in}}%
\pgfpathlineto{\pgfqpoint{3.576752in}{1.045500in}}%
\pgfpathlineto{\pgfqpoint{3.577363in}{1.108691in}}%
\pgfpathlineto{\pgfqpoint{3.578583in}{1.321355in}}%
\pgfpathlineto{\pgfqpoint{3.579193in}{1.228548in}}%
\pgfpathlineto{\pgfqpoint{3.580414in}{1.003420in}}%
\pgfpathlineto{\pgfqpoint{3.581024in}{1.049898in}}%
\pgfpathlineto{\pgfqpoint{3.581329in}{1.066472in}}%
\pgfpathlineto{\pgfqpoint{3.581939in}{1.046108in}}%
\pgfpathlineto{\pgfqpoint{3.582855in}{0.947808in}}%
\pgfpathlineto{\pgfqpoint{3.583465in}{0.991985in}}%
\pgfpathlineto{\pgfqpoint{3.584075in}{1.080243in}}%
\pgfpathlineto{\pgfqpoint{3.584685in}{1.035790in}}%
\pgfpathlineto{\pgfqpoint{3.585906in}{0.860570in}}%
\pgfpathlineto{\pgfqpoint{3.586516in}{0.903198in}}%
\pgfpathlineto{\pgfqpoint{3.590483in}{1.280646in}}%
\pgfpathlineto{\pgfqpoint{3.590788in}{1.277703in}}%
\pgfpathlineto{\pgfqpoint{3.594754in}{0.783531in}}%
\pgfpathlineto{\pgfqpoint{3.595670in}{0.901600in}}%
\pgfpathlineto{\pgfqpoint{3.596585in}{1.053066in}}%
\pgfpathlineto{\pgfqpoint{3.597500in}{0.976960in}}%
\pgfpathlineto{\pgfqpoint{3.597806in}{0.961407in}}%
\pgfpathlineto{\pgfqpoint{3.598111in}{0.978187in}}%
\pgfpathlineto{\pgfqpoint{3.599331in}{1.120859in}}%
\pgfpathlineto{\pgfqpoint{3.600246in}{1.072363in}}%
\pgfpathlineto{\pgfqpoint{3.600552in}{1.059109in}}%
\pgfpathlineto{\pgfqpoint{3.600857in}{1.076329in}}%
\pgfpathlineto{\pgfqpoint{3.602077in}{1.320445in}}%
\pgfpathlineto{\pgfqpoint{3.602993in}{1.203967in}}%
\pgfpathlineto{\pgfqpoint{3.605128in}{1.091341in}}%
\pgfpathlineto{\pgfqpoint{3.606654in}{0.970732in}}%
\pgfpathlineto{\pgfqpoint{3.607264in}{1.013081in}}%
\pgfpathlineto{\pgfqpoint{3.608485in}{1.116821in}}%
\pgfpathlineto{\pgfqpoint{3.609095in}{1.055123in}}%
\pgfpathlineto{\pgfqpoint{3.610010in}{0.968493in}}%
\pgfpathlineto{\pgfqpoint{3.610926in}{1.000330in}}%
\pgfpathlineto{\pgfqpoint{3.611536in}{1.007463in}}%
\pgfpathlineto{\pgfqpoint{3.612146in}{0.997855in}}%
\pgfpathlineto{\pgfqpoint{3.612451in}{0.995329in}}%
\pgfpathlineto{\pgfqpoint{3.612756in}{1.005129in}}%
\pgfpathlineto{\pgfqpoint{3.614587in}{1.339140in}}%
\pgfpathlineto{\pgfqpoint{3.615197in}{1.216278in}}%
\pgfpathlineto{\pgfqpoint{3.616418in}{1.024354in}}%
\pgfpathlineto{\pgfqpoint{3.617333in}{1.032923in}}%
\pgfpathlineto{\pgfqpoint{3.618859in}{0.833991in}}%
\pgfpathlineto{\pgfqpoint{3.619469in}{0.901090in}}%
\pgfpathlineto{\pgfqpoint{3.623435in}{1.311511in}}%
\pgfpathlineto{\pgfqpoint{3.623741in}{1.293971in}}%
\pgfpathlineto{\pgfqpoint{3.625876in}{1.127301in}}%
\pgfpathlineto{\pgfqpoint{3.626487in}{1.127737in}}%
\pgfpathlineto{\pgfqpoint{3.627707in}{1.085395in}}%
\pgfpathlineto{\pgfqpoint{3.628317in}{1.103397in}}%
\pgfpathlineto{\pgfqpoint{3.628622in}{1.117649in}}%
\pgfpathlineto{\pgfqpoint{3.629233in}{1.091485in}}%
\pgfpathlineto{\pgfqpoint{3.630758in}{0.874840in}}%
\pgfpathlineto{\pgfqpoint{3.631369in}{0.927303in}}%
\pgfpathlineto{\pgfqpoint{3.633809in}{1.209422in}}%
\pgfpathlineto{\pgfqpoint{3.634115in}{1.196347in}}%
\pgfpathlineto{\pgfqpoint{3.636250in}{1.015749in}}%
\pgfpathlineto{\pgfqpoint{3.637166in}{1.061178in}}%
\pgfpathlineto{\pgfqpoint{3.638996in}{1.227199in}}%
\pgfpathlineto{\pgfqpoint{3.639607in}{1.149931in}}%
\pgfpathlineto{\pgfqpoint{3.640827in}{1.041267in}}%
\pgfpathlineto{\pgfqpoint{3.641437in}{1.055596in}}%
\pgfpathlineto{\pgfqpoint{3.642048in}{1.033514in}}%
\pgfpathlineto{\pgfqpoint{3.643268in}{0.899560in}}%
\pgfpathlineto{\pgfqpoint{3.644183in}{0.952786in}}%
\pgfpathlineto{\pgfqpoint{3.644794in}{0.985066in}}%
\pgfpathlineto{\pgfqpoint{3.645709in}{0.966425in}}%
\pgfpathlineto{\pgfqpoint{3.646319in}{1.002393in}}%
\pgfpathlineto{\pgfqpoint{3.647540in}{1.190520in}}%
\pgfpathlineto{\pgfqpoint{3.648150in}{1.139599in}}%
\pgfpathlineto{\pgfqpoint{3.649370in}{1.000946in}}%
\pgfpathlineto{\pgfqpoint{3.649981in}{1.042789in}}%
\pgfpathlineto{\pgfqpoint{3.653032in}{1.195385in}}%
\pgfpathlineto{\pgfqpoint{3.653947in}{1.135640in}}%
\pgfpathlineto{\pgfqpoint{3.655778in}{0.997974in}}%
\pgfpathlineto{\pgfqpoint{3.656388in}{1.015179in}}%
\pgfpathlineto{\pgfqpoint{3.657609in}{1.089841in}}%
\pgfpathlineto{\pgfqpoint{3.658219in}{1.068369in}}%
\pgfpathlineto{\pgfqpoint{3.658524in}{1.056818in}}%
\pgfpathlineto{\pgfqpoint{3.659134in}{1.075695in}}%
\pgfpathlineto{\pgfqpoint{3.659744in}{1.105205in}}%
\pgfpathlineto{\pgfqpoint{3.660355in}{1.071784in}}%
\pgfpathlineto{\pgfqpoint{3.661270in}{1.019280in}}%
\pgfpathlineto{\pgfqpoint{3.661880in}{1.068359in}}%
\pgfpathlineto{\pgfqpoint{3.665237in}{1.299204in}}%
\pgfpathlineto{\pgfqpoint{3.666762in}{1.014532in}}%
\pgfpathlineto{\pgfqpoint{3.668898in}{0.828796in}}%
\pgfpathlineto{\pgfqpoint{3.669203in}{0.834462in}}%
\pgfpathlineto{\pgfqpoint{3.670424in}{0.995055in}}%
\pgfpathlineto{\pgfqpoint{3.671949in}{1.227751in}}%
\pgfpathlineto{\pgfqpoint{3.672865in}{1.204574in}}%
\pgfpathlineto{\pgfqpoint{3.673170in}{1.204419in}}%
\pgfpathlineto{\pgfqpoint{3.673780in}{1.213679in}}%
\pgfpathlineto{\pgfqpoint{3.674085in}{1.206265in}}%
\pgfpathlineto{\pgfqpoint{3.675916in}{1.064519in}}%
\pgfpathlineto{\pgfqpoint{3.676831in}{1.092939in}}%
\pgfpathlineto{\pgfqpoint{3.677746in}{1.121987in}}%
\pgfpathlineto{\pgfqpoint{3.678357in}{1.109156in}}%
\pgfpathlineto{\pgfqpoint{3.682323in}{0.946256in}}%
\pgfpathlineto{\pgfqpoint{3.685374in}{1.246923in}}%
\pgfpathlineto{\pgfqpoint{3.686595in}{1.435073in}}%
\pgfpathlineto{\pgfqpoint{3.687205in}{1.368635in}}%
\pgfpathlineto{\pgfqpoint{3.690256in}{0.924903in}}%
\pgfpathlineto{\pgfqpoint{3.690561in}{0.925108in}}%
\pgfpathlineto{\pgfqpoint{3.691477in}{1.000315in}}%
\pgfpathlineto{\pgfqpoint{3.692087in}{1.053943in}}%
\pgfpathlineto{\pgfqpoint{3.692697in}{0.994102in}}%
\pgfpathlineto{\pgfqpoint{3.693918in}{0.838404in}}%
\pgfpathlineto{\pgfqpoint{3.694528in}{0.895067in}}%
\pgfpathlineto{\pgfqpoint{3.697884in}{1.296723in}}%
\pgfpathlineto{\pgfqpoint{3.698189in}{1.288192in}}%
\pgfpathlineto{\pgfqpoint{3.699715in}{1.015827in}}%
\pgfpathlineto{\pgfqpoint{3.700630in}{1.155601in}}%
\pgfpathlineto{\pgfqpoint{3.701241in}{1.208817in}}%
\pgfpathlineto{\pgfqpoint{3.701851in}{1.126374in}}%
\pgfpathlineto{\pgfqpoint{3.703071in}{0.864782in}}%
\pgfpathlineto{\pgfqpoint{3.703987in}{0.942567in}}%
\pgfpathlineto{\pgfqpoint{3.704597in}{0.995060in}}%
\pgfpathlineto{\pgfqpoint{3.705512in}{0.956601in}}%
\pgfpathlineto{\pgfqpoint{3.705817in}{0.953692in}}%
\pgfpathlineto{\pgfqpoint{3.710394in}{1.348961in}}%
\pgfpathlineto{\pgfqpoint{3.711004in}{1.292083in}}%
\pgfpathlineto{\pgfqpoint{3.714056in}{0.868195in}}%
\pgfpathlineto{\pgfqpoint{3.714361in}{0.876755in}}%
\pgfpathlineto{\pgfqpoint{3.715581in}{1.005941in}}%
\pgfpathlineto{\pgfqpoint{3.715581in}{1.005941in}}%
\pgfusepath{stroke}%
\end{pgfscope}%
\begin{pgfscope}%
\pgfpathrectangle{\pgfqpoint{0.664400in}{0.567611in}}{\pgfqpoint{3.051181in}{1.188976in}}%
\pgfusepath{clip}%
\pgfsetrectcap%
\pgfsetroundjoin%
\pgfsetlinewidth{2.007500pt}%
\definecolor{currentstroke}{rgb}{0.000000,0.000000,0.000000}%
\pgfsetstrokecolor{currentstroke}%
\pgfsetdash{}{0pt}%
\pgfpathmoveto{\pgfqpoint{0.816959in}{1.097795in}}%
\pgfpathlineto{\pgfqpoint{0.821231in}{1.113439in}}%
\pgfpathlineto{\pgfqpoint{0.824587in}{1.113566in}}%
\pgfpathlineto{\pgfqpoint{0.828859in}{1.106432in}}%
\pgfpathlineto{\pgfqpoint{0.830994in}{1.105561in}}%
\pgfpathlineto{\pgfqpoint{0.834351in}{1.094688in}}%
\pgfpathlineto{\pgfqpoint{0.836792in}{1.091140in}}%
\pgfpathlineto{\pgfqpoint{0.838622in}{1.091929in}}%
\pgfpathlineto{\pgfqpoint{0.843199in}{1.102724in}}%
\pgfpathlineto{\pgfqpoint{0.848386in}{1.109797in}}%
\pgfpathlineto{\pgfqpoint{0.850522in}{1.113553in}}%
\pgfpathlineto{\pgfqpoint{0.852048in}{1.115548in}}%
\pgfpathlineto{\pgfqpoint{0.852658in}{1.115135in}}%
\pgfpathlineto{\pgfqpoint{0.854794in}{1.111528in}}%
\pgfpathlineto{\pgfqpoint{0.859981in}{1.091877in}}%
\pgfpathlineto{\pgfqpoint{0.860591in}{1.092208in}}%
\pgfpathlineto{\pgfqpoint{0.863032in}{1.094226in}}%
\pgfpathlineto{\pgfqpoint{0.868219in}{1.107959in}}%
\pgfpathlineto{\pgfqpoint{0.868829in}{1.107736in}}%
\pgfpathlineto{\pgfqpoint{0.873711in}{1.107762in}}%
\pgfpathlineto{\pgfqpoint{0.877067in}{1.111678in}}%
\pgfpathlineto{\pgfqpoint{0.879813in}{1.108830in}}%
\pgfpathlineto{\pgfqpoint{0.886221in}{1.094954in}}%
\pgfpathlineto{\pgfqpoint{0.887136in}{1.096054in}}%
\pgfpathlineto{\pgfqpoint{0.894459in}{1.110920in}}%
\pgfpathlineto{\pgfqpoint{0.895069in}{1.110542in}}%
\pgfpathlineto{\pgfqpoint{0.898426in}{1.106933in}}%
\pgfpathlineto{\pgfqpoint{0.899646in}{1.108946in}}%
\pgfpathlineto{\pgfqpoint{0.900867in}{1.109220in}}%
\pgfpathlineto{\pgfqpoint{0.901172in}{1.108816in}}%
\pgfpathlineto{\pgfqpoint{0.910020in}{1.095550in}}%
\pgfpathlineto{\pgfqpoint{0.910630in}{1.096003in}}%
\pgfpathlineto{\pgfqpoint{0.913376in}{1.101319in}}%
\pgfpathlineto{\pgfqpoint{0.917953in}{1.109699in}}%
\pgfpathlineto{\pgfqpoint{0.919174in}{1.108656in}}%
\pgfpathlineto{\pgfqpoint{0.923750in}{1.101180in}}%
\pgfpathlineto{\pgfqpoint{0.924056in}{1.101332in}}%
\pgfpathlineto{\pgfqpoint{0.927412in}{1.102153in}}%
\pgfpathlineto{\pgfqpoint{0.929243in}{1.099294in}}%
\pgfpathlineto{\pgfqpoint{0.933819in}{1.087222in}}%
\pgfpathlineto{\pgfqpoint{0.934430in}{1.087864in}}%
\pgfpathlineto{\pgfqpoint{0.936565in}{1.093361in}}%
\pgfpathlineto{\pgfqpoint{0.941142in}{1.105914in}}%
\pgfpathlineto{\pgfqpoint{0.943888in}{1.103046in}}%
\pgfpathlineto{\pgfqpoint{0.947550in}{1.095574in}}%
\pgfpathlineto{\pgfqpoint{0.948160in}{1.096092in}}%
\pgfpathlineto{\pgfqpoint{0.951821in}{1.100784in}}%
\pgfpathlineto{\pgfqpoint{0.952126in}{1.100534in}}%
\pgfpathlineto{\pgfqpoint{0.954262in}{1.094945in}}%
\pgfpathlineto{\pgfqpoint{0.957924in}{1.086303in}}%
\pgfpathlineto{\pgfqpoint{0.958229in}{1.086613in}}%
\pgfpathlineto{\pgfqpoint{0.960670in}{1.092958in}}%
\pgfpathlineto{\pgfqpoint{0.965246in}{1.104725in}}%
\pgfpathlineto{\pgfqpoint{0.967077in}{1.104304in}}%
\pgfpathlineto{\pgfqpoint{0.972874in}{1.099337in}}%
\pgfpathlineto{\pgfqpoint{0.973180in}{1.099661in}}%
\pgfpathlineto{\pgfqpoint{0.975315in}{1.100439in}}%
\pgfpathlineto{\pgfqpoint{0.982638in}{1.091851in}}%
\pgfpathlineto{\pgfqpoint{0.982943in}{1.092132in}}%
\pgfpathlineto{\pgfqpoint{0.985384in}{1.098012in}}%
\pgfpathlineto{\pgfqpoint{0.988435in}{1.102540in}}%
\pgfpathlineto{\pgfqpoint{0.992402in}{1.103158in}}%
\pgfpathlineto{\pgfqpoint{1.000335in}{1.105488in}}%
\pgfpathlineto{\pgfqpoint{1.003081in}{1.096915in}}%
\pgfpathlineto{\pgfqpoint{1.006437in}{1.089744in}}%
\pgfpathlineto{\pgfqpoint{1.007658in}{1.092316in}}%
\pgfpathlineto{\pgfqpoint{1.011930in}{1.102278in}}%
\pgfpathlineto{\pgfqpoint{1.015286in}{1.101989in}}%
\pgfpathlineto{\pgfqpoint{1.018032in}{1.099878in}}%
\pgfpathlineto{\pgfqpoint{1.018337in}{1.100117in}}%
\pgfpathlineto{\pgfqpoint{1.023219in}{1.104210in}}%
\pgfpathlineto{\pgfqpoint{1.023524in}{1.103865in}}%
\pgfpathlineto{\pgfqpoint{1.025965in}{1.098753in}}%
\pgfpathlineto{\pgfqpoint{1.029931in}{1.089369in}}%
\pgfpathlineto{\pgfqpoint{1.031152in}{1.090058in}}%
\pgfpathlineto{\pgfqpoint{1.037559in}{1.097111in}}%
\pgfpathlineto{\pgfqpoint{1.040611in}{1.097170in}}%
\pgfpathlineto{\pgfqpoint{1.043052in}{1.100673in}}%
\pgfpathlineto{\pgfqpoint{1.045798in}{1.103994in}}%
\pgfpathlineto{\pgfqpoint{1.047933in}{1.102763in}}%
\pgfpathlineto{\pgfqpoint{1.050069in}{1.099375in}}%
\pgfpathlineto{\pgfqpoint{1.053731in}{1.087245in}}%
\pgfpathlineto{\pgfqpoint{1.054341in}{1.087794in}}%
\pgfpathlineto{\pgfqpoint{1.059528in}{1.095436in}}%
\pgfpathlineto{\pgfqpoint{1.061664in}{1.093643in}}%
\pgfpathlineto{\pgfqpoint{1.063800in}{1.092376in}}%
\pgfpathlineto{\pgfqpoint{1.069597in}{1.097004in}}%
\pgfpathlineto{\pgfqpoint{1.072953in}{1.102278in}}%
\pgfpathlineto{\pgfqpoint{1.074174in}{1.101075in}}%
\pgfpathlineto{\pgfqpoint{1.080581in}{1.088167in}}%
\pgfpathlineto{\pgfqpoint{1.081191in}{1.089027in}}%
\pgfpathlineto{\pgfqpoint{1.086989in}{1.098789in}}%
\pgfpathlineto{\pgfqpoint{1.088819in}{1.096612in}}%
\pgfpathlineto{\pgfqpoint{1.091565in}{1.094547in}}%
\pgfpathlineto{\pgfqpoint{1.092786in}{1.096070in}}%
\pgfpathlineto{\pgfqpoint{1.097973in}{1.110272in}}%
\pgfpathlineto{\pgfqpoint{1.099193in}{1.109247in}}%
\pgfpathlineto{\pgfqpoint{1.100719in}{1.105570in}}%
\pgfpathlineto{\pgfqpoint{1.104991in}{1.091333in}}%
\pgfpathlineto{\pgfqpoint{1.105296in}{1.091658in}}%
\pgfpathlineto{\pgfqpoint{1.109872in}{1.101570in}}%
\pgfpathlineto{\pgfqpoint{1.111093in}{1.100377in}}%
\pgfpathlineto{\pgfqpoint{1.114449in}{1.096378in}}%
\pgfpathlineto{\pgfqpoint{1.115059in}{1.096991in}}%
\pgfpathlineto{\pgfqpoint{1.116890in}{1.100870in}}%
\pgfpathlineto{\pgfqpoint{1.120246in}{1.108961in}}%
\pgfpathlineto{\pgfqpoint{1.121772in}{1.109312in}}%
\pgfpathlineto{\pgfqpoint{1.122077in}{1.108922in}}%
\pgfpathlineto{\pgfqpoint{1.128790in}{1.099926in}}%
\pgfpathlineto{\pgfqpoint{1.129095in}{1.100182in}}%
\pgfpathlineto{\pgfqpoint{1.133061in}{1.103182in}}%
\pgfpathlineto{\pgfqpoint{1.133367in}{1.102859in}}%
\pgfpathlineto{\pgfqpoint{1.137638in}{1.096296in}}%
\pgfpathlineto{\pgfqpoint{1.138248in}{1.097215in}}%
\pgfpathlineto{\pgfqpoint{1.140384in}{1.106011in}}%
\pgfpathlineto{\pgfqpoint{1.143741in}{1.121228in}}%
\pgfpathlineto{\pgfqpoint{1.144046in}{1.121093in}}%
\pgfpathlineto{\pgfqpoint{1.145876in}{1.116871in}}%
\pgfpathlineto{\pgfqpoint{1.151674in}{1.104325in}}%
\pgfpathlineto{\pgfqpoint{1.158386in}{1.107854in}}%
\pgfpathlineto{\pgfqpoint{1.161437in}{1.102097in}}%
\pgfpathlineto{\pgfqpoint{1.162658in}{1.103778in}}%
\pgfpathlineto{\pgfqpoint{1.168760in}{1.114630in}}%
\pgfpathlineto{\pgfqpoint{1.170286in}{1.112767in}}%
\pgfpathlineto{\pgfqpoint{1.174557in}{1.107104in}}%
\pgfpathlineto{\pgfqpoint{1.184321in}{1.103931in}}%
\pgfpathlineto{\pgfqpoint{1.192559in}{1.114686in}}%
\pgfpathlineto{\pgfqpoint{1.193780in}{1.113202in}}%
\pgfpathlineto{\pgfqpoint{1.195611in}{1.109655in}}%
\pgfpathlineto{\pgfqpoint{1.198662in}{1.103256in}}%
\pgfpathlineto{\pgfqpoint{1.198967in}{1.103368in}}%
\pgfpathlineto{\pgfqpoint{1.202018in}{1.104258in}}%
\pgfpathlineto{\pgfqpoint{1.204459in}{1.104299in}}%
\pgfpathlineto{\pgfqpoint{1.206595in}{1.106135in}}%
\pgfpathlineto{\pgfqpoint{1.207205in}{1.105336in}}%
\pgfpathlineto{\pgfqpoint{1.209951in}{1.103680in}}%
\pgfpathlineto{\pgfqpoint{1.212392in}{1.104444in}}%
\pgfpathlineto{\pgfqpoint{1.214223in}{1.107032in}}%
\pgfpathlineto{\pgfqpoint{1.218800in}{1.114529in}}%
\pgfpathlineto{\pgfqpoint{1.220935in}{1.113705in}}%
\pgfpathlineto{\pgfqpoint{1.226122in}{1.100397in}}%
\pgfpathlineto{\pgfqpoint{1.226733in}{1.100685in}}%
\pgfpathlineto{\pgfqpoint{1.233750in}{1.102611in}}%
\pgfpathlineto{\pgfqpoint{1.236191in}{1.100935in}}%
\pgfpathlineto{\pgfqpoint{1.237412in}{1.103076in}}%
\pgfpathlineto{\pgfqpoint{1.241378in}{1.110716in}}%
\pgfpathlineto{\pgfqpoint{1.242904in}{1.110145in}}%
\pgfpathlineto{\pgfqpoint{1.244735in}{1.105782in}}%
\pgfpathlineto{\pgfqpoint{1.249006in}{1.093008in}}%
\pgfpathlineto{\pgfqpoint{1.250227in}{1.094134in}}%
\pgfpathlineto{\pgfqpoint{1.253888in}{1.097809in}}%
\pgfpathlineto{\pgfqpoint{1.254193in}{1.097465in}}%
\pgfpathlineto{\pgfqpoint{1.258465in}{1.092516in}}%
\pgfpathlineto{\pgfqpoint{1.258770in}{1.092748in}}%
\pgfpathlineto{\pgfqpoint{1.261821in}{1.099261in}}%
\pgfpathlineto{\pgfqpoint{1.265178in}{1.104555in}}%
\pgfpathlineto{\pgfqpoint{1.266703in}{1.103335in}}%
\pgfpathlineto{\pgfqpoint{1.269144in}{1.097499in}}%
\pgfpathlineto{\pgfqpoint{1.272500in}{1.089402in}}%
\pgfpathlineto{\pgfqpoint{1.273721in}{1.090436in}}%
\pgfpathlineto{\pgfqpoint{1.275857in}{1.092757in}}%
\pgfpathlineto{\pgfqpoint{1.276162in}{1.092577in}}%
\pgfpathlineto{\pgfqpoint{1.278603in}{1.088624in}}%
\pgfpathlineto{\pgfqpoint{1.280739in}{1.086136in}}%
\pgfpathlineto{\pgfqpoint{1.281044in}{1.086282in}}%
\pgfpathlineto{\pgfqpoint{1.284400in}{1.090252in}}%
\pgfpathlineto{\pgfqpoint{1.289892in}{1.097377in}}%
\pgfpathlineto{\pgfqpoint{1.290807in}{1.095336in}}%
\pgfpathlineto{\pgfqpoint{1.296910in}{1.078041in}}%
\pgfpathlineto{\pgfqpoint{1.297825in}{1.078766in}}%
\pgfpathlineto{\pgfqpoint{1.299961in}{1.078815in}}%
\pgfpathlineto{\pgfqpoint{1.303928in}{1.077147in}}%
\pgfpathlineto{\pgfqpoint{1.311861in}{1.082059in}}%
\pgfpathlineto{\pgfqpoint{1.314607in}{1.084440in}}%
\pgfpathlineto{\pgfqpoint{1.318573in}{1.080966in}}%
\pgfpathlineto{\pgfqpoint{1.328032in}{1.074599in}}%
\pgfpathlineto{\pgfqpoint{1.330168in}{1.075137in}}%
\pgfpathlineto{\pgfqpoint{1.337185in}{1.082208in}}%
\pgfpathlineto{\pgfqpoint{1.339931in}{1.082403in}}%
\pgfpathlineto{\pgfqpoint{1.342372in}{1.078902in}}%
\pgfpathlineto{\pgfqpoint{1.346339in}{1.074976in}}%
\pgfpathlineto{\pgfqpoint{1.348170in}{1.074613in}}%
\pgfpathlineto{\pgfqpoint{1.348475in}{1.075093in}}%
\pgfpathlineto{\pgfqpoint{1.351221in}{1.077705in}}%
\pgfpathlineto{\pgfqpoint{1.354882in}{1.074675in}}%
\pgfpathlineto{\pgfqpoint{1.355493in}{1.075618in}}%
\pgfpathlineto{\pgfqpoint{1.358849in}{1.079102in}}%
\pgfpathlineto{\pgfqpoint{1.367697in}{1.081766in}}%
\pgfpathlineto{\pgfqpoint{1.369223in}{1.079303in}}%
\pgfpathlineto{\pgfqpoint{1.371969in}{1.076944in}}%
\pgfpathlineto{\pgfqpoint{1.376546in}{1.074075in}}%
\pgfpathlineto{\pgfqpoint{1.378681in}{1.075976in}}%
\pgfpathlineto{\pgfqpoint{1.382038in}{1.080060in}}%
\pgfpathlineto{\pgfqpoint{1.384174in}{1.078331in}}%
\pgfpathlineto{\pgfqpoint{1.385394in}{1.078605in}}%
\pgfpathlineto{\pgfqpoint{1.385699in}{1.079035in}}%
\pgfpathlineto{\pgfqpoint{1.389361in}{1.081489in}}%
\pgfpathlineto{\pgfqpoint{1.396683in}{1.078160in}}%
\pgfpathlineto{\pgfqpoint{1.398209in}{1.079740in}}%
\pgfpathlineto{\pgfqpoint{1.398819in}{1.079137in}}%
\pgfpathlineto{\pgfqpoint{1.405227in}{1.072587in}}%
\pgfpathlineto{\pgfqpoint{1.406447in}{1.072945in}}%
\pgfpathlineto{\pgfqpoint{1.411024in}{1.080634in}}%
\pgfpathlineto{\pgfqpoint{1.411939in}{1.080002in}}%
\pgfpathlineto{\pgfqpoint{1.420788in}{1.076782in}}%
\pgfpathlineto{\pgfqpoint{1.422619in}{1.078573in}}%
\pgfpathlineto{\pgfqpoint{1.423229in}{1.077883in}}%
\pgfpathlineto{\pgfqpoint{1.427500in}{1.073404in}}%
\pgfpathlineto{\pgfqpoint{1.427806in}{1.073791in}}%
\pgfpathlineto{\pgfqpoint{1.431772in}{1.077790in}}%
\pgfpathlineto{\pgfqpoint{1.432077in}{1.077497in}}%
\pgfpathlineto{\pgfqpoint{1.438180in}{1.072033in}}%
\pgfpathlineto{\pgfqpoint{1.441231in}{1.074439in}}%
\pgfpathlineto{\pgfqpoint{1.444892in}{1.075075in}}%
\pgfpathlineto{\pgfqpoint{1.446723in}{1.073520in}}%
\pgfpathlineto{\pgfqpoint{1.448859in}{1.072599in}}%
\pgfpathlineto{\pgfqpoint{1.454961in}{1.074306in}}%
\pgfpathlineto{\pgfqpoint{1.455876in}{1.074329in}}%
\pgfpathlineto{\pgfqpoint{1.456181in}{1.073910in}}%
\pgfpathlineto{\pgfqpoint{1.460148in}{1.069913in}}%
\pgfpathlineto{\pgfqpoint{1.461979in}{1.070946in}}%
\pgfpathlineto{\pgfqpoint{1.463504in}{1.074471in}}%
\pgfpathlineto{\pgfqpoint{1.464725in}{1.077200in}}%
\pgfpathlineto{\pgfqpoint{1.465640in}{1.076359in}}%
\pgfpathlineto{\pgfqpoint{1.471743in}{1.072697in}}%
\pgfpathlineto{\pgfqpoint{1.476930in}{1.073081in}}%
\pgfpathlineto{\pgfqpoint{1.482422in}{1.069673in}}%
\pgfpathlineto{\pgfqpoint{1.485168in}{1.070460in}}%
\pgfpathlineto{\pgfqpoint{1.488524in}{1.073343in}}%
\pgfpathlineto{\pgfqpoint{1.488829in}{1.073043in}}%
\pgfpathlineto{\pgfqpoint{1.492491in}{1.069800in}}%
\pgfpathlineto{\pgfqpoint{1.492796in}{1.070091in}}%
\pgfpathlineto{\pgfqpoint{1.497067in}{1.073731in}}%
\pgfpathlineto{\pgfqpoint{1.498593in}{1.072010in}}%
\pgfpathlineto{\pgfqpoint{1.503170in}{1.063643in}}%
\pgfpathlineto{\pgfqpoint{1.503780in}{1.064352in}}%
\pgfpathlineto{\pgfqpoint{1.507746in}{1.071773in}}%
\pgfpathlineto{\pgfqpoint{1.508662in}{1.070564in}}%
\pgfpathlineto{\pgfqpoint{1.511103in}{1.067711in}}%
\pgfpathlineto{\pgfqpoint{1.511408in}{1.067905in}}%
\pgfpathlineto{\pgfqpoint{1.517205in}{1.072101in}}%
\pgfpathlineto{\pgfqpoint{1.521172in}{1.070471in}}%
\pgfpathlineto{\pgfqpoint{1.525443in}{1.066420in}}%
\pgfpathlineto{\pgfqpoint{1.526054in}{1.067438in}}%
\pgfpathlineto{\pgfqpoint{1.529105in}{1.071175in}}%
\pgfpathlineto{\pgfqpoint{1.530630in}{1.069886in}}%
\pgfpathlineto{\pgfqpoint{1.534597in}{1.062450in}}%
\pgfpathlineto{\pgfqpoint{1.534902in}{1.062731in}}%
\pgfpathlineto{\pgfqpoint{1.537038in}{1.068813in}}%
\pgfpathlineto{\pgfqpoint{1.540699in}{1.077323in}}%
\pgfpathlineto{\pgfqpoint{1.541309in}{1.076642in}}%
\pgfpathlineto{\pgfqpoint{1.544971in}{1.069403in}}%
\pgfpathlineto{\pgfqpoint{1.545886in}{1.070314in}}%
\pgfpathlineto{\pgfqpoint{1.549243in}{1.072446in}}%
\pgfpathlineto{\pgfqpoint{1.549548in}{1.072100in}}%
\pgfpathlineto{\pgfqpoint{1.553514in}{1.069708in}}%
\pgfpathlineto{\pgfqpoint{1.558091in}{1.070072in}}%
\pgfpathlineto{\pgfqpoint{1.562057in}{1.075221in}}%
\pgfpathlineto{\pgfqpoint{1.562973in}{1.074347in}}%
\pgfpathlineto{\pgfqpoint{1.568465in}{1.067855in}}%
\pgfpathlineto{\pgfqpoint{1.569075in}{1.068682in}}%
\pgfpathlineto{\pgfqpoint{1.572737in}{1.073790in}}%
\pgfpathlineto{\pgfqpoint{1.574262in}{1.071114in}}%
\pgfpathlineto{\pgfqpoint{1.577313in}{1.066995in}}%
\pgfpathlineto{\pgfqpoint{1.579754in}{1.066661in}}%
\pgfpathlineto{\pgfqpoint{1.581280in}{1.067901in}}%
\pgfpathlineto{\pgfqpoint{1.585857in}{1.076553in}}%
\pgfpathlineto{\pgfqpoint{1.586467in}{1.076164in}}%
\pgfpathlineto{\pgfqpoint{1.587993in}{1.076435in}}%
\pgfpathlineto{\pgfqpoint{1.590739in}{1.078281in}}%
\pgfpathlineto{\pgfqpoint{1.592874in}{1.075660in}}%
\pgfpathlineto{\pgfqpoint{1.595010in}{1.073944in}}%
\pgfpathlineto{\pgfqpoint{1.595315in}{1.074138in}}%
\pgfpathlineto{\pgfqpoint{1.597451in}{1.074261in}}%
\pgfpathlineto{\pgfqpoint{1.601723in}{1.069740in}}%
\pgfpathlineto{\pgfqpoint{1.602028in}{1.070136in}}%
\pgfpathlineto{\pgfqpoint{1.606605in}{1.075840in}}%
\pgfpathlineto{\pgfqpoint{1.612097in}{1.074391in}}%
\pgfpathlineto{\pgfqpoint{1.615758in}{1.079428in}}%
\pgfpathlineto{\pgfqpoint{1.616369in}{1.078749in}}%
\pgfpathlineto{\pgfqpoint{1.618809in}{1.078387in}}%
\pgfpathlineto{\pgfqpoint{1.622776in}{1.075211in}}%
\pgfpathlineto{\pgfqpoint{1.625827in}{1.073138in}}%
\pgfpathlineto{\pgfqpoint{1.626132in}{1.073493in}}%
\pgfpathlineto{\pgfqpoint{1.631930in}{1.079406in}}%
\pgfpathlineto{\pgfqpoint{1.640778in}{1.083874in}}%
\pgfpathlineto{\pgfqpoint{1.644744in}{1.079096in}}%
\pgfpathlineto{\pgfqpoint{1.645660in}{1.079994in}}%
\pgfpathlineto{\pgfqpoint{1.650847in}{1.083565in}}%
\pgfpathlineto{\pgfqpoint{1.653898in}{1.086238in}}%
\pgfpathlineto{\pgfqpoint{1.655424in}{1.087159in}}%
\pgfpathlineto{\pgfqpoint{1.655729in}{1.086928in}}%
\pgfpathlineto{\pgfqpoint{1.658170in}{1.081893in}}%
\pgfpathlineto{\pgfqpoint{1.659085in}{1.080751in}}%
\pgfpathlineto{\pgfqpoint{1.659695in}{1.081410in}}%
\pgfpathlineto{\pgfqpoint{1.663662in}{1.093170in}}%
\pgfpathlineto{\pgfqpoint{1.664882in}{1.090706in}}%
\pgfpathlineto{\pgfqpoint{1.668849in}{1.084375in}}%
\pgfpathlineto{\pgfqpoint{1.679223in}{1.085629in}}%
\pgfpathlineto{\pgfqpoint{1.680748in}{1.086157in}}%
\pgfpathlineto{\pgfqpoint{1.687461in}{1.092298in}}%
\pgfpathlineto{\pgfqpoint{1.688071in}{1.091387in}}%
\pgfpathlineto{\pgfqpoint{1.692953in}{1.081197in}}%
\pgfpathlineto{\pgfqpoint{1.693869in}{1.082429in}}%
\pgfpathlineto{\pgfqpoint{1.697225in}{1.090065in}}%
\pgfpathlineto{\pgfqpoint{1.697835in}{1.089647in}}%
\pgfpathlineto{\pgfqpoint{1.700276in}{1.084980in}}%
\pgfpathlineto{\pgfqpoint{1.702107in}{1.082872in}}%
\pgfpathlineto{\pgfqpoint{1.702412in}{1.083076in}}%
\pgfpathlineto{\pgfqpoint{1.705158in}{1.087926in}}%
\pgfpathlineto{\pgfqpoint{1.707904in}{1.089211in}}%
\pgfpathlineto{\pgfqpoint{1.711260in}{1.090581in}}%
\pgfpathlineto{\pgfqpoint{1.716447in}{1.087340in}}%
\pgfpathlineto{\pgfqpoint{1.716752in}{1.087737in}}%
\pgfpathlineto{\pgfqpoint{1.720414in}{1.090254in}}%
\pgfpathlineto{\pgfqpoint{1.722244in}{1.089544in}}%
\pgfpathlineto{\pgfqpoint{1.727126in}{1.084732in}}%
\pgfpathlineto{\pgfqpoint{1.727737in}{1.085320in}}%
\pgfpathlineto{\pgfqpoint{1.732619in}{1.089175in}}%
\pgfpathlineto{\pgfqpoint{1.732924in}{1.088834in}}%
\pgfpathlineto{\pgfqpoint{1.736890in}{1.085779in}}%
\pgfpathlineto{\pgfqpoint{1.739026in}{1.086410in}}%
\pgfpathlineto{\pgfqpoint{1.741467in}{1.086665in}}%
\pgfpathlineto{\pgfqpoint{1.743908in}{1.086305in}}%
\pgfpathlineto{\pgfqpoint{1.746959in}{1.089024in}}%
\pgfpathlineto{\pgfqpoint{1.747569in}{1.088161in}}%
\pgfpathlineto{\pgfqpoint{1.751231in}{1.083485in}}%
\pgfpathlineto{\pgfqpoint{1.752756in}{1.086277in}}%
\pgfpathlineto{\pgfqpoint{1.755197in}{1.089420in}}%
\pgfpathlineto{\pgfqpoint{1.757638in}{1.087443in}}%
\pgfpathlineto{\pgfqpoint{1.760994in}{1.086729in}}%
\pgfpathlineto{\pgfqpoint{1.764046in}{1.089037in}}%
\pgfpathlineto{\pgfqpoint{1.765571in}{1.089630in}}%
\pgfpathlineto{\pgfqpoint{1.765876in}{1.089364in}}%
\pgfpathlineto{\pgfqpoint{1.768317in}{1.088799in}}%
\pgfpathlineto{\pgfqpoint{1.771369in}{1.088169in}}%
\pgfpathlineto{\pgfqpoint{1.775945in}{1.085529in}}%
\pgfpathlineto{\pgfqpoint{1.780217in}{1.088216in}}%
\pgfpathlineto{\pgfqpoint{1.786319in}{1.087682in}}%
\pgfpathlineto{\pgfqpoint{1.789370in}{1.090251in}}%
\pgfpathlineto{\pgfqpoint{1.790591in}{1.088718in}}%
\pgfpathlineto{\pgfqpoint{1.793337in}{1.084412in}}%
\pgfpathlineto{\pgfqpoint{1.793642in}{1.084635in}}%
\pgfpathlineto{\pgfqpoint{1.796693in}{1.087703in}}%
\pgfpathlineto{\pgfqpoint{1.797304in}{1.086744in}}%
\pgfpathlineto{\pgfqpoint{1.800660in}{1.080692in}}%
\pgfpathlineto{\pgfqpoint{1.801270in}{1.081249in}}%
\pgfpathlineto{\pgfqpoint{1.804626in}{1.082575in}}%
\pgfpathlineto{\pgfqpoint{1.808898in}{1.081386in}}%
\pgfpathlineto{\pgfqpoint{1.813475in}{1.088921in}}%
\pgfpathlineto{\pgfqpoint{1.814695in}{1.087531in}}%
\pgfpathlineto{\pgfqpoint{1.816831in}{1.087191in}}%
\pgfpathlineto{\pgfqpoint{1.822018in}{1.086623in}}%
\pgfpathlineto{\pgfqpoint{1.828426in}{1.081085in}}%
\pgfpathlineto{\pgfqpoint{1.830561in}{1.084399in}}%
\pgfpathlineto{\pgfqpoint{1.832087in}{1.085212in}}%
\pgfpathlineto{\pgfqpoint{1.832392in}{1.084968in}}%
\pgfpathlineto{\pgfqpoint{1.835443in}{1.083839in}}%
\pgfpathlineto{\pgfqpoint{1.845512in}{1.088494in}}%
\pgfpathlineto{\pgfqpoint{1.847953in}{1.078915in}}%
\pgfpathlineto{\pgfqpoint{1.850089in}{1.076194in}}%
\pgfpathlineto{\pgfqpoint{1.851920in}{1.078845in}}%
\pgfpathlineto{\pgfqpoint{1.854971in}{1.082184in}}%
\pgfpathlineto{\pgfqpoint{1.856496in}{1.080475in}}%
\pgfpathlineto{\pgfqpoint{1.858022in}{1.079216in}}%
\pgfpathlineto{\pgfqpoint{1.858327in}{1.079526in}}%
\pgfpathlineto{\pgfqpoint{1.865650in}{1.088199in}}%
\pgfpathlineto{\pgfqpoint{1.866260in}{1.087701in}}%
\pgfpathlineto{\pgfqpoint{1.869006in}{1.083200in}}%
\pgfpathlineto{\pgfqpoint{1.872363in}{1.079686in}}%
\pgfpathlineto{\pgfqpoint{1.875719in}{1.079870in}}%
\pgfpathlineto{\pgfqpoint{1.879685in}{1.082423in}}%
\pgfpathlineto{\pgfqpoint{1.883347in}{1.080688in}}%
\pgfpathlineto{\pgfqpoint{1.885483in}{1.085253in}}%
\pgfpathlineto{\pgfqpoint{1.888534in}{1.088946in}}%
\pgfpathlineto{\pgfqpoint{1.890670in}{1.087634in}}%
\pgfpathlineto{\pgfqpoint{1.894636in}{1.084722in}}%
\pgfpathlineto{\pgfqpoint{1.896772in}{1.084516in}}%
\pgfpathlineto{\pgfqpoint{1.901654in}{1.081266in}}%
\pgfpathlineto{\pgfqpoint{1.905620in}{1.082024in}}%
\pgfpathlineto{\pgfqpoint{1.907146in}{1.082251in}}%
\pgfpathlineto{\pgfqpoint{1.913859in}{1.088270in}}%
\pgfpathlineto{\pgfqpoint{1.916605in}{1.086475in}}%
\pgfpathlineto{\pgfqpoint{1.923622in}{1.079086in}}%
\pgfpathlineto{\pgfqpoint{1.926369in}{1.080704in}}%
\pgfpathlineto{\pgfqpoint{1.929420in}{1.081047in}}%
\pgfpathlineto{\pgfqpoint{1.931556in}{1.080862in}}%
\pgfpathlineto{\pgfqpoint{1.935522in}{1.085252in}}%
\pgfpathlineto{\pgfqpoint{1.938573in}{1.089186in}}%
\pgfpathlineto{\pgfqpoint{1.939489in}{1.088317in}}%
\pgfpathlineto{\pgfqpoint{1.949252in}{1.080486in}}%
\pgfpathlineto{\pgfqpoint{1.952914in}{1.082579in}}%
\pgfpathlineto{\pgfqpoint{1.953524in}{1.081491in}}%
\pgfpathlineto{\pgfqpoint{1.954744in}{1.081044in}}%
\pgfpathlineto{\pgfqpoint{1.955050in}{1.081341in}}%
\pgfpathlineto{\pgfqpoint{1.962067in}{1.088296in}}%
\pgfpathlineto{\pgfqpoint{1.963593in}{1.085974in}}%
\pgfpathlineto{\pgfqpoint{1.968780in}{1.073818in}}%
\pgfpathlineto{\pgfqpoint{1.969695in}{1.074736in}}%
\pgfpathlineto{\pgfqpoint{1.972746in}{1.077133in}}%
\pgfpathlineto{\pgfqpoint{1.976713in}{1.076931in}}%
\pgfpathlineto{\pgfqpoint{1.983731in}{1.082431in}}%
\pgfpathlineto{\pgfqpoint{1.987087in}{1.085542in}}%
\pgfpathlineto{\pgfqpoint{1.989223in}{1.082908in}}%
\pgfpathlineto{\pgfqpoint{1.992579in}{1.079445in}}%
\pgfpathlineto{\pgfqpoint{1.995325in}{1.077559in}}%
\pgfpathlineto{\pgfqpoint{1.997766in}{1.073676in}}%
\pgfpathlineto{\pgfqpoint{1.998376in}{1.074697in}}%
\pgfpathlineto{\pgfqpoint{2.001428in}{1.078927in}}%
\pgfpathlineto{\pgfqpoint{2.005089in}{1.081520in}}%
\pgfpathlineto{\pgfqpoint{2.008750in}{1.085280in}}%
\pgfpathlineto{\pgfqpoint{2.010276in}{1.084498in}}%
\pgfpathlineto{\pgfqpoint{2.015768in}{1.074923in}}%
\pgfpathlineto{\pgfqpoint{2.016989in}{1.073592in}}%
\pgfpathlineto{\pgfqpoint{2.017599in}{1.074553in}}%
\pgfpathlineto{\pgfqpoint{2.019124in}{1.075733in}}%
\pgfpathlineto{\pgfqpoint{2.019430in}{1.075489in}}%
\pgfpathlineto{\pgfqpoint{2.022786in}{1.074606in}}%
\pgfpathlineto{\pgfqpoint{2.026752in}{1.075983in}}%
\pgfpathlineto{\pgfqpoint{2.033465in}{1.084397in}}%
\pgfpathlineto{\pgfqpoint{2.035296in}{1.083814in}}%
\pgfpathlineto{\pgfqpoint{2.037431in}{1.080503in}}%
\pgfpathlineto{\pgfqpoint{2.041093in}{1.073484in}}%
\pgfpathlineto{\pgfqpoint{2.041398in}{1.073642in}}%
\pgfpathlineto{\pgfqpoint{2.043229in}{1.075041in}}%
\pgfpathlineto{\pgfqpoint{2.043839in}{1.074247in}}%
\pgfpathlineto{\pgfqpoint{2.047500in}{1.071196in}}%
\pgfpathlineto{\pgfqpoint{2.050857in}{1.072274in}}%
\pgfpathlineto{\pgfqpoint{2.057264in}{1.081936in}}%
\pgfpathlineto{\pgfqpoint{2.058790in}{1.080422in}}%
\pgfpathlineto{\pgfqpoint{2.063977in}{1.072237in}}%
\pgfpathlineto{\pgfqpoint{2.065807in}{1.073426in}}%
\pgfpathlineto{\pgfqpoint{2.067943in}{1.073545in}}%
\pgfpathlineto{\pgfqpoint{2.071300in}{1.072467in}}%
\pgfpathlineto{\pgfqpoint{2.074656in}{1.074747in}}%
\pgfpathlineto{\pgfqpoint{2.078012in}{1.078911in}}%
\pgfpathlineto{\pgfqpoint{2.080453in}{1.079546in}}%
\pgfpathlineto{\pgfqpoint{2.085335in}{1.077789in}}%
\pgfpathlineto{\pgfqpoint{2.092048in}{1.070208in}}%
\pgfpathlineto{\pgfqpoint{2.095099in}{1.069611in}}%
\pgfpathlineto{\pgfqpoint{2.096624in}{1.072449in}}%
\pgfpathlineto{\pgfqpoint{2.100591in}{1.078033in}}%
\pgfpathlineto{\pgfqpoint{2.104557in}{1.078613in}}%
\pgfpathlineto{\pgfqpoint{2.110050in}{1.075236in}}%
\pgfpathlineto{\pgfqpoint{2.115542in}{1.072267in}}%
\pgfpathlineto{\pgfqpoint{2.121949in}{1.074666in}}%
\pgfpathlineto{\pgfqpoint{2.125306in}{1.074018in}}%
\pgfpathlineto{\pgfqpoint{2.131408in}{1.077304in}}%
\pgfpathlineto{\pgfqpoint{2.132933in}{1.075713in}}%
\pgfpathlineto{\pgfqpoint{2.138731in}{1.069064in}}%
\pgfpathlineto{\pgfqpoint{2.149105in}{1.073299in}}%
\pgfpathlineto{\pgfqpoint{2.150020in}{1.074115in}}%
\pgfpathlineto{\pgfqpoint{2.150630in}{1.073308in}}%
\pgfpathlineto{\pgfqpoint{2.152156in}{1.071198in}}%
\pgfpathlineto{\pgfqpoint{2.152766in}{1.071593in}}%
\pgfpathlineto{\pgfqpoint{2.155512in}{1.072254in}}%
\pgfpathlineto{\pgfqpoint{2.157038in}{1.069990in}}%
\pgfpathlineto{\pgfqpoint{2.160089in}{1.065467in}}%
\pgfpathlineto{\pgfqpoint{2.161615in}{1.064281in}}%
\pgfpathlineto{\pgfqpoint{2.162225in}{1.064993in}}%
\pgfpathlineto{\pgfqpoint{2.165276in}{1.066608in}}%
\pgfpathlineto{\pgfqpoint{2.169548in}{1.069937in}}%
\pgfpathlineto{\pgfqpoint{2.170158in}{1.068997in}}%
\pgfpathlineto{\pgfqpoint{2.172599in}{1.068314in}}%
\pgfpathlineto{\pgfqpoint{2.178701in}{1.069177in}}%
\pgfpathlineto{\pgfqpoint{2.183278in}{1.065744in}}%
\pgfpathlineto{\pgfqpoint{2.189991in}{1.065740in}}%
\pgfpathlineto{\pgfqpoint{2.195178in}{1.067033in}}%
\pgfpathlineto{\pgfqpoint{2.197008in}{1.068975in}}%
\pgfpathlineto{\pgfqpoint{2.199449in}{1.069866in}}%
\pgfpathlineto{\pgfqpoint{2.211349in}{1.063401in}}%
\pgfpathlineto{\pgfqpoint{2.216841in}{1.067605in}}%
\pgfpathlineto{\pgfqpoint{2.218672in}{1.065406in}}%
\pgfpathlineto{\pgfqpoint{2.221418in}{1.063622in}}%
\pgfpathlineto{\pgfqpoint{2.223554in}{1.064018in}}%
\pgfpathlineto{\pgfqpoint{2.225384in}{1.062413in}}%
\pgfpathlineto{\pgfqpoint{2.225994in}{1.063017in}}%
\pgfpathlineto{\pgfqpoint{2.228435in}{1.064389in}}%
\pgfpathlineto{\pgfqpoint{2.231792in}{1.064019in}}%
\pgfpathlineto{\pgfqpoint{2.235758in}{1.064307in}}%
\pgfpathlineto{\pgfqpoint{2.238199in}{1.064408in}}%
\pgfpathlineto{\pgfqpoint{2.245217in}{1.063080in}}%
\pgfpathlineto{\pgfqpoint{2.248268in}{1.063247in}}%
\pgfpathlineto{\pgfqpoint{2.251319in}{1.060128in}}%
\pgfpathlineto{\pgfqpoint{2.254370in}{1.058967in}}%
\pgfpathlineto{\pgfqpoint{2.255896in}{1.059217in}}%
\pgfpathlineto{\pgfqpoint{2.263219in}{1.065215in}}%
\pgfpathlineto{\pgfqpoint{2.266575in}{1.064465in}}%
\pgfpathlineto{\pgfqpoint{2.269626in}{1.062771in}}%
\pgfpathlineto{\pgfqpoint{2.271152in}{1.065078in}}%
\pgfpathlineto{\pgfqpoint{2.271762in}{1.064101in}}%
\pgfpathlineto{\pgfqpoint{2.274203in}{1.061555in}}%
\pgfpathlineto{\pgfqpoint{2.276644in}{1.061932in}}%
\pgfpathlineto{\pgfqpoint{2.282746in}{1.065348in}}%
\pgfpathlineto{\pgfqpoint{2.287628in}{1.064446in}}%
\pgfpathlineto{\pgfqpoint{2.289764in}{1.064456in}}%
\pgfpathlineto{\pgfqpoint{2.293426in}{1.062483in}}%
\pgfpathlineto{\pgfqpoint{2.295867in}{1.063290in}}%
\pgfpathlineto{\pgfqpoint{2.296172in}{1.062831in}}%
\pgfpathlineto{\pgfqpoint{2.299223in}{1.059472in}}%
\pgfpathlineto{\pgfqpoint{2.299528in}{1.059688in}}%
\pgfpathlineto{\pgfqpoint{2.311122in}{1.072117in}}%
\pgfpathlineto{\pgfqpoint{2.319056in}{1.066648in}}%
\pgfpathlineto{\pgfqpoint{2.323632in}{1.064003in}}%
\pgfpathlineto{\pgfqpoint{2.323937in}{1.064418in}}%
\pgfpathlineto{\pgfqpoint{2.325768in}{1.067713in}}%
\pgfpathlineto{\pgfqpoint{2.326683in}{1.066929in}}%
\pgfpathlineto{\pgfqpoint{2.330650in}{1.066468in}}%
\pgfpathlineto{\pgfqpoint{2.334617in}{1.071020in}}%
\pgfpathlineto{\pgfqpoint{2.335837in}{1.069095in}}%
\pgfpathlineto{\pgfqpoint{2.339193in}{1.065004in}}%
\pgfpathlineto{\pgfqpoint{2.342244in}{1.066224in}}%
\pgfpathlineto{\pgfqpoint{2.348347in}{1.061240in}}%
\pgfpathlineto{\pgfqpoint{2.348957in}{1.062098in}}%
\pgfpathlineto{\pgfqpoint{2.355670in}{1.074357in}}%
\pgfpathlineto{\pgfqpoint{2.356890in}{1.072172in}}%
\pgfpathlineto{\pgfqpoint{2.358416in}{1.072828in}}%
\pgfpathlineto{\pgfqpoint{2.359636in}{1.071683in}}%
\pgfpathlineto{\pgfqpoint{2.364518in}{1.066789in}}%
\pgfpathlineto{\pgfqpoint{2.369095in}{1.064639in}}%
\pgfpathlineto{\pgfqpoint{2.371536in}{1.063267in}}%
\pgfpathlineto{\pgfqpoint{2.373061in}{1.066532in}}%
\pgfpathlineto{\pgfqpoint{2.377028in}{1.073761in}}%
\pgfpathlineto{\pgfqpoint{2.382520in}{1.070324in}}%
\pgfpathlineto{\pgfqpoint{2.384351in}{1.069614in}}%
\pgfpathlineto{\pgfqpoint{2.387097in}{1.067890in}}%
\pgfpathlineto{\pgfqpoint{2.391369in}{1.064009in}}%
\pgfpathlineto{\pgfqpoint{2.396861in}{1.065131in}}%
\pgfpathlineto{\pgfqpoint{2.402963in}{1.073281in}}%
\pgfpathlineto{\pgfqpoint{2.406014in}{1.071458in}}%
\pgfpathlineto{\pgfqpoint{2.406624in}{1.072605in}}%
\pgfpathlineto{\pgfqpoint{2.408760in}{1.073388in}}%
\pgfpathlineto{\pgfqpoint{2.410286in}{1.071702in}}%
\pgfpathlineto{\pgfqpoint{2.415473in}{1.061752in}}%
\pgfpathlineto{\pgfqpoint{2.415778in}{1.061891in}}%
\pgfpathlineto{\pgfqpoint{2.418829in}{1.065364in}}%
\pgfpathlineto{\pgfqpoint{2.422185in}{1.066608in}}%
\pgfpathlineto{\pgfqpoint{2.423711in}{1.066005in}}%
\pgfpathlineto{\pgfqpoint{2.424016in}{1.066534in}}%
\pgfpathlineto{\pgfqpoint{2.426762in}{1.069666in}}%
\pgfpathlineto{\pgfqpoint{2.431949in}{1.071617in}}%
\pgfpathlineto{\pgfqpoint{2.438967in}{1.061281in}}%
\pgfpathlineto{\pgfqpoint{2.439577in}{1.061756in}}%
\pgfpathlineto{\pgfqpoint{2.442323in}{1.065723in}}%
\pgfpathlineto{\pgfqpoint{2.443849in}{1.067832in}}%
\pgfpathlineto{\pgfqpoint{2.444459in}{1.067608in}}%
\pgfpathlineto{\pgfqpoint{2.448120in}{1.067928in}}%
\pgfpathlineto{\pgfqpoint{2.452087in}{1.074474in}}%
\pgfpathlineto{\pgfqpoint{2.453918in}{1.071492in}}%
\pgfpathlineto{\pgfqpoint{2.457274in}{1.066169in}}%
\pgfpathlineto{\pgfqpoint{2.457579in}{1.066311in}}%
\pgfpathlineto{\pgfqpoint{2.459410in}{1.065566in}}%
\pgfpathlineto{\pgfqpoint{2.463681in}{1.062174in}}%
\pgfpathlineto{\pgfqpoint{2.475581in}{1.071705in}}%
\pgfpathlineto{\pgfqpoint{2.477107in}{1.070143in}}%
\pgfpathlineto{\pgfqpoint{2.478022in}{1.069176in}}%
\pgfpathlineto{\pgfqpoint{2.478632in}{1.069769in}}%
\pgfpathlineto{\pgfqpoint{2.480768in}{1.070328in}}%
\pgfpathlineto{\pgfqpoint{2.484124in}{1.069619in}}%
\pgfpathlineto{\pgfqpoint{2.487786in}{1.062252in}}%
\pgfpathlineto{\pgfqpoint{2.488701in}{1.063652in}}%
\pgfpathlineto{\pgfqpoint{2.492057in}{1.068282in}}%
\pgfpathlineto{\pgfqpoint{2.492668in}{1.067835in}}%
\pgfpathlineto{\pgfqpoint{2.494498in}{1.068561in}}%
\pgfpathlineto{\pgfqpoint{2.496634in}{1.071237in}}%
\pgfpathlineto{\pgfqpoint{2.498160in}{1.073267in}}%
\pgfpathlineto{\pgfqpoint{2.498770in}{1.072962in}}%
\pgfpathlineto{\pgfqpoint{2.504567in}{1.068621in}}%
\pgfpathlineto{\pgfqpoint{2.508229in}{1.063445in}}%
\pgfpathlineto{\pgfqpoint{2.510365in}{1.061147in}}%
\pgfpathlineto{\pgfqpoint{2.511280in}{1.059832in}}%
\pgfpathlineto{\pgfqpoint{2.511890in}{1.060407in}}%
\pgfpathlineto{\pgfqpoint{2.517077in}{1.070720in}}%
\pgfpathlineto{\pgfqpoint{2.518298in}{1.069703in}}%
\pgfpathlineto{\pgfqpoint{2.520433in}{1.067748in}}%
\pgfpathlineto{\pgfqpoint{2.521044in}{1.068417in}}%
\pgfpathlineto{\pgfqpoint{2.525620in}{1.072298in}}%
\pgfpathlineto{\pgfqpoint{2.528367in}{1.069281in}}%
\pgfpathlineto{\pgfqpoint{2.532028in}{1.062939in}}%
\pgfpathlineto{\pgfqpoint{2.533554in}{1.063997in}}%
\pgfpathlineto{\pgfqpoint{2.539351in}{1.068399in}}%
\pgfpathlineto{\pgfqpoint{2.541792in}{1.070259in}}%
\pgfpathlineto{\pgfqpoint{2.544538in}{1.070193in}}%
\pgfpathlineto{\pgfqpoint{2.548504in}{1.074669in}}%
\pgfpathlineto{\pgfqpoint{2.550640in}{1.072698in}}%
\pgfpathlineto{\pgfqpoint{2.552166in}{1.069228in}}%
\pgfpathlineto{\pgfqpoint{2.554302in}{1.065478in}}%
\pgfpathlineto{\pgfqpoint{2.554607in}{1.065627in}}%
\pgfpathlineto{\pgfqpoint{2.558878in}{1.066141in}}%
\pgfpathlineto{\pgfqpoint{2.561014in}{1.064970in}}%
\pgfpathlineto{\pgfqpoint{2.562540in}{1.065760in}}%
\pgfpathlineto{\pgfqpoint{2.566201in}{1.070718in}}%
\pgfpathlineto{\pgfqpoint{2.569863in}{1.071918in}}%
\pgfpathlineto{\pgfqpoint{2.573219in}{1.073298in}}%
\pgfpathlineto{\pgfqpoint{2.575965in}{1.073194in}}%
\pgfpathlineto{\pgfqpoint{2.579931in}{1.068984in}}%
\pgfpathlineto{\pgfqpoint{2.582983in}{1.068472in}}%
\pgfpathlineto{\pgfqpoint{2.584203in}{1.068081in}}%
\pgfpathlineto{\pgfqpoint{2.584508in}{1.068361in}}%
\pgfpathlineto{\pgfqpoint{2.591831in}{1.075847in}}%
\pgfpathlineto{\pgfqpoint{2.592441in}{1.075106in}}%
\pgfpathlineto{\pgfqpoint{2.596408in}{1.071578in}}%
\pgfpathlineto{\pgfqpoint{2.596713in}{1.071838in}}%
\pgfpathlineto{\pgfqpoint{2.599764in}{1.072784in}}%
\pgfpathlineto{\pgfqpoint{2.604646in}{1.069509in}}%
\pgfpathlineto{\pgfqpoint{2.610138in}{1.070924in}}%
\pgfpathlineto{\pgfqpoint{2.612274in}{1.071077in}}%
\pgfpathlineto{\pgfqpoint{2.617461in}{1.073107in}}%
\pgfpathlineto{\pgfqpoint{2.620817in}{1.072819in}}%
\pgfpathlineto{\pgfqpoint{2.622953in}{1.072931in}}%
\pgfpathlineto{\pgfqpoint{2.624784in}{1.071095in}}%
\pgfpathlineto{\pgfqpoint{2.626309in}{1.069647in}}%
\pgfpathlineto{\pgfqpoint{2.626920in}{1.070117in}}%
\pgfpathlineto{\pgfqpoint{2.630276in}{1.070412in}}%
\pgfpathlineto{\pgfqpoint{2.632412in}{1.071564in}}%
\pgfpathlineto{\pgfqpoint{2.635463in}{1.072668in}}%
\pgfpathlineto{\pgfqpoint{2.637294in}{1.072818in}}%
\pgfpathlineto{\pgfqpoint{2.640955in}{1.074812in}}%
\pgfpathlineto{\pgfqpoint{2.653770in}{1.072725in}}%
\pgfpathlineto{\pgfqpoint{2.657431in}{1.074986in}}%
\pgfpathlineto{\pgfqpoint{2.659262in}{1.072338in}}%
\pgfpathlineto{\pgfqpoint{2.660178in}{1.072063in}}%
\pgfpathlineto{\pgfqpoint{2.660483in}{1.072530in}}%
\pgfpathlineto{\pgfqpoint{2.663839in}{1.080162in}}%
\pgfpathlineto{\pgfqpoint{2.664754in}{1.079018in}}%
\pgfpathlineto{\pgfqpoint{2.668111in}{1.073439in}}%
\pgfpathlineto{\pgfqpoint{2.668416in}{1.073598in}}%
\pgfpathlineto{\pgfqpoint{2.672993in}{1.074925in}}%
\pgfpathlineto{\pgfqpoint{2.674823in}{1.075275in}}%
\pgfpathlineto{\pgfqpoint{2.677264in}{1.075167in}}%
\pgfpathlineto{\pgfqpoint{2.680315in}{1.073160in}}%
\pgfpathlineto{\pgfqpoint{2.680620in}{1.073548in}}%
\pgfpathlineto{\pgfqpoint{2.684587in}{1.077923in}}%
\pgfpathlineto{\pgfqpoint{2.684892in}{1.077703in}}%
\pgfpathlineto{\pgfqpoint{2.689164in}{1.075349in}}%
\pgfpathlineto{\pgfqpoint{2.693130in}{1.073836in}}%
\pgfpathlineto{\pgfqpoint{2.693435in}{1.074333in}}%
\pgfpathlineto{\pgfqpoint{2.696792in}{1.077776in}}%
\pgfpathlineto{\pgfqpoint{2.700453in}{1.075943in}}%
\pgfpathlineto{\pgfqpoint{2.706250in}{1.078245in}}%
\pgfpathlineto{\pgfqpoint{2.709302in}{1.074797in}}%
\pgfpathlineto{\pgfqpoint{2.709912in}{1.075535in}}%
\pgfpathlineto{\pgfqpoint{2.712048in}{1.077150in}}%
\pgfpathlineto{\pgfqpoint{2.712353in}{1.076963in}}%
\pgfpathlineto{\pgfqpoint{2.717845in}{1.074317in}}%
\pgfpathlineto{\pgfqpoint{2.726998in}{1.076363in}}%
\pgfpathlineto{\pgfqpoint{2.728524in}{1.075308in}}%
\pgfpathlineto{\pgfqpoint{2.728829in}{1.075618in}}%
\pgfpathlineto{\pgfqpoint{2.731880in}{1.080150in}}%
\pgfpathlineto{\pgfqpoint{2.732491in}{1.079576in}}%
\pgfpathlineto{\pgfqpoint{2.737067in}{1.075328in}}%
\pgfpathlineto{\pgfqpoint{2.739813in}{1.074322in}}%
\pgfpathlineto{\pgfqpoint{2.742865in}{1.071572in}}%
\pgfpathlineto{\pgfqpoint{2.743475in}{1.072680in}}%
\pgfpathlineto{\pgfqpoint{2.746831in}{1.077446in}}%
\pgfpathlineto{\pgfqpoint{2.751408in}{1.077283in}}%
\pgfpathlineto{\pgfqpoint{2.753849in}{1.079731in}}%
\pgfpathlineto{\pgfqpoint{2.754154in}{1.079267in}}%
\pgfpathlineto{\pgfqpoint{2.757815in}{1.074525in}}%
\pgfpathlineto{\pgfqpoint{2.758120in}{1.074682in}}%
\pgfpathlineto{\pgfqpoint{2.760561in}{1.074480in}}%
\pgfpathlineto{\pgfqpoint{2.765138in}{1.071993in}}%
\pgfpathlineto{\pgfqpoint{2.767274in}{1.073471in}}%
\pgfpathlineto{\pgfqpoint{2.771546in}{1.077780in}}%
\pgfpathlineto{\pgfqpoint{2.779784in}{1.073833in}}%
\pgfpathlineto{\pgfqpoint{2.784056in}{1.072812in}}%
\pgfpathlineto{\pgfqpoint{2.786802in}{1.069516in}}%
\pgfpathlineto{\pgfqpoint{2.790158in}{1.072150in}}%
\pgfpathlineto{\pgfqpoint{2.799006in}{1.075925in}}%
\pgfpathlineto{\pgfqpoint{2.801447in}{1.075070in}}%
\pgfpathlineto{\pgfqpoint{2.806939in}{1.073000in}}%
\pgfpathlineto{\pgfqpoint{2.809380in}{1.067964in}}%
\pgfpathlineto{\pgfqpoint{2.809991in}{1.068565in}}%
\pgfpathlineto{\pgfqpoint{2.812431in}{1.074354in}}%
\pgfpathlineto{\pgfqpoint{2.813347in}{1.073040in}}%
\pgfpathlineto{\pgfqpoint{2.815788in}{1.068664in}}%
\pgfpathlineto{\pgfqpoint{2.816093in}{1.068961in}}%
\pgfpathlineto{\pgfqpoint{2.819449in}{1.076244in}}%
\pgfpathlineto{\pgfqpoint{2.820670in}{1.074757in}}%
\pgfpathlineto{\pgfqpoint{2.823111in}{1.070585in}}%
\pgfpathlineto{\pgfqpoint{2.823721in}{1.071382in}}%
\pgfpathlineto{\pgfqpoint{2.826467in}{1.075359in}}%
\pgfpathlineto{\pgfqpoint{2.826772in}{1.075086in}}%
\pgfpathlineto{\pgfqpoint{2.830128in}{1.072797in}}%
\pgfpathlineto{\pgfqpoint{2.831959in}{1.073300in}}%
\pgfpathlineto{\pgfqpoint{2.832264in}{1.072853in}}%
\pgfpathlineto{\pgfqpoint{2.835926in}{1.069153in}}%
\pgfpathlineto{\pgfqpoint{2.837451in}{1.072369in}}%
\pgfpathlineto{\pgfqpoint{2.839587in}{1.074113in}}%
\pgfpathlineto{\pgfqpoint{2.843554in}{1.075635in}}%
\pgfpathlineto{\pgfqpoint{2.845079in}{1.076057in}}%
\pgfpathlineto{\pgfqpoint{2.845384in}{1.075716in}}%
\pgfpathlineto{\pgfqpoint{2.848741in}{1.073447in}}%
\pgfpathlineto{\pgfqpoint{2.851487in}{1.074997in}}%
\pgfpathlineto{\pgfqpoint{2.851792in}{1.074581in}}%
\pgfpathlineto{\pgfqpoint{2.856979in}{1.067987in}}%
\pgfpathlineto{\pgfqpoint{2.860030in}{1.071319in}}%
\pgfpathlineto{\pgfqpoint{2.861861in}{1.072642in}}%
\pgfpathlineto{\pgfqpoint{2.862166in}{1.072352in}}%
\pgfpathlineto{\pgfqpoint{2.863386in}{1.071962in}}%
\pgfpathlineto{\pgfqpoint{2.863691in}{1.072324in}}%
\pgfpathlineto{\pgfqpoint{2.867048in}{1.074915in}}%
\pgfpathlineto{\pgfqpoint{2.870709in}{1.074778in}}%
\pgfpathlineto{\pgfqpoint{2.872845in}{1.074528in}}%
\pgfpathlineto{\pgfqpoint{2.875286in}{1.072793in}}%
\pgfpathlineto{\pgfqpoint{2.878337in}{1.068439in}}%
\pgfpathlineto{\pgfqpoint{2.881693in}{1.065907in}}%
\pgfpathlineto{\pgfqpoint{2.883829in}{1.062819in}}%
\pgfpathlineto{\pgfqpoint{2.884134in}{1.063031in}}%
\pgfpathlineto{\pgfqpoint{2.887796in}{1.069029in}}%
\pgfpathlineto{\pgfqpoint{2.890542in}{1.071175in}}%
\pgfpathlineto{\pgfqpoint{2.895119in}{1.077791in}}%
\pgfpathlineto{\pgfqpoint{2.895729in}{1.076727in}}%
\pgfpathlineto{\pgfqpoint{2.900000in}{1.069550in}}%
\pgfpathlineto{\pgfqpoint{2.902136in}{1.067198in}}%
\pgfpathlineto{\pgfqpoint{2.905798in}{1.060593in}}%
\pgfpathlineto{\pgfqpoint{2.906103in}{1.060708in}}%
\pgfpathlineto{\pgfqpoint{2.910374in}{1.064578in}}%
\pgfpathlineto{\pgfqpoint{2.912205in}{1.065534in}}%
\pgfpathlineto{\pgfqpoint{2.915561in}{1.068624in}}%
\pgfpathlineto{\pgfqpoint{2.918918in}{1.069839in}}%
\pgfpathlineto{\pgfqpoint{2.922274in}{1.069576in}}%
\pgfpathlineto{\pgfqpoint{2.930207in}{1.061916in}}%
\pgfpathlineto{\pgfqpoint{2.930512in}{1.062252in}}%
\pgfpathlineto{\pgfqpoint{2.931733in}{1.062873in}}%
\pgfpathlineto{\pgfqpoint{2.932343in}{1.062262in}}%
\pgfpathlineto{\pgfqpoint{2.935089in}{1.062087in}}%
\pgfpathlineto{\pgfqpoint{2.936615in}{1.064692in}}%
\pgfpathlineto{\pgfqpoint{2.939666in}{1.068296in}}%
\pgfpathlineto{\pgfqpoint{2.942107in}{1.069023in}}%
\pgfpathlineto{\pgfqpoint{2.943632in}{1.070234in}}%
\pgfpathlineto{\pgfqpoint{2.943937in}{1.069942in}}%
\pgfpathlineto{\pgfqpoint{2.950345in}{1.061920in}}%
\pgfpathlineto{\pgfqpoint{2.951260in}{1.063030in}}%
\pgfpathlineto{\pgfqpoint{2.954006in}{1.063608in}}%
\pgfpathlineto{\pgfqpoint{2.955837in}{1.062851in}}%
\pgfpathlineto{\pgfqpoint{2.960414in}{1.061739in}}%
\pgfpathlineto{\pgfqpoint{2.968042in}{1.072737in}}%
\pgfpathlineto{\pgfqpoint{2.968652in}{1.072069in}}%
\pgfpathlineto{\pgfqpoint{2.971703in}{1.065489in}}%
\pgfpathlineto{\pgfqpoint{2.972924in}{1.066439in}}%
\pgfpathlineto{\pgfqpoint{2.975975in}{1.066201in}}%
\pgfpathlineto{\pgfqpoint{2.979026in}{1.060984in}}%
\pgfpathlineto{\pgfqpoint{2.980857in}{1.061874in}}%
\pgfpathlineto{\pgfqpoint{2.984823in}{1.065830in}}%
\pgfpathlineto{\pgfqpoint{2.987569in}{1.072278in}}%
\pgfpathlineto{\pgfqpoint{2.988485in}{1.071046in}}%
\pgfpathlineto{\pgfqpoint{2.993367in}{1.063897in}}%
\pgfpathlineto{\pgfqpoint{2.996418in}{1.063397in}}%
\pgfpathlineto{\pgfqpoint{2.998248in}{1.063955in}}%
\pgfpathlineto{\pgfqpoint{3.000994in}{1.063384in}}%
\pgfpathlineto{\pgfqpoint{3.004351in}{1.058779in}}%
\pgfpathlineto{\pgfqpoint{3.005266in}{1.060334in}}%
\pgfpathlineto{\pgfqpoint{3.008317in}{1.065109in}}%
\pgfpathlineto{\pgfqpoint{3.008622in}{1.064951in}}%
\pgfpathlineto{\pgfqpoint{3.011674in}{1.064300in}}%
\pgfpathlineto{\pgfqpoint{3.016556in}{1.067083in}}%
\pgfpathlineto{\pgfqpoint{3.022048in}{1.064036in}}%
\pgfpathlineto{\pgfqpoint{3.025709in}{1.059712in}}%
\pgfpathlineto{\pgfqpoint{3.029065in}{1.061202in}}%
\pgfpathlineto{\pgfqpoint{3.039744in}{1.066219in}}%
\pgfpathlineto{\pgfqpoint{3.051034in}{1.061060in}}%
\pgfpathlineto{\pgfqpoint{3.055916in}{1.065751in}}%
\pgfpathlineto{\pgfqpoint{3.061408in}{1.068174in}}%
\pgfpathlineto{\pgfqpoint{3.063849in}{1.068954in}}%
\pgfpathlineto{\pgfqpoint{3.065680in}{1.065818in}}%
\pgfpathlineto{\pgfqpoint{3.068731in}{1.064895in}}%
\pgfpathlineto{\pgfqpoint{3.071477in}{1.063620in}}%
\pgfpathlineto{\pgfqpoint{3.075443in}{1.061133in}}%
\pgfpathlineto{\pgfqpoint{3.077274in}{1.060527in}}%
\pgfpathlineto{\pgfqpoint{3.077884in}{1.060353in}}%
\pgfpathlineto{\pgfqpoint{3.078494in}{1.061506in}}%
\pgfpathlineto{\pgfqpoint{3.081241in}{1.064967in}}%
\pgfpathlineto{\pgfqpoint{3.086428in}{1.069916in}}%
\pgfpathlineto{\pgfqpoint{3.088869in}{1.067764in}}%
\pgfpathlineto{\pgfqpoint{3.091615in}{1.064669in}}%
\pgfpathlineto{\pgfqpoint{3.093140in}{1.065900in}}%
\pgfpathlineto{\pgfqpoint{3.094361in}{1.066507in}}%
\pgfpathlineto{\pgfqpoint{3.094666in}{1.066003in}}%
\pgfpathlineto{\pgfqpoint{3.098022in}{1.061762in}}%
\pgfpathlineto{\pgfqpoint{3.099243in}{1.063124in}}%
\pgfpathlineto{\pgfqpoint{3.101683in}{1.065862in}}%
\pgfpathlineto{\pgfqpoint{3.103209in}{1.066686in}}%
\pgfpathlineto{\pgfqpoint{3.103819in}{1.065934in}}%
\pgfpathlineto{\pgfqpoint{3.107481in}{1.064406in}}%
\pgfpathlineto{\pgfqpoint{3.109617in}{1.066035in}}%
\pgfpathlineto{\pgfqpoint{3.112668in}{1.066815in}}%
\pgfpathlineto{\pgfqpoint{3.115719in}{1.065043in}}%
\pgfpathlineto{\pgfqpoint{3.122126in}{1.055299in}}%
\pgfpathlineto{\pgfqpoint{3.123347in}{1.058038in}}%
\pgfpathlineto{\pgfqpoint{3.124262in}{1.059808in}}%
\pgfpathlineto{\pgfqpoint{3.125178in}{1.058735in}}%
\pgfpathlineto{\pgfqpoint{3.128534in}{1.056572in}}%
\pgfpathlineto{\pgfqpoint{3.133721in}{1.059823in}}%
\pgfpathlineto{\pgfqpoint{3.135246in}{1.061109in}}%
\pgfpathlineto{\pgfqpoint{3.135552in}{1.060789in}}%
\pgfpathlineto{\pgfqpoint{3.140433in}{1.052400in}}%
\pgfpathlineto{\pgfqpoint{3.141044in}{1.053274in}}%
\pgfpathlineto{\pgfqpoint{3.145315in}{1.058466in}}%
\pgfpathlineto{\pgfqpoint{3.147146in}{1.059629in}}%
\pgfpathlineto{\pgfqpoint{3.148672in}{1.060902in}}%
\pgfpathlineto{\pgfqpoint{3.149282in}{1.060252in}}%
\pgfpathlineto{\pgfqpoint{3.152638in}{1.056340in}}%
\pgfpathlineto{\pgfqpoint{3.152943in}{1.056583in}}%
\pgfpathlineto{\pgfqpoint{3.157825in}{1.063132in}}%
\pgfpathlineto{\pgfqpoint{3.158741in}{1.061627in}}%
\pgfpathlineto{\pgfqpoint{3.161181in}{1.059681in}}%
\pgfpathlineto{\pgfqpoint{3.164843in}{1.057609in}}%
\pgfpathlineto{\pgfqpoint{3.168504in}{1.051935in}}%
\pgfpathlineto{\pgfqpoint{3.170335in}{1.051851in}}%
\pgfpathlineto{\pgfqpoint{3.172471in}{1.052341in}}%
\pgfpathlineto{\pgfqpoint{3.175522in}{1.052189in}}%
\pgfpathlineto{\pgfqpoint{3.177048in}{1.053888in}}%
\pgfpathlineto{\pgfqpoint{3.179183in}{1.057091in}}%
\pgfpathlineto{\pgfqpoint{3.179794in}{1.056683in}}%
\pgfpathlineto{\pgfqpoint{3.182845in}{1.056628in}}%
\pgfpathlineto{\pgfqpoint{3.187117in}{1.056513in}}%
\pgfpathlineto{\pgfqpoint{3.189252in}{1.053291in}}%
\pgfpathlineto{\pgfqpoint{3.190168in}{1.052597in}}%
\pgfpathlineto{\pgfqpoint{3.190778in}{1.053544in}}%
\pgfpathlineto{\pgfqpoint{3.193829in}{1.059094in}}%
\pgfpathlineto{\pgfqpoint{3.194134in}{1.058769in}}%
\pgfpathlineto{\pgfqpoint{3.197796in}{1.054967in}}%
\pgfpathlineto{\pgfqpoint{3.206034in}{1.057077in}}%
\pgfpathlineto{\pgfqpoint{3.213662in}{1.056626in}}%
\pgfpathlineto{\pgfqpoint{3.217018in}{1.055824in}}%
\pgfpathlineto{\pgfqpoint{3.223120in}{1.058811in}}%
\pgfpathlineto{\pgfqpoint{3.224646in}{1.057517in}}%
\pgfpathlineto{\pgfqpoint{3.226172in}{1.056195in}}%
\pgfpathlineto{\pgfqpoint{3.226782in}{1.056775in}}%
\pgfpathlineto{\pgfqpoint{3.230748in}{1.059526in}}%
\pgfpathlineto{\pgfqpoint{3.232579in}{1.058311in}}%
\pgfpathlineto{\pgfqpoint{3.233189in}{1.059061in}}%
\pgfpathlineto{\pgfqpoint{3.235630in}{1.060710in}}%
\pgfpathlineto{\pgfqpoint{3.238376in}{1.056648in}}%
\pgfpathlineto{\pgfqpoint{3.239597in}{1.056490in}}%
\pgfpathlineto{\pgfqpoint{3.239902in}{1.056941in}}%
\pgfpathlineto{\pgfqpoint{3.244174in}{1.065385in}}%
\pgfpathlineto{\pgfqpoint{3.244784in}{1.064685in}}%
\pgfpathlineto{\pgfqpoint{3.248140in}{1.058439in}}%
\pgfpathlineto{\pgfqpoint{3.248750in}{1.058795in}}%
\pgfpathlineto{\pgfqpoint{3.256073in}{1.060509in}}%
\pgfpathlineto{\pgfqpoint{3.259124in}{1.059889in}}%
\pgfpathlineto{\pgfqpoint{3.264311in}{1.064090in}}%
\pgfpathlineto{\pgfqpoint{3.267057in}{1.067634in}}%
\pgfpathlineto{\pgfqpoint{3.267363in}{1.067397in}}%
\pgfpathlineto{\pgfqpoint{3.271024in}{1.065522in}}%
\pgfpathlineto{\pgfqpoint{3.275601in}{1.065836in}}%
\pgfpathlineto{\pgfqpoint{3.277431in}{1.064817in}}%
\pgfpathlineto{\pgfqpoint{3.278042in}{1.065440in}}%
\pgfpathlineto{\pgfqpoint{3.285670in}{1.069577in}}%
\pgfpathlineto{\pgfqpoint{3.286280in}{1.068346in}}%
\pgfpathlineto{\pgfqpoint{3.288721in}{1.066251in}}%
\pgfpathlineto{\pgfqpoint{3.290246in}{1.068893in}}%
\pgfpathlineto{\pgfqpoint{3.291467in}{1.070122in}}%
\pgfpathlineto{\pgfqpoint{3.292077in}{1.069604in}}%
\pgfpathlineto{\pgfqpoint{3.295433in}{1.065467in}}%
\pgfpathlineto{\pgfqpoint{3.296349in}{1.066626in}}%
\pgfpathlineto{\pgfqpoint{3.297874in}{1.068064in}}%
\pgfpathlineto{\pgfqpoint{3.298485in}{1.067489in}}%
\pgfpathlineto{\pgfqpoint{3.302451in}{1.061957in}}%
\pgfpathlineto{\pgfqpoint{3.303672in}{1.063643in}}%
\pgfpathlineto{\pgfqpoint{3.305197in}{1.064683in}}%
\pgfpathlineto{\pgfqpoint{3.305807in}{1.064203in}}%
\pgfpathlineto{\pgfqpoint{3.309774in}{1.060713in}}%
\pgfpathlineto{\pgfqpoint{3.310079in}{1.061012in}}%
\pgfpathlineto{\pgfqpoint{3.318928in}{1.069174in}}%
\pgfpathlineto{\pgfqpoint{3.327166in}{1.067727in}}%
\pgfpathlineto{\pgfqpoint{3.328996in}{1.069605in}}%
\pgfpathlineto{\pgfqpoint{3.329302in}{1.069383in}}%
\pgfpathlineto{\pgfqpoint{3.332353in}{1.065728in}}%
\pgfpathlineto{\pgfqpoint{3.333573in}{1.067060in}}%
\pgfpathlineto{\pgfqpoint{3.339065in}{1.069655in}}%
\pgfpathlineto{\pgfqpoint{3.344252in}{1.069266in}}%
\pgfpathlineto{\pgfqpoint{3.346388in}{1.065694in}}%
\pgfpathlineto{\pgfqpoint{3.346998in}{1.066387in}}%
\pgfpathlineto{\pgfqpoint{3.349134in}{1.068835in}}%
\pgfpathlineto{\pgfqpoint{3.349744in}{1.067978in}}%
\pgfpathlineto{\pgfqpoint{3.352185in}{1.064872in}}%
\pgfpathlineto{\pgfqpoint{3.352491in}{1.065049in}}%
\pgfpathlineto{\pgfqpoint{3.358593in}{1.068085in}}%
\pgfpathlineto{\pgfqpoint{3.361644in}{1.069100in}}%
\pgfpathlineto{\pgfqpoint{3.363475in}{1.071017in}}%
\pgfpathlineto{\pgfqpoint{3.364085in}{1.070452in}}%
\pgfpathlineto{\pgfqpoint{3.366831in}{1.069280in}}%
\pgfpathlineto{\pgfqpoint{3.370493in}{1.070035in}}%
\pgfpathlineto{\pgfqpoint{3.375680in}{1.063581in}}%
\pgfpathlineto{\pgfqpoint{3.376290in}{1.064375in}}%
\pgfpathlineto{\pgfqpoint{3.379036in}{1.069324in}}%
\pgfpathlineto{\pgfqpoint{3.379646in}{1.068723in}}%
\pgfpathlineto{\pgfqpoint{3.382087in}{1.065869in}}%
\pgfpathlineto{\pgfqpoint{3.382697in}{1.066499in}}%
\pgfpathlineto{\pgfqpoint{3.385443in}{1.068494in}}%
\pgfpathlineto{\pgfqpoint{3.387579in}{1.068686in}}%
\pgfpathlineto{\pgfqpoint{3.391546in}{1.072693in}}%
\pgfpathlineto{\pgfqpoint{3.391851in}{1.072430in}}%
\pgfpathlineto{\pgfqpoint{3.394292in}{1.067331in}}%
\pgfpathlineto{\pgfqpoint{3.397343in}{1.063105in}}%
\pgfpathlineto{\pgfqpoint{3.401615in}{1.066233in}}%
\pgfpathlineto{\pgfqpoint{3.404056in}{1.067065in}}%
\pgfpathlineto{\pgfqpoint{3.408327in}{1.065788in}}%
\pgfpathlineto{\pgfqpoint{3.414124in}{1.071314in}}%
\pgfpathlineto{\pgfqpoint{3.417481in}{1.068271in}}%
\pgfpathlineto{\pgfqpoint{3.420227in}{1.065108in}}%
\pgfpathlineto{\pgfqpoint{3.426024in}{1.065061in}}%
\pgfpathlineto{\pgfqpoint{3.427244in}{1.065656in}}%
\pgfpathlineto{\pgfqpoint{3.427855in}{1.065095in}}%
\pgfpathlineto{\pgfqpoint{3.429991in}{1.063856in}}%
\pgfpathlineto{\pgfqpoint{3.430296in}{1.064151in}}%
\pgfpathlineto{\pgfqpoint{3.435178in}{1.069092in}}%
\pgfpathlineto{\pgfqpoint{3.437008in}{1.068351in}}%
\pgfpathlineto{\pgfqpoint{3.439449in}{1.067882in}}%
\pgfpathlineto{\pgfqpoint{3.441585in}{1.069843in}}%
\pgfpathlineto{\pgfqpoint{3.442195in}{1.068847in}}%
\pgfpathlineto{\pgfqpoint{3.445246in}{1.064136in}}%
\pgfpathlineto{\pgfqpoint{3.445552in}{1.064316in}}%
\pgfpathlineto{\pgfqpoint{3.448603in}{1.065657in}}%
\pgfpathlineto{\pgfqpoint{3.452874in}{1.064710in}}%
\pgfpathlineto{\pgfqpoint{3.459892in}{1.069298in}}%
\pgfpathlineto{\pgfqpoint{3.462028in}{1.074195in}}%
\pgfpathlineto{\pgfqpoint{3.462943in}{1.072868in}}%
\pgfpathlineto{\pgfqpoint{3.469656in}{1.064058in}}%
\pgfpathlineto{\pgfqpoint{3.473317in}{1.065141in}}%
\pgfpathlineto{\pgfqpoint{3.477284in}{1.067851in}}%
\pgfpathlineto{\pgfqpoint{3.480640in}{1.066730in}}%
\pgfpathlineto{\pgfqpoint{3.482166in}{1.068059in}}%
\pgfpathlineto{\pgfqpoint{3.485522in}{1.072540in}}%
\pgfpathlineto{\pgfqpoint{3.487048in}{1.070976in}}%
\pgfpathlineto{\pgfqpoint{3.489489in}{1.069055in}}%
\pgfpathlineto{\pgfqpoint{3.493760in}{1.066896in}}%
\pgfpathlineto{\pgfqpoint{3.496811in}{1.064868in}}%
\pgfpathlineto{\pgfqpoint{3.497117in}{1.065332in}}%
\pgfpathlineto{\pgfqpoint{3.498642in}{1.066584in}}%
\pgfpathlineto{\pgfqpoint{3.498947in}{1.066300in}}%
\pgfpathlineto{\pgfqpoint{3.502304in}{1.064634in}}%
\pgfpathlineto{\pgfqpoint{3.507796in}{1.070298in}}%
\pgfpathlineto{\pgfqpoint{3.510542in}{1.070829in}}%
\pgfpathlineto{\pgfqpoint{3.512983in}{1.070071in}}%
\pgfpathlineto{\pgfqpoint{3.517865in}{1.068998in}}%
\pgfpathlineto{\pgfqpoint{3.519695in}{1.067562in}}%
\pgfpathlineto{\pgfqpoint{3.522746in}{1.063360in}}%
\pgfpathlineto{\pgfqpoint{3.523052in}{1.063541in}}%
\pgfpathlineto{\pgfqpoint{3.524577in}{1.067298in}}%
\pgfpathlineto{\pgfqpoint{3.527628in}{1.072595in}}%
\pgfpathlineto{\pgfqpoint{3.528849in}{1.071637in}}%
\pgfpathlineto{\pgfqpoint{3.530985in}{1.069262in}}%
\pgfpathlineto{\pgfqpoint{3.531290in}{1.069417in}}%
\pgfpathlineto{\pgfqpoint{3.533120in}{1.071694in}}%
\pgfpathlineto{\pgfqpoint{3.534646in}{1.073697in}}%
\pgfpathlineto{\pgfqpoint{3.534951in}{1.073359in}}%
\pgfpathlineto{\pgfqpoint{3.538002in}{1.070955in}}%
\pgfpathlineto{\pgfqpoint{3.540443in}{1.069278in}}%
\pgfpathlineto{\pgfqpoint{3.543800in}{1.065916in}}%
\pgfpathlineto{\pgfqpoint{3.548071in}{1.067002in}}%
\pgfpathlineto{\pgfqpoint{3.551122in}{1.069452in}}%
\pgfpathlineto{\pgfqpoint{3.554479in}{1.069498in}}%
\pgfpathlineto{\pgfqpoint{3.557225in}{1.071042in}}%
\pgfpathlineto{\pgfqpoint{3.560276in}{1.074328in}}%
\pgfpathlineto{\pgfqpoint{3.562107in}{1.071650in}}%
\pgfpathlineto{\pgfqpoint{3.565158in}{1.067821in}}%
\pgfpathlineto{\pgfqpoint{3.567294in}{1.070364in}}%
\pgfpathlineto{\pgfqpoint{3.568209in}{1.070492in}}%
\pgfpathlineto{\pgfqpoint{3.568514in}{1.070104in}}%
\pgfpathlineto{\pgfqpoint{3.571565in}{1.067928in}}%
\pgfpathlineto{\pgfqpoint{3.573396in}{1.069096in}}%
\pgfpathlineto{\pgfqpoint{3.579498in}{1.074637in}}%
\pgfpathlineto{\pgfqpoint{3.582550in}{1.074103in}}%
\pgfpathlineto{\pgfqpoint{3.587126in}{1.069884in}}%
\pgfpathlineto{\pgfqpoint{3.589872in}{1.069477in}}%
\pgfpathlineto{\pgfqpoint{3.594449in}{1.070508in}}%
\pgfpathlineto{\pgfqpoint{3.595975in}{1.069585in}}%
\pgfpathlineto{\pgfqpoint{3.596280in}{1.069897in}}%
\pgfpathlineto{\pgfqpoint{3.599026in}{1.071260in}}%
\pgfpathlineto{\pgfqpoint{3.602077in}{1.071495in}}%
\pgfpathlineto{\pgfqpoint{3.606044in}{1.074825in}}%
\pgfpathlineto{\pgfqpoint{3.609705in}{1.074596in}}%
\pgfpathlineto{\pgfqpoint{3.611841in}{1.072741in}}%
\pgfpathlineto{\pgfqpoint{3.614282in}{1.069163in}}%
\pgfpathlineto{\pgfqpoint{3.615197in}{1.070300in}}%
\pgfpathlineto{\pgfqpoint{3.618248in}{1.071386in}}%
\pgfpathlineto{\pgfqpoint{3.621300in}{1.070315in}}%
\pgfpathlineto{\pgfqpoint{3.621605in}{1.070749in}}%
\pgfpathlineto{\pgfqpoint{3.626181in}{1.075481in}}%
\pgfpathlineto{\pgfqpoint{3.630758in}{1.074483in}}%
\pgfpathlineto{\pgfqpoint{3.633504in}{1.074900in}}%
\pgfpathlineto{\pgfqpoint{3.634725in}{1.075062in}}%
\pgfpathlineto{\pgfqpoint{3.635030in}{1.074607in}}%
\pgfpathlineto{\pgfqpoint{3.637776in}{1.071955in}}%
\pgfpathlineto{\pgfqpoint{3.639607in}{1.073613in}}%
\pgfpathlineto{\pgfqpoint{3.641437in}{1.073968in}}%
\pgfpathlineto{\pgfqpoint{3.646014in}{1.072642in}}%
\pgfpathlineto{\pgfqpoint{3.649981in}{1.072634in}}%
\pgfpathlineto{\pgfqpoint{3.651506in}{1.073380in}}%
\pgfpathlineto{\pgfqpoint{3.656388in}{1.077270in}}%
\pgfpathlineto{\pgfqpoint{3.658829in}{1.076432in}}%
\pgfpathlineto{\pgfqpoint{3.662491in}{1.070976in}}%
\pgfpathlineto{\pgfqpoint{3.664016in}{1.073220in}}%
\pgfpathlineto{\pgfqpoint{3.666762in}{1.076671in}}%
\pgfpathlineto{\pgfqpoint{3.667067in}{1.076380in}}%
\pgfpathlineto{\pgfqpoint{3.671034in}{1.072068in}}%
\pgfpathlineto{\pgfqpoint{3.671339in}{1.072376in}}%
\pgfpathlineto{\pgfqpoint{3.675611in}{1.078390in}}%
\pgfpathlineto{\pgfqpoint{3.676221in}{1.077679in}}%
\pgfpathlineto{\pgfqpoint{3.679272in}{1.074347in}}%
\pgfpathlineto{\pgfqpoint{3.684764in}{1.074448in}}%
\pgfpathlineto{\pgfqpoint{3.687205in}{1.075982in}}%
\pgfpathlineto{\pgfqpoint{3.689036in}{1.077527in}}%
\pgfpathlineto{\pgfqpoint{3.689341in}{1.077308in}}%
\pgfpathlineto{\pgfqpoint{3.692697in}{1.076464in}}%
\pgfpathlineto{\pgfqpoint{3.695138in}{1.074500in}}%
\pgfpathlineto{\pgfqpoint{3.696969in}{1.074705in}}%
\pgfpathlineto{\pgfqpoint{3.702156in}{1.077512in}}%
\pgfpathlineto{\pgfqpoint{3.715276in}{1.073833in}}%
\pgfpathlineto{\pgfqpoint{3.715581in}{1.074174in}}%
\pgfpathlineto{\pgfqpoint{3.715581in}{1.074174in}}%
\pgfusepath{stroke}%
\end{pgfscope}%
\begin{pgfscope}%
\pgfsetrectcap%
\pgfsetmiterjoin%
\pgfsetlinewidth{0.803000pt}%
\definecolor{currentstroke}{rgb}{0.000000,0.000000,0.000000}%
\pgfsetstrokecolor{currentstroke}%
\pgfsetdash{}{0pt}%
\pgfpathmoveto{\pgfqpoint{0.664400in}{0.567611in}}%
\pgfpathlineto{\pgfqpoint{0.664400in}{1.756587in}}%
\pgfusepath{stroke}%
\end{pgfscope}%
\begin{pgfscope}%
\pgfsetrectcap%
\pgfsetmiterjoin%
\pgfsetlinewidth{0.803000pt}%
\definecolor{currentstroke}{rgb}{0.000000,0.000000,0.000000}%
\pgfsetstrokecolor{currentstroke}%
\pgfsetdash{}{0pt}%
\pgfpathmoveto{\pgfqpoint{3.715581in}{0.567611in}}%
\pgfpathlineto{\pgfqpoint{3.715581in}{1.756587in}}%
\pgfusepath{stroke}%
\end{pgfscope}%
\begin{pgfscope}%
\pgfsetrectcap%
\pgfsetmiterjoin%
\pgfsetlinewidth{0.803000pt}%
\definecolor{currentstroke}{rgb}{0.000000,0.000000,0.000000}%
\pgfsetstrokecolor{currentstroke}%
\pgfsetdash{}{0pt}%
\pgfpathmoveto{\pgfqpoint{0.664400in}{0.567611in}}%
\pgfpathlineto{\pgfqpoint{3.715581in}{0.567611in}}%
\pgfusepath{stroke}%
\end{pgfscope}%
\begin{pgfscope}%
\pgfsetrectcap%
\pgfsetmiterjoin%
\pgfsetlinewidth{0.803000pt}%
\definecolor{currentstroke}{rgb}{0.000000,0.000000,0.000000}%
\pgfsetstrokecolor{currentstroke}%
\pgfsetdash{}{0pt}%
\pgfpathmoveto{\pgfqpoint{0.664400in}{1.756587in}}%
\pgfpathlineto{\pgfqpoint{3.715581in}{1.756587in}}%
\pgfusepath{stroke}%
\end{pgfscope}%
\end{pgfpicture}%
\makeatother%
\endgroup%

    \fonte{SIM FUI EU QUE FIZ}
    % \label{fig:my_label}
\end{figure}
\lipsum[2]


\chapter{Graphene and silicene nanodomains in a ultra-thin SiC layer for water splitting and hydrogen storage. A first principle study}

First-principles calculations within the density functional theory (DFT) have been addressed to investigate the energetic stability, electronic and optical properties of graphene and silicene nanodomains in a SiC single layer (h-SiC). We observe that graphene domains form a planar structure and give rise to an occupied and an empty electronic levels inside the h-SiC band gap, leading the h-SiC to present a strong optical absorption peak in the visible region. On the other hand, when a silicene nanodomain is present the system is no longer planar and present a corrugated structure similar to the silicene structure. The silicene nanodomain introduce three empty electronic levels within the band gap, leading the h-SiC with optical absorption in the visible region. These results show that a graphene nanodomain in h-SiC is appropriate for optical devices, while silicene nanodomains form almost sp$^3$ quantum dots. This finding suggest that the graphene and silicene nanodomains in a SiC single layer increase the possibility to use h-SiC to produce new electronic and optical devices as well for energy storage by hydrogen adsorption. In fact, we study the H2 and O2 adsorption on the pristine system and on the nanodomains, we observe that the presence of the nanodomais increase the binding energies of the adsorbed molecules \cite{kremer2020graphene}.
\chapter{Two-dimensional nanodomains as quantum dots models in an ultra-thin hydrogenated SiC layer}

First-principles calculations within the density functional theory (DFT) are addressed to study the energetic stability and the electronic, magnetic, and optical properties of embedded nanodomains (NDs) formed by threefold coordinated Si and C atoms within a hydrogenated silicon carbide (H-SiC) monolayer. The total energy calculations show that these nanodomains have low formation energy and act as two-dimensional quantum dots (2D QDs), giving rise to localized electronic levels inside the H-SiC bandgap. The stability of the QDs is ruled by their size and shape. For NDs where the number of threefold Si and C atoms are the same, the system is a nonmagnetic semiconductor, whereas if the number of threefold coordinated Si and C atoms is different, the system is a magnetic semiconductor with a magnetic moment of 1 $\mu_{\text{B}}$ per unpaired (Si or C) atom present in the QDs. The calculated optical spectra show that there is a strong absorption optical in the visible region, and the position of the optical absorption peaks presents a dependence with the size and shape of the QDs. These findings are in accordance with previous works where 2D SiC QDs were investigated and the results suggest that 2D SiC QDs are potential materials for optical applications. Furthermore, our DFT results can be used to obtain 2D SiC QDs with desirable electronic, magnetic, and optical properties to be employed in nanodevices \cite{kremer2021two}.
%
%% REFERENCIAS
%
\postextual
\bibliography{ref.bib}
%
%% APENDICES 
%
% \begin{apendicesenv}
% \include{apend-KH-teo}
% \end{apendicesenv}
\end{document}
