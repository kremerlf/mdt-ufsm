\chapter{Two-dimensional nanodomains as quantum dots models in an ultra-thin hydrogenated SiC layer}

First-principles calculations within the density functional theory (DFT) are addressed to study the energetic stability and the electronic, magnetic, and optical properties of embedded nanodomains (NDs) formed by threefold coordinated Si and C atoms within a hydrogenated silicon carbide (H-SiC) monolayer. The total energy calculations show that these nanodomains have low formation energy and act as two-dimensional quantum dots (2D QDs), giving rise to localized electronic levels inside the H-SiC bandgap. The stability of the QDs is ruled by their size and shape. For NDs where the number of threefold Si and C atoms are the same, the system is a nonmagnetic semiconductor, whereas if the number of threefold coordinated Si and C atoms is different, the system is a magnetic semiconductor with a magnetic moment of 1 $\mu_{\text{B}}$ per unpaired (Si or C) atom present in the QDs. The calculated optical spectra show that there is a strong absorption optical in the visible region, and the position of the optical absorption peaks presents a dependence with the size and shape of the QDs. These findings are in accordance with previous works where 2D SiC QDs were investigated and the results suggest that 2D SiC QDs are potential materials for optical applications. Furthermore, our DFT results can be used to obtain 2D SiC QDs with desirable electronic, magnetic, and optical properties to be employed in nanodevices \cite{kremer2021two}.